\documentclass{bmstu}

\begin{document}

\chapter*{Философия немецкого экзистенциализма}

\subsubsection*{Небольшое введение про экзистенциализм}

Термин <<экзистенциализм>> ввел датский теолог и религиозный философ Серен Кьеркегор. Он объяснял его как <<отрицание слишком абстрактного или слишком научного подхода>>. Кьеркегор считал правильным для философии концентрироваться на самой <<экзистенции>>~---~человеческом существовании.

Экзистенциализм~---~это индивидуалистская философия, согласно которой люди обладают свободной волей и сами определяют свою судьбу. Представители течения считают, что человек самостоятельно наделяет жизнь смыслом в изначально бессмысленном мире. Никакие моральные и ценностные категории, а также нормы поведения, существующие в обществе, не имеют для экзистенциалистов значения, поскольку все они искусственны. Также, поскольку общество воспринимается ими как конструкт, всякая принадлежность к социальной группе условна. По мнению экзистенциалистов, человек ни на кого не может переложить ответственность за самого себя.

Экзистенциализм~---~это течение европейской философии. Основные его представители в Германии~---~Мартин Хайдеггер (1889–1976) и Карл Ясперс (1883–1969).

\subsubsection*{Мартин Хайдеггер}

Основоположник немецкого экзистенциализма Мартин Хайдеггер (1889–1976) был учеником Гуссерля. Формирование его философского мировоззрения пришлось на сложное время (1910–1920 годы), полное не только внешними событиями~---~время, когда вышел в свет полный Ницше, возвратилось внимание к Гегелю и Шеллингу, были переведены на немецкий Киркегор и Достоевский, началось издание собрания сочинений Дильтея, был заново открыт Гельдерлин, Вернер Гейзенберг сформулировал соотношение неопределенностей в квантовой механике.

Опираясь на гуссерлевскую феноменологию Хайдеггер стремился раскрыть <<смысл бытия>> через рассмотрение человеческого бытия. Появление в 1927 году самой знаменитой книги Хайдеггера <<Бытие и время>> ознаменовало возникновение нового направления с четко выраженной программой.

Хайдеггер считает, что смысл человеческому существованию придает его конечность, временность. Поэтому время должно рассматриваться как самая существенная характеристика бытия (в экзистенциализме вообще проблема времени становится одной из центральных).

Задача заключается в том, чтобы раскрыть связь между бытием, временем и человеком. Утверждая единство времени и бытия, Хайдеггер доказывает, что ничто сущее, кроме человека, не знает о своей конечности, а значит только человеку ведома временность, а с нею и само бытие. Причем время неразрывно связано не просто с бытием, а именно с человеческим бытием. Сущность времени можно раскрыть лишь в его отношении к человеку.

Время, относимое к человеку, Хайдеггер называет <<первоначальным>>, а время, не связанное непосредственно с человеком,~---~<<производным>>. Иными словами, деление им времени на <<первоначальное>> и <<производное>> соответствует различению психологического, переживаемого человеком времени, и объективного, физического времени.

Время переживаемое человеком Хайдеггер называет 
\textbf{<<временностью>>}. Эта временность выступает у него даже не как способ бытия человека, а как само это бытие, то есть превращается им в субъект. Время имеет свой источник в конечности человека, жизнь которого заключена между рождением и смертью. Получается, что не существует мира вне и независимо от человеческого существования. Временность всегда <<наша>> и <<в нас>>, благодаря временности раскрывается бытие.

Временность Хайдеггер характеризует такими свойствами, как конечность, экстатичность, горизонтальность, направленность к смерти. Конечность ее связана с конечностью человеческого существования. Экстатичность заключается в том, что прошлое, настоящее и будущее существуют одновременно и представляют собой <<экстазы>>~---~моменты субъективных переживаний человека, его выходы в прошлое, настоящее и будущее. Горизонтальность означает, что <<первоначальное>> время никак не связано с процессом развития, движения от менее развитого состояния к более развитому. Для него свойственно не развитие, а повторение.

Хайдеггер считает, что в европейской философской традиции происходит абсолютизация одного из моментов времени~---~настоящего. На самом же деле прошлое, настоящее и будущее взаимно проникают друг друга. Прошлое постоянно присутствует и влияет на настоящее и будущее, которые также всегда есть, влияют друг на друга и на прошлое.

Рассматриваемое как настоящее, время раскрывается как <<вечное присутствие>>; в форме будущего оно связано с <<заботой>>, <<страхом>>, <<ожиданием>>. Только сосредоточенность на будущем дает личности подлинное существование. <<Перед>> и <<вперед>>~---~утверждает Мартин Хайдеггер,~---~показывают настоящее как такое, какое вообще впервые делает возможным для присутствия быть так, что речь для него идет о его способности быть. Основанное в настоящем бросание себя на <<ради себя самого>> есть сущностная черта экзистенциальности. Ее первичный смысл есть будущее. Перевес же настоящего приводит к тому, что мир повседневности заслоняет для человека его конечность. Именно понятия <<вина>>, <<совесть>>, <<решимость>>, <<страх>>, <<забота>> выражают духовный опыт личности, чувствующей свою неповторимость, однократность и смертность.

Человеческое существование соединяет собою различные моменты времени, соединяет бытие со временем. Более того, человек, по Хайдеггеру, сам оказывается создателем времени, ибо настоящее и будущее (а следовательно, и прошлое) детерминированы его поведением и планами.

Для обозначения времени человека в совместном его существовании с другими людьми Хайдеггер вводит понятие <<мирового времени>>, которое носит безличный, общественный характер. Мировое или публичное время связано с <<озабоченным>> бытием. Тем не менее в открытом мире, в повседневности это время измеряется независимо от человека и является в отличие от временности бесконечным.

Время вначале дано человеку в повседневной практической деятельности и понимается им исходя из обстоятельств и потребностей этой деятельности. На такой основе становится возможным теоретическое (абстрактное) понимание времени. В ходе деятельного общения человека с миром возникает, согласно Хайдеггеру, <<естественная>> датируемость времени на <<раньше>>, <<теперь>>, <<потом>>.

Но поскольку деятельное общение человека осуществляется с предметами, находящимися в настоящем, то и деятельность человека является только настоящей. Соответственно и все три измерения мирового времени рассматриваются сквозь призму настоящего.

Бытие, понятое в горизонте времени, это и есть история. А так как время для немецкого мыслителя является формой переживания целесообразной индивидуальной деятельности, то историческое становится индивидуально-историческим. <<Мировая история>> для него~---~это <<вторично-историческое>>, а история человека, под которой понимается субъективное переживание человеком своего индивидуального существования, это <<первично-историческое>>. <<Расхожая характеристика времени как бесконечной, уходящей, необратимой череды теперь возникает,~---~по мнению Хайдеггера,~---~из временности падающего присутствия... Потому и история обычно и чаще понимается публично как внутривременное событие>>. Иначе говоря, история включена во время человеческой жизни. Она, таким образом, превращается в биографию.

Бытие человека в мире определяется, по Хайдеггеру, следующими обстоятельствами.

\begin{enumerate}
	\item Личность “заброшена” в мир, не имеющий смысла, кроме того, который она сама ему придает.
	\item Возникновение личности навязано извне.
	\item Личность существует из-за себя и для себя.
	\item Хотя личность и нуждается в других людях, но никакая подлинная связь между ними невозможна (коллектив подавляет личность).
	\item Личность~---~это бытие, направленное к смерти.
\end{enumerate}

Итак, человек свободен и одинок. Перед ним ряд путей. Выбор зависит только от человека. При этом человеческое бытие может быть <<подлинным>> или <<неподлинным>>. В первом случае для него на передний план выдвигается будущее, направленность к смерти, во втором~---~настоящее, <<обреченность вещам>>, повседневности.

Традиционное понимание человека, полагает Хайдеггер, основывалось на объяснении человека по аналогии с вещами, на забвении им своей временности, историчности, конечности. <<Неподлинное>> существование приводит к так называемому объективному взгляду на личность, когда она оказывается вполне заменимой любой другой личностью. Причем <<другой>>~---~это уже не кто-то вполне определенный, а <<любой другой>>, <<другой>> вообще. Возникает фикция среднего, простого человека.

<<Подлинное>> же существование выступает как осознание человеком своей историчности, конечности и свободы. Оно достижимо только перед <<лицом смерти>>. Человек вырывается за пределы неподлинного существования, переживая <<экзистенциальный страх>>. В основе всякого страха лежит страх смерти, раскрывающий перед человеком последнюю перспективу~---~смерть. Посмотреть в глаза смерти~---~это, с точки зрения Хайдеггера, единственное средство вырваться из сферы обыденности и обратиться к самому себе.

<<Бытие и время>> Хайдеггера~---~одно из самых сложных и загадочных произведений мировой философии. В книге отсутствуют конкретные приметы эпохи и более или менее отчетливые характеристики ее политических, экономических, социальных проблем. Автор не называет своих предшественников и своих идейных противников. Хотя, несомненно, развитие им духовной традиции, связанной с именами Экхарта, Киркегора, Дильтея, Зиммеля, Ницше.

Хайдеггер попытался преодолеть, сложившийся в 18--19 веках культ истории (или историцизм), нашедший наиболее яркое выражение в <<Философии истории>> Гегеля. Согласно этой позиции, история сама знает, куда ей идти, то есть несет в самой себе разумную необходимость. Она окупает, оправдывает зло, несправедливость, насилие как действенные орудия прогресса. Что бы ни делали люди, история не может ни потерпеть крушение, ни изменить свое направление.

Полное неприятие Хайдеггером подобной концепции истории и, как следствие, перенос акцента на человека, его самостоятельность и свободу выбора, многое проясняет в его философских построениях.


\subsubsection*{Карл Ясперс}

Другой крупнейший представитель немецкого экзистенциализма~---~Карл Ясперс (1883-1969)~---~начинал свою научную карьеру как врач-психиатр. Получил степень доктора медицины и доктора психологии. Затем становится профессором философии в Гейдельбергском университете. С 1937 по 1945 годы Ясперс был отстранен от преподавания и лишен права издавать свои работы в Германии. В послевоенные годы становится одним из духовных лидеров Германии. Больше всего его занимает вопрос~---~как спасти человечество от тоталитаризма, главной опасности 20 века?

Основные философские произведения Карла Ясперса: <<Психология мировоззрений>> (1919), <<Духовная ситуация времени>> (1931), <<Философия>> (три тома 1931--1932), <<Истоки истории и ее цель>> (1948), <<Философская вера>> (1948) и др.

Экзистенциальная философия Ясперса с самого начала привлекла к себе внимание новой манерой выражения. Он избрал форму свободного размышления и не стремился вывести все содержание мысли из единого общего принципа.

Темой его творчества становятся человек и история. Причем история рассматривается как изначальное измерение человеческого бытия. Чтобы понять историю надо уяснить, что же такое человек? А человеческое существование раскрывается, в свою очередь, через время, через историчность. <<Вместе со скачком в историю,~---~пишет Ясперс,~---~осознается преходящий характер всего. Всему в мире отведено определенное время, и все обречено на гибель. Но только человек знает, что он должен умереть. Наталкиваясь на эту пограничную ситуацию, он познает вечность во времени, историчность как явление бытия, уничтожение времени во времени. Его осознание истории становится тождественным осознанию вечности>>.

Бытие в концепции Ясперса подразделяется на три стороны или три части (так называемая онтологическая триада Ясперса).

\begin{enumerate}
	\item Предметное бытие или <<бытие-в-мире>>, то есть мир явлений, чувственный мир.
	\item Экзистенция, то есть необъективируемая человеческая самость (Я--бытие).
	\item Трансценденция, или непостижимый предел всякого бытия и мышления (для одних~---~это Бог, для других~---~нечто сверхчувственное).
\end{enumerate}

Если наличное бытие человека изучается такими науками, как антропология, психология, социология, история и так далее, то экзистенция в принципе не доступна научному познанию. Познавать можно только то, что является объектом, а экзистенция необъективируема, ее нельзя отождествлять ни с сознанием, ни с духом. <<Экзистенция,~---~считает Ясперс,~---~есть то, что никогда не становится объектом, есть источник моего мышления и действия, о котором я говорю в таком ходе мысли, где ничего не познается; экзистенция это то, что относится к самому себе и тем самым к своей трансценденции>>.

Экзистенция, по мнению Ясперса, неразрывно связана с трансценденцией (или Божеством). Экзистенция и есть свобода, так как только в свободе коренится бытие личности, я. А достигается свобода только через соприкосновение с трансценденцией, выходящей за пределы чувственного мира.

Экзистенция как бытийное ядро личности с особой силой открывается самому человеку в так называемых пограничных ситуациях. Наиболее яркий случай пограничной ситуации~---~это смерть. Когда человек оказывается перед лицом собственной смерти, смерти близкого человека, тогда смерть из абстрактной возможности становится пограничной ситуацией. Не только смерть, но и смертельная болезнь, страдание, вина, борьба тоже ставят личность в пограничную ситуацию, делая неизбежным осознание собственной конечности, вырывая человека из мира повседневности, заботы. Страсти и огорчения теперь обнаруживают свою несущественность.

Лишь по-настоящему пережив хрупкость и конечность своего существования, человек может открыть для себя трансцендентный мир, точнее, его существование, связанное таинственным образом с человеческим бытием.

Ясперс утверждает, что экзистенции разных людей соотносятся в акте <<коммуникации>>, то есть общения в истине, взаимного понимания. Это возможность быть понятым другим. <<Коммуникация>> рассматривается и как критерий истины. Истинно то, что можно сообщить другому, вернее, то сообщение другому, которое объединяет меня с ним, что служит средством единения.

Смысл философии, по Ясперсу,~---~в создании путей общечеловеческой <<коммуникации>> между странами и веками поверх всех границ культурных кругов. Исходя из этого он предпринимает довольно интересную попытку дать толкование мирового исторического процесса с позиций философской веры (<<Истоки истории и ее цель>>). Философская вера, с его точки зрения, находится как бы на границе между верой религиозной и научным знанием, то есть может считаться предшественницей и религии, и науки.

В отличие от очень популярной в Европе первой половины 20 века теории культурных циклов (О.~Шпенглер, А.~Тойнби) Ясперс доказывает, что человечество имеет единое происхождение и единый путь развития. Не отвергая значения экономических факторов, он все же убежден, что история как человеческая реальность определяется в наибольшей степени факторами духовными и прежде всего теми, которые связаны экзистенциальной жизнью, то есть с толкованием трансцендентного.

Таким образом, в полемике со Шпенглером он настаивает на единстве мирового исторического процесса, а в полемике с марксизмом~---~на приоритете его <<духовной составляющей>>.

Обращаясь к линейной схеме истории, Ясперс отказывается видеть ее <<ось>> в боговоплощении (то есть Христе). По его мнению, историческая ось должна иметь значение для всего человечества, тогда как явление Христа значимо только для христиан.

Возникает вопрос~---~возможна ли вера, общая для всего человечества, которая объединяла бы, а не разъединяла разные культурные регионы планеты?

Такую веру не смогли предложить мировые религии~---~ни буддизм, ни христианство, ни ислам, ни иудаизм.

Общей для человечества верой, приходит к выводу Ясперс, может быть только \textbf{философская вера}. Она имеет глубокие корни в исторической традиции, она древнее, чем христианство или ислам.

Поэтому время рождения философской веры~---~это и есть искомая <<ось мировой истории>> или <<осевая эпоха>>. Это~---~время примерно между 800 и 200 годами до н.~э.~(8--3 вв. до н.э.). В этот промежуток времени возникли параллельно в Китае, Индии, Персии, Палестине и Древней Греции духовные движения, которые сформировали тип человека, существующий и поныне. <<Осевая эпоха>>~---~время рождения и мировых религий, пришедших на смену язычеству, и философии, пришедшей на смену мифологии.

Почти одновременно, независимо друг от друга, образовалось несколько духовных центров, внутренне родственных. Их сближал прорыв мифологического мировоззрения. Человек здесь впервые пробудился к ясному, отчетливому мышлению, возникла рефлексия, недоверие к опытному знанию. Пробуждение духа стало началом общей истории человечества, которое до того было разделено на локальные, не связанные между собой культуры.

Согласно Ясперсу, подлинная связь между народами~---~духовная, а не родовая, не природная. Духовное же единство человечества питается из таинственного, трансцендентного источника. <<Тем, что свершилось тогда, что было создано и продумано в то время, человечество живет вплоть до сего дня. В каждом своем порыве люди, вспоминая, обращаются к осевому времени, воспламеняются идеями той эпохи>>.

<<Осевое время>>~---~это священная эпоха мировой истории. А священная история потому и священна, что хотя она и происходит на земле, но корни и смысл ее~---~неземные. Напротив для верующего человека (не только в религиозном смысле) она сама является последней основой того, что совершалось, совершается и свершится на Земле.

Для Ясперса концепция <<осевого времени>>~---~это положение веры и в тоже время допущение разума. Поскольку он признает важное значение разума и науки, он на стороне рационализма и Просвещения, но поскольку он ограничивает знания, чтобы оставить место вере, он против рационализма.

Поэтому, обращаясь к трем основным наукам о человеке~---~социологии, психологии и антропологии,~---~Ясперс предостерегает от претензий на безграничные возможности рационалистического познания и недооценки нравственно-религиозных установлений. К таким теориям он относит в сфере социологии марксизм, в сфере психологии~---~психоанализ Фрейда, в сфере антропологии~---~расовую теорию.

Условием общечеловеческой коммуникации Ясперс считает общий духовный исток всего человечества~---~<<осевую эпоху>> как корень и почву общеисторического бытия. Если человечество отречется от этой своей общности в судьбе и вере, спасающих человека в самых трудных, пограничных ситуациях, то возможность человеческого общения и взаимопонимания оборвется, а это чревато мировой катастрофой, атомным пожаром или экологическим катаклизмом. Следовательно, проблема общечеловеческих ценностей, взаимопонимания, открытости друг другу различных типов обществ, народов, религий - жизненная необходимость.


\subsection*{Типа вывод}

Отсюда и особая роль философии в современном обществе. Она, считает Ясперс, перестает быть делом узких кружков или университетских курсов, она сегодня должна связать всех людей с помощью философской веры, которая будет противоядием против различного рода рационалистических утопий, разрушающих нравственные и культурные традиции. 

\textit{Активное противление абсурду окружающего мира~---~это то, из чего напрямую исходит западная политическая культура второй половины 20 века. Ключевые для экзистенциализма понятия свободы как ответственности и личного выбора как двигателя всех происходящих событий легли в основу студенческих протестов в Париже 1968 года и освободительного движения Мартина Лютера Кинга.}

\end{document}