\documentclass{bmstu}

\begin{document}

\chapter*{Философия французского экзистенциализма}

\subsubsection*{Небольшое введение про экзистенциализм}

Философия экзистенциализма, основоположником которой является датский теолог и религиозный мыслитель Серен Кьеркегор, представляет собой индивидуалистическое направление, предполагающее, что люди обладают свободой воли и самостоятельно определяют свою судьбу. Экзистенциалисты считают, что именно человек придает значение жизни в изначально бессодержательном мире, и не признают значимости моральных и ценностных категорий и общественных норм, считая их созданными искусственно. С их точки зрения, социальная принадлежность условна, и никто не может делегировать другому ответственность за собственную жизнь.

Экзистенциализм~---~это течение европейской философии. Основные его представители во Франции~---~Жан-Поль Сартр (1905--1980), Альбер Камю (1913--1960), Габриэль Марсель (1889--1973) и Симона де Бовуар (1908--1986).

TO DO

\subsection*{Типа вывод}

Философия французского экзистенциализма предлагает уникальный подход к пониманию человеческой свободы, ответственности и смысла жизни. Она акцентирует внимание на индивидуальном опыте и способности человека принимать решения и действовать автономно, подчеркивая важность личностного роста и саморазвития.

Французский экзистенциализм оказал большое влияние на современную философию и культуру. Его идеи продолжают развиваться и обсуждаться в академических кругах и за их пределами.

\end{document}