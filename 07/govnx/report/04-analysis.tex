\chapter{Аналитический раздел}

В данном разделе приведен обзор функциональности и составляющих статического сервера.

\section{Протокол HTTP}
HTTP~---~широко распространённый протокол передачи данных, изначально
предназначенный для передачи гипертекстовых документов, то есть документов, которые могут содержать ссылки, позволяющие организовать переход к другим документам.

Клиенты и серверы взаимодействуют, обмениваясь одиночными сообщениями, а не потоком данных. Сообщения, отправленные клиентом, обычно веб-браузером, называются запросами, а сообщения, отправленные сервером, называются ответами \cite{http}. 

HTTP это клиент-серверный протокол, соединение всегда устанавливается
клиентом. Открыть соединение в HTTP~---~значит установить соединение через соответствующий транспорт, обычно TCP.

\section{Сокеты}
Сокеты~---~универсальное средство взаимодействия параллельных процессов. Универсальность заключается в том, что сокеты используются как и на локальной машине, так и в распределенной системе, в отличие от, например, разделяемой памяти, которая применима только на отдельно стоящей машине~\cite{ryaznu}.

Сокеты создаются системным вызовом \texttt{socket()} со следующими параметрами.

\begin{enumerate}[label=\arabic*)]
	\item \texttt{Family/Domain}~---~домен соединения& Некоторые значения для домена: \texttt{AF\_UNIX}, \texttt{AF\_LOCAL}, \texttt{AF\_INET}, \texttt{AF\_NETLINK}. В данной работе будет использован \texttt{AF\_INET}, как работающий по протоколу IPv4.
	\item \texttt{Type}~---~задает семантику коммуникации. Некоторые типы: \texttt{SOCK\_STREAM}, \texttt{SOCK\_DGRAM}, \texttt{SOCK\_RAW}.
		\item \texttt{Protocol}~---~задает конкретный протокол, который работает с сокетом. Может быть определен автоматически, при заданной семантике. Например, для \texttt{SOCK\_STREAM} используется TCP.
\end{enumerate}

Взаимодействие на сокетах осуществляется по модели <<клиент-сервер>>: сервер предоставляет ресурсы и службы одному или нескольким клиентам, которые обращаются к серверу. 

\section{Мультиплексирование}
Для сокращения времени блокировки сервера в ожидании соединения используется мультиплексирование, так как время установления соединения со многими клиентами меньше, чем с каждым конкретным клиентом в определенной последовательности~\cite{ryaznu}. 

При мультиплексировании происходит обращение к одному системному вызову~---~мультиплексору, который производит опрос соединений. Когда соединение готово, оно фиксируется ядром. В рамках данной работы используется мультиплексор \texttt{epoll}~\cite{manual}.

\section{\texttt{prefork}}

\texttt{prefork} представляет собой выделенный пул процессов (\texttt{fork()}), которые создаются при запуске сервера. Каждый такой процесс может принимать новые соединения и обрабатывать клиентские запросы~\cite{prefork}.