\chapter{Конструкторский раздел}

В данном разделе описан алгоритм работы статического веб-сервера.

\section{Алгоритм работы статического веб-сервера}

Алгоритм работы статического веб-сервера следующий:

\begin{enumerate}[label=\arabic*)]
	\item мультиплексирование с использованием \texttt{epoll\_wait()};
	\item для каждого из полученных обновлений;
	\item если доступный сокет равен основному сокету, то это соединение с клиентом: создание сокета для соединения с помощью \texttt{accept()}, добавление в \texttt{epoll};
	\item иначе обработка HTTP-запроса: парсинг тела запроса, отдача файла клиенту.
\end{enumerate}

\section{Алгоритм работы \texttt{prefork}}

Вышеописанный алгоритм работает в каждом дочернем процессе, и в основном процессе. При этом, из-за изолированных пространстве имен файловых дескрипторов в каждом процессе~\cite{широков2021методические}, каждый процесс обладает своим дескриптором \texttt{epollFD}~\cite{gammo2004comparing,manual} и следит за своим набором сокетов.

Таким образом, алгоритм работы \texttt{prefork} следующий:

\begin{enumerate}[label=\arabic*)]
	\item создание сокета (\texttt{socket()});
	\item назначение адреса сокету (\texttt{bind()});
	\item прослушивание сокета (\texttt{listen()});
	\item цикл по количеству процессов в пуле;
	\item создание процесса (\texttt{fork()});
	\item создание нового дескриптора \texttt{epoll} (\texttt{epoll\_create1()});
	\item добавление в \texttt{epoll} начального сокета (\texttt{epoll\_ctl()});
	\item работа статического веб-сервера.
\end{enumerate}