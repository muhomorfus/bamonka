\chapter{Практический раздел}

\section{Равномерное распределение}

На рис.\ref{fig:1}--\ref{fig:2} представлены графики распределения и плотности случайной величины, распределенной по равномерному закону.

\begin{figure}[H]
    \begin{center}
    \includegraphics[width=0.7\linewidth]{../code/out/uniform/distribution/all.pdf}
    \caption{График функции распределения}
    \label{fig:1}
    \end{center}
\end{figure}

\begin{figure}[H]
    \begin{center}
    \includegraphics[width=0.7\linewidth]{../code/out/uniform/density/all.pdf}
    \caption{График функции плотности}
    \label{fig:2}
    \end{center}
\end{figure}

При меньшей длине интервала, плотность достигает больших значений.

\newpage

\section{Гиперэкспоненциальное распределение}

На рис.\ref{fig:3}--\ref{fig:4} представлены графики распределения и плотности случайной величины, распределенной по гиперэкспоненциальному закону.

\begin{figure}[H]
    \begin{center}
    \includegraphics[width=0.7\linewidth]{../code/out/hyper_exponential/distribution/all.pdf}
    \caption{График функции распределения}
    \label{fig:3}
    \end{center}
\end{figure}

\begin{figure}[H]
    \begin{center}
    \includegraphics[width=0.7\linewidth]{../code/out/hyper_exponential/density/all.pdf}
    \caption{График функции плотности}
    \label{fig:4}
    \end{center}
\end{figure}

Графики зависят не только от параметров экспоненциальных распределений, но и от их вероятностей. Кроме того, при количестве распределений, равном одному, распределение становится экспоненциальным.


\newpage

