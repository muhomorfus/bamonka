\chapter{Теоретический раздел}

\section{Режимы работы}

Всего возможно два режима работы информационного центра. 

\textbf{Режим нормального обслуживания}~---~есть свободные операторы, клиент выбирает свободного оператора с максимальной производительностью.

\textbf{Режим отказа в обслуживании}~---~свободных операторов нет, клиенту отказывается в обслуживании.

\section{Переменные имитационной модели}

\textbf{Эндогенные переменные}: время обработки задания $i$-ым оператором, время решения задания $j$-ым компьютером.

\textbf{Экзогенные переменные}: число обслуженных клиентов и число клиентов, получивших отказ.

\section{Уравнения имитационной модели}

\begin{equation}
	P_{\text{отк}} = \frac{C_{\text{отк}}}{C_{\text{отк}} + C_{\text{обсл}}}.
\end{equation}

\section{Схемы}

\begin{figure}[ht]
    \centering
    \includegraphics[width=0.7\textwidth]{assets/struct.pdf}
    \caption{Структурная схема модели}
    \label{fig:struct}
\end{figure}

\newpage

\begin{figure}[ht]
    \centering
    \includegraphics[width=0.7\textwidth]{assets/smo.pdf}
    \caption{Схема модели в терминах СМО}
    \label{fig:smo}
\end{figure}

\section{GPSS}

GPSS (англ. General Purpose Simulation System~---~система моделирования общего назначения)~---~язык моделирования используемый для имитационного моделирования различных систем, в основном систем массового обслуживания. Динамическим элементом модели является транзакт~---~абстрактный объект, который перемещается между статическими элементами, воспроизводя различные события реального моделируемого объекта. В процессе работы модели накапливается статистика, автоматически выводимая по завершении процесса моделирования. Статические элементы модели: источники транзактов, устройства, очереди и другие.

