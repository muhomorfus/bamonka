\chapter{Теоретический раздел}

Обычно под термином <<случайные числа>> подразумевают последовательность независимых случайных величин с заданным распределением.

Равномерным распределением на конечном множестве чисел называется такое распределение, при котором любое из возможных чисел имеет одинаковую вероятность появления.

\begin{equation}
	f(x) = \begin{cases}
        1, x \in (a; b), \\
        0, \text{иначе}.
    \end{cases}
\end{equation}

Существует несколько категорий способов генерации случайных чисел. Одними из них являются табличный и алгоритмический способы.

Табличный способ заключается в использовании специальным образом составленных таблиц. Генерация чисел производится следующим способом: вычисляется начальная позиция в файле, после чего происходит чтение нужного количества цифр числа из файла. При этом переход на следующую позицию возможен множеством способов: переходом на следующую позицию, на следующую строку в столбце, на ближайшую четную позицию.

Алгоритмический способ заключается в генерации последовательности чисел так, что каждый следующий элемент последовательности зависит от предыдущего. Таким образом, фактически последовательность не является случайной, но выглядит таковой для пользователя. Числа, полученные алгоритмическим способом, называются псевдослучайными.

В данной работе реализован критерий сериальной корреляции для проверки того, насколько последовательность можно считать случайной. Расчитывается коэффициент корреляции между соседними группами измерений. Больший коэффициент корреляции означает зависимость, близкую к линейной.