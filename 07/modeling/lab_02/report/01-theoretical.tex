\chapter{Теоретический раздел}

Случайный процесс называют \textbf{марковским}, если во всякий момент времени вероятность нахождения системы в некотором состоянии после него не зависит от состояния системы до него.

Предельной вероятностью нахождения системы в $i$-ом состоянии называют
число

\begin{equation}
	p_i = \lim_{t \rightarrow +\infty} p_i(t).
\end{equation}

Предельная вероятность $p_i$ нахождения системы в $i$-ом состоянии
может быть найдена путем решения СЛАУ

\begin{equation}
	\begin{cases}
        \sum_{j = 1, j \ne i}^n p_j \lambda_{ji}
            = p_i \sum_{j = 1, j \ne i}^n \lambda_{ij}; \\
        p_1 + p_2 + ... + p_n = 1.
    \end{cases}
\end{equation}
