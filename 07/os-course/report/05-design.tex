\chapter{Конструкторский раздел}

\section{IDEF0-диаграмма нулевого уровня}

На рисунке~\ref{img:idef0} представлена диаграмма IDEF0 нулевого уровня разрабатываемого ПО.

\begin{figure}[h!]
    \centering
    \includegraphics[scale=0.8]{assets/idef0}
    \caption{IDEF0-диаграмма нулевого уровня}
    \label{img:idef0}
\end{figure}

\section{Структура, хранящая информацию о текущем системном вызове}

Для сбора информации о времени исполнения текущего системного вызова необходимо хранить следующую информацию:

\begin{itemize}
	\item время вызова функции;
	\item системный вызов;
	\item PID вызывающего.
\end{itemize}

\newpage

\section{Структура, хранящая статистику о времени исполнения системного вызова}

Для статистики времени исполнения системного вызова необходимо хранить следующую информацию:

\begin{itemize}
	\item количество вызовов;
	\item суммарную длительность;
	\item максимальную длительность.
\end{itemize}

\section{Алгоритм установки перехватчиков системных вызовов}

На рисунке \ref{img:set_interceptor} представлена схема алгоритма установки перехватчиков системных вызовов.

\begin{figure}[h!]
    \centering
    \includegraphics[width=0.6\textwidth]{assets/set_interceptor.pdf}
    \caption{Схема алгоритма установки перехватчиков системных вызовов}
    \label{img:set_interceptor}
\end{figure}

\newpage

\section{Алгоритм обработчика, выполняемого при вызове функции}

На рисунке \ref{img:start_handler} представлена схема алгоритма обработчика, выполняемого при вызове функции.

\begin{figure}[h!]
    \centering
    \includegraphics[width=0.4\textwidth]{assets/start_handler.pdf}
    \caption{Схема алгоритма обработчика, выполняемого при вызове функции}
    \label{img:start_handler}
\end{figure}

\newpage

\section{Алгоритм обработчика, выполняемого после возврата из функции}

На рисунке \ref{img:return_handler} представлена схема алгоритма обработчика, выполняемого после возврата из функции.

\begin{figure}[h!]
    \centering
    \includegraphics[width=0.7\textwidth]{assets/return_handler.pdf}
    \caption{Схема алгоритма обработчика, выполняемого после возврата из функции}
    \label{img:return_handler}
\end{figure}

\newpage

\section{Структура программного обеспечения}

На рисунке \ref{img:structure} представлена структура программного обеспечения.

\begin{figure}[h!]
    \centering
    \includegraphics[width=0.8\textwidth]{assets/structure.pdf}
    \caption{Структура ПО}
    \label{img:structure}
\end{figure}

ПО можно разделить на следующие логические части:

\begin{itemize}
	\item \texttt{handlers}~---~обрабатывает вход и выход из перехватываемых системных вызовов и собирает статистику;
	\item \texttt{store}~---~хранит статистику;
	\item \texttt{view}~---~предоставляет статистику пользователю.
\end{itemize}

Из-за небольшого размера, все компоненты будут размещены в одном модуле.


