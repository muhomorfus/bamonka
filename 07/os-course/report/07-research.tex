\chapter{Исследовательский раздел}

С помощью разработанного модуля были протестированы две программы, имеющих схожий функционал. Первая программа использует небуферизованный ввод-вывод, вторая буферизованный. Код программ приведет на листингах~\ref{lst:nobuf}--\ref{lst:buf}.

\begin{lstlisting}[language=C, label=lst:nobuf, caption={Программа, использующая небуферизованный ввод-вывод}]
#include <fcntl.h>
#include <unistd.h>

int main()
{
  char c;    

  int fd1 = open("alphabet.txt", O_RDONLY);
  int fd2 = open("alphabet.txt", O_RDONLY);

  while((read(fd1, &c, 1) == 1) && (read(fd2, &c, 1) == 1))
  {
    write(1, &c, 1);
    write(1, &c, 1);
  }

  return 0;
}
\end{lstlisting}


\begin{lstlisting}[language=C, label=lst:buf, caption={Программа, использующая буферизованный ввод-вывод}]
#include <stdio.h>
#include <fcntl.h>

#define BUF_SIZE 20
#define FILENAME "alphabet.txt"

int main()
{
  int fd = open(FILENAME, O_RDONLY);

  FILE* fs1 = fdopen(fd, "r");
  char buff1[BUF_SIZE];
  setvbuf(fs1, buff1, _IOFBF, BUF_SIZE);

  FILE* fs2 = fdopen(fd, "r");
  char buff2[BUF_SIZE];
  setvbuf(fs2, buff2, _IOFBF, BUF_SIZE);

  int flag1 = 1, flag2 = 2;
  while(flag1 == 1 || flag2 == 1)
  {
    char c;

    flag1 = fscanf(fs1, "%c", &c);
    if (flag1 == 1)
    {
      fprintf(stdout, "%c", c);     
    }

    flag2 = fscanf(fs2, "%c", &c);
    if (flag2 == 1)
    {
      fprintf(stdout, "%c", c); 
    }
  }

  return 0;
}
\end{lstlisting}

\newpage

Результаты тестирования представлены в таблице~4.1. Единицы измерения времени~---~наносекунды.

\begin{table}[h!]
\label{tbl:res}
\caption{Замеры количества и время исполнения системных вызовов для программ, использующих буферизованный и небуферизованный ввод-вывод}
\begin{tabular}{|l|r|r|}
\hline
                                & Небуф. & Буф. \\ \hline
Количество вызовов open         & 5                                           & 4                                         \\ \hline
Максимальная длительность open  & 72164                                       & 74830                                     \\ \hline
Средняя длительность open       & 32340                                       & 26207                                     \\ \hline
Количество вызовов read         & 10003                                       & 254                                       \\ \hline
Максимальная длительность read  & 249946                                      & 20041                                     \\ \hline
Средняя длительность read       & 363                                         & 561                                       \\ \hline
Количество вызовов write        & 10000                                       & 2                                         \\ \hline
Максимальная длительность write & 54081                                            & 3667                                      \\ \hline
Средняя длительность write      & 258                                         & 2021                                      \\ \hline
\end{tabular}
\end{table}