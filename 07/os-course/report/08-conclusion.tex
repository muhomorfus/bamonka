\chapter*{ЗАКЛЮЧЕНИЕ}
\addcontentsline{toc}{chapter}{ЗАКЛЮЧЕНИЕ}

В ходе выполнения курсовой работы был разработан загружаемый модуль ядра, позволяющий получать информацию о количестве и времени исполнения системных вызовов \texttt{open}, \texttt{read}, \texttt{write} для заданного процесса. В процессе выполнения работы были решены следующие задачи.

\begin{itemize}
	\item Проведен обзор способов перехвата системных вызовов: модификация таблицы системных вызовов, kernel probes, function trace. В результате сравнения был выбран kernel probes, как наиболее подходящий для решения поставленной задачи.
	\item Определен способ взаимодействия пользователя с модулем ядра~---~файловая система \texttt{/proc}.
	\item Проведен обзор структур и функций, необходимых для определения обработчиков входа и выхода из системного вызова.
	\item Разработаны алгоритмы установки перехватчиков системных вызовов, обработчика входа в системный вызов и возврата их системного вызова.
	\item Реализован загружаемый модуль ядра.
	\item Работа модуля ядра протестирована на программах, использующих небуферизованный и буферизованный ввод-вывод.
\end{itemize}

По итогам тестирования модуля можно сказать, что суммарное время исполнения вызова \texttt{read} в программе с буферизованным вводом в 24.5 раза меньше, чем в программе с небуферизованным вводом. Максимальное время исполнения вызова \texttt{read} в программе с буферизованным вводом в 12.4 раза меньше, чем в программе с небуферизованным вводом.