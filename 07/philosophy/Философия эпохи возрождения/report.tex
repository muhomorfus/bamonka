\documentclass[14pt]{extarticle}
\usepackage[T1]{fontenc}
\usepackage[utf8]{inputenc}
\usepackage{amsmath,amssymb}
\usepackage[russian]{babel}

\begin{document}
\title{\text{Философия эпохи возрождения}}
\maketitle

\section{Общая характеристика философии эпохи Возрождения}

Эпоха Возрождения длилась приблизительно с середины XIV века до середины XVII века. Она получила такое название потому, что в то время возник огромный интерес к античной культуре, включая и философию. Некоторым людям даже казалось, что античность возникла снова, возродилась и что они живут в античном обществе. Но время необратимо, и прошедшее не возвращается. Происходило не возрождение античной культуры, а ее усвоение и использование людьми последующего времени в соответствии с их интересами.

В эпоху Возрождения центром философских изы­сканий становится человек~---~наступает вре­мя антропоцентризма. Возникла новая система цен­ностей, где на первом месте человек и природа, а затем религия и ее проблемы. Эстетическое, (что в переводе с греческого озна­чает относящееся к чувству), доминирует в филосо­фии эпохи Возрождения. Мыслителей больше инте­ресует творчество и красота человеческой личности, а не религиозные догмы. 

Основания антропоцен­тризма Возрождения лежит в изменении экономиче­ских отношений. Разделение сельского хозяйства и ремесла, бурное развитие мануфактурного производ­ства, знаменовало переход от феодализма к раннему капитализму.

В философии эпохи Возрождения выделяются следующие на­правления:

\begin{enumerate}
	\item Гуманистическое (14--18 вв.)~---~решались проблемы человека, утверждались его величие и могущество, отрицались догматы церкви (Петрарка, Валли).
	\item Неоплатоническое (15--16 вв.)~---~с позиций идеализ­ма пытались познать явления природы, космос, проблемы человека, развивали учение Платона (Кузанскнй, Марандола, Парацельс).
	\item Натурфилософское (16--нач. 17 вв.)~---~опираясь не научные и астрономические открытия, сделали по­пытку изменить представление об устройстве Все­ленной, Космосе и основ мироздания (Николай Ко­перник, Джордано Бруно, Г. Галилей).
	\item Реформационное (14--17 вв.)~---~попытка пересмотра церковной идеологии и взаимоотношений между людьми и церковью (Эразм Роттердамский, Жан Кальвин, Марти Лютер, Томас Мюнцер, Усенлиф).
	\item Политическое (15 в.)~---~связано с проблемами управления государством (Николо Макиавелли).
	\item Утопическо-социалистическое (15-17 вв.)~---~поиски идеального общества на основе регулирования всех взаимоотношений со стороны государства при условии отсутствия частной собственности (Томас Мор, Томмазо Кемпанелла).
\end{enumerate}

\section{Предыстория эпохи Возрождения}

В то время существовала тесная связь философии, художественной литературы, искусства и науки. Эпоха Возрождения~---~время интенсивного развития искусства, создания великолепных художественных произведений. Выдающиеся представители культуры эпохи Возрождения: английский поэт и драматург Уильям Шекспир, итальянский художник, ученый, изобретатель и философ Леонардо да Винчи, итальянский скульптор, художник, архитектор и поэт Микеланджело Буонарроти, ученые Николай Коперник (Польша), Галилео Галилей (Италия), Иоганн Кеплер (Германия) и другие.

Для характеристики философии того времени необходимо отметить ее основные понятия: человек, природа и разум. Интерес прогрессивных мыслителей эпохи Возрождения к этим понятиям был прямо или косвенно направлен против религиозного мировоззрения, которое господствовало в средние века. В центре мировоззрения людей средневековья находилось понятие Бога. Их мировоззрение носило теоцентрический характер. А в центре мировоззрения прогрессивных, образованных людей эпохи Возрождения находилось понятие человека. Их мировоззрение носило антропоцентрический характер.

\section{Гуманизм как основная характеристика эпохи Возрождения}

Не случайно в эпоху Возрождения впервые возник гуманизм как система идей и движение социальной мысли. Слово «гуманизм» происходит от латинского humanus~---~человечный, человеческий. Гуманизм~---~отношение к человеку как высшей ценности и человечеству как наивысшей ценности. Один из первых гуманистов эпохи Возрождения~---~итальянский философ Джованни Пико делла Мирандола. Религиозные философы, конечно, тоже писали о человеке. Но они, как правило, характеризовали его отрицательными выражениями. По мнению известного раннехристианского философа Августина, человек~---~это «малейшая часть» Божественного творения. А Пико делла Мирандола написал произведение «О достоинстве человека», где дана высокая оценка человека~---~свободной и деятельной личности. К числу мыслителей-гуманистов эпохи Возрождения относятся также Эразм Роттердамский (немецкий философ конца XV-начала XVI века), Мишель Монтень, (французский философ XVI века) и другие. В своем основном произведении «Опыты» Монтень изложил не только обобщающие мысли о человеке, но и собственный жизненный опыт.

Если мыслители средневековья видели в человеке «образ и подобие Бога», то философы эпохи Возрождения понимали человека в тесной связи с природой. Многие образованные люди все больше осознавали великое значение природы для жизни человека. Природа становилась не только предметом познания, но и предметом любования, эстетического (художественного) восприятия.

Усиление интереса к природе способствовало развитию естественных наук. А для развития естествознания нужны не только научные опыты, но и мышление, разум, при помощи которого происходит обобщение опытов. Поэтому в философии эпохи Возрождения понятия человека и природы тесно связаны с понятием разума. Если в средние века основным свойством человека считали веру, особенно веру в Бога, то многие мыслители эпохи Возрождения из всех свойств человека ставили на первое место разум. Они высоко ценили разум, его способности познавать мир и руководить человеческим поведением. В то же время они беспощадно критиковали и высмеивали противоположность разума~---~глупость. Например, Эразм Роттердамский написал произведение «Похвала глупости». Это заглавие надо понимать иронически: автор не хвалит, а высмеивает глупость и ее носителей.

\section{Неоплатонизм в эпоху Возрождения}

Мыслители эпохи Возрождения не всегда были последовательными в своих размышлениях о человеке, природе и разуме. Они не могли сразу освободиться от всех религиозных идей и представлений, которые распространялись длительное время и были привычными для многих. Поэтому нет ничего удивительного в том, что некоторые философы делали уступки религиозным традициям и учению церкви. Они искали такое мировоззрение, в котором научное исследование природы можно было бы сочетать с верой в Бога. Такое мировоззрение они видели в пантеизме. Слово «пантеизм» происходит от двух древнегреческих слов: «пан»~---~все, «теос»~---~Бог. Пантеизм~---~философское учение, отождествляющее Бога и мир. По мнению пантеистов, Бог и мир~---~это одно и тоже, иначе говоря, они совпадают.

Выдающимся философом неоплатонического направления являлся Николай Кузанский. Философское решение Кузанским главной пробле­мы~---~отношения Бога и мира~---~является теоцентрическим, но в то же время содержит элементы и тенден­ции, отличающиеся от средневекового католического богословия. Он считал, что познание вещей возможно при помощи чувств, разума и интеллекта, однако зна­ние о конечных вещах всегда выходит за свои пределы. Безусловное знание мы можем постичь лишь символически. Основой этой символики для Кузанского являются математические символы. Николай Кузанский существенно переосмыслил положения, лежа­щие в основе неоплатонизма. Если у Платона и его последователей единое определялось как противостоящее многому, то  Ку­занский отказался от такого противопоставления и поставил в основу своих идей утверждение «единое есть все». Другими словами, с точки зрения этого мыслителя единому ничто не противоположно.

Чтобы проиллюстрировать наиболее значимые положения своего учения, Кузанский обращается к математике. При увеличении радиуса круга до бесконечности круг постепенно превращается в бесконеч­ную прямую линию. Поскольку диаметр круга бесконечен и в то же время бесконечна прямая, они совпадают друг с другом. Кроме того, центр круга, то есть определенная точка, также совпадает с кругом. Наконец, если обратиться к треугольнику, то при увеличении одной его стороны до бесконечности другие стороны также становятся бесконечными; поскольку линия, образовавшаяся при увеличении радиуса круга до бесконечности, бесконечна, очевидно, что треугольник совпадает с кругом. Мир~---~не изолированный шар в лоне абсолюта, но он (мир) бесконечен и является беско­нечным шаром. Уже этим его воззрение отличается от геоцентрического, ибо бесконечный шар не имеет определенного центра, он имеет центр везде и нигде. Нигде нет ничего устойчивого и абсолютного, нет так­же и абсолютного покоя. Абсолютной является лишь бесконечность.

Некоторые пантеисты отождествляли Бога и природу, «растворяли» Бога в природе. Итальянский философ~---~пантеист конца XVI--начала XVII в. Лючилио Ванини утверждал: «Сама природа есть Бог». Из таких мнений некоторые делали тот вывод, что научное исследование природы не удаляет людей от Бога, а, наоборот, приближает их к Богу. В условиях эпохи Возрождения, когда религия еще господствовала в сознании многих людей, такое понимание природы могло способствовать развитию науки.

\section{Естественнонаучные изыскания европейской философии эпохи Возрождения}

Большую роль для переосмысления идей соотношения мира и Бога сыграли натурфилософские изыскания Николая Коперника, который предложил теорию об обращении Земли вокруг Солнца и о суточном вращении Земли вокруг своей оси. Используя галерею как обсерваторию, он проводил астрономические наблюдения с помощью изготовленных им самим инструментов. Примерно в 1515 изложил свои идеи в сочинении Комментарии, написанном на латинском языке. В нем Коперник кратко рассматривает объяснение движения планет, которое давали астрономы древности, а затем формулирует основные положения гелиоцентрической системы мира. Некоторое время его труд свободно распространялся среди ученых. Только тогда, когда у Коперника появились последователи, его учение было объявлено ересью, а книга внесена в <<Индекс>> запрещенных книг.

Данная теория означала переход от геоцентрической системе к гелиоцентрической системе мира. Тем самым опровергались основанные на религии представления о Земле как «избранной Богом» арене, на которой разыгрывалась борьба божественных и дьявольских сил за человеческие души. Идеи Коперника продолжил Галилео Галилей, который считал, что мир бесконечен, материя вечна, природа едина.

Галилео Галилей создал телескоп с 32-кратным увеличением. Он установил, что Солнце вращается вокруг своей оси. Активно защищал гелиоцентрическую систему мира, за что был, подвергнут суду инквизиции, вынудившей его отречься от учения Коперника. Был обвинен в безбожии и отлучен от церкви.

Мировоззренческие идеи Коперника и Галилея стали основой пантеистической философии при­роды Джордано Бруно.

Бруно понимал мир как Единый, который состоит из множества самостоятельных единиц. Космос есть структура, состоящая из дискретных частей, атомов, существующих в непрерывной бесконечности. Вселенная является бесконечной. Тезис о бесконечности вселенной имеет основопо­лагающее значение для космологии Бруно. Космос~---~одновременно пустая и одновременно наполненная бесконечность. Вне космоса нет ничего иного, он является всем бытием, вечным, несотворенным богом. Бесконечность мира не является божественным атри­бутом, как это доказывает теология. Бруно отвергает также представление о том, что мир находится, на некотором особенном месте, окруженном пустым пространством, или Богом. Бруно создает новую космологию, которая восходит к гениальным открытиям Коперника, и делает из гелиоцентрического понимания мира радикальные философские выводы. Мир однороден во всех своих частях, ни одно тело не имеет привилегированного положения, не су­ществует никакого размещенного в центре внешнего источника движения (первого двигателя). Следстви­ем концепции физического единства вселенной у Бруно является гипотеза, выражающая возможность существования жизни и на других планетах.

Официальная церковь относилась к пантеизму отрицательно, так как он подрывал церковное учение о Боге как сверхъестественном духе, существующем где-то вне природы, до природы, и независимо от нее. Церковные власти жестоко расправлялись с пантеистами.

Начало освобождения от духовного господства религии, развитие материального производства и естествознания стимулировали разработку материалистической философии. Эпоха Возрождения~---~время интенсивного развития материализма, особенно в Англии. Это объясняется тем, что в Англии того времени был наибольший прогресс промышленности и естествознания.

Почему материализм тесно связан с естествознанием? Потому что многие ученые~---~естествоиспытатели стихийно (а некоторые~---~осознанно) приходят к материалистическим взглядам и выводам. В ходе научного исследования природы они убеждаются в том, что природа существует сама по себе, независимо от всякого сознания и что в мире нет ничего сверхъестественного. В этом и состоит материализм, изложенный в наиболее кратком виде.

Прогресс естествознания и материалистической философии в эпоху Возрождения способствовал развитию медицины. Из медиков того, времени для философии наиболее интересны Парацельс и Кардано. Немецкий врач, естествоиспытатель и философ Гогенгеим больше известен под псевдонимом Парацельс.

По его мнению, медицина есть более искусство (мастерство), чем наука. Мастерство же основано на опыте. С целью ознакомления с опытом других людей и накопления собственного опыта Парацельс много путешествовал, посетил ряд стран Востока. В этих путешествиях он старался усвоить всевозможные знания и приемы деятельности, которые можно использовать в медицине.

Итальянский врач, изобретатель, математик и философ Джироламо Кардано продолжал в медицине традиции, заложенные еще Гиппократом и Галеном, добавляя к ним элементы астрологии. Вместе с тем он придавал решающее значение врачебному опыту. 

Интерес к действительному миру, к реальной человеческой жизни, к усвоению и применению профессионального и жизненного опыта присущ многим образованным, творчески мыслящим людям эпохи Возрождения.

\section{Реформационное направление философии Возрождения}

Эпоха Возрождения дала начало такому движению как реформация~---~широкое движение, связанное с отказом принимать догматы католической цер­кви. Одной из причин реформации стала широко распространенная практика продажи индульгенций в католической церкви, а также господство католической церкви в церковно-государственных отношениях. К XVI веку католическая церковь сосредоточила в своих руках большие богатства и являлась крупнейшим собственником в Европе. Такое богатство церкви входило в противоречие с древними христианскими учениями, осуждающие богатство и стяжательство. Кроме того, у многих вызвала недовольство практика продажи индульгенций, которая давала возможность состоятельным людям отпускать грехи, что опять подрывало идеи древнего христианства. Начало реформации положил Мартин Лютер, который в 1517 году выступил против существующей практики отпущения грехов.

Он полагал, что люди могут обрести спасение только личной верой в Бога. По мнению Лютера, которое кардинально отличалось от мнения католической церкви, верующий дос­тигает спасения души не потому, что выполняет церковные ритуалы, а лишь благодаря вере, которую он получает непосредственно от бога. Мартин Лютер также перевел Библию на разговорный немецкий язык. До этого Библию могли читать только священнослужители, которые владели латинским языком. Лютеранство, которое окончательно сложилось после смерти его основателя как самостоятельное общественно-религиозное дви­жение Германии, отрицает сословие духовенства как наделенного бла­годатью посредника между Богом и человеком.

Продолжателем идей реформации был Жан Кальвин, который существенно упростил порядок богослужения. В кальви­нистской церкви первичная религиозная община верующих приобретала значительные права, которые проявлялись в том, что она само­стоятельно избирала себе помощника на ограниченный срок. Кроме того, кальвинисты стремились к тому, чтобы новая церковь определя­ла религиозные и общественные нравы людей, а светские власти в своей деятельности руководствовались церковными предписаниями.

В целом реформация привела к появлению протестантизма в христианском вероучении, где значительно снижен статус церковных служителей и церкви в целом. Главное отличие протестантизма от католичества и православия состоит в учении о непосредственной связи Бога и человека, человек может общаться с Богом лично без посредничества священнослужителей и церкви. Благодаря движению протестантов Библия была переведена на все разговорные языки мира. На сегодняшний день протестантизм представлен множеством течений и направлений.


\section{Социальные и политические взгляды философии эпохи возрождения}

Эпоха Возрождения дала толчок развитию философии политики, которая наибольшее развитие в данный период получила в трудах Никколо Макиавелли.

Одно из главных сочинений Макиавелли~---~«Государь». В нем автор рисует образ идеального правителя в государстве. По мнению философа, государь должен быть всегда справедливым, но он может пренебречь нормами морали, если это необходимо для блага государства. Философия Макиавелли почти вся посвящена идее создания сильного и справедливого государства. Правитель должен быть абсолютным властелином, деспотом. Он не дол­жен быть связан никакими априорными схемами, пра­вовыми предписаниями, религией или своим собствен­ным словом. Он должен руководствоваться строго анализированными реальными фактами, может быть жестоким, хитрым, грешным, беспощадным. 

Мораль силы Макиавелли часто определяется как образец «циничности» и аморальности в политике. Термин «макиавеллизм» со временем стал синонимом политики, которая руководствуется принципом «цель оправдывает средства». Макиавеллизм осуждается как теория и практика бесконтрольного использования власти, не подчиненной никаким «высшим» мораль­ным критериям, как деятельность, единственным за­коном которой является успех любой ценой.

Однако так называемый ма­киавеллизм, приписанный Макиавелли, нельзя сводить к вульгарной реализации принципа «цель оправдывает средства». В действительности этот принцип имеет более позднее, иезуитское происхождение. Макиавелли его никогда не формулировал, он не вытекает из кон­текста его творчества. Необходимо понимать творчество Макиавелли не абстрактно, вне конкретных, исторических условий эпохи, а в связи с интересами прогрессивных сил тогдашнего итальян­ского общества, которое испытывало потребность установить государ­ственную власть антифеодального типа, объединить страну в централизованное государство во главе с абсолютным правителем. Макиавелли жил в раздробленной стране, раздираемой войнами и заговорами. Он от всей души желал построения стабильного и единого государства. Осуществить эту цель может только сильный, умный и хитрый государь, по своей сути~---~особый человек~---~политический, которому некогда заниматься нравственностью.

В период образования первых зародышей капи­тализма, связанных с первоначальным накоплением капитала, в философской мысли возникают идеи со­циального равенства людей. Часто это такое предвидение имеет утопический и иллюзорный харак­тер, так как отражает объективно несуществующие общественные условия и силы тогдашнего общества.

Утопические учения XVI в. связаны, прежде всего, с трудами английского гуманиста Томаса Мора, итальянского монаха Томмазо Кампанеллы.

Творчество Мора является ярким выражением гуманистического нравственного идеала, учением о достоинстве человека и его свободе. В своем главном произведении «Книжка поистине золотая и равно полезная, как и забавная, о наилуч­шем устройстве государства и острове Утопия» он рассуждает о социальных и политических проблемах эпохи, которой представляются устройство и жизнь на вымышлен­ном острове Утопия. Идеалом, который он конкретно демонст­рирует на примере отношений на острове Утопия, были общественная собственность, высокоорганизованное производство, целесообразное руководство, гарантирующее справедливое и равное распределе­ние общественного богатства. Все люди должны иметь право и обязаны работать и т.~д.

Томмазо Кампанелла был одним из представителей итальянской философии природы; од­нако более значительную роль сыграло его социаль­ное учение и его труд «Города Солнца». Он выражает мысль о необходимости больших общественных преобразований, направленных на реа­лизацию царства божьего на земле, призывает в соот­ветствии с христианской совестью к ликвидации част­ной собственности и эксплуатации. Государственное устройство Солнечного города представляет собой идеализированную теократиче­скую систему, во главе которой стоит жрец, первый духовник, Метафизик, отмеченный солнечным симво­лом. Его помощники~---~Власть, Мудрость и Любовь~---~занимаются вопросами войны и мира, военным ис­кусством и ремеслом; свободными искусствами, нау­ками, школьным образованием; вопросами контроля рождаемости, воспитания, медициной, земледелием и скотоводством. Политическая, светская власть пере­плетается с Церковной, духовной. Религия граждан города Солнца сливается с философией природы, за­дача состоит в их объединении. Он предсказывал огромную роль науки, говорил об образовании народа, о ликви­дации войн, частной собственности, о справедливом и разумном управлении.

Мор и Кампанелла принадлежат к прогрессив­ным мыслителям, их социалистические утопии пред­ставляют собой идейно целое и плодотворное течение социально-политических концепций Ренессанса.

\section{Вывод}

Таким образом, философия Возрождения занимает видное ме­сто в истории философской мысли. Во всех областях культуры Ренессанса в течение всего периода старые идеи, традиции, концепции сталкиваются с новыми. 

Основным признаком философии Ренессанса явля­ется его светская, земная направленность. Если пред­метом средневековой философии был бог, то ныне на первое место выступает природа. Значение философии Ренессанса можно кратко представить в том смысле, что в целом она, собствен­но, создала основу философии Нового времени. 

Пе­риод философии Ренессанса представляет собой необ­ходимый и закономерный переход от средневековых философских традиций к философии Нового времени.

\end{document}