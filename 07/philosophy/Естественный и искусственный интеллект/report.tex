\documentclass[14pt]{extarticle}
\usepackage[T1]{fontenc}
\usepackage[utf8]{inputenc}
\usepackage{amsmath,amssymb}
\usepackage[russian]{babel}

\begin{document}
\title{\text{Естественный и искусственный интеллект}}
\maketitle

В понятии «искусственный интеллект» выражается попытка осмыслить проблему интеллекта с разных сторон~---~естественнонаучной, психологической и философской. И это вполне правомерно. Человеческий разум представляет собой уникальное явление на нашей планете. Науки о человеке (физиология, психология, социология) рассматривают человеческое сознание как природное и социальное явление, однако они не затрагивают вопроса о его искусственном воспроизведении. Кибернетика же ставит эту проблему. И это имеет важное значение для познания конкретных механизмов естественного (человеческого) разума.

Следует отметить, что термин «естественный интеллект» неточно выражает смысл понятия человеческого интеллекта. Последний, если рассматривать не материальный субстрат (мозг), а способность его отражать внешний мир, выступает в значительной мере не природным, а социальным, то есть формируется в результате человеческой деятельности.

Это свидетельствует о том, что понятие «естественный интеллект», так же как и «искусственный интеллект», характеризует только один аспект категории «интеллект». Последняя становится основанием для рассмотрения диалектической взаимосвязи естественного и искусственного в интеллекте.

Наряду с понятием «искусственный интеллект» употребляется термин «машинный интеллект». В литературе нет единого мнения о специфическом содержании этих понятий. Одни считают, что «машинный интеллект»~---~это показатель того, насколько кибернетическая машина приспособлена к решению разнообразных задач и к эффективному взаимодействию с человеком, а «искусственный интеллект»~---~это модель мозга и высших форм психической деятельности. Другие трактуют эти термины иначе и даже противопоставляют их.

На наш взгляд, несовпадение трактовок в данном случае не является принципиальным, так как общим для «машинного» и «искусственного» интеллекта (на современном этапе развития наук об искусственном интеллекте) является то, что интеллект «принадлежит» машине и различаются они главным образом по способу задания (построения) интеллекта. Последний может быть ориентирован на моделирование особенностей человеческого интеллекта или может развивать алгоритмические структуры ЭВМ без непосредственной связи их со структурами человеческого мышления. «Машинное мышление», полученное путем кибернетического моделирования естественного интеллекта, больше соответствует понятию искусственного интеллекта. Итак, методологически важным становится определение понятий «интеллект», «естественный интеллект» и «искусственный интеллект».

В литературе нет четкого определения понятия «интеллект». Нам представляется, что в выработке такого определения известную помощь может дать сравнительный анализ свойств «искусственного» и «естественного», то есть человеческого, интеллекта. Для выявления инвариантного содержания этих систем важно провести их сопоставление по структурно-функциональным свойствам, так как субстратные характеристики (у человека и ЭВМ) заведомо различны. Онтологическим основанием для такого сопоставления процессов, характеризующих качественно различные формы движения материи, служит всеобщее свойство отражения, структурно-функциональная «родственность» уровней которого доказана развитием философии и естествознания. Принцип отражения позволяет решать проблему взаимоотношения человека и машины («одну из великих проблем», как назвал ее Винер) не только философски, но и с позиций естествознания и математики, то есть не только с качественной, но и с количественной стороны. Успехи количественного познания сложных явлений зависят от того, насколько удается их формализовать.

Формализация предполагает анализ структуры интеллекта. Следует иметь в виду, что попытки вычленения в интеллекте различных структурных элементов предпринимались неоднократно. Так, еще Аристотель различал «пассивный» и «активный» разум, Кузанский~---~рассудок и интеллект, Д. Бруно — разум и интеллект. Дальнейшее обоснование деления мышления на рассудочное и разумное нашло в философских системах Канта и Гегеля. Правомерность такого различения, как это известно, признавал Энгельс. Этот подход к мышлению приобретает эвристическое значение в свете кибернетических теорий «искусственного интеллекта». Если разум представляет собой высшую форму теоретического освоения действительности, для которой характерно осознанное оперирование понятиями, исследование их природы, творчески активное, целенаправленное отражение действительности, то рассудок, также оперируя абстракциями, не вникает в их содержание и природу. Ему присущ известный автоматизм. «Рассудочная деятельность,~---~писал Копнин,~---~имеет как бы три слоя: ее элементы у высших животных, рассудок человека и замена рассудочной деятельности человека машиной. В последнем случае рассудок выступает в чистом виде, он не за-гемнен никакими другими моментами и поражает человека точностью, быстротой в выполнении определенных операций мышления. В этом отношении машина как рассудок превосходит рассудок индивидуума».

При диалектическом подходе к этому явлению необходимо иметь в виду взаимосвязь и взаимопереходы рассудочного и разумного. То, что на данном уровне развитая мышления выступает разумным, со временем может стать рассудочным, в свою очередь рассудочное было когда-то разумным. Разум переходит в рассудок путем формализации. Это превращение происходит в каждом случае передачи функции человеческого мышления машине посредством создания алгоритма.

Таким образом, в структуре интеллекта наряду с искусственным и естественным необходимо различать рассудочное и разумное. Природу интеллекта можно рассматривать и в плане различных уровней его структуры. Основная задача познания «искусственного интеллекта» выглядит в таком случае как переход от поверхностных структур, которые им моделируются, к глубинной структуре, представленной в естественном интеллекте, и как «идентификация» структур машинного и человеческого мышления.

Классический вопрос «может ли машина мыслить?» необходимо обсуждать как в философском, так и в естественнонаучном и математическом аспектах. Полезность сопоставлений того и другого подхода способствует уточнению определений основных понятий. Некоторые ученые, возражая против правомерности расчленения интеллекта на его структурные элементы, обосновывают это тем, что разум не расчленяется на отдельные уровни. С этим вряд ли можно согласиться. Дело в том, что моделировать интеллект вообще, как таковой, не выделяя его конкретных качеств, нельзя. Поэтому специалисты в области «искусственного интеллекта» при выработке исходных принципов определяют интеллект с помощью операциональных критериев. Чтобы быть «разумной», машина должна обладать многими функциональными способностями человека. Но при этом едва ли необходимо, чтобы она была похожа на человека.

В теоретических работах по «искусственному интеллекту» такие понятия, как «интеллект», обычно относимые к человеку, употребляются в специально-научном смысле. И это согласуется с исторической практикой формирования научных (физических, математических) понятий, таких, как тело, энергия и тому подобное. Метафорические (основанные на переносе значения) и операциональные характеристики понятия «искусственный интеллект» служат отправным моментом в разработке этого вопроса теории. Сравнивая человеческий интеллект с «машинным мышлением», метафорическое употребление понятия «искусственный интеллект» вместе с тем облегчает понимание человеческого разума, хотя у этих образований помимо общих свойств имеется множество других, по которым они совершенно различны.

Эта метафора помогает строить гипотезы относительно системы оригинала, но при формировании понятий необходимо помнить об их метафорическом характере. Метафорические аналогии между искусственным и естественным интеллектом оправданы тогда, когда они понимаются не очень буквально. Такие аналогии имеют значение для гипотетико-дедуктивных построений теорий «искусственного интеллекта»; аксиоматическое изложение предполагает использование неопределяемых понятий.

Понятие «искусственный интеллект», возникшее в кибернетике, позволяет классифицировать сложные кибернетические системы по функциональному критерию. Оно удачно объединяет целый ряд эффективных свойств специальных программ для ЭВМ, которые аналогичны (гомоморфны) качествам человеческого интеллекта. Оно показывает также, что многие различия между интеллектуальными программами ЭВМ и человека, которые казались существенными, по сути являются количественными. Однако поведение человека, его память, восприятие, способность к обучению и самоорганизации, несомненно, богаче, чем у эвристических программ ЭВМ. Необходимо иметь в виду ограниченные возможности современных автоматов, являющихся «эмбриональными» кибернетическими системами, в силу чего многие из интеллектуальных функций в настоящее время могут выполняться лишь в принципе. Это означает, что, хотя в существенных частях эти функции могут быть реализованы, осуществление их в целом остается под вопросом из-за больших материальных затрат.

При сравнении человеческого интеллекта с машинным важно четко различать, на каком уровне проводится аналогия~---~принципиальном или фактическом. Ясно, что фактическое сравнение не всегда оправдано, так как эти объекты существенно различны. С точки зрения учения об «искусственном интеллекте», видимо, нежелательно обыденные представления об интеллекте возводить в ранг серьезных аргументов. Сравнение мозга и машины может оказаться неадекватным в оценке либо мозга, либо вычислительной машины. Дело в том, что последние создаются преимущественно для решения задач, которые человек сам решить не в состоянии. Лишь творческая мысль и интуиция человека, дополненная кибернетической машиной, способны выполнять трудные задачи. Взаимодействие человека и вычислительной машины основано на том, что последняя~---~это не просто сверхмощный и быстродействующий арифмометр; в определенных отношениях она усиливает интеллектуальные способности человека. Это значит, что вычислительные машины создаются для усиления, а не для замены человеческого интеллекта.

Для выявления особенностей структуры человеческого интеллекта нужны соответствующие понятия. Ее нельзя вывести из понятийной структуры, описывающей менее глубокие уровни действительности. Первая должна интегрироваться в самоорганизующую систему, подчиняющуюся имманентным закономерностям. Это означает, что такая система располагает внутренними механизмами саморазвития, которые позволяют ей обучаться, совершенствоваться, самовоспроизводиться. Последнее обстоятельство нередко расценивается как аргумент против признания любой формы «машинного интеллекта». При этом отмечают, что машина получает способности от своего создателя. С этим, разумеется, нельзя не согласиться, но следует иметь в виду, что и человеческий интеллект развивается аналогично. Предпосылкой интеллекта служит его связь с внешним миром.

Подчинение самоорганизующихся систем имманентным законам развития относительно. Иерархический принцип самоорганизации действует и за пределами системы, поскольку последняя является составной частью вышестоящих материальных структур. Следовательно, при изучении интеллекта как самоорганизующейся системы важно выявить диалектику внутреннего и внешнего, которая выражается во взаимодействии моделей двух типов~---~модели системы самой себя и модели внешнего мира (человеческий разум моделирует сам себя в ЭВМ, и он же создает модели внешнего мира). В таком взаимодействии выделяется уровень информационных отношений, на котором система переводит и интегрирует внешнее во внутреннее. Существенную роль в такого рода отношении играет обратная связь, ее значение существенно в человеческом поведении. Использование обратной связи плодотворно в исследовании работы мозга и машины. Особенно полезна отрицательная обратная связь, уменьшающая рассогласование между действительным и желаемым поведением.

При сравнительном анализе мозга и машины возникают некоторые трудности, связанные с их сложностью. Существует точка зрения, согласно которой, пока структурная сложность машин не достигла уровня сложности мозга, до тех пор не может быть и речи об «искусственном интеллекте», поскольку сложности одной и другой систем несопоставимы. С этим мнением нельзя согласиться. Дело в том, что кибернетика позволяет перевести проблему структурной сложности «на язык» сложности функционального порядка. Если сложность функций различных систем сопоставима, то можно сделать вывод и о сопоставимости сложности самих систем. Здесь происходит своего рода оборачивание метода: то, что исторически было первичным, на логическом уровне анализа оказывается вторичным. Структура и функция объективно находятся в неразрывном единстве, хотя функция системы определяется ее структурой. Однако в научном исследовании в различные периоды может становиться существенным или тот, или другой аспект. В данном случае функциональный аспект для познания имеет большее значение, чем субстратно-структурный подход. В самоорганизующихся системах функциональное сходство приобретает решающее значение: в том случае, когда вычислительная машина может самоорганизоваться, способ первоначального соединения элементов, если он не обеспечивает эффективного решения стоящих перед системой задач, подвергается «пересмотру». Вообще принципы самоорганизации (в особенности эвристической самоорганизации~---~на основе «отсечения» плохих вариантов поведения) служат базой для конкретной разработки задач «искусственного интеллекта».

Весомый вклад в понимание особенностей интеллекта и выделение его характеристик можно ожидать на пути создания «машинного мышления». В связи с этим (даже при допущении того, что практически не удастся создать «машинный интеллект») большой интерес представляет анализ и выделение логических, гносеологических и эвристических принципов разума. Разумное поведение система будет иметь только тогда, когда она будет в состоянии создавать оптимальную модель среды. И наоборот, разум будет ограничен, если эта модель слишком груба и не дает достаточного описания среды или если она неполно отражает взаимодействие между ее элементами. Интеллект присущ системам, которые осуществляют целенаправленное поведение, обладают необходимой информационно-логической структурой, обеспечивающей продуктивное мышление.

Междисциплинарный характер кибернетического подхода вызывает определенные трудности и в сопоставлении порождаемых им общенаучных понятий с традиционными философскими категориями. Это объясняется известной специфичностью кибернетики как науки. Кибернетический подход к познанию мира, с одной стороны, предполагает создание систем логико-математических абстракций и упрощающих идеализации, а с другой~---~позволяет создавать устройства, функционирующие в реальном масштабе времени.

Исследование философских аспектов проблемы «искусственного интеллекта» требует глубокого анализа и уточнения самих понятий «интеллект», «разум», «мышление» в плане сопоставления особенностей человеческого мышления с возможностями его кибернетических аналогов. Такое исследование должно опираться на\\ диалектика-материалистические методологические основы понимания сущности мышления. На этом пути представляет большой интерес анализ логических, гносеологических и эвристических принципов разума. Выделить структуру и понять принципы организации интеллекта~---~это значит вскрыть реальные основания проблемы, показать ее глубокую специфичность. Иначе говоря, необходимо изучить. исторические, научно-технические и гносеологические аспекты проблемы «искусственного интеллекта», непосредственно опираясь на диалектико-материалистическую философию. Философское осмысление научной проблемы должно помогать исследователям в ее разрешении, направлять научный поиск. В этом проявляется эвристическая роль философии.

\section*{Вывод}

Таким образом, стремление моделировать «духовные процессы» в системе «искусственного интеллекта» заслуживает поощрения. В этом плане важно создание машин, обладающих функциями мышления. В частности, рассмотрение некоторых аспектов «искусственного интеллекта» свидетельствует о том, что на путях создания «машинного мышления» наука может продвинуться и в понимании человеческого интеллекта. В свою очередь тот факт, что проявления «человеческого духа» могут быть воспроизведены, служит новым аргументом в борьбе против философского идеализма и агностицизма.

Таким образом, рассмотрение предмета и метода кибернетики, ее центральных принципов и идей позволяет раскрыть диалектические аспекты этой науки, установить ее закономерные взаимосвязи с материалистической диалектикой. Ее развитие идет в русле синтетических тенденций, выражающих необходимость взаимосвязи общественных, естественных и технических наук.

\end{document}