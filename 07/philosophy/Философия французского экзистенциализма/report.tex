\documentclass{bmstu}

\begin{document}

\chapter*{Философия французского экзистенциализма}

\subsubsection*{Небольшое введение про экзистенциализм}

Философия экзистенциализма, основоположником которой является датский теолог и религиозный мыслитель Серен Кьеркегор, представляет собой индивидуалистическое направление, предполагающее, что люди обладают свободой воли и самостоятельно определяют свою судьбу. Экзистенциалисты считают, что именно человек придает значение жизни в изначально бессодержательном мире, и не признают значимости моральных и ценностных категорий и общественных норм, считая их созданными искусственно. С их точки зрения, социальная принадлежность условна, и никто не может делегировать другому ответственность за собственную жизнь.

В начале 40-х годов двадцатого столетия центр экзистенциалистского движения перемещается во Францию. Именно в этот период создают свои наиболее значительные произведения Жан-Поль Сартр (1905--1980), Альбер Камю (1913--1960), Габриэль Марсель (1889--1973) и Симона де Бовуар (1908--1986).

\subsubsection*{Воспоминания о немецкой философии}

Также как и немецкий экзистенциализм французская <<философия существования>> в центр внимания помещает человека. Существование понимается как <<конкретное>>, то есть единичное, индивидуальное, неповторимое и противопоставляется всему общему, закономерному, рациональному.

Для французских экзистенциалистов характерна активная литературно-художественная деятельность. Свои идеи они излагают не только в философских трактатах, но и в многочисленных драматургических произведениях, новеллах, романах, мемуарах. Это в немалой степени способствовало расширению сферы влияния экзистенциализма во второй половине 20 века сначала во Франции, а затем и во всем мире.

\subsubsection*{Жан-Поль Сартр}

Самый значительный представитель французского экзистенциализма~---~Жан-Поль Сартр, автор теоретического труда <<Бытие и ничто>> (1943), а также романов и пьес (<<Дороги свободы>>, <<Слова>>, <<Тошнота>>, <<Мухи>>, <<Мертвые без погребения>>, <<Стена>> и др.). В довоенные годы Сартр был откровенно аполитичен. Но все меняется с начала 40-х годов. Он начинает активно интересоваться политической жизнью, принимает участие в движении Сопротивления. На рубеже 40--50-х годов Сартр стремится создать партию, объединяющую левую интеллигенцию, выступает за сотрудничество с французскими коммунистами. Он дважды приезжал в СССР, симпатизировал революции на Кубе, осуждал французский колониализм и войну во Вьетнаме. В 1964 году ему была присуждена Нобелевская премия по литературе (однако он от нее отказывается). В период студенческих волнений 1968 года во Франции Сартр~---~признанный кумир бунтующей молодежи.

Философия Сартра формировалась под влиянием идей Гуссерля, Хайдеггера и Гегеля. В своей программной лекции <<Экзистенциализм~---~это гуманизм>> (1946) французский мыслитель заявляет, что <<если даже Бога нет, то есть по крайней мере одно бытие, которое существует прежде, чем его можно определить каким-нибудь понятием, и этим бытием является человек>>.

Бытие, по Сартру, в человеческой реальности проявляется через три формы: <<бытие-в-себе>>, <<бытие-для-себя>> и <<бытие-для-другого>>. Это три стороны единой человеческой реальности, разделяемые лишь в абстракции.

Миру как <<бытию-в-себе>> противостоит человек в качестве чистого <<бытия-для-себя>>. <<Бытие-для-себя>>~---~непосредственная жизнь самосознания и есть чистое <<ничто>> по сравнению с миром. Оно может существовать только как отталкивание, отрицание, <<отверстие>> в бытии как таковом.

<<Бытие-для-другого>> обнаруживает конфликтность межличностных отношений. Субъективность самосознания приобретает внешнюю предметность только тогда, когда существование личности входит в кругозор, в поле зрения другого сознания. И отношение к другому, с точки зрения Сартра,~---~это борьба за признание свободы личности в глазах другого.

Возражая Хайдеггеру, Сартр заявляет, что человеческое бытие и экзистенцию с самого начала надо описывать, имея в виду сознание, ибо сознание есть мера человеческого бытия (кстати говоря, ни Сартр, ни Хайдеггер себя идеалистами не считали. А Хайдеггер не считал свою философию экзистенциалистской).

Сознание, согласно Сартру, не из чего не выводимо, оно отлично от всех других мировых явлений и процессов. Оно есть насквозь существование, индивидуальный опыт существования. Сознание~---~ничто в том смысле, что нет такой <<данности>>, про которую мы могли бы сказать, что это сознание. Сознание существует только как сознание вещи, на которую оно направлено. Но сознание о чем-то есть в то же время и сознание самого себя как отличного от того, на что оно направлено.

Понятие <<ничто>> играет в сартровской философии определяющую роль (не случайно и основное его философское произведение названо <<Бытие и ничто>>). <<Ничто>> указывает в первую очередь на специфический способ существования сознания и человеческого бытия. Именно исходя из понимания сознания как ничто, Сартр и определяет такие его характеристики, как свобода, временность, тревога, ответственность и т.~п.

Другими словами, человеческая реальность такова, что имеется бытие. Мир стал <<иметься>> только с появлением сознания.

Существование сознания и человеческой реальности~---~это факт, исходя из которого Сартр прямо утверждает первенство бытия перед ничто. Ничто возможно потому, что есть бытие. Но это не значит, что сознание возникает из бытия, бытие не является причиной сознания. Да и вообще такая проблема, будучи метафизической, с точки зрения Сартра, не только неразрешима, но и бессмысленна.

Также как и Хайдеггер, Сартр уделяет много внимания проблеме времени. Время у него лишено объективности и полагается <<Я>>. Временность дана через <<ничто>>, которое, по существу, и оказывается источником временности. Для себя бытие <<овременяет>> свое существование. Иначе говоря, временность признается только как приходящая в мир через человека, только как свойство переживающей человеческой души. Время субъективизируется, оно возможно лишь в форме многих отдельных <<временностей>>, существующих как отношения <<бытия-для-себя>> и <<бытия-в-себе>>, соединенных <<ничто>>.

Особое значение Сартр придает анализу соотношения прошлого, настоящего и будущего. Время и возникает, по его мнению, в процессе постоянного ускользания сознания от тождества с самим собой. <<Бытие-для-себя>> существует в форме трех временных состояний или экстазов (прошлого, настоящего, будущего). Эти временные экстазы существуют как разделенные моменты изначального синтеза, который осуществляется субъектом. Именно поэтому прошлое, как и будущее, не существует вне связи с настоящим. Прошлое~---~это всегда чье-то прошлое, прошлое кого-то или чего-то (предмета, человека, народа, общества и т.~д.).

Сартр считает, что прошлое всегда оказывается настоящим, но только таким, которое выражает прошлое: <<Я есть мое прошлое,~---~пишет он,~---~и если меня нет, мое прошлое не будет существовать дольше меня или кого-то еще. Оно не будет больше иметь связей с настоящим. Это определенно не означает, что оно не будет существовать, но только то, что его бытие будет неоткрытым. Я единственный, в ком мое прошлое существует в этом мире>>. Выходит, что во всякий момент своей жизни человек свободно определяет, что же такое в действительности его прошлое, он перетолковывает прошлое.

Само же настоящее, по Сартру,~---~это только мгновенное постижение <<теперь>> или <<ничто>>, которое ориентирует человека в его отношении к своему прошлому и своему будущему. Но <<если мы,~---~утверждает Сартр,~---~изолируем человека на мгновенном острове его настоящего и если все модусы его бытия окажутся предназначены природой к вечному настоящему, то мы радикально устраним все методы его рассудочного отношения к прошлому>>. Настоящее, таким образом, есть чистое <<ничто>>, не имеющее каких-либо положительных определений. Такое понимание настоящего Сартр противопоставляет <<вещистскому>>, по его словам, пониманию человека.

Соответственно, будущее приходит в мир только с человеческим существованием, как <<проект>> будущего, где прошлое через заполнение <<ничто>> настоящего может стать чем-то. Но будущее на самом деле никогда не сможет реализоваться, так как в самый момент осуществления цели оно переходит в прошлое. <<Действительное будущее,~---~отмечает Сартр,~---~есть возможность такого настоящего, которое я продолжаю в себе и которое есть продление действительного в себе. Мое будущее вовлекает как будущее сосуществование очертания будущего мира ... будущее в-себе, которое обнаруживается моим будущим, существует в направлении прямо соединенном с реальностью, в которой я существую>>.

Субъективизация времени, категории очень важной для понимания истории, приводит Сартра к отрицанию истории природы и истории общества. Абсолютизируя свободу в духе индетерминизма (человек ставится вне всякой закономерности и причинной зависимости), Сартр растворяет историю в хаосе произвольных отдельных действий и отрицает объективность исторических событий.

Свобода не терпит ни причины, ни основания. Свобода сохраняется в любой обстановке и выражается в возможности выбирать свое отношение к данной ситуации (узник или раб свободен, самоопределяя свое отношение к своему положению).

Свобода, в сартровском понимании, предполагает и независимость по отношению к прошлому. Проектируемое сознанием будущее, а не реальное настоящее выступает в таком случае критерием свободы. Причем это будущее берется вне связи с возможностью его претворения в действительность. Так как время становится эгоцентрическим, то смещается и вся историческая перспектива. Время личности, по сути дела, противопоставляется времени истории. Только в <<проектировании>> человек схватывает время как промежуток, отделяющий его нынешнее состояние от возможного и желательного состояния.

Таким образом, у Сартра способ постижения времени объективного мира превращается и в способ его творения.

Стоит еще сказать об отношении Сартра к Богу и религии. Выступая против философского рационализма, он свою позицию называет последовательно атеистической и видит одну из задач своей философии в критике непоследовательного атеизма. Такой атеизм, нападая на религию, сам оказывается во внутренней зависимости от нее. Это происходит по причине веры в разумность самого бытия. Отрицание личного Бога христианства здесь оборачивается утверждением Бога в качестве структуры и смысла этого мира. Такая установка находит завершение в отождествлении Бога и природы.

Поэтому нужно освободиться,~---~по мнению Сартра,~---~от любых представлений об упорядоченности мира, о наличии в нем закономерности. Только так можно добиться обезбоженности мира. Быть реальным~---~значит оказываться чуждым сознанию, оказываться совершенно <<случайным>>. Тайна человеческого поведения состоит в его абсолютной необусловленности, независимости, спонтанности. От Бога у Сартра остается только <<направленный на меня взгляд>>, пронзающий и вездесущий, наблюдающий за человеком из самых глубин сознания. Это тот <<другой>>, который не просто люди или даже человечество.

Отвергая веру в Бога, Сартр в основу своей этической концепции закладывает все ту же абсолютную свободу личности. Стало быть человек~---~единственный источник, критерий и цель нравственности. Причем каждый отдельный человек, <<я>>. Моя личная свобода является единственной основой моральных ценностей. Итак, пользуясь своей свободой, будь самим собой! В таком случае, конечно, не остается места для общезначимости, социальности моральных норм.

\subsubsection*{Альбер Камю}

Популярность и влиятельность французского экзистенциализма во многом также обязана творчеству Альбера Камю~---~писателя, публициста, философа. В годы войны он участвовал в Сопротивлении, сотрудничал в подпольной газете. Тогда же он пишет повесть <<Посторонний>> (1942) и философское сочинение <<Миф о Сизифе>> (1942). В послевоенной Европе с большим успехом идут постановки его пьес, очень популярны роман <<Чума>> (1947), философское эссе <<Бунтующий человек>> (1951). В 1957 году Камю становится лауреатом Нобелевской премии.

В работах Камю осуществляется синтез философии и литературы. Художественные произведения наполняются экзистенциалистским философским содержанием. Он сознательно отказывается от традиционной формы изложения философских идей.

Его нередко называют <<философом абсурда>>. Камю приходит к заключению, что трагическое переживание <<смерти Бога>> в двадцатом веке (упадок христианства) знаменует утрату смысла бытия и жизни человека.

Мир противостоит человеку как неразумное начало, как некая условная абсурдная декорация. Мыслящий индивид стремится понять себя исходя из абсурдности и неустранимости бытия. Абсурд кроется в несовместимости одновременного присутствия различных факторов мира.

Абсурд, согласно Камю,~---~это экзистенциал. Абсурд~---~основное понятие нашей эпохи. <<Сам по себе мир,~---~утверждает он,~---~просто неразумен, и это все, что о нем можно сказать. Абсурдно столкновение между иррациональностью и исступленным желанием ясности, зов которого отдается в самых глубинах человеческой души. Абсурд равно зависит и от человека, и от мира. Пока он~---~единственная связь между ними. Абсурд скрепляет их так прочно, как умеет приковывать одно живое существо к другому только ненависть>>. Личность ощущает всю трагичность своей связи с миром, отсутствие своего завтра, будущего. Человек очень остро переживает свой разлад с миром, свою чуждость миру.

Опыт человеческого существования, неминуемо завершающийся смертью, приводит мыслящую личность к открытию <<абсурда>> как конечной правды своего существования на Земле. Но эта истина должна не обезоруживать, а, напротив, пробуждать в душе мужественное достоинство продолжать жить вопреки всему и без всяких доводов в пользу такого выбора.

Подлинное мужество заключается не в бегстве от жизни (суицид) и не в примирении с действительностью. Нужно схватиться с миром, бороться, бунтовать.

Отчаяние коренится в крушении надежд, возлагаемых человеком на историю. Но история не сказка со счастливым концом, ибо нет Бога, который бы так ее устроил. Отчаяние есть расплата за эту иллюзию.

<<Мир>>, по определению Камю, <<безрассудно молчалив>>; это значит, что история сама по себе не содержит ни задания, ни запроса, ни оправдания человеческих поступков, нелепо искать в ней ответ на вопрос о нашем предназначении, о том, ради чего нам следует жить.

<<Безрассудному молчанию мира>> Камю противопоставляет <<порыв>>, или <<бунт>>, самого человека. Так как историческая действительность не имеет разумно-целостного строения (она абсурдна и бессвязна), то нет оснований рассматривать <<бунт>> как слепой и стихийный. <<Бунтующий человек>> не разрушитель, напротив, своим действием он впервые вносит в мир некоторую гармонию.

В своей этике Камю пытался обосновать учение о некоем <<праведничестве без Бога>>, опирающемся на заповеди христианского милосердия и противопоставленном нравственности, исходящей из социально-исторических установок.

Вообще взгляды Камю эволюционировали от <<философии абсурда>> в направлении к гуманизму.

\subsubsection*{Антуан де Сент-Экзюпери}

Влияние экзистенциализма заметно ощущается и в книгах другого известного французского писателя Антуана де Сент-Экзюпери (особенно в его новелле <<Военный летчик>>, написанной в 1941 году).

\subsubsection*{Габриэль Марсель}


Создателем религиозного (католического) варианта экзистенциальной философии является Габриэль Марсель~---~французский драматург, критик, философ. Основные его труды: <<Метафизический дневник>> (1927), <<Быть и иметь>> (1935), <<Человек-скиталец>> (1945), <<Таинство бытия>> (1951), а также пьесы: <<Иконоборец>>, <<Разбитый мир>> и др.

В центре внимания Марселя стоит проблема бытия, преломленная через индивидуальный опыт, существование отдельного человека. Он проводит резкое различие между миром <<объективности>> (физический мир) и миром <<существования>> (личностный мир), или неподлинным миром обладания и подлинным миром бытия. Действительность рассматривается не через взаимоотношение между субъектом и объектом, а через форму взаимосвязи между переживающим и переживаемым. Причем они нераздельны, одно без другого невозможно.

Марсель порывает с традицией католической схоластики, восходящей к Фоме Аквинскому и его принципу обоснования веры разумом. Французский мыслитель считает невозможным какое-либо рациональное обоснование веры.

Он противопоставляет рациональному познанию (проблеме) интуитивное, эмоциональное познание (таинство). <<Бытие есть таинство>>,~---~заявляет он. Проблема находится в сфере логического, рационального. Человек подходит к проблеме как к чему-то, находящемуся вне и независимо от него. Таинство же включает, вовлекает самого человека, сливает воедино <<я>> и <<не-я>>, познающее и познаваемое, стирает грань между субъектом и объектом.

Бытие непознаваемо средствами науки, в мире нет закономерностей. Рациональность мира, по его мнению, это лишь ложная проекция науки на реальность жизни.

Причем для Марселя в основе познания лежит не интеллектуальная интуиция, а эмоциональная, имеющая чувственную направленность. На место декартовского <<Я мыслю, следовательно, существую>> он ставит <<Я чувствую, следовательно, существую>>. Мышление берет свое начало, таким образом, прежде всего из чувств. На место <<вещных>> отношений, причинной связи приходят любовь, вера, привязанность, ответственность, послушание.

Любовь к людям основывается на любви к Богу и отношении к другим как к <<детям Божьим>>. <<Братство людей>>~---~это братство во Христе. А мир в целом для Марселя представляет собой связующее звено с Богом. Отношение человека к Богу имеет эмоциональный, интимный характер и основывается на вере, надежде, преклонении, не нуждаясь в рациональном обосновании.

Человека Марсель понимает как единство духа и тела или <<воплощенное бытие>>. Через опыт телесного <<присутствия>>, <<вовлеченности в бытие>> человек входит в поток времени, истории. <<Человек-странник>> изначально одинок.

Христианское мировоззрение рассматривается Марселем как средство преодоления этого одиночества и как возможность преодоления реальных общественных противоречий. Он видит в Боге источник культурно-исторического творчества, обеспечивающий сплочение людей для созидания многообразного мира культуры.

В последний период своего творчества Марсель резко критикует <<научно-технический разум>> и обвиняет его во всех бедствиях европейской истории. <<Расколотый мир>>, по его словам, порождается <<вещными отношениями>> общества, приводящего к столкновению запросов развития цивилизации и всемерного совершенствования способностей личности.

Марсель видит вину философии 17-19 веков, и не без оснований, в успокоенности, упоении успехами разума и отсутствии <<мудрости>>, способной объединить различные способы человеческого освоения мира. <<Научно-технический разум>> в это время вступает в откровенное противоборство с религиозно-философской <<мудростью>>. Это противоречие он называет базисным конфликтом буржуазной эпохи.

Марселя страшит <<массовое общество>> и его продукт~---~<<человек массы>>, который полностью утерял личностное начало и всецело подчинен <<механическим законам>> социальной жизни. <<Человек массы>> живет в царстве социальных отношений, лишенных личностного измерения. Он становится всего лишь совокупностью функций, у него исчезает подлинно гуманное отношение к ближнему.

Чрезвычайно отрицательно оценивает Марсель и преувеличение роли техники, составляющей <<душу>> цивилизации Нового времени и современности (<<технической цивилизации>>). Человек, почувствовавший себя центром Вселенной, приходит к утверждению мощи своего разума через покорение внешнего мира, что грозит в конце концов полной его дегуманизацией.

Отсюда задача~---~разрушить иллюзии <<духа абстракции>>, рационально-технического разума, сделать человека трагически мудрым. А пробуждение творческого начала личности связано с пробуждением в ней добродетели христианской мудрости. Таковы основные черты католического экзистенциализма Габриэля Марселя.


\subsection*{Типа вывод}

В 60--70 годы прошлого века экзистенциализм, став одним из самых популярных направлений западной философии, приобретает множество последователей и сторонников по всему миру. Особенно восприимчивой к его идеям оказывается художественная интеллигенция и студенческая молодежь. Важное значение для человечества конца двадцатого века имели предупреждения экзистенциализма против безоглядной веры в науку и технику, против недобросовестности приспосабливающегося к экономическим и социальным реалиям сознания (т.~е. против конформизма). Экзистенциализм по-новому и весьма продуктивно поставил и разрешил проблему свободы человека, выбора жизненных путей.

Французский экзистенциализм оказал большое влияние на современную философию и культуру. Его идеи продолжают развиваться и обсуждаться в академических кругах и за их пределами.

\end{document}