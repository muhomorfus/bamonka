\documentclass[14pt]{extarticle}
\usepackage[T1]{fontenc}
\usepackage[utf8]{inputenc}
\usepackage{amsmath,amssymb}
\usepackage[russian]{babel}

\begin{document}
\title{\text{Структура сознания. Сознание и язык}}
\maketitle

Сознание~---~это способность человеческой психики познавать окружающий мир, самоосознавать себя, вырабатывать эмоциональное отношение и осуществлять целенаправленную деятельность как практического, так и духовного характера.

Основной проблемой сознания в философии является вопрос его отношения к бытию. Этот вопрос имеет две стороны:

\begin{enumerate}
	\item онтологическую, в рамках которой решается вопрос первичности материи или сознания по отношению друг к другу;
	\item гносеологическую, в рамках которой решается вопрос о принципиальной возможности познания мир.
\end{enumerate}

Онтологическая сторона основного вопроса философии (первичность материи или сознания) решается следующими путями.

\begin{enumerate}
	\item Материя первична, а сознание является лишь свойством высокоорганизованной материи (материализм).
	\item Сознание существует независимо и от материального мира, и от человека, оно принадлежит Высшей Сущности (Нусу, Логосу, Богу, Абсолюту, Мировому Разуму, Абсолютному духу и так далее). Сознание первично, потому что именно Высшая Сущность своим сознанием создает и организует мировой порядок, используя для этого материю как сырой материал (объективный идеализм).
	\item Сознание формирует для человека мир через комплекс поставляемых самому себе ощущений, и поэтому активное индивидуальное сознание человека, обнаруживающее пассивную материю и представляющее её в осмысленно организованной форме, первично (субъективный идеализм).
	\item Сознание существует отдельно от материи, а материя существует независимо от сознания, ничто из них не первично другому, но их взаимодействие определяет картину бытия (дуализм).
\end{enumerate}

Гносеологическая сторона основного вопроса философии (познаваемость мира) решается двумя противоположными путями.

\begin{itemize}
	\item Мир познаваем.
	\item Мир не познаваем.
\end{itemize}

Помимо своего основного вопроса, философия пытается в отношении сознания решить также проблемы природы сознания и механизмов его работы.

Относительно природы сознания общая позиция всех философских течений состоит в том, что сознание не материально по своей природе и не может быть объектом предметно-практической деятельности человека.

Таким образом, сознание изначально всегда понимается только как субъективная реальность, только как нечто идеальное, присущее только внутреннему миру человека.

Что же касается механизмов работы сознания, то философия ограничивает здесь сферу своих интересов процессами интеллектуального порядка, а процессы, относящиеся к различным душевным состояниям, переживаниям и способностям, она оставляет другим наукам (психология, нейрофизиология). То есть философию интересует, и она исследует только то, что касается умственной деятельности человека.

Умственная деятельность~---~это по своей форме есть диалог сознания с самим собой в процессе своего взаимодействия с действительностью.

В процессе же взаимодействия с действительностью, умственная деятельность сознания характеризуется наличием двух компонентов.

\begin{itemize}
	\item Компонент чувственных ощущений (внутренних о себе и внешних, о мире).
	\item Компонент смыслообразования (преобразование чувственных ощущений в интеллектуальные абстракции).
\end{itemize}

Переход компонента ощущений в компонент смыслообразования, то есть восхождение сознания от единичных чувственных ощущений действительности (внешней и внутренней) на уровень сложного интеллектуального знания о ней, называется мышлением.

В результате интеллектуализации мышлением чувственных ощущений, мышление создает знание о действительности, которое невозможно получить только средствами чувственного восприятия.

Мышление, создавая интеллектуальное знание о действительности, осуществляет, тем самым, объективацию сознания. Объективация сознания~---~это превращение мышлением содержания сознания в объекты, с которым мышление может работать. Для получения таких объектов мышление выделяет в сознании элементы, формирующие содержание, и наделяет эти элементы свойствами объектов познания, для чего воплощает их в те или иные мыслимые формы: понятия, суждения, идеи, гипотезы, теории, модели, образы и так далее.

Таким образом, мышление своей деятельностью объективирует сознание в различные формы идеальных представлений, с помощью которых, затем, конструирует или метафизические идеи, не связанные с реальным опытом, или абстрактные образы реальных объектов действительности.

Способность мышления создавать метафизические идеи и абстрактные модели реальной действительности называется разумом. Разум создает метафизические идеи и модели реальной действительности из набора готовых интеллектуальных сведений, которые ему поставляет рассудок.

Рассудок~---~это способность мышления расчленять действительность на отдельные смысловые факты, классифицировать их по отличительным признакам, понятийно наделять определениями и тестировать на соответствие сложившемуся порядку вещей, то есть здравому смыслу. Специфика рассудка состоит в том, что рассудок работает только с готовыми, единожды и навсегда выработанными мышлением понятиями, не преобразуя их во что-либо новое, и не синтезируя между собой.

Преобразует устойчивые понятия в новое, и синтезирует их в систему знания~---~разум. Таким образом, разум~---~это способность мышления к преобразованию интеллектуального материала и к его творческому синтезу в различные системы знаний о действительности.

Исходя из этого, можно сказать, что рассудок и разум~---~это два уровня мышления, с помощью взаимодействия которых оно осуществляет свою деятельность по объективации сознания.

Деятельность мышления осуществляется в различных формах, исходными из которых являются: понятие, суждение и умозаключение.

Понятие~---~это терминологически сформулированное средствами языка представление о чем-либо, фиксирующее в себе наиболее существенные признаки объекта или явления (например, понятия: «вода», «вино», «молоко» и так далее).

Отличительным свойством Понятия является его способность собирать в себя различные объекты и объединять их в некие классы по существенным общим признакам (например, понятия вода, вино и кровь могут выражаться по общему признаку текучести единым понятием «жидкость»). Наиболее общими понятиями, вбирающими в себя не только существенные, но и всеобщие свойства действительности, являются категории (материя, причина, качество, движение и так далее). Таким образом, в понятиях происходит объективация сознания в форме мысленного обнаружения и терминологического определения отдельных объектов и явлений действительности.

Суждение~---~это мысль, выраженная предложением, и содержащая в себе ложное или истинное утверждение («Вода замерзает и испаряется»~---~истинное, «Вода горит»~---~ложное).

Суждение может выражаться не только в предложениях языка, но и в символах ($2 + 2 = 4$~---~истинное суждение, $2 + 2 = 6$~---~ложное).

Всё, что не может оцениваться с точки зрения истинности или ложности, не является суждением, и относится к другим формам мысли (<<Принесите мне воды!>>, <<Вода холодная?>>, <<Вода~---~сколько смысла в этом слове...>>, $2 + 3$, $4$, $5$, $6$, $48$).

Таким образом, структура суждения должна содержать в себе такие понятия и смысловые связки между ними, которые могут быть доказаны или опровергнуты с точки зрения объективности. 

Суждение~---~это объективация сознания в форме выявления истинности или ошибочности существующих между понятиями связей и отношений, которые мышление обнаруживает или создает самостоятельно.

Умозаключение~---~это форма мышления, посредством которой из одного или нескольких суждений логически выводится новое суждение. Исходные суждения в составе умозаключения называются посылками, а новое суждение, получаемое логически из посылок, называется заключением (или следствием). Например.

\begin{enumerate}
	\item Все преступления наказуемы законом (1-я посылка).
	\item Воровство есть преступление (2-я посылка).
	\item Воровство наказуемо законом (заключение из двух посылок).
\end{enumerate}

Все умозаключения подчиняются одному условию: если истинны исходные посылки, то истинно и выводимое из них заключение. Истинное заключение делает умозаключение правильным. Правильное умозаключение, таким образом, представляет собой истинное (или правдоподобное) выводное знание о действительности. 

Умозаключение~---~это объективация сознания в результатах осмысления мышлением действительности.

Таким образом, мышление объективирует сознание, воплощая и представляя его содержание в различных результатах своей работы.

Но сами результаты работы мышления требуют, в свою очередь, дополнительной собственной объективации для того, чтобы стать продуктом информационного обмена между людьми. Без этой объективации все результаты мышления оставались бы субъективным достоянием отдельных личностей (субъектов), неведомым для других субъектов.

Объективация субъективного мышления человека в формы, объективно понятные другому субъекту (человеку), осуществляется средствами языка.

Язык~---~это знаковая система, хранящая и передающая информацию. Языки бывают естественными (речь), или искусственными (алфавит, математические формулы, ноты, цифры, условные сигналы и так далее). Благодаря языку устанавливаются точные и всеобще принятые обозначения объектов и явлений действительности, что создает условия для понимания процессов мышления отдельных субъектов другими субъектами и для обмена результатами этого мышления между ними.

Таким образом, язык является средством объективации сознания, в процессе которой мысли приобретают какую-либо материальную форму своего выражения, общепринятую для всех и понятную всем.

Исходя из этого, элементы языка (слова, предложения, знаки, формулы и так далее)~---~это средства, существующие в структуре языка, которые обозначают соответствующие понятия, суждения, заключения, образы и так далее, существующие в структуре мышления.

Но при этом нельзя говорить о тождестве языка и мышления, поскольку структура языка и структура мышления специфически разные:

\begin{itemize}
	\item языковой знак не имеет смысла вне структуры своего языка;
	\item элемент мышления обладает универсальным смыслом вне любой структуры сознания и в любой структуре сознания.
\end{itemize}

Следовательно, языковые функции относительно объективации сознания можно сформулировать двунаправлено.

\begin{enumerate}
	\item Формулирование мыслей и результатов мышления в универсально понятных формах для хранения и обмена.
	\item Коммуникация мыслей и результатов мышления.
\end{enumerate}

\subsection*{Вывод}

Таким образом, соотнесенность языка и мышления выражается в том, что мышление объективирует содержание сознания в мыслимые формы, а язык обеспечивает их универсальное понимание, хранение и коммуникацию.


\end{document}