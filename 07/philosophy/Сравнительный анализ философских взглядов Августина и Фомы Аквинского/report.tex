\documentclass[14pt]{extarticle}
\usepackage[T1]{fontenc}
\usepackage[utf8]{inputenc}
\usepackage{amsmath,amssymb}
\usepackage[russian]{babel}

\begin{document}
\title{\text{Сравнительный анализ философских} \newline \text{взглядов Августина и Фомы Аквинского}}
\maketitle

\section{Оcновные черты философии Средневековья}
Средневековые философы были в начале пути, ведущего к пониманию взаимосвязи двух экзистенциальных сфер. Средневековье предложило свою модель их взаимоотношений, точнее, ряд моделей, которые были основаны на общих предпосылках, но ведущих к разным выводам. Основная предпосылка относилась к пониманию цели и смысла человеческого существования. Земная жизнь не должна занимать в жизни человека центральное место. Быть человеком — значит жить не только в «горизонтальной» плоскости (между вещами и людьми), но и прежде всего в «вертикальной» плоскости, постоянно стремясь к Богу, помня о Нем и в радости, и в печали, постоянно ощущая Его присутствие. 

Для верующего Бог есть жизнь; он источник жизни, тот, кто дает жизнь; отступление от Бога, согласно христианским воззрениям, убивает душу, теряется связь с Бытием, живой смысл бытия. Поэтому целью человека является общение с Богом и познание Бога. Все остальные этапы человеческого существования, как и мирские познания, должны быть подчинены делам богопознания, спасения души. Это оригинальный тезис христианской философии, разделяемый всеми (независимо от их отношения к тому или иному измерению) западноевропейского Средневековья.

И все же у философов возникли разногласия при обсуждении вопроса о том, помогает ли разумное знание продвижению христианина на пути богопознания или, напротив, только отвлекает его от поисков истины. В западном средневековье можно найти два противоположных ответа на этот вопрос, изложенные в двух философских течениях:


\begin{itemize}
	\item схоластике;
	\item патристике.
\end{itemize}

Основными идеями средневековой философии являются:

\begin{itemize}
	\item Теоцентризм~—~принцип, согласно которому Бог является центром, фокусом средневековых философских и религиозных представлений.
	\item Моноцентризм~—~Бог един, а не множествен.
	\item Креационизм~—~учение о сотворении мира Богом из ничего.
	\item Провиденциализм~—~предполагает, что событие совершается по воле бога, предопределено им.
	\item Средневековый антропоцентризм: человек обладает не двумя измерениями, а именно телом и душой, как считали в античности, а тремя. К первым двум добавляется «дух» (духовность)~—~причастность к божественному посредством веры. Суть человека~—~вера, надежда, любовь, «но любовь из них больше» (апостол Павел). Быть человеком~—~значит жить по этическим правилам, изложенным Христом в Нагорной проповеди.
	\item Концепция линейного времени. Идея историзма.
\end{itemize}


\section{Схоластика Фомы Аквинского}
Схоластика~---~это систематическая европейская средневековая философия, сосредоточенная вокруг университетов и представляющая собой синтез христианского (католического) богословия и логики Аристотеля. Для схоластики характерно сочетание богословских и догматических предпосылок с рационалистической методологией и интересом к формально-логическим проблемам. 

 Схоластика, или «школьная» философия, появилась, когда христианские мыслители стали понимать, что догматы веры допускают рациональное обоснование и даже нуждаются в нем. Схоластика рассматривала в качестве пути постижения бога разум, логические рассуждения, а не мистическое созерцание и чувство. Целью «служанки богословия» становится философское обоснование и систематизация христианского вероучения. Характерной чертой схоластики была слепая вера в непререкаемые «авторитеты». Источники схоластики~---~учение Платона, а так же идеи Аристотеля, из которых устранены все его материалистические взгляды, Библия, писание «отцов церкви».

Наиболее крупным представителем схоластики является Фома Аквинский. Философия Фомы Аквинского, как и его последователей, является объективным идеализмом. В поле притяжения объектов идеализма находятся различные оттенки спиритуализма, утверждающего, что вещи и явления~---~это лишь проявления душ. Философия Фомы Аквинского признает существование не только душ, но и целой иерархии чистых духов, или ангелов.

Фома полагал, что существует три вида познания Бога: через разум, через откровение и через интуицию по вещам, которые были ранее познаны посредством откровения. Другими словами, он утверждал, что познание Бога может опираться не только на веру, но и на рассудок. Фома Аквинский сформулировал 5 доказательств бытия Бога.

\begin{enumerate}
	\item Доказательство от движения. Факт того, что все вещи изменяются в мире, ведет нас к мысли о том, что движимое двигается не иначе как с силой иной. Двигаться~---~значит приводить потенцию в акт. Вещь может быть приведена в действие тем, кто уже активен. Следовательно, все что двигается, кем-то движимо. Иначе говоря, все, что движется, движется по воле Бога.
	\item Доказательство первой причины. Оно основано на невозможности бесконечного регресса: у любого явления есть причина, которая, в свою очередь, также имеет причину и т.д. до бесконечности. Поскольку бесконечный регресс невозможен, в какой-то момент объяснение должно остановиться. Эта конечная причина, по мнению Аквинского, и является Богом.
	\item Путь возможности. В природе есть вещи, бытие которых возможно, но они могут и не быть. Если бы не было ничего, то ничто не могло бы начаться. Не все сущее только возможно, должно быть нечто, существование чего необходимо. Следовательно, мы не можем не принять существования того, кто имеет собственную необходимость в себе, то есть Бога.
	\item Путь степеней совершенства. Мы обнаруживаем в мире различные степени совершенства, которые должны иметь свой источник в чем-то абсолютно совершенном. Другими словами, поскольку есть вещи, совершенные в разной степени, необходимым является предположении существования чего-то, обладающего максимумом совершенства.
	\item Доказательство того, что мы обнаруживаем, как даже безжизненные вещи служат цели, которая должна быть целью, установленной неким существом вне их, ибо лишь живые существа могут иметь внутреннюю цель.
\end{enumerate}

Фома рассматривал мир как иерархическую систему, основой и смыслом которой является Бог. Духовной сфере противостоит материальная природа, а человек является существом, соединяющим в себе духовное и материальное начала и наиболее близко стоящим к Богу. Любое явление мира обладает сущностью и существованием. Для человека и явлений живой и неживой природы сущность не равна существованию, сущность не вытекает из их единичной сути, поскольку они сотворены, а следовательно, их существование обусловлено. Лишь Бог, будучи несотворенным и ничем не обусловленным, характеризуется тем, что его сущность и существование тождественны друг другу.

Фома Аквинский различает в субстанциях 3 рода форм или универсалий:

\begin{enumerate}
	\item Универсалия, содержащаяся в вещи, в качестве ее сущности, непосредственная универсалия.
	\item Универсалия, абстрагированная от субстанции, то есть существующая в человеческом уме. В этом виде она реально существует только в уме, а в вещи имеет лишь свою основу. Эту универсалию Фома называет рефлексивной.
	\item Универсалия, не зависимая от вещи в божественном уме. Универсалии в уме творца~---~это неизменные, постоянные, вечные формы, или основы вещей.
\end{enumerate}

Вводя градацию форм, Фома дает философское обоснование не только мира природы, но и общественного порядка. Критерием, отличающим 1 вещь от другой, выступают не их естественные особенности, а различия в совершенстве форм, являющихся «нечем иным, как подобием Бога, которому вещи сопричастны».

В это время возревает и материалистическая концепция, которая нашла свое первое выражение в концепции номинализма. Одним из крупнейших вопросов схоластики был вопрос о природе общих понятий, по которому были выдвинуты две основные противоположные концепции. С точки зрения реализма (ее придерживался, например, Фома Аквинский) общие понятия, или универсалии, существуют объективно, вне человеческого сознания и вне вещей. С позиции номинализма универсалии есть только названия, даваемые нами сходным вещам.

Фома Аквинский считал, что разум должен превосходить над верой, то есть люди должны в первую очередь понять, а после поверить. В свободной воле человека он не видел греха, а лишь проявление Бога как мирового творца.

\section{Патристика Августина Аврелия}
Патристика~---~это термин, обозначающий совокупность богословских, философских и социальных учений, разработанных великими христианскими мыслителями (отцами церкви) в период со II по VIII в.

Наиболее яркий представитель патристики~---~Августин Аврелий (Блаженный) (354~--~430 гг.). Его главные труды: «Исповедь», «О граде Божьем». В произведениях Августина мифологические и библейские сюжеты сочетаются с религиозно-философскими размышлениями.

Августин~---~крупнейший систематизатор христианского вероучения, стоявший на позициях неоплатонизма.

\textbf{Учение о Боге и мире.} Бог рассматривается им как начало всего сущего, как единственная причина возникновения вещей. Бог вечен и неизменен, он есть нечто постоянное. Мир созданных богом вещей изменчив и пребывает во времени. Мир представляет собой лестницу, где есть высшее (бестелесное и божественное) и низшее (телесное и материальное). То есть в мире существует иерархия~---~жесткий, установленный Богом порядок.

\textbf{Учение о познании.} Внешний изменчивый мир не может быть источником истины, таковым может быть только вечное, то есть Бог. Познание Бога должно составлять смысл и содержание всей жизни человека. Постичь истину можно только путем откровения. Таким образом, Августин выдвигает тезис о превосходстве веры над разумом («верить, чтобы понимать»~---~суть теории познания Августина). Разум постигает явления видимого мира, а вера приводит к осознанию вечного.

\textbf{Учение о душе.} Душа, по Августину, есть только у человека~---~это ставит его выше всех живых существ. Душа бессмертна, она бестелесна, нематериальна и рассеяна по всему телу. Ее важнейшие способности~---~разум, воля и память.

\textbf{Проблема свободы воли.} Августин развивал идею божественной предопределенности. Но в мире существует добро и зло, поэтому возникает вопрос о природе зла. Августин утверждал, что Бог творит только добро, зло~---~это отсутствие добра и возникает в результате человеческой деятельности, так как от рождения человеку дана свобода воли. Он считал её главным источников греха в жизни человека.

\textbf{Взгляды на общественную жизнь.} Социальное неравенство Августин рассматривает как результат грехопадения человечества и считает его основным принципом бытия общества. Государство должно носить теократический характер и служить интересам Церкви. Историю человечества Августин представлял как борьбу двух царств~---~Божьего и земного. В Божье царство входит меньшая часть человечества~---~это люди искренне верующие, живущие «по духу». Град земной составляют люди, живущие «по плоти» (неверующие, язычники). Представителем града Божьего на земле является церковь, следовательно, ее власть выше светской. 

\section{Сравнительный анализ (aka Вывод)}
Таким образом, взгляды Фомы Аквинского и Августина довольно сильно отличаются. 

Первый (Фома) считал, что разум должен превосходить над верой, и путь к вере лежит через понимание. Второй (Августин) же считал, что вера превосходит над разумом, и понять можно только через веру.

Фома Аквинский выдвигал пять онтологических доказательств существования Бога, в том время, как Августин выдвигал единственное, по которому считается, что основой мира является Бог, и, если бы не он, то не существовало бы ничего в этом мире.

Фома Аквинский не видел ничего плохого в свободной воле, в то время, как Августин считал ее главным источником грехов.

\end{document}