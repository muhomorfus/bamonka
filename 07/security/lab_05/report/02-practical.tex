\chapter{Практический раздел}

\section{Листинг алгоритма Хаффмана}

\begin{lstlisting}[language=C, label=lst:sha1, caption={Реализация алгоритма Хаффмана}]
int HuffmanCompress(char *filenameIn, char *filenameOut) {
    fclose(fopen(filenameOut, "wb"));

    FILE *f = fopen(filenameIn, "rb");
    if (f == NULL) {
        return -1;
    }
    uint8_t character[maxCodeValue] = {0};
    int freq[maxCodeValue] = {0};
    uint8_t temp = '\0';
    int empty = 1;
    while (!feof(f)) {
        if (fread(&temp, 1, 1, f)) {
            character[temp] = temp;
            freq[temp]++;
            empty = 0;
        }
    }

    if (empty) {
        fclose(f);
        return 0;
    }

    Tree *t = buildHuffmanTree(character, freq, maxCodeValue);

    int arr[MaxCode] = {0}, top = 0;
    HuffmanCode codes[maxCodeValue] = {0};
    getCodes(t->nodes[0], arr, codes, top);

    saveCodes(filenameOut, codes, freq, maxCodeValue);
    encodeData(filenameIn, filenameOut, codes, maxCodeValue);

    deleteTree(t);

    fclose(f);

    return 0;
}

int HuffmanDecompress(char *filenameIn, char *filenameOut) {
    FILE *f;
    f = fopen(filenameIn, "rb");
    if (f == NULL) {
        printf("fopen()\n");
        return -1;
    }
    HuffmanCode codes[maxCodeValue] = {0};
    int freq[maxCodeValue] = {0};
    int lastUsedBits = 0;

    if (readCodes(f, codes, freq, maxCodeValue, &lastUsedBits) != 0) {
        printf("readCodes()\n");
        return -1;
    }

    uint8_t character[maxCodeValue] = {0};
    for (int i = 0; i < maxCodeValue; i++) {
        character[i] = i;
    }
    Tree *t = buildHuffmanTree(character, freq, maxCodeValue);

    decodeData(f, filenameOut, t, codes, maxCodeValue, lastUsedBits);

    deleteTree(t);
    fclose(f);

    return 0;
}
\end{lstlisting}


\section{Тестирование}

Корректность алгоритма проверялось путем применения дешифрации на шифрованное сообщение.

Тестирование было проведено на файлах с типами:

\begin{enumerate}[label=\arabic*)]
	\item текстовый (txt);
	\item графический (jpg, png);
	\item архив (zip).
\end{enumerate}



В таблице \ref{tbl:test} представлены тестовые данные.

\begin{table}[H]
	\begin{center}
		\centering
		\caption{Тестовые данные}
		\label{tbl:test}
		\begin{tabular}{|c|c|c|} 
			
			\hline
			\multicolumn{1}{|c|}{Номер теста}
			&  \multicolumn{1}{c|}{Тип файла} &  \multicolumn{1}{c|}{Содержимое файла}\\
			\hline
			
			1& txt &  {\specialcell{Стоят два бобра в речке...}} \\ \hline
			
			2& txt &  {\specialcell{@alexodnodvorcev}} \\ \hline
			
			3& txt &  \\ \hline
			
			4& png & {\specialcell{\includegraphics[width=1in]{../data/4.png}}} \\ \hline
			
			5& jpg & {\specialcell{\includegraphics[width=1in]{../data/5.jpg}}} \\ \hline
			
			6& zip & Архив с файлами предыдущих тестов \\ \hline
			
		\end{tabular}
	\end{center}
\end{table}

\newpage

\chapter*{Заключение}

В результате выполнения данной лабораторной работы был реализован в виде программы на языке Си алгоритм сжатия Хаффмана.