\chapter{Практический раздел}

\section{Листинг алгоритма SHA1}

\begin{lstlisting}[language=C, label=lst:sha1, caption={Реализация алгоритма SHA1}]
int sha1(Byte *filename, Byte *result) {
    int size;
    Byte *data;
    if (ReadFile((char *) filename, &data, &size)) {
        Errorln("cant ReadFile()");
        return -1;
    }

    Block digest[5];
    Block w[80];
    Block a, b, c, d, e, f;

    unsigned int i, j;
    Byte *buffer;
    Block bits;
    Block temp, k;
    Block lb = size * 8;

    bits = padding(size);
    buffer = (Byte *) malloc((bits / 8) + 8);
    memcpy(buffer, data, size);

    *(buffer + size) = 0x80;
    for (i = size + 1; i < (bits / 8); i++) {
        *(buffer + i) = 0x00;
    }

    *(buffer + (bits / 8) + 4 + 0) = (lb >> 24) & 0xFF;
    *(buffer + (bits / 8) + 4 + 1) = (lb >> 16) & 0xFF;
    *(buffer + (bits / 8) + 4 + 2) = (lb >> 8) & 0xFF;
    *(buffer + (bits / 8) + 4 + 3) = (lb >> 0) & 0xFF;

    digest[0] = 0x67452301;
    digest[1] = 0xEFCDAB89;
    digest[2] = 0x98BADCFE;
    digest[3] = 0x10325476;
    digest[4] = 0xC3D2E1F0;

    for (i = 0; i < ((bits + 64) / 512); i++) {
        for (j = 0; j < 80; j++) {
            w[j] = 0x00;
        }

        for (j = 0; j < 16; j++) {
            w[j] = buffer[j * 4 + 0];
            w[j] = w[j] << 8;
            w[j] |= buffer[j * 4 + 1];
            w[j] = w[j] << 8;
            w[j] |= buffer[j * 4 + 2];
            w[j] = w[j] << 8;
            w[j] |= buffer[j * 4 + 3];
        }

        for (j = 16; j < 80; j++) {
            w[j] = (rotateLeft(1, (w[j - 3] ^ w[j - 8] ^ w[j - 14] ^ w[j - 16])));
        }

        a = digest[0];
        b = digest[1];
        c = digest[2];
        d = digest[3];
        e = digest[4];

        for (j = 0; j < 80; j++) {
            if (j < 20) {
                f = ((b) & (c)) | ((~(b)) & (d));
                k = 0x5A827999;

            } else if (j < 40) {
                f = (b) ^ (c) ^ (d);
                k = 0x6ED9EBA1;
            } else if (j < 60) {
                f = ((b) & (c)) | ((b) & (d)) | ((c) & (d));
                k = 0x8F1BBCDC;
            } else {
                f = (b) ^ (c) ^ (d);
                k = 0xCA62C1D6;
            }

            temp = rotateLeft(5, (a)) + (f) + (e) + k + w[j];
            e = (d);
            d = (c);
            c = rotateLeft(30, (b));
            b = (a);
            a = temp;
        }

        digest[0] += a;
        digest[1] += b;
        digest[2] += c;
        digest[3] += d;
        digest[4] += e;

        buffer = buffer + 64;
    }

    memcpy(result, digest, 20);
    return 0;
}
\end{lstlisting}


\section{Листинг алгоритма RSA}

\begin{lstlisting}[language=C, label=lst:rsa, caption={Реализация алгоритма RSA}]
int MakeKeyPair(Byte *fileName) {
    Byte *privateFileName = fileName;
    Byte publicFileName[100];
    sprintf((char *) publicFileName, "%s.pub", fileName);

    bignum *p = bignum_alloc();
    bignum *q = bignum_alloc();
    bignum *n = bignum_alloc();
    bignum *d = bignum_alloc();
    bignum *e = bignum_alloc();
    bignum *phi = bignum_alloc();

    randPrime(32, p);
    randPrime(32, q);

    bignum_multiply(n, p, q);
    bignum_isubtract(p, &NUMS[1]);
    bignum_isubtract(q, &NUMS[1]);
    bignum_multiply(phi, p, q);

    getE(phi, e);
    getD(e, phi, d);

    rsaKey private = {
        .exp = e,
        .mod = n,
    };

    rsaKey public = {
        .exp = d,
        .mod = n,
    };

    if (writeRSAKey(privateFileName, &private)) {
        return -1;
    }

    if (writeRSAKey(publicFileName, &public)) {
        return -1;
    }

    return 0;
}

int rsaInFile(const Byte *buf, int size, rsaKey *key, Byte *result) {
    bignum *res, *plain;
    int resSize;

    plain = bignum_from_buf(buf, size);
    res = bignum_alloc();
    bignum_pow_mod(res, plain, key->exp, key->mod);

    resSize = res->length * sizeof(Block);

    for (int i = 0; i < resSize; i++)
        result[i] = ((Byte *) res->data)[i];

    return resSize;
}
\end{lstlisting}


\section{Листинг алгоритма цифровой подписи}

\begin{lstlisting}[language=C, label=lst:rsa, caption={Реализация алгоритма цифровой подписи на основе SHA1 и RSA}]
int Sign(Byte *filename, Byte *keyFileName) {
    char signFileName[512];
    sprintf(signFileName, "%s.signed", filename);

    Byte hash[512] = {0};
    if (sha1(filename, hash)) {
        Errorln("cant sha1()");
        return -1;
    }

    Byte sign_content[512] = {0};
    int size;
    if ((size = rsa(hash, 20, keyFileName, sign_content)) < 0) {
        Errorln("cant rsa()");
        return -1;
    }

    if (WriteFile(signFileName, sign_content, size)) {
        Errorln("cant WriteFile()");
        return -1;
    }

    return 0;
}

int Check(Byte *filename, Byte *keyFileName) {
    char signFileName[512];
    sprintf(signFileName, "%s.signed", filename);

    Byte hash[512] = {0};
    if (sha1(filename, hash) != 0) {
        Errorln("cant sha1()");
        return -1;
    }
    InfoSHA1ln("SHA1 of file", hash);

    Byte *signData;
    int signSize;
    if (ReadFile(signFileName, &signData, &signSize)) {
        Errorln("cant ReadFile()");
        return -1;
    }

    Byte toCompare[512] = {0};
    if (rsa(signData, signSize, keyFileName, toCompare) < 0) {
        Errorln("cant rsa()");
        return -1;
    }
    InfoSHA1ln("SHA1 from signature", toCompare);

    for (int i = 0; i < 20; i++) {
        if (toCompare[i] != hash[i]) {
            Errorln("file is broken");
            return -1;
        }
    }

    Infoln("file is valid");
    return 0;
}
\end{lstlisting}


\section{Тестирование}

Корректность алгоритма проверялось путем применения дешифрации на шифрованное сообщение.

Тестирование было проведено на файлах с типами:

\begin{enumerate}[label=\arabic*)]
	\item текстовый (txt);
	\item графический (jpg, png);
	\item архив (zip).
\end{enumerate}



В таблице \ref{tbl:test} представлены тестовые данные.

\begin{table}[H]
	\begin{center}
		\centering
		\caption{Тестовые данные}
		\label{tbl:test}
		\begin{tabular}{|c|c|c|} 
			
			\hline
			\multicolumn{1}{|c|}{Номер теста}
			&  \multicolumn{1}{c|}{Тип файла} &  \multicolumn{1}{c|}{Содержимое файла}\\
			\hline
			
			1& txt &  {\specialcell{Стоят два бобра в речке...}} \\ \hline
			
			2& txt &  {\specialcell{@alexodnodvorcev}} \\ \hline
			
			3& txt &  \\ \hline
			
			4& png & {\specialcell{\includegraphics[width=1in]{../data/4.png}}} \\ \hline
			
			5& jpg & {\specialcell{\includegraphics[width=1in]{../data/5.jpg}}} \\ \hline
			
			6& zip & Архив с файлами предыдущих тестов \\ \hline
			
		\end{tabular}
	\end{center}
\end{table}

\newpage

\chapter*{Заключение}

В результате выполнения данной лабораторной работы был реализован в виде программы на языке Си алгоритм шифрования RSA и механизм цифровой подписи на основе алгоритма хеширования SHA1.