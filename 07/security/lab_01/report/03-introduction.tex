\chapter*{ВВЕДЕНИЕ}
\addcontentsline{toc}{chapter}{ВВЕДЕНИЕ}

<<ЭНИГМА>> является криптографической машиной, которая была создана в 1920-х годах в Германии как способ военных безопасно обеспечивать свои коммуникации~\cite{bib2}.

\subsection*{Цель работы}

Программная реализация аналога шифровальной машины <<ЭНИГМА>> на языке Си.

\subsection*{Задачи работы}

Для достижения поставленной цели необходимо выполнить следующие задачи.

\begin{enumerate}[label=\arabic*)]
	\item Изучить алгоритм работы шифровальной машины <<ЭНИГМА>>.
	\item Спроектировать алгоритм работы шифровальной машины <<ЭНИГМА>>.
	\item Реализовать алгоритм работы шифровальной машины <<ЭНИГМА>> на языке Си.
	\item Протестировать реализацию алгоритма.
\end{enumerate}