\chapter{Технологический раздел}

\section{Реализация алгоритма шифрования}

В листинге \ref{lst:l1} приведена реализация алгоритма шифрования машины <<ЭНИГМА>>.

\begin{lstlisting}[language=C, label=lst:l1, caption={Реализация алгоритма шифрования}]
void e_shift_right(int* rotor, int size) {
    int last = rotor[size - 1];

    for (int i = size - 1; i >= 1; i--) {
        rotor[i] = rotor[i - 1];
    }

    rotor[0] = last;
}

int e_find_index(const int* rotor, int size, int val) {
    for (int i = 0; i < size; i++) {
        if (rotor[i] == val) {
            return i;
        }
    }

    return -1;
}

void e_shift_rotors(enigma_t* e) {
    e_shift_right(e->rotors[0], e->alphabet_count);
    for (int j = 1; j < e->rotors_count; j++) {
        if (e->counter % (e->alphabet_count * j) == 0) {
            e_shift_right(e->rotors[j], e->alphabet_count);
        }
    }
}

unsigned char* e_encode(enigma_t* e, const unsigned char* data, size_t size) {
    unsigned char* res = calloc(size, sizeof(unsigned char));

    e->counter = 1;
    for (int i = 0; i < size; i++, e->counter++) {
        int code = data[i];

        code = e->panel[code];

        for (int j = 0; j < e->rotors_count; j++) {
            code = e->rotors[j][code];
        }

        code = e->reflector[code];

        for (int j = e->rotors_count - 1; j >= 0; j--) {
            code = e_find_index(e->rotors[j], e->alphabet_count, code);
        }

        code = e->panel[code];
        res[i] = code;
        
        e_shift_rotors(e);
    }

    return res;
}
\end{lstlisting}

\section{Тестирование}
Корректность алгоритма проверялось путем применения дешифрации на шифрованное сообщение.

Тестирование было проведено на файлах с типами:

\begin{enumerate}[label=\arabic*)]
	\item текстовый (txt);
	\item графический (jpg, png);
	\item архив (zip).
\end{enumerate}



В таблице \ref{tbl:test} представлены тестовые данные.

\begin{table}[H]
	\begin{center}
		\centering
		\caption{Тестовые данные}
		\label{tbl:test}
		\begin{tabular}{|c|c|c|} 
			
			\hline
			\multicolumn{1}{|c|}{Номер теста}
			&  \multicolumn{1}{c|}{Тип файла} &  \multicolumn{1}{c|}{Содержимое файла}\\
			\hline
			
			1& txt &  {\specialcell{Стоят два бобра в речке...}} \\ \hline
			
			2& txt &  {\specialcell{@alexodnodvorcev}} \\ \hline
			
			3& txt &  \\ \hline
			
			4& png & {\specialcell{\includegraphics[width=1in]{../data/4.png}}} \\ \hline
			
			5& jpg & {\specialcell{\includegraphics[width=1in]{../data/5.jpg}}} \\ \hline
			
			6& zip & Архив с файлами предыдущих тестов \\ \hline
			
		\end{tabular}
	\end{center}
\end{table}
