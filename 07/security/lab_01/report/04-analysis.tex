\chapter{Аналитический раздел}

Машина под названием ENIGMA, разработанная немцем Артуром Шербиусом для обеспечения безопасности коммерческой информации работает, подавая электрический ток при нажатии любой клавиши. Механические части машины двигаясь, рандомизируют электрический контур каждый раз при нажатии клавиши, создавая разные буквы. Шифровальная машина <<ЭНИГМА>> состоит из роторов, рефлектора и коммутационной панели \cite{bib2}.

\section{Роторы}

Роторы реализуют полиалфавитный алгоритм шифрования, а их определённо выстроенная позиция представляет собой один из основных ключей шифрования. Для Энигмы было разработано восемь различных роторов, и каждый ротор выполнял чётко поставленную задачу в плане коммуникации. Выбор позиций роторов тоже имел значение, образовывая свойство некоммутативности. Каждый ротор обладал 26 гранями, где каждая грань представляла собой нумерацию английского алфавита. Выбор одной определённой грани из 26 также представлял собой инициализацию ключа шифрования. 

\section{Рефлектор}

Статичный механизм, позволяющий на одной машине реализовывать как шифрование, так и расшифрование. За счет связи 1 к 1 для открытого и закрытого символов, в <<ЭНИГМЕ>> расшифрование есть то же самое, что и шифрование.

\section{Коммутационная панель}

Динамический механизм, представляет собой также парный шифр. Коммутаторы, можно их рассматривать как некие кабеля, вставляются в коммутационную панель, на которой изображены символы английского алфавита. Один коммутатор имеет два конца, каждый из которых вставляется в два отверстия коммутационной панели. Так, например, если коммутатор был вставлен в два отверстия $(A, B)$, то $A$ и $B$ становятся парными символами при шифровании. 

\section{Количество вариантов ключа}

Учитывая все вышеописанные механизмы можно вычислить количество всех возможных ключей, которое будет равно $[((5!) / (2!)) \cdot (26³)] \cdot [(26!) / (6! \cdot 10! \cdot 210)] = 158 962 555 217 826 360 000$.