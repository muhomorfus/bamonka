\chapter{Теоретический раздел}

DES (in English, Data Encryption Standard)~---~алгоритм для симметричного шифрования, разработанный фирмой IBM и утверждённый правительством США в 1977 году как официальный стандарт. Размер блока для DES равен 64 битам. В основе алгоритма лежит сеть Фейстеля с 16 циклами (раундами) и ключом, имеющим длину 56 бит. Алгоритм использует комбинацию нелинейных ($S$-блоки) и линейных (перестановки $E$, $IP$, $IP^{-1}$) преобразований. Для DES рекомендовано несколько режимов:

\begin{itemize}
	\item ECB (electronic code book)~---~режим «электронной кодовой книги» (простая замена);
	\item CBC (cipher block chaining)~---~режим сцепления блоков;
	\item PCBC (propagating cipher block chaining)~---~режим распространяющегося сцепления блоков;
	\item CFB (cipher feed back)~---~режим обратной связи по шифротексту;
	\item OFB (output feed back)~---~режим обратной связи по выходу;
	\item Counter Mode (CM)~---~режим счётчика.
\end{itemize}
     
Прямым развитием DES в настоящее время является алгоритм Triple DES (3DES). В 3DES шифрование/расшифровка выполняются путём троекратного выполнения алгоритма DES. 

\section{Алгоритм DES}

Алгоритм шифрования DES состоит из следующих шагов.

\begin{enumerate}
	\item Совершить начальную перестановку с помощью функции $IP$.
	\item Создать 16 ключей.
	
	\begin{enumerate}
		\item С помощью функции $G$ провести удаление проверочных битов из ключа и провести перестановку. Получить 56-битный ключ из 64-битного блока.
		\item Разбить получившийся ключ на две половины $C_0$ (старшая) и $D_0$ (младшая).
		\item Для каждой итерации (от 1 до 16) совершить циклический сдвиг половин влево на величину, заданную в таблице. Величина зависит от номера итерации.
		\item Склеить половинки в 56-битный ключ.
		\item Провести перестановку и изменение размера ключа до 48 бит с помощью функции $H$.
	\end{enumerate}
	
	\item Разбить блок на две половины по 32 бита $L_0$ (старшая) и $R_0$ (младшая).
	\item На каждой итерации от 1 до 16 вычислить новые значения $L_{i} = R_{i-1}$ и $R_{i} = L_{i-1} \oplus f(R_{i-1}, K_i)$.
	
	\begin{enumerate}
		\item Провести перестановку с увеличение размера блока с 32 до 48 бит с помощью функции $E$.
		\item Сложить получившийся блок по модулю 2 с ключом.
		\item Разбить результат на 8 блоков по 6 бит.
		\item Для каждого блока, получить номер строки, который представляет собой два бита~---~первый и последний биты блока из 6 бит.
		\item Получить номер столбца, как 4 серединных бита.
		\item По номеру строки и столбца, найти соответствующее значение функции $S_i$. Значение представляет собой 4-битное число.
		\item Получить восемь 4-битных блоков.
		\item Склеить блоки в 32-битный блок.
		\item Провести перестановку с помощью функции $P$.
	\end{enumerate}
	
	\item Совершить конечную перестановку с помощью функции $IP^{-1}$.
\end{enumerate}

Расшифрование представляет собой тот же алгоритм, запущенный в обратном порядке.

\newpage

\section{Алгоритм PCBC}

На рисунках \ref{fig:pcbc_enc.png}--\ref{fig:pcbc_dec.png} изображена схемы шифрования и расшифрования с использованием алгоритма PCBC.

\begin{figure}[h!]
\centering
\includegraphics[width=0.75\textwidth]{assets/pcbc_enc.png}
\caption{Шифрование с использованием PCBC}
\label{fig:pcbc_enc.png}
\end{figure}

\begin{figure}[h!]
\centering
\includegraphics[width=0.75\textwidth]{assets/pcbc_dec.png}
\caption{Расшифрование с использованием PCBC}
\label{fig:pcbc_dec.png}
\end{figure}

