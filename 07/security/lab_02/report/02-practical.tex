\chapter{Практический раздел}

\section{Листинг алгоритма DES}

\begin{lstlisting}[language=C, label=lst:des, caption={Реализация алгоритма DES}]
Block f(Block r, Block k) {
    r = E(r) ^ k;

    Block b[8];
    for (int i = 8; i > 0; i--) {
        b[i - 1] = r & mask(6);
        r >>= 6;
    }

    for (int i = 8; i > 0; i--) {
        b[i - 1] = sTable[i - 1][(bit(b[i - 1], 5) << 1 | bit(b[i - 1], 0))][((b[i - 1] >> 1) & mask(4))];
    }

    r = 0;
    for (int i = 1; i <= 8; i++) {
        r <<= 4;
        r |= b[i - 1];
    }

    return P(r);
}

Block shift(Block block, int size, int shift) {
    shift %= size;

    Block k = block >> (size - shift);
    block <<= shift;

    block &= mask(size);
    block |= k;

    return block;
}

void generateKeys(Block k[17], Block key) {
    k[0] = G(key);

    Block c[17], d[17];
    c[0] = (k[0] >> 28) & mask(28);
    d[0] = k[0] & mask(28);

    for (int i = 1; i <= 16; i++) {
        c[i] = shift(c[i - 1], 28, shifts[i - 1]);
        d[i] = shift(d[i - 1], 28, shifts[i - 1]);

        k[i] = H(d[i] | (c[i] << 28));
    }
}

Block encryptDES(Block t, Block key) {
    Block t0 = IP(t);

    Block r[17], l[17];
    r[0] = t0 & mask(32);
    l[0] = (t0 >> 32) & mask(32);

    Block k[17];
    generateKeys(k, key);

    for (int i = 1; i <= 16; i++) {
        l[i] = r[i-1];
        r[i] = l[i-1] ^ f(r[i-1], k[i]);
    }

    Block t16 = l[16] | (r[16] << 32);

    return IPReversed(t16);
}

Block decryptDES(Block t, Block key) {
    Block t16 = IP(t);

    Block r[17], l[17];
    l[16] = t16 & mask(32);
    r[16] = (t16 >> 32) & mask(32);

    Block k[17];
    generateKeys(k, key);

    for (int i = 16; i > 0; i--) {
        r[i-1] = l[i];
        l[i-1] = r[i] ^ f(l[i], k[i]);
    }

    Block t0 = r[0] | (l[0] << 32);

    return IPReversed(t0);
}
\end{lstlisting}

\section{Листинг алгоритма PCBC}

\begin{lstlisting}[language=C, label=lst:pcbc, caption={Реализация алгоритма PCBC}]
Block* encryptPCBC(Block* blocks, int len, Block key, Block iv) {
    Block* result = calloc(len, sizeof(Block));
    if (result == NULL) {
        printf("calloc()\n");
        return NULL;
    }

    result[0] = encryptDES(blocks[0] ^ iv, key);
    for (int i = 1; i < len; i++) {
        result[i] = encryptDES(result[i - 1] ^ blocks[i - 1] ^ blocks[i], key);
    }

    return result;
}

Block* decryptPCBC(Block* blocks, int len, Block key, Block iv) {
    Block* result = calloc(len, sizeof(Block));
    if (result == NULL) {
        printf("calloc()\n");
        return NULL;
    }

    result[0] = decryptDES(blocks[0], key) ^ iv;
    for (int i = 1; i < len; i++) {
        result[i] = decryptDES(blocks[i], key) ^ result[i - 1] ^ blocks[i - 1];
    }

    return result;
}
\end{lstlisting}	



\section{Тестирование}

Корректность алгоритма проверялось путем применения дешифрации на шифрованное сообщение.

Тестирование было проведено на файлах с типами:

\begin{enumerate}[label=\arabic*)]
	\item текстовый (txt);
	\item графический (jpg, png);
	\item архив (zip).
\end{enumerate}



В таблице \ref{tbl:test} представлены тестовые данные.

\begin{table}[H]
	\begin{center}
		\centering
		\caption{Тестовые данные}
		\label{tbl:test}
		\begin{tabular}{|c|c|c|} 
			
			\hline
			\multicolumn{1}{|c|}{Номер теста}
			&  \multicolumn{1}{c|}{Тип файла} &  \multicolumn{1}{c|}{Содержимое файла}\\
			\hline
			
			1& txt &  {\specialcell{Стоят два бобра в речке...}} \\ \hline
			
			2& txt &  {\specialcell{@alexodnodvorcev}} \\ \hline
			
			3& txt &  \\ \hline
			
			4& png & {\specialcell{\includegraphics[width=1in]{../data/4.png}}} \\ \hline
			
			5& jpg & {\specialcell{\includegraphics[width=1in]{../data/5.jpg}}} \\ \hline
			
			6& zip & Архив с файлами предыдущих тестов \\ \hline
			
		\end{tabular}
	\end{center}
\end{table}

\newpage

\chapter*{Заключение}

В результате выполнения данной лабораторной работы был реализован в виде программы на языке Си алгоритм шифрование DES с режимом шифрования PCBC.