\chapter{Практический раздел}

\section{Листинг алгоритма AES}

\begin{lstlisting}[language=C, label=lst:aes, caption={Реализация алгоритма AES}]
void subBytes(Byte *state) {
    for (Byte i = 0; i < WordSize; i++) {
        for (Byte j = 0; j < NB; j++) {
            state[NB * i + j] = sTable[state[NB * i + j]];
        }
    }
}

void shiftRows(Byte *state) {
    Byte s, tmp;
    for (Byte i = 1; i < WordSize; i++) {
        s = 0;
        while (s < i) {
            tmp = state[NB * i + 0];
            for (Byte k = 1; k < NB; k++) {
                state[NB * i + k - 1] = state[NB * i + k];
            }
            state[NB * i + NB - 1] = tmp;
            s++;
        }
    }
}

void mixColumns(Byte *state) {
    Byte matrix[] = {0x02, 0x01, 0x01, 0x03};

    Byte column[WordSize], res[WordSize];

    for (Byte j = 0; j < NB; j++) {
        for (Byte i = 0; i < WordSize; i++) {
            column[i] = state[NB * i + j];
        }

        matrixMul(matrix, column, res);

        for (Byte i = 0; i < WordSize; i++) {
            state[NB * i + j] = res[i];
        }
    }
}

void addRoundKey(Byte *state, const Byte *expanded_key, Byte round_num) {
    for (Byte c = 0; c < NB; c++) {
        state[NB * 0 + c] = state[NB * 0 + c] ^ expanded_key[WordSize * NB * round_num + WordSize * c + 0];
        state[NB * 1 + c] = state[NB * 1 + c] ^ expanded_key[WordSize * NB * round_num + WordSize * c + 1];
        state[NB * 2 + c] = state[NB * 2 + c] ^ expanded_key[WordSize * NB * round_num + WordSize * c + 2];
        state[NB * 3 + c] = state[NB * 3 + c] ^ expanded_key[WordSize * NB * round_num + WordSize * c + 3];
    }
}

void subWord(Byte *key) {
    for (Byte i = 0; i < WordSize; i++) {
        key[i] = sTable[key[i]];
    }
}

void rotWord(Byte *key) {
    Byte tmp;
    tmp = key[0];
    for (Byte i = 0; i < 3; i++) {
        key[i] = key[i + 1];
    }
    key[3] = tmp;
}

void expandKey(const Byte *key, Byte *out) {
    for (Byte i = 0; i < NK; i++) {
        out[WordSize * i + 0] = key[WordSize * i + 0];
        out[WordSize * i + 1] = key[WordSize * i + 1];
        out[WordSize * i + 2] = key[WordSize * i + 2];
        out[WordSize * i + 3] = key[WordSize * i + 3];
    }

    for (int i = NK; i < NB * (NR + 1); i++) {
        Byte tmp[WordSize];

        tmp[0] = out[WordSize * (i - 1) + 0];
        tmp[1] = out[WordSize * (i - 1) + 1];
        tmp[2] = out[WordSize * (i - 1) + 2];
        tmp[3] = out[WordSize * (i - 1) + 3];
        if (i % NK == 0) {
            rotWord(tmp);
            subWord(tmp);

            Byte rCon[4] = {rConByte(i / NK), 0, 0, 0};

            workXor(tmp, rCon, tmp);
        } else if (NK > 6 && i % NK == WordSize) {
            subWord(tmp);
        }

        out[WordSize * i + 0] = out[WordSize * (i - NK) + 0] ^ tmp[0];
        out[WordSize * i + 1] = out[WordSize * (i - NK) + 1] ^ tmp[1];
        out[WordSize * i + 2] = out[WordSize * (i - NK) + 2] ^ tmp[2];
        out[WordSize * i + 3] = out[WordSize * (i - NK) + 3] ^ tmp[3];
    }
}

void aes(const Byte *buf, Byte *result, Byte *key) {
    Byte state[WordSize * NB];
    toState(buf, state);

    Byte expanded_key[WordSize * NB * (NR + 1)];
    expandKey(key, expanded_key);
    addRoundKey(state, expanded_key, 0);

    for (Byte r = 1; r < NR; r++) {
        subBytes(state);
        shiftRows(state);
        mixColumns(state);
        addRoundKey(state, expanded_key, r);
    }

    subBytes(state);
    shiftRows(state);
    addRoundKey(state, expanded_key, NR);

    fromState(state, result);
}\end{lstlisting}

\section{Листинг алгоритма OFB}

\begin{lstlisting}[language=C, label=lst:ofb, caption={Реализация алгоритма OFB}]
void ofb(Byte *data, int num_blocks, Byte *iv, Byte *key) {
    for (int i = 0; i < num_blocks; i++) {
        aes(iv, iv, key);

        for (int j = 0; j < BlockSize; j++) {
            data[BlockSize * i + j] = data[BlockSize * i + j] ^ iv[j];
        }
    }
}
\end{lstlisting}	

\section{Тестирование}

Корректность алгоритма проверялось путем применения дешифрации на шифрованное сообщение.

Тестирование было проведено на файлах с типами:

\begin{enumerate}[label=\arabic*)]
	\item текстовый (txt);
	\item графический (jpg, png);
	\item архив (zip).
\end{enumerate}



В таблице \ref{tbl:test} представлены тестовые данные.

\begin{table}[H]
	\begin{center}
		\centering
		\caption{Тестовые данные}
		\label{tbl:test}
		\begin{tabular}{|c|c|c|} 
			
			\hline
			\multicolumn{1}{|c|}{Номер теста}
			&  \multicolumn{1}{c|}{Тип файла} &  \multicolumn{1}{c|}{Содержимое файла}\\
			\hline
			
			1& txt &  {\specialcell{Стоят два бобра в речке...}} \\ \hline
			
			2& txt &  {\specialcell{@alexodnodvorcev}} \\ \hline
			
			3& txt &  \\ \hline
			
			4& png & {\specialcell{\includegraphics[width=1in]{../data/4.png}}} \\ \hline
			
			5& jpg & {\specialcell{\includegraphics[width=1in]{../data/5.jpg}}} \\ \hline
			
			6& zip & Архив с файлами предыдущих тестов \\ \hline
			
		\end{tabular}
	\end{center}
\end{table}

\newpage

\chapter*{Заключение}

В результате выполнения данной лабораторной работы был реализован в виде программы на языке Си алгоритм шифрования AES с режимом шифрования OFB.