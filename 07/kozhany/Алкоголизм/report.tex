\documentclass[14pt]{extarticle}
\usepackage[T1]{fontenc}
\usepackage[utf8]{inputenc}
\usepackage{amsmath,amssymb}
\usepackage[russian]{babel}
\usepackage{graphicx}

\begin{document}
\title{\text{Алкоголизм}}
\maketitle


\section{Введение}

Алкоголизм~---~одна из самых острых проблем современного общества, которая затрагивает в наши дни не только мужчин, но даже женщин, подростков и детей. Для одних алкоголь~---~лекарство от всех болезней, для других~---~средство для снятия стресса и психологического напряжения, для третьих~---~необходимое условие веселого праздника.

Какую угрозу несет алкоголизм? Самое страшное в алкоголизме то, что человек, страдающий алкогольной зависимостью, никогда не признает свою болезнь. Он считает, что он абсолютно здоров, и думает, что в любой момент может перестать пить по собственному желанию.
Со временем пьющий человек теряет способность адекватно оценивать свое состояние и считает, что окружающие сильно завышают свои требования по отношению к нему, хотят от него чего-то нереального. Появляются всевозможные отговорки в свое оправдание, такие как:

\begin{itemize}
	\item все пили, я тоже выпил;
	\item захочу~---~выпью, не захочу~---~не выпью;
	\item сам прекращу выпивать, когда мне захочется.
\end{itemize}	

Алкоголь наносит непоправимый вред всем органам и системам человеческого организма. Со временем развиваются тяжелые хронические заболевания, самым грозным из которых является цирроз печени, так как он неизлечим и неминуемо приводит к смерти. Нарушается и психика человека, что проявляется сначала в проблемах взаимоотношений с близкими, а затем развиваются тяжелые формы психозов.

Алкоголизм~---~это просто проблема общества и социальная беда, это еще и медицинская проблема. Но не стоит опускать руки, в наши дни, сочетая профессионализм и опыт врачей с современными методиками, можно добиться излечения от этого недуга.
\section{Понятие, виды алкоголизма}

\subsection{Понятие алкоголизма}

Алкоголизм~---~очень страшное заболевание, опасность которого заключается в том, что на первичные его симптомы, как правило, не обращают никакого внимания, позволяя болезни прогрессировать, переходить в хронические состояния. Поэтому выявление признаков алкогольной зависимости необходимо именно на начальной стадии болезни. Какие же бывают симптомы алкоголизма?

Первые симптомы алкоголизма~---~проявление дискомфорта в трезвом состоянии. Испытывая психическую неудовлетворенность при попытке «трезво взглянуть на мир», человек спешит вернуться в состояние алкогольного опьянения, где всего его расстройства нивелируются выпитым спиртным. Причем дозировка алкоголя для получения этого состояния постоянно растет. Известные многим провалы памяти во время алкогольного опьянения~---~также один из первых симптомов алкогольной зависимости. А вскоре к нему добавляются тяжелое утреннее состояние, плохой сон, нарушение речи и координации движения, усталость и апатия. Все это нельзя переводить в юмор утреннего похмелья и не стоит бежать с друзьями за пивом, ведь речь идет о прогрессирующем алкоголизме! Если не предпринимать никаких мер~---~болезнь начинает быстро прогрессировать. Алкоголик становится раздражительным, периоды агрессии быстро сменяются долгими периодами апатии, человек утрачивает интерес к нормальной жизни и всему, что не связано со спиртным. Неоправданная агрессивность создает много конфликтных ситуаций, как дома, так и на работе. В таком состоянии организм уже не в состоянии перерабатывать алкоголь в прежних объемах, а для опьянения достаточно небольших доз. В этой стадии алкоголизма начинают проявляться различные заболевания печени, поджелудочной железы, почек, головного мозга, центральной нервной системы. Поводом бить тревогу может послужить тот факт, что организм утрачивает способность к самозащите и теряется рвотный рефлекс. Если после бурной вечеринки вы испытываете жуткое похмелье, но при этом не чувствуете рвотных позывов~---~стоит задуматься о визите к наркологу. Это означает, что часть организма уже привыкла к употреблению спиртного в больших количествах, и не считает нужным защищаться от него.

Алкоголизм страшен тем, что его развитие проходит для человека незаметно. Причем это незаметно как для самого алкоголика, так и для окружающих, часто списывающих симптомы алкоголизма на многочисленные традиции, праздники и прочее. Алкоголизм~---~это страшное заболевание, ведущее к деструкции мозга и нервной системы, приводящее, в конечном итоге, к летальному исходу.

\subsection{Виды алкоголизма}

Сам алкоголизм распознать не так уж и просто. При слове «алкоголик» чаще всего представляют опустившегося, грязного человека с небритостью и стеклянными глазами. Однако алкоголиком может быть фактически любой: начиная от элегантной женщины и заканчивая примерным подростком. Острое опьянение может случиться с любым человеком при превышении нормы употребления спиртного. Синдром острого алкоголизма чаще бывает у непьющих людей, так как организм не толерантен к этанолу. Тождественность острого и хронического алкоголизма равна нулю.

\subsubsection*{Хронический алкоголизм}

При хроническом алкоголизме спиртные напитки употребляются регулярно. Этот вид алкогольной зависимости развивается очень быстрыми темпами. Зачастую окружающие знают о проблеме человека со спиртным. Алкоголь употребляется в рамках своеобразного ритуала: например, принимается рюмка после трудового дня, как вознаграждение за «трудный день». Возможно принятие спиртного с определенным человеком при каждой встрече. Процесс лечения от хронического алкоголизма начинается с разрушения ритуалов, связанных с выпивкой. При хроническом алкоголизме изменяется реакция организма:

\begin{itemize}
	\item теряется самоконтроль;
	\item развивается похмельный синдром;
	\item изменяется стойкость организма к алкоголю;
	\item проявляется алкогольный психоз.
\end{itemize}
	
На самых ранних стадиях он проявляется потерей самоконтроля. Главный признак хронического алкоголизма~---~похмельный синдром. Этот синдром дает о себе знать дрожанием, потливостью, тахикардией и подавленным настроением. Могут возникать судороги и галлюцинации, которые смягчаются после опохмеления.

\subsubsection*{Запойный алкоголизм}

Запойный алкоголизм характерен для людей, у которых отсутствует возможность пить регулярно. В рабочие дни человек может себе позволить выпить немного, но если произойдет психологическая дестабилизация~---~он уходит в запой на несколько дней. Особо тяжелый случай, если запои продолжаются месяцами. Выйдя из запоя (с помощью медиков или самостоятельно), они возвращаются к нормальному образу жизни. Через какое-то время запой повторяется. Окружающие не всегда могут распознать запойного алкоголика. Этот вид алкоголизма очень опасен для здоровья человека, так как стойкость организма к спиртному очень низкая. Для начала лечения необходимо определить причину или «пусковую кнопку», с которой начинается запой.

\subsubsection*{Тайный и пивной алкоголизм}

Тайный алкоголизм может быть хроническим или запойным. Главное отличие состоит в том, что его тщательно скрывают от окружающих. Тайный вид алкоголизма чаще всего присущий состоятельным людям и женщинам, которые считают эту зависимость позором. Людям часто успешно удается, проявляя изобретательность, скрывать свою болезнь. Это формирует ложную уверенность о незнании окружающих. Однако спустя время алкоголь может проявиться: оказать влияние на внешность или стать причиной заболеваний. Скрывая свою зависимость, люди могут прибегать к двум способам употребления алкоголя: постоянно понемногу пить слабоалкогольные напитки или в определенное время пить много крепких напитков.

Пивной алкоголизм является итоговой и неизбежной стадией многолетнего употребления пива. При пивном алкоголизме человек выпивает не меньше 1 л пива каждый день или несколько раз в неделю. Этот вид коварен тем, что многие считают употребление пива невинным занятием, которое не приносит вред здоровью. Зачастую пивной алкоголизм сопровождает хронический или запойный. Пивная зависимость также характеризуется ритуальностью: с годами формируется привычка употреблять пиво в определенных ситуациях.

Так же алкоголизм классифицируется по типам:

\begin{itemize}
	\item альфа-алкоголизм~---~ежедневно употребляются слабоалкогольные напитки;
	\item бета-алкоголизм~---~слабоалкогольные напитки употребляются изредка;
	\item гамма-алкоголизм~---~употребляются крепкие напитки редко, но в больших дозах.
\end{itemize}
	
\subsubsection*{Алкогольная зависимость у женщин и детей}

Для женщин алкоголизм особенно опасен. У них процесс привыкания к спиртному происходит гораздо быстрее, чем у мужчин. Так как развитие симптомов зависимости очень стремительное, лечение алкоголизма осложняется. Самая распространенная патология для женщин, употребляющих алкоголь~---~поражение поджелудочной железы и печени. Действие алкоголя часто приводит женщин к беспорядочной половой жизни. Отсутствие гигиены во время интимной близости приводит к венерическим заболеваниям.

При длительном злоупотреблении алкоголем у женщин развиваются психические нарушения, меняется характер, появляется нервозность и агрессивное поведение. Причины женского алкоголизма:

\begin{itemize}
	\item Проблемы социального характера: трудности в материальном плане, проблемы на работе.
	\item Эмоциональное потрясение, стресс.
	\item Общение с пьющими мужчинами.
	\item Заболевания нервного или психического характера.
	\item Работа в алкогольной сфере.
	\item Криминал, занятие проституцией.
\end{itemize}
	
Так как женщины редко признаются в своем пристрастии к алкоголю, процесс лечения у них проходит гораздо тяжелее.

Детским называется алкоголизм, признаки которого проявляются у ребенка возрастом до 18 лет. Прием алкоголя детьми не только наносит вред здоровью, но и служит причиной девиантного поведения. Дети становятся агрессивными, и родители теряют над ними контроль. Этот вид алкоголизма наносит большой вред обществу. Под влиянием спиртного часто совершаются преступления насильственного характера. Детский алкоголизм характерен такими особенностями:

\begin{itemize}
	\item быстрое привыкание к алкоголю;
	\item употребление спиртного в больших дозах и тайком;
	\item низкая эффективность лечения;
	\item быстрое развитие запойного пьянства.
\end{itemize}
	
Чаще всего дети употребляют алкоголь в обществе сверстников. Подростки уверены, что принимая алкоголь, они кажутся взрослее. Тяжелая степень опьянения у детей является результатом отсутствия самоконтроля. Список причин алкогольной зависимости у детей очень большой. Самые распространенные из них:

\begin{itemize}
	\item Ребенок пытается самоутвердиться в компании товарищей.
	\item Неприятности в школе, непонимание родителей, неудачи личного характера.
	\item Пьянство родителей.
	\item Неконтролируемые деньги.
	\item Влияние со стороны.
\end{itemize}	

\section{Последствия алкоголя}

Последствия алкоголизма условно можно разделить на два класса. Первый~---~это негативные последствия для самого алкоголика, связанные с ухудшением его здоровья и деградацией личности. Второй~---~отрицательные последствия для общества, а именно~---~увеличение количества социальных проблем, связанных со злоупотреблением спиртным.

Алкоголика можно узнать по внешним признакам. Они выглядят старше своих лет, их волосы тусклые и взлохмаченные. Лицо алкоголика равномерного розоватого оттенка, который в сочетании с пастозностью вызывает эффект «распаренности». С годами сосуды лица становятся постоянно переполненными кровью, а когда человек какой-то период времени воздерживается от алкголя, эта краснота исчезает. Зато на фоне общей бледности проступает телеангиэктазии~---~постоянное расширение мелких сосудов кожи~---~на щеках, краях носа, шее и верхней части груди. Кожа становится дряблой.

Мышечный тонус восстанавливается при приеме спиртного. Расслабленность круговой мышцы лица придает алкоголикам особый облик. Небрежность в одежде и нечистоплотность дополняют характерный образ алкоголика.

Алкоголизм имеет очень серьезные психические последствия. Это~---~астения, психопатизация, снижение личностных качеств (утрата интересов и нравственных ценностей, огрубение.

К психическим последствия также относятся аффективные расстройства~---~перепады настроения, агрессивность, депрессии и дисфории, склонность к суициду. В тяжелых случаях алкоголизм приводит к слабоумию.

Для алкоголиков характерен плоский, бестактный юмор. Его даже называют «алкогольным». Галлюцинации, бред ревности~---~проявления, типичные для алкоголиков. Не говоря уже о делириозных и галлюцинаторно-параноидных синдромах.

Кроме психических расстройств, алкоголизм чреват неврологическими нарушениями. У людей, постоянно принимающих большое количество спиртного, наблюдаются острые мозговые синдромы~---~эпилептиформный, мозжечковый, синдром Гайе-Вернике. Может также возникнуть атрофия зрительного или слухового нервов~---~особенно часто это происходит при употреблении суррогатов.

\subsection{Последствия женского алкоголизма}

Женский алкоголизм имеет еще более серьезные последствия, потому что женщины становятся матерями. Если во время беременности будущая мать продолжает пить, большая вероятность того, что она родит плод с алкогольным синдромом~---~грубыми морфологическими нарушениями. Это могут быть неправильные пропорции и размеры головы, лицевой и мозговой области черепа, тела и конечностей. У новорожденного могут быть шарообразные глаза, или посаженные глубоко, утопленное основание носа или широкая переносица, недоразвитие челюстных костей и другие патологии.

Кроме внешних признаков, у таких детей наблюдается врожденная низкая мозговая недостаточность, которая выражается гиперподвижностью, отсутствием сосредоточенности, агрессивностью, склонностью к разрушению. Моторное и психическое развитие у детей, рожденных алкоголичками, неудовлетворительное или замедленное, им сложно овладевать практическими навыками.

При пьянстве родителей дети растут в сложной обстановке, травмирующей психику ребенка, поэтому у них часто возникают энурезы, заикания, ночные страхи, агрессивность, упрямство, уходы из дома. Эмоциональное поведение таких детей неустойчиво~---~часто возникают тревоги, депрессии, наблюдается склонность к суициду. Нарушения в психическом развитии вызывает трудности в обучении и контактах со сверстниками.

\subsection{Последствия детского алкоголизма}

Если женский организм в связи с физиологическими особенностями больше подвержен алкоголизму, чем мужской, то детский организм практически не имеет защиты от спиртного. Для того, чтобы стать алкоголиком, ребенку достаточно нескольких месяцев употребления алкоголя. И последствия этого явления просто ужасают.

Алкоголь разрушает неокрепшие органы ребенка, особенно страдают печень и сердечно-сосудистая система.

У ребенка замедляется умственное развитие, растет агрессивность. Под действием спиртного дети просто «теряют тормоза» и их поступки поражают своей жестокостью. Не имея денег для того, чтобы купить новую порцию спиртного, юные алкоголики начинают попрошайничать или идут на преступления. Если денег не хватает на водку, вино или пиво, малолетние алкоголики покупают клей. Становясь токсикоманами, они стремительно деградируют психически и уничтожают себя физически.
\section{Лечение алкоголизма}

\subsection{Методы лечения алкогольной зависимости}

Существуют разнообразные методы лечения алкоголизма, но в каждом случае во время лечения больной должен полностью отказаться от приема спиртного. Будет этот отказ добровольным или принудительным, зависит от самого больного. Борьба с алкогольной зависимостью предполагает комплексный подход к проблеме зависимости от спиртного. На главном этапе терапии применяются препараты для лечения алкоголизма, а на его завершающем этапе самой важной является помощь больному в условиях адаптации к трезвой жизни, полностью исключающей спиртное.

Методы избавления от алкогольной зависимости, используемые современной медициной, весьма эффективны, однако полное избавление от алкоголизма возможно только тогда, когда к лечению подключается психотерапия.

На больного воздействуют психологически, чтобы создать у него негатив по отношению к спиртному. Таким методом является кодирование, подтвердившее свою эффективность уже давно и успешно применяющееся в лечении алкоголизма в настоящее время.

Но такой способ избавления от алкогольной зависимости имеет один недостаток~---~высокую вероятность срыва, когда больной снова начинает пить, и все лечение оказывается бесполезным. К тому же, психологические способы лечения алкоголизма зависят от психики конкретного человека, поэтому такую помощь больному должны оказывать только опытные психиатры.

В отличие от лечения алкоголизма кодированием, более безопасным считается применение лекарственных запрещающих методов. Однако и здесь есть свои подводные камни. Безопасность метода гарантирована только в случаях абсолютной трезвости больного в период приема этих препаратов.

Лечение алкоголизма~---~это комплексный подход к здоровью больного. Важным моментом противоалкогольной терапии является восстановление организма человека, длительно и в больших количествах употреблявшего спиртное. Для этих целей используются препараты, нормализующие работу внутренних органов, сердечнососудистой и нервной систем больного.

В лечении больных алкоголизмом нельзя забывать об определяющей роли психологического фактора. Препараты для лечения алкоголизма необходимо сочетать с помощью психологов. Только уверенный в своих силах человек сможет полностью избавиться от пагубного пристрастия к спиртному. Иначе болезнь может возвратиться, и тогда лечить ее будет гораздо сложнее.
\section{Выводы}

Обобщая, можно сказать, что смысл жизни в самой жизни, и экзистенция сама жизнь. Вопрос ценности жизненных проявлений, вещей, ситуаций, смысла жизни, экзистенции в целом учетом ее составляющих форм и сфер требует дальнейшей разработки как общего фактора существования человека и включения ее в психотерапевтический в наркологии процесс с целью формирования психической устойчивости, налаживания полноценной деятельности. Систематическое употребление спиртного обусловленное стремлением и желанием жить в состоянии опьянения, при этом перенос значимой жизни в него, включенность в опьянение глубинной своей сущности, активизация содержания своего «я», повышение эмоционального переживания своего бытия в этом состоянии, предвкушении его, стремление и желание этого состояния. А. Форель назвал эти переживания «находится под влиянием своих болезненных похотей».

\end{document}