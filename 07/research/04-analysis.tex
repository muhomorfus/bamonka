\chapter{Анализ предметной области}

В данной части рассматриваются актуальность задачи, виды запросов к базам данных, производятся анализ СУБД PostgreSQL и Greenplum и сравнение способов составления запросов к базам данных.

\section{Актуальность задачи}

Последнее время веб-технологии переживают стремительный рост. Различные сервисы позволили значительно повлиять на многие сферы современной жизни~\cite{murodilov2023web}. В связи с этим хранение и обработка большого объема информации стало актуальной задачей~\cite{картавец2019актуальность}. СУБД используются с целью централизовать работу с данными~\cite{стасышин2022проектирование}.

СУБД предоставляют специальные языки запросов для получения, фильтрации и изменения данных~\cite{song2023testing}. Такие языки позволяют работать с базами данных, включающих в себя множественные отношения между сущностями, делать сложные запросы на получение информации~\cite{hohenstein1992sql}. Но, на практике, наиболее популярные языки запросов, такие как SQL, в случае сложных запросов оказываются очень сложными для понимания, что усложняет разработку и может приводить к возникновению ошибок~\cite{date2011sql}.

Для упрощения работы с БД, в том числе для неквалифицированных пользователей, активно ведутся разработки таких методов, как, например,  методы генерации запросов к СУБД на основе естественного языка, что говорит об актуальности задачи~\cite{fu2023catsql,naik2023sql,sun2023sql}.


\section{Виды запросов к базам данных}

Для определения видов запросов к БД необходимо рассмотреть виды отношений между сущностями и то, каким образом будут использованы эти отношения при запросах к СУБД.

\subsection{Виды отношений между сущностями}

Отношения между сущностями представляют собой способ связывания данных, обеспечивающий целостность и согласованность хранимой информации~\cite{чаглей2023сравнение}. Существует три вида отношений между сущностями. Ниже приведено описание каждого из них~\cite{стасышин2022проектирование}.

\begin{enumerate}
	\item \textbf{Один-к-одному.} Данный тип связи подразумевает связь единственного экземпляра сущности с единственным экземпляром другой сущности.
	\item \textbf{Один-ко-многим.} Данный тип связи подразумевает связь единственного экземпляра сущности с одним или более экземплярами другой сущности.
	\item \textbf{Многие-ко-многим.} Данный тип связи подразумевает связь одного или более экземпляров сущности с одним или более экземплярами другой сущности.
\end{enumerate}

\subsection{Виды использования связей при запросах к БД}

Использование связи при запросах к базам данных характеризуется количеством связанных сущностей, учитываемом при построении запроса. Запрос может осуществляться как относительно одной сущности без учета связей, так и относительно нескольких связанных сущностей~\cite{li2021teaching}.

\begin{enumerate}
	\item \textbf{Запросы относительно одной сущности.} Данный тип запросов не учитывает какие-либо связи сущности с другими. Условия выборки и набор запрашиваемых данных задаются относительно одной сущности.
	\item \textbf{Запросы относительно нескольких сущностей.} Данный тип запросов подразумевает использование в запросе объединения двух–трех сущностей. 
	\item \textbf{Запросы относительно большого количества сущностей.} Данный тип запросов подразумевает использование в запросе объединения более, чем трех сущностей. В особо редких случаях, количество сущностей в запросе может доходить до 100~\cite{mancini2022efficient}.
\end{enumerate}

\section{Анализ СУБД PostgreSQL и Greenplum}

\section{Сравнение способов составления запросов к базам данных}

