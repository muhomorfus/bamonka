\chapter{Анализ предметной области}

В настоящее время веб-технологии переживают стремительный рост. Различные сервисы позволили значительно повлиять на многие сферы современной жизни~\cite{murodilov2023web}. В связи с этим хранение и обработка большого объема информации стало актуальной задачей~\cite{картавец2019актуальность}. СУБД используются с целью централизовать работу с данными~\cite{стасышин2022проектирование}.

СУБД предоставляют специальные языки запросов для получения, фильтрации и изменения данных~\cite{song2023testing}. Такие языки позволяют работать с базами данных, включающих в себя множественные отношения между сущностями, делать сложные запросы на получение информации~\cite{hohenstein1992sql}. Но, на практике, наиболее популярные способы составления запросов, такие как SQL, в случае сложных запросов оказываются очень нетривиальными для понимания, что может приводить к возникновению ошибок~\cite{date2011sql}.

Для упрощения работы с БД, в том числе для неквалифицированных пользователей, активно ведутся разработки таких методов, как, например,  методы генерации запросов к СУБД на основе естественного языка~\cite{fu2023catsql}\cite{naik2023sql}\cite{sun2023sql}.


\section{Виды отношений между сущностями}

Отношения между сущностями представляют собой способ связывания данных, обеспечивающий целостность и согласованность хранимой информации~\cite{чаглей2023сравнение}. Существует следующие виды отношений между сущностями~\cite{стасышин2022проектирование}.

\textbf{Бинарная связь.} Данный тип связи подразумевает какую-либо связь экземпляров двух сущностей. В свою очередь, бинарный тип связи можно разделить на несколько подвидов.


\begin{enumerate}
	\item \textbf{Один-к-одному.} Данный тип связи подразумевает связь единственного экземпляра сущности с единственным экземпляром другой сущности.
	\item \textbf{Один-ко-многим.} Данный тип связи подразумевает связь единственного экземпляра сущности с одним или более экземплярами другой сущности.
	\item \textbf{Многие-ко-многим.} Данный тип связи подразумевает связь одного или более экземпляров сущности с одним или более экземплярами другой сущности.
\end{enumerate}

\textbf{n-арная связь.} Данный тип связи подразумевает совместную связь нескольких экземпляров различных сущностей с несколькими экземплярами каких-либо других сущностей.

\section{Виды запросов к БД}

Запросы к базам данных позволяют осуществлять различные манипуляции с хранимыми данными. Соответственно, запросы можно разделить на виды, соответствующие выполняемой операции~\cite{truica2015performance}. 

\begin{enumerate}
	\item Запросы создания данных (Create). 
	\item Запросы чтения данных (Read) можно разделить на виды по использованию связей. Использование связи характеризуется количеством связанных сущностей, учитываемом при построении запроса. Запрос может осуществляться как относительно одной сущности без учета связей, так и относительно нескольких связанных сущностей~\cite{li2021teaching}\cite{mancini2022efficient}. 
	\item Запросы изменения данных (Update).
	\item Запросы удаления данных (Delete).
\end{enumerate}
