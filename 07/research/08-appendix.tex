\begin{appendices}

\chapter{Этапы опроса определения «понятности» способов составления запросов}

\section{Вопрос 1}

В листинге \ref{1:q} приведен запрос из вопроса 1, составленный с помощью SQL.

\begin{lstlisting}[label=1:q,caption=Вопрос 1]
select contact_name from customers where customer_id = 'ANTON';
\end{lstlisting}

Варианты ответа:

\begin{enumerate}
	\item \textbf{Antonio Moreno;}
	\item Pirkko Koskitalo;
	\item Stas Baluev;
	\item Palle Ibsen;
	\item Thomas Hardy;
	\item bebebe;
	\item Papa Johns;
	\item Thomas Parvs.
\end{enumerate}

\section{Вопрос 2}

В листинге \ref{2:q} приведен запрос из вопроса 2, составленный с помощью Prolog/Datalog.

\begin{lstlisting}[label=2:q,caption=Вопрос 2]
-- select product_name from products where product_id = 1;

products(1, ProductName, _, _, _, _, _, _, _, _).
\end{lstlisting}

Варианты ответа:

\begin{enumerate}
	\item \textbf{Chai;}
	\item Daet;
	\item Mozzarella di Giovanni;
	\item Pivo;
	\item Outback Lager;
	\item Chocolade;
	\item Papa Johns;
	\item Liqueour.
\end{enumerate}

%\section{Вопрос 3}
%
%В листинге \ref{3:q} приведен запрос из вопроса 3, составленный с помощью Datalog.
%
%\begin{lstlisting}[label=3:q,caption=Вопрос 3]
%-- select unit_price from products where product_name = 'Tofu';
%
%??
%\end{lstlisting}
%
%Варианты ответа:
%
%\begin{enumerate}
%	\item \textbf{23.25;}
%	\item 11;
%	\item 20;
%	\item 678;
%	\item 1000;
%	\item 20;
%	\item 13;
%	\item 14.
%\end{enumerate}

\section{Вопрос 3}

В листинге \ref{3:q} приведен запрос из вопроса 3, составленный с помощью ORM.

\begin{lstlisting}[label=3:q,caption=Вопрос 3]
-- select employee_id from employees where last_name = 'King';

type Employee struct {
	EmployeeID int `gorm:"employee_id"`
}
var employee Employee
db.Table("employees e").Select("e.employee_id").Where("e.last_name = ?", "King").First(&employee)
\end{lstlisting}

Варианты ответа:

\begin{enumerate}
	\item \textbf{7;}
	\item 0;
	\item 1;
	\item 2;
	\item 3;
	\item 10;
	\item 11;
	\item 6.
\end{enumerate}

\section{Вопрос 4}

В листинге \ref{4:q} приведен запрос из вопроса 4, составленный с помощью SQL.

\begin{lstlisting}[label=4:q,caption=Вопрос 4]
select last_name from employees where title = 'Sales Representative' and title_of_courtesy = 'Mrs.';
\end{lstlisting}

Варианты ответа:

\begin{enumerate}
	\item \textbf{Peacock;}
	\item Ivan;
	\item Stas;
	\item Hanks;
	\item Davolio;
	\item Fuller;
	\item Henry;
	\item King.
\end{enumerate}

\section{Вопрос 5}

В листинге \ref{5:q} приведен запрос из вопроса 5, составленный с помощью Prolog/Datalog.

\begin{lstlisting}[label=5:q,caption=Вопрос 5]
-- select quantity from order_details where order_id = 10248 and product_id = 11;

order_details(10248, 11, _, Quantity, _).
\end{lstlisting}

Варианты ответа:

\begin{enumerate}
	\item \textbf{12;}
	\item 4;
	\item 8;
	\item 100;
	\item 13;
	\item 10;
	\item 7;
	\item 3.
\end{enumerate}

%\section{Вопрос 7}
%
%В листинге \ref{7:q} приведен запрос из вопроса 7, составленный с помощью Datalog.
%
%\begin{lstlisting}[label=7:q,caption=Вопрос 7]
%-- select company_name from suppliers where city = 'Melbourne' and contact_title = 'Marketing Manager';
%
%??
%\end{lstlisting}
%
%Варианты ответа:
%
%\begin{enumerate}
%	\item \textbf{Pavlova, Ltd.;}
%	\item Mayumi's;
%	\item Tokyo Traders;
%	\item Plutzer Lebensmittelgroßmärkte AG;
%	\item Karkki Oy;
%	\item PoelPospal inc.;
%	\item Gai pâturage;
%	\item Sockets.
%\end{enumerate}

\section{Вопрос 6}

В листинге \ref{6:q} приведен запрос из вопроса 6, составленный с помощью ORM.

\begin{lstlisting}[label=6:q,caption=Вопрос 6]
-- select order_id from orders where customer_id = 'QUEEN' and order_date = '1997-10-14';

type Order struct {
	OrderID int `gorm:"order_id"`
}
var order Order
db.Table("orders o").Select("o.order_id").Where("o.customer_id = ?", "QUEEN").Where("o.order_date = ?", "1997-10-14").First(&order)

\end{lstlisting}

Варианты ответа:

\begin{enumerate}
	\item \textbf{10704;}
	\item 9123;
	\item 12103;
	\item 10000;
	\item 9192;
	\item 13012;
	\item 11111;
	\item 12280.
\end{enumerate}

\section{Вопрос 7}

В листинге \ref{7:q} приведен запрос из вопроса 7, составленный с помощью SQL.

\begin{lstlisting}[label=7:q,caption=Вопрос 7]
select c.contact_title from orders o join customers c on o.customer_id = c.customer_id where o.order_id = 10274;
\end{lstlisting}

Варианты ответа:

\begin{enumerate}
	\item \textbf{Accounting Manager;}
	\item Sales Manager;
	\item Marketologist;
	\item Beer Somelier;
	\item Owner;
	\item Sales Agent;
	\item Representative Manager;
	\item Marketing Assistant.
\end{enumerate}

\section{Вопрос 8}

В листинге \ref{8:q} приведен запрос из вопроса 8, составленный с помощью Prolog/Datalog.

\begin{lstlisting}[label=8:q,caption=Вопрос 8]
-- select u.state_name from customers c join us_states u on c.region = u.state_abbr where customer_id = 'LONEP';

us_states(_, StateName, StateAbbr, _), customers("LONEP", _, _, _, _, StateAbbr, _, _, _, _).
\end{lstlisting}

Варианты ответа:

\begin{enumerate}
	\item \textbf{Oregon;}
	\item New Orlean;
	\item Wyoming;
	\item Idaho;
	\item Alabama;
	\item New Mexico;
	\item Alaska;
	\item Africa.
\end{enumerate}

%\section{Вопрос 11}
%
%В листинге \ref{11:q} приведен запрос из вопроса 11, составленный с помощью Datalog.
%
%\begin{lstlisting}[label=11:q,caption=Вопрос 11]
%-- select p.product_name from order_details o join products p on o.product_id = p.product_id where o.order_id = 10248 and quantity = 5;
%
%??
%\end{lstlisting}
%
%Варианты ответа:
%
%\begin{enumerate}
%	\item \textbf{Mozzarella di Giovanni;}
%	\item Cheese;
%	\item Pizza;
%	\item Olives;
%	\item Singaporean Hokkien Fried Mee;
%	\item Fried Cheese;
%	\item Queso Cabrales;
%	\item Röd Kaviar.
%\end{enumerate}

\section{Вопрос 9}

В листинге \ref{9:q} приведен запрос из вопроса 9, составленный с помощью ORM.

\begin{lstlisting}[label=9:q,caption=Вопрос 9]
-- select o.quantity from order_details o join products p on o.product_id = p.product_id where o.order_id = 10250 and p.product_id = 51;

type Order struct {
	Quantity int `gorm:"quantity"`
}
var order Order
db.Table("order_details o").Select("o.quantity").Joins("join products p on o.product_id = p.product_id").Where("o.order_id = ?", 10250).Where("p.product_id = ?", 51).First(&order)
\end{lstlisting}

Варианты ответа:

\begin{enumerate}
	\item \textbf{35;}
	\item 21;
	\item 19;
	\item 10;
	\item 30;
	\item 167;
	\item 9;
	\item 0.
\end{enumerate}

\section{Вопрос 10}

В листинге \ref{10:q} приведен запрос из вопроса 10, составленный с помощью SQL.

\begin{lstlisting}[label=10:q,caption=Вопрос 10]
select t.territory_description from employees e join employee_territories et on e.employee_id = et.employee_id join territories t on t.territory_id = et.territory_id where e.first_name = 'Robert' and t.territory_id = '95060';
\end{lstlisting}

Варианты ответа:

\begin{enumerate}
	\item \textbf{Santa Cruz;}
	\item Hoffman Estates;
	\item Chicago;
	\item Denver;
	\item Colorado Springs;
	\item Santa Monica;
	\item Menlo Park;
	\item Campbell.
\end{enumerate}

\section{Вопрос 11}

В листинге \ref{11:q} приведен запрос из вопроса 11, составленный с помощью Prolog/Datalog.

\begin{lstlisting}[label=11:q,caption=Вопрос 11]
--- select c.category_name from categories c join products p on p.category_id = c.category_id join order_details od on p.product_id = od.product_id where od.order_id = 10248 and od.quantity = 10;

categories(CategoryID, Name, _, _), products(ProductID, _, _, CategoryID, _, _, _, _, _, _), order_details(10248, ProductID, _, 10, _).
\end{lstlisting}

Варианты ответа:

\begin{enumerate}
	\item \textbf{Grains/Cereals;}
	\item Beer;
	\item Confections;
	\item Denver;
	\item Produce;
	\item Seafood;
	\item Dairy Products;
	\item Delisious.
\end{enumerate}

%\section{Вопрос 15}
%
%В листинге \ref{15:q} приведен запрос из вопроса 15, составленный с помощью Datalog.
%
%\begin{lstlisting}[label=15:q,caption=Вопрос 15]
%--- select r.region_description from territories t join region r on r.region_id = t.region_id join employee_territories et on t.territory_id = et.territory_id where et.employee_id = 2
%
%???
%\end{lstlisting}
%
%Варианты ответа:
%
%\begin{enumerate}
%	\item \textbf{Eastern;}
%	\item Beer;
%	\item Confections;
%	\item Denver;
%	\item Southern;
%	\item Northern;
%	\item Western;
%	\item Delisious.
%\end{enumerate}

\section{Вопрос 12}

В листинге \ref{12:q} приведен запрос из вопроса 12, составленный с помощью ORM.

\begin{lstlisting}[label=12:q,caption=Вопрос 12]
--- select s.phone from customers c join orders o on c.customer_id = o.customer_id join shippers s on s.shipper_id = o.ship_via where c.customer_id = 'VICTE' and o.order_id = 10450

type Shipper struct {
	Phone string `gorm:"phone"`
}
var shipper Shipper
db.Table("customers c").Select("s.phone").Joins("join orders o on c.customer_id = o.customer_id").Joins("join shippers s on s.shipper_id = o.ship_via").Where("c.customer_id = ?", "VICTE").Where("o.order_id = ?", 10450).First(&shipper)
\end{lstlisting}

Варианты ответа:

\begin{enumerate}
	\item \textbf{(503) 555-3199;}
	\item Localhost;
	\item 4;
	\item (503) 111-9931;
	\item (503) 222-9831;
	\item (999) 333-9831;
	\item (891) 112-1566;
	\item Eleven.
\end{enumerate}

\section{Вопрос 13}

В листинге \ref{13:q} приведен запрос из вопроса 13, составленный с помощью SQL.

\begin{lstlisting}[label=13:q,caption=Вопрос 13]
select e.employee_id from employees e join employee_territories et on e.employee_id = et.employee_id where et.territory_id = '03801'
\end{lstlisting}

Варианты ответа:

\begin{enumerate}
	\item \textbf{9;}
	\item 1;
	\item 4;
	\item 5;
	\item 15;
	\item 22;
	\item 7;
	\item Eleven.
\end{enumerate}

\section{Вопрос 14}

В листинге \ref{14:q} приведен запрос из вопроса 14, составленный с помощью Prolog/Datalog.

\begin{lstlisting}[label=14:q,caption=Вопрос 14]
-- select s.company_name from shippers s join orders o on s.shipper_id = o.ship_via where o.order_id = 10253

shippers(ShipperID, Company, _), orders(10253, _, _, _, _, _, ShipperID, _, _, _, _, _, _, _, _).
\end{lstlisting}

Варианты ответа:

\begin{enumerate}
	\item \textbf{United Package;}
	\item DHL;
	\item Federal Shipping;
	\item UPS;
	\item SDEK;
	\item Pochta Rossii;
	\item LPR;
	\item LLepr.
\end{enumerate}

%\section{Вопрос 19}
%
%В листинге \ref{19:q} приведен запрос из вопроса 19, составленный с помощью Datalog.
%
%\begin{lstlisting}[label=19:q,caption=Вопрос 19]
%-- select e.last_name from orders o join employees e on o.employee_id = e.employee_id where o.ship_city = 'Caracas' and o.order_date = '1996-07-30'
%
%???
%\end{lstlisting}
%
%Варианты ответа:
%
%\begin{enumerate}
%	\item \textbf{Callahan;}
%	\item Mimi;
%	\item Davolio;
%	\item Muller;
%	\item Buchanan;
%	\item Peacock;
%	\item Butcher;
%	\item King.
%\end{enumerate}

\section{Вопрос 15}

В листинге \ref{15:q} приведен запрос из вопроса 15, составленный с помощью ORM.

\begin{lstlisting}[label=15:q,caption=Вопрос 15]
-- select c.category_name from categories c join products p on c.category_id = p.category_id where p.product_name = 'Chai'

type Category struct {
	CategoryName string `gorm:"category_name"`
}
var category Category
db.Table("categories c").Select("c.category_name").Joins("join products p on c.category_id = p.category_id").Where("p.product_name = ?", "Chai").First(&category)
\end{lstlisting}

Варианты ответа:

\begin{enumerate}
	\item \textbf{Beverages;}
	\item Grains/Cereals;
	\item Tea;
	\item Denver;
	\item Produce;
	\item Seafood;
	\item Dairy Products;
	\item Delisious.
\end{enumerate}

\section{Вопрос 16}

В листинге \ref{16:q} приведен запрос из вопроса 16, составленный с помощью SQL.

\begin{lstlisting}[label=16:q,caption=Вопрос 16]
select unit_price from products where product_name = 'Tofu';
\end{lstlisting}

Варианты ответа:

\begin{enumerate}
	\item \textbf{23.25;}
	\item 11;
	\item 20;
	\item 678;
	\item 1000;
	\item 26.1;
	\item 13;
	\item 14.
\end{enumerate}

\section{Вопрос 17}

В листинге \ref{17:q} приведен запрос из вопроса 17, составленный с помощью Prolog/Datalog.

\begin{lstlisting}[label=17:q,caption=Вопрос 17]
-- select region_description from region where region_id = 3

region(3, Description).
\end{lstlisting}

Варианты ответа:

\begin{enumerate}
	\item \textbf{Northern;}
	\item Western;
	\item Eastern;
	\item Southern;
	\item Africa;
	\item Asia;
	\item Russia;
	\item USA.
\end{enumerate}

\section{Вопрос 18}

В листинге \ref{18:q} приведен запрос из вопроса 18, составленный с помощью ORM.

\begin{lstlisting}[label=18:q,caption=Вопрос 18]
-- select company_name from shippers where shipper_id = 5

type Shipper struct {
	CompanyName string `gorm:"company_name"`
}
var shipper Shipper
db.Table("shippers s").Select("s.company_name").Where("s.shipper_id = ?", 5).First(&shipper)
\end{lstlisting}

Варианты ответа:

\begin{enumerate}
	\item \textbf{UPS;}
	\item Speedy Express;
	\item United Package;
	\item Federal Shipping;
	\item Africa;
	\item Asia;
	\item Alliance Shippers;
	\item DHL.
\end{enumerate}

\section{Вопрос 19}

В листинге \ref{19:q} приведен запрос из вопроса 19, составленный с помощью SQL.

\begin{lstlisting}[label=19:q,caption=Вопрос 19]
select c.category_name from categories c join products p on c.category_id = p.category_id join order_details od on p.product_id = od.product_id join orders o on o.order_id = od.order_id where o.order_date = '1996-07-04' and od.quantity = 5
\end{lstlisting}

Варианты ответа:

\begin{enumerate}
	\item \textbf{Dairy Products;}
	\item Tea;
	\item Stas;
	\item Federal Shipping;
	\item Condiments;
	\item Confections;
	\item Grains/Cereals;
	\item Produce.
\end{enumerate}

\section{Вопрос 20}

В листинге \ref{20:q} приведен запрос из вопроса 20, составленный с помощью Prolog/Datalog.

\begin{lstlisting}[label=20:q,caption=Вопрос 20]
-- select e.last_name from orders o join employees e on e.employee_id = o.employee_id join employee_territories et on e.employee_id = et.employee_id join territories t on t.territory_id = et.territory_id where t.territory_description = 'New York' and o.ship_city = 'Albuquerque'

orders(_, _, EmployeeID, _, _, _, _, _, _, _, "Albuquerque", _, _, _), employees(EmployeeID, LastName, _, _, _, _, _, _, _, _, _, _, _, _, _, _, _, _), employee_territories(EmployeeID, TerritoryID), territories(TerritoryID, "New York", _).
\end{lstlisting}

Варианты ответа:

\begin{enumerate}
	\item \textbf{Buchanan;}
	\item Fuller;
	\item Stas;
	\item Muller;
	\item Leverling;
	\item Peacock;
	\item Suyama;
	\item King.
\end{enumerate}

\section{Вопрос 21}

В листинге \ref{21:q} приведен запрос из вопроса 21, составленный с помощью ORM.

\begin{lstlisting}[label=21:q,caption=Вопрос 21]
-- select c.contact_name from customers c join employees e on c.city = e.city join orders o on e.employee_id = o.employee_id join order_details od on o.order_id = od.order_id where od.quantity = 91


type Customer struct {
	ContactName string `gorm:"contact_name"`
}
var customer Customer
db.Table("customers c").Select("c.contact_name").Joins("join employees e on c.city = e.city").Joins("join orders o on e.employee_id = o.employee_id").Joins("join order_details od on o.order_id = od.order_id").Where("od.quantity = ?", 91).First(&customer)
\end{lstlisting}

Варианты ответа:

\begin{enumerate}
	\item \textbf{Helvetius Nagy;}
	\item Ana Trujillo;
	\item Thomas Hardy;
	\item Stas Baluev;
	\item Hanna Moos;
	\item Alex Odnodvorcev;
	\item Frédérique Citeaux;
	\item Pedro Afonso.
\end{enumerate}


\chapter{Общее описание способов описания запроса}

\section{SQL}

Общий синтаксис языка SQL можно описать так:

\begin{lstlisting}[label=sql-syntax,caption=Синтаксис запроса выборки в SQL]
select <t>.<field1>, <t>.<field2> from <table> <t> where <condition>
\end{lstlisting}

\noindent где \texttt{<field1>}, \texttt{<field2>}~---~названия полей, при выборке перечисляются через запятую, \texttt{<t>}~---~псевдоним таблицы, по которому различать поля с одинаковым названием из разных таблиц в одном запроса, \texttt{<table>}~---~название таблицы, \texttt{<condition>}~---~условие выборки.

\subsubsection*{Пример}

\begin{lstlisting}[label=sql-syntax-ex-1,caption=Пример запроса на SQL]
select c.category_name from categories c where c.category_id = 1
\end{lstlisting}

В данном запросе производится поиск строк таблицы \texttt{categories} с условием на поле \texttt{c.category\_id = 1}. Из результата считывается только поле \texttt{c.category\_name}. При выполнении такого запроса в базе Northwind, будет получен ответ \texttt{Beverages}.

Таблица может быть задана с помощью объединения нескольких таблиц. Для получения объединенной таблицы используется следующий синтаксис:

\begin{lstlisting}[label=sql-syntax-join,caption=Синтаксис объединения таблиц в SQL]
<table1> <t1> join <table2> <t2> on <condition>
\end{lstlisting}

\noindent где \texttt{<table1>}, \texttt{<table2>}~---~названия таблиц,  \texttt{<t1>}, \texttt{<t2>}~---~псевдонимы таблиц, \texttt{<condition>}~---~условие объединения. 

Так как результатом объединения является таблица, то ее тоже можно объединять с другими таблицами. Таким образом, можно объединять более двух таблиц.

\newpage

\subsubsection*{Пример}

\begin{lstlisting}[label=sql-syntax-ex-2,caption=Пример запроса с объединением на SQL]
select c.category_name, p.product_name from categories c join products p on c.category_id = p.category_id
\end{lstlisting}

В данном запросе происходит объединение таблиц \texttt{categories} и \texttt{products} по условию \texttt{c.category\_id = p.category\_id}. Таким образом, строки этих двух таблиц с одинаковым значением поля \texttt{category\_id} будут объединены. В результирующей таблице информация о продукте будет <<обогащена>> подробной информации о категории продукта, вместо одного поля \texttt{category\_id} в исходной таблице. Так как производится выбор полей \texttt{c.category\_name} и \texttt{p.product\_name}, то результатом запроса будет список названий продукта и его категории.

\section{Prolog/Datalog}

Prolog является логическим языком программирования. По сути, <<запрос>> в Prolog представляет собой набор утверждений, на которые система может ответить, истинны ли они или ложны. Утверждения могут содержать, кроме конкретных значений, переменные. В этом случае система ищет такие значения переменных, при которых утверждения будут истинными. При этом, существует специальный символ \texttt{\_}, которому может быть сопоставлено любое значение.

\subsubsection*{Пример}

\begin{lstlisting}[label=prolog-syntax-ex-1,caption=Пример запроса на Prolog]
categories(1, Name, _, _).
\end{lstlisting}

Данный запрос аналогичен примеру из SQL. Описано утверждение о существовании категории с первым полем (\texttt{category\_id}) равным 1. Вместо имени категории (второе поле) указана переменная, то есть система будет искать такие значения имени, при котором будет существовать категория, с идентификатором, равным 1 и значением имени.

Названия переменных похожи на названия полей, но назвать их можно как угодно, в данном случае значение имеет порядок полей. Здесь и далее порядок полей аналогичен порядку полей в таблице.

Система может принимать как единичные утверждения, так и наборы утверждений, перечисленных через запятую. В этом случае запятая играет роль оператора \texttt{И}. Для выполнения набора нужно, чтобы выполнялись все утверждения этого набора. Таким образом может быть реализовано объединение таблиц.

\begin{lstlisting}[label=prolog-syntax-ex-2,caption=Пример запроса с объединением на Prolot]
categories(CategoryID, CategoryName, _, _), products(_, ProductName, _, CategoryID, _, _, _, _, _, _).
\end{lstlisting}

Данный запрос аналогичен примеру из SQL. В перечисленных утверждениях используется 3 переменных: \texttt{CategoryID}, \texttt{CategoryName} и \texttt{ProductName}. Система будет искать такие значения переменных, чтобы оба этих утверждения одновременно были истинными. При этом, переменная \texttt{CategoryID} используется в двух утверждениях, являясь, по сути, условием объединения: будет производиться поиск таких \texttt{CategoryID}, чтобы ей соответствовало значение и продукта, и категории.

\section{ORM}

ORM представляет собой технологию, позволяющую связывать сущности базы данных с элементами объектно-ориентированного программирования. Существует множество реализаций данной технологии. В данном случае, примеры будут приведены для библиотеки GORM.

\subsubsection*{Пример}

\begin{lstlisting}[label=orm-syntax-ex-1,caption=Пример запроса на ORM]
type Category struct {
	CategoryName string `gorm:"category_name"`
}
var category Category
db.Model(&Category{}).Select("category_name").Where("category_id = 1").First(&category)
\end{lstlisting}

В GORM код преобразуется в SQL, с помощью него можно выполнять аналогичные с SQL операции, например, объединения.

\begin{lstlisting}[label=orm-syntax-ex-2,caption=Пример запроса с объединением на ORM]
type CategoryProduct struct {
	CategoryName string `gorm:"category_name"`
	ProductName  string `gorm:"product_name"`
}
var categoryProducts []CategoryProduct
db.Table("categories c").Select("c.category_name, p.product_name").Joins("join products p on c.category_id = p.category_id").Scan(&categoryProducts)
\end{lstlisting}

\end{appendices}