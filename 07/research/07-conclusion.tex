\chapter*{ЗАКЛЮЧЕНИЕ}
\addcontentsline{toc}{chapter}{ЗАКЛЮЧЕНИЕ}

По результатам опроса, можно сделать следующие выводы.

\begin{itemize}
	\item Среднее время ответа на запросы одной сущности в вопросах с Prolog/Datalog на 27\% меньше, чем в вопросах с SQL и ORM для респондентов, связанных со сферой информационных технологий.
	\item Среднее время ответа на запросы одной сущности в вопросах с Prolog/Datalog на 48\% меньше, чем в вопросах с SQL и ORM для респондентов, не связанных со сферой информационных технологий.
	\item Среднее время ответа на запросы нескольких сущностей в вопросах с Prolog/Datalog на 4\% меньше, чем в вопросах с SQL и на 15\% меньше, чем в вопросах с ORM для респондентов, не связанных со сферой информационных технологий.
	\item Среднее время ответа на запросы нескольких сущностей в вопросах с Prolog/Datalog на 18\% меньше, чем в вопросах с SQL и на 34\% меньше, чем в вопросах с ORM для респондентов, связанных со сферой информационных технологий.
	\item Среднее количество ошибок в ответах на запросы одной сущности в вопросах с Prolog/Datalog на 3 процентных пункта больше, чем в вопросах с SQL и на 2 процентных пункта больше, чем в вопросах с ORM для респондентов, связанных со сферой информационных технологий.
	\item Среднее количество ошибок в ответах на запросы одной сущности в вопросах с Prolog/Datalog на 4 процентных пункта меньше, чем в вопросах с SQL и такое же, как в вопросах с ORM для респондентов, не связанных со сферой информационных технологий.
	\item Среднее количество ошибок в ответах на запросы нескольких сущностей в вопросах с Prolog/Datalog на 6 процентных пунктов меньше, чем в вопросах с SQL и на 1 процентный пункт больше, чем в вопросах с ORM для респондентов, связанных со сферой информационных технологий.
	\item Среднее количество ошибок в ответах на запросы нескольких сущностей в вопросах с Prolog/Datalog на 8 процентных пунктов больше, чем в вопросах с SQL и на 9 процентных пунктов меньше, чем в вопросах с ORM для респондентов, не связанных со сферой информационных технологий.
\end{itemize}

Процент ошибок в большинстве случаев для разных способов составления запросов отличается не более, чем на 7 процентных пункта. Однако, в случае запросов нескольких сущностей, среди респондентов, не связанных со сферой информационных технологий, процент ошибок на запросах с ORM на 17 процентных пункта выше, чем на запросах с SQL, что может говорить о меньшей <<понятности>> ORM относительно SQL.
 
Разница в среднем времени ответа на вопрос выше, и среди всех групп респондентов для разных видов запросов среднее время на вопросы с Prolog/Datalog меньше, что может говорить о большей <<понятности>>.


Цель работы~---~классификация запросов к базам данных, была выполнена. Были решены следующие задачи:

\begin{itemize}
	\item проведен обзор существующих видов запросов в произвольной предметной области;
	\item проведен обзор СУБД PostgreSQL и Greenplum;
	\item сформулированы критерии сравнения способов составления запросов к базам данных;
	\item проведено сравнение SQL, ORM, Prolog и Datalog для выявления критерия <<понятности>>.
\end{itemize}