\chapter*{ВВЕДЕНИЕ}
\addcontentsline{toc}{chapter}{ВВЕДЕНИЕ}

Появление в середине XX века запоминающих устройств относительно большой емкости дало возможность создания долговременно хранимых структур данных в программах. Но реализация логики хранения данных на стороне приложения сильно усложнило код и привело к увеличению стоимости разработки. В связи с этим возникла идея централизовать организацию хранения информации с использованием систем управления базами данных~\cite{стасышин2022проектирование}.

Для управления СУБД и манипуляции данных, хранящимися в СУБД, используются специальные языки запросов. Изучение таких языков может быть довольно сложной задачей~\cite{невский2022применение}. Недостаточное знание или понимание языка запросов может приводить к ошибкам при работе с данными~\cite{гарскова2005базы}.

Целью данной научно-исследовательской работы является классификация запросов к предметной области на примере анализа запросов к базам данных. Для достижения поставленной цели необходимо решить следующие задачи:

\begin{itemize}
	\item провести обзор существующих видов запросов в произвольной предметной области;
	\item провести анализ СУБД PostgreSQL и Greenplum;
	\item сформулировать критерии сравнения способов составления запросов к базам данных;
	\item провести сравнение SQL, ORM, Prolog и Datalog для выявления критерия <<понятности>>;
	\item классифицировать запросы к предметной области на примере анализа запросов к базам данных.
\end{itemize}