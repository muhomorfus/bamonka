\chapter{Обзор СУБД PostgreSQL и Greenplum}

В данной части рассматриваются системы управления базами данных PostgreSQL и Greenplum. PostgreSQL на текущий момент является одной из самых популярных СУБД в мире, и в России в частности~\cite{штомпель2019вакансия}. При этом, данная СУБД не обладает такими функциями, как автоматическое восстановление кластера после сбоев, для чего существуют такие внешние инструменты, как Patroni~\cite{kumar2021postgresql}. Greenplum использует в качестве своих рабочих узлов СУБД PostgreSQL, при этом реализует недостающий в PostgreSQL функционал по работе в кластерном режиме, например, автоматическое восстановление и распределенные транзакции~\cite{lyu2021greenplum}.

\section{PostgreSQL}

PostgreSQL~---~объектно-реляционная система управлениями базами данных с открытым исходным кодом~\cite{makris2021mongodb}. Обладает следующими возможностями~\cite{drake2002practical}.

\begin{itemize}
	\item \textbf{Пользовательские операции и функции.} В БД поддерживается объявление пользовательских операций, функций, процедур и типов данных.
	\item \textbf{Поддержка языка SQL.} Поддерживается весь стандартный синтаксис языка SQL. Кроме того, имеются специфичные для PostgreSQL расширения языка.
	\item \textbf{Принципы ACID.} PostgreSQL является транзакционной базой данных~---~OLTP, соответственно, поддерживает принципы атомарности, целостности, изолированности и надежности.
	\item \textbf{Расширяемый API.} Для PostgreSQL можно создавать дополнения на разных языках программирования. Эти дополнения позволяют расширять стандартный функционал базы данных. Например, расширение PostGIS добавляет поддержку работы с геоданными~\cite{obe2021postgis}.
	\item \textbf{Интеграция процедурных языков программирования.} Кроме языка SQL, для работы с данными в базе данных могут быть использованы некоторые процедурные языки программирования, например, Python, Perl и другие.
	\item \textbf{Клиент-серверная архитектура.} PostgreSQL использует клиент-серверную архитектуру. Основной процесс создает дочерний процесс для обработки клиентского соединения.
	\item \textbf{WAL.} После каждой операции с базой данных, еще до изменения данных на диске, происходит запись в лог. Этот лог используется для повышения отказоустойчивости базы данных, так как он может послужить опорной точкой для восстановления данных в случае аварийного завершения работы СУБД. Кроме того, лог используется для репликации базы данных.
	\item \textbf{Репликация.} В PostgreSQL реализованы две стратегии репликации: синхронная и асинхронная. В случае синхронной репликации, изменения данных с основной реплики копируются на реплики-копии сразу после изменения данных. В случае асинхронной репликации, изменения будут скопированы только после применения всей транзакции на основной реплике~\cite{boszormenyi2013postgresql}.
\end{itemize}



\section{Greenplum}


Greenplum~---~масштабируемая система для организации хранилища данных.  Обладает следующими возможностями~\cite{lyu2021greenplum}.

\begin{itemize}
	\item \textbf{Массивно-параллельная архитектура.} Greenplum разделяет данные на фрагменты, которые хранит на рабочих узлах. Запрос к данным выполняется сначала на сегментах, затем результат от сегментов агрегируется на основном узле.
	\item \textbf{Разделение сегментов.} Узлы кластера Greenplum не разделяют память и другие ресурсы. Обмен данными между ними происходит только по сети.
	\item \textbf{Репликация.} У узлов кластера могут быть так называемые «узлы-зеркала»~---~узлы, не участвующие в выполнении запросов, но получающие обновления WAL с основных узлов.
	\item \textbf{Распределенные транзакции.} Для выполнения принципов ACID в рамках распределенной системы, в Greenplum реализован двухфазный протокол применения транзакции.
\end{itemize}


