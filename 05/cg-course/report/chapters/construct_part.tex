\chapter{Конструкторская часть}
В данном разделе представлены используемые алгоритмы, а так же приведено обоснование выбора типов и структур данных.

\section{Разработка алгоритма обратной трассировки лучей}

\subsection{Входные данные}

\begin{itemize}
	\item массив записей с информацией об объектах $Objects$, для каждого объекта сцены хранится информация о полигонах объекта;
	\item массив записей с информацией об источниках света $Lights$, для каждого источника света хранится информация о местоположении источника, а так же его интенсивности;
	\item целочисленные длина и ширина ожидаемого изображения;
	\item радиус-вектор положения наблюдателя $S$;
	\item координата $z_m$ экрана.
\end{itemize}

\subsection{Выходные данные}

\begin{itemize}
	\item растровое изображение заданной сцены $Image$.
\end{itemize}

\subsection{Алгоритм}

\subsubsection{Общий алгоритм}

Алгоритм обратной трассировки лучей можно разделить на следующие этапы.

\begin{enumerate}
	\item Для каждого пиксела $Image$ с координатами $x, y$:
	\begin{enumerate}
		\item Сформировать вектор 
		
		\begin{equation}
			V = \{x, y, z_m\} - S,
		\end{equation}
		
		где $V$~---~вектор направления взгляда.
		
		\item Найти пересечение с ближайшим объектом сцены.
		\item Найти освещенность данной точки с учетом света от источников.
		\item Установить цвет пиксела $x, y$ изображения $Image$.
	\end{enumerate}
	\item Вернуть $Image$.
\end{enumerate}

\subsubsection{Нахождение пересечения с ближайшим объектом сцены}
\begin{enumerate}
	\item Для каждого объекта сцены $Object$ из $Objects$:
	\begin{enumerate}
		\item Проверить, пересекает ли луч $V$ сферическую оболочку объекта $Object$.
		
		Тест используется для оптимизации. Каждый объект сцены описывается в сферу. Соответственно, если луч не пересекает сферическую оболочку объекта, то оно точно не пересекает объект. При этом, проверка пересечения со сферой требует меньших вычислительных затрат, чем проверка пересечения со множеством полигонов.
		
		\item Если не пересекает, то перейти на следующий шаг.
		\item Найти пересечение луча с ближайшим полигоном объекта.
		\item Поместить найденный полигон в промежуточную переменную $Nearest$.
	\end{enumerate}
	\item Вернуть $Nearest$.
\end{enumerate}

\subsubsection{Нахождение пересечения с ближайшим полигоном объекта}

\begin{enumerate}
	\item Для каждого полигона $Polygon$ из объекта $Object$:
	\begin{enumerate}
		\item Найти точку пересечения луча $V$ и полигона $Polygon$.
		\item Если пересечения нет, то перейти на следующий шаг.
		\item Если есть пересечение, то сравнить координату $z$ точки пересечения с сохраненной координатой $z_{max}$.
		\item Если больше, то сохранить точку пересечения и номер полигона в промежуточные переменные $Intersection_{nearest}$ и $Index_{nearest}$ соответственно.
	\end{enumerate}
	\item Вернуть $Intersection_{nearest}$ и $Index_{nearest}$.
\end{enumerate}

\subsubsection{Нахождение точки пересечения луча и полигона}

Пусть:

\begin{itemize}
	\item уравнение плоскости, на котором лежит полигон имеет вид
	
	\begin{equation}
		a \cdot x + b \cdot y + c \cdot z + d = 0;
	\end{equation}
	
	\item вектор нормали к этой плоскости $n = \{a, b, c\}$ нормализован;
	\item луч представлен в виде
	
	\begin{equation}
		V(t) = S + V \cdot t,
	\end{equation}
	
	где 
		
		\begin{itemize}
			\item $S$~---~положение наблюдателя;
			\item $V$~---~вектор направления взгляда;
			\item $t$~---~параметр.
		\end{itemize}
\end{itemize}

Тогда для нахождения точки пересечения луча и полигона надо произвести следующие действия.

\begin{enumerate}
	\item Найти такой параметр $t$, при котором луч пересекает плоскость.
	\begin{equation}
		a \cdot (S_x + V_x \cdot t) + b \cdot (S_y + V_y \cdot t) + c \cdot (S_z + V_z \cdot t) + d = 0
	\end{equation}
	
	\begin{equation}
		t = - \frac{a \cdot S_x + b \cdot S_y + c \cdot S_z + d}{a \cdot V_x + b \cdot V_y + c \cdot V_z}
	\end{equation}
	
	\item Проверить, что точка пересечения лежит внутри полигона. 
	
	Проверка пересечения производится с помощью следующего теста: вычисляется векторное произведение вектора, образованного стороной полигона, и вектора, образованного начало первого вектора и проверяемой точкой. Если для всех таких произведений, для каждой стороны, знаки векторного произведения совпадают, то точка лежит внутри полигона \cite{bib:13}.
	
	\item Если точка лежит внутри полигона, вернуть точку пересечения.
\end{enumerate}

\subsubsection{Проверка пересечения луча и сферической оболочки}

Для проверки пересечения луча и полигона можно воспользоваться геометрическим методом. На рис. \ref{img:sphere} изображен чертеж к нахождению пересечения луча и сферы.

\noindent
\begin{figure}[h!]
	\centering
    \includegraphics[width=0.7\linewidth]{sphere.pdf}
    \caption{Пересечение луча со сферой}
    \label{img:sphere}
\end{figure}

\begin{enumerate}
	\item Определить, лежит ли начало луча внутри сферы. Для этого необходимо вычислить вектор 
	
	\begin{equation}
		SO = O - S.
	\end{equation}
	
	Если длина вектора $SO$ меньше или равна радиусу окружности $R$, то начало луча лежит внутри сферы, то есть луч пересекает сферу. В этом случае вернуть <<Да>>.
	
	\item Найти ближайшую к центру точку луча. Расстояния до нее можно обозначить как $D$. Вычислить
	
	\begin{equation}
		t_{ca} = SO \cdot V,
	\end{equation}
	
	где $t_{ca}$~---~расстояние от начала луча до ближайшей к центру сферы точки.
	
	\item Если $t_{ca} < 0$, вернуть <<Нет>>.
	\item Вычислить
	
	\begin{equation}
		D^2 = SO^2 - t_{ca}^2,
	\end{equation}
	
	\begin{equation}
		t_{hc}^2 = R^2 - D^2 = r^2 - SO^2 + t_{ca}^2.
	\end{equation}
	
	\item Если $t_{hc}^2 < 0$, вернуть <<Нет>>.
	\item Вернуть <<Да>>.
\end{enumerate}

\subsubsection{Нахождение освещенности точки с учетом света от источников света}

\begin{enumerate}	
	\item Инициализировать освещенность $Intensity$ значением $0$.

	\item Для каждого источника света $Light$ из $Lights$:
	
	\begin{enumerate}
		\item Сформировать вектор
		
		\begin{equation}
			V = \{x, y, z\} - Light,
		\end{equation}
		
		где $x, y, z$~---~координаты точки, для которой проверяется освещенность, $V$~---~вектор направления взгляда от источника света к точке.
		
		\item Найти пересечение с ближайшим объектом сцены.
		\item Если пересечение не совпадает с проверяемой точкой, перейти на следующий шаг.
		\item Вычислить значение $I$~---~интенсивность света от одного источника в данной точке:
		
		\begin{equation}
			I = K_a \cdot I_a + K_d \cdot (n \cdot V),
		\end{equation}
		
		где 
		
		\begin{itemize}
			\item $K_a$~---~коэффициент фонового освещения;
			\item $I_a$~---~интенсивность фонового освещения;
			\item $K_d$~---~коэффициент диффузного отражения;
			\item $n$~---~вектор нормали к рассматриваемому полигону.
		\end{itemize}
		
		\item Сложить $I$ к итоговому значению $Intencity$.
	\end{enumerate}
	
	\item Вернуть $Intencity$.
\end{enumerate}

\section{Разработка алгоритма генерации лесистой местности}
\subsection{Входные данные}
\begin{itemize}
\item точки $(x_{min}, z_{min})$, $(x_{max}, z_{max})$, задающие площадку, на которой будет расположен фрагмент лесистой местности;
\item величина $y$~---~координата площадки по оси $Y$;
\item число с плавающей точкой, соотношение питания одной ветки к другой $ratio$;
\item коэффициент густоты кроны $spread$;
\item длина ветки, при которой происходит разделение $splitSize$;
\item количество деревьев на площадке $N$.
\end{itemize}


\subsection{Выходные данные}
\begin{itemize}
\item список описаний деревьев на площадке.
\end{itemize}


\subsection{Алгоритм}

\subsubsection{Рост дерева}
Пусть

\begin{itemize}
	\item $feed$~---~количество питательных веществ, полученной веткой.
\end{itemize}

\begin{enumerate}
	\item Если ветка не имеет потомков (является листом):
	\begin{enumerate}
		\item Вычислить
		
		\begin{equation}
		lenAddition = \sqrt[3]{feed}	
		\end{equation}
		
		\item Добавить $lenAddition$ к длине ветки.
		\item Добавить остаток от $feed$ к площади ветки.
		\item Если ветка имеет достаточную для разделения длину, разделить ее.

	\end{enumerate}
	
	\item Если ветка имеет потомков:
	
	\begin{enumerate}
		\item Вычислить
		
		\begin{equation}
		K = \frac{childArea}{childArea + area},
		\end{equation}
		
		где 
		
		\begin{itemize}
			\item $childArea$~---~суммарная площадь потомков данной ветки;
			\item $area$~---~площадь данной ветки.
		\end{itemize}
		
		\item Добавить к площади ветки значение $\frac{K \cdot feed}{len}$, где $len$~---~текущая длина ветки.
		\item Распределить остаток питания по потомкам в соотношении $ratio$.
		
		

	\end{enumerate}
\end{enumerate}

\subsubsection{Деление ветки}
\begin{enumerate}
	\item Создать ветки-потомки $A$ и $B$.
	\item Вычислить вектор нормали $N$ к текущему направлению ветки.
	\item Вычислить
	\begin{equation}
		Ns = N * randSign * spread,
	\end{equation}	
	
	где $randSign$~---~случайный знак, $+1$ или $-1$.
	
	\item Вычислить
	\begin{equation}
		Ms = -Ns.
	\end{equation}
	
	\item Рассчитать направления $A$ как линейную интерполяцию между векторами $Ns$ и $Dir$ для значения $ratio$.
	\item Рассчитать направления $B$ как линейную интерполяцию между векторами $Ms$ и $Dir$ для значения $1.0 - ratio$. Таким образом, чем больше $ratio$, тем больше ветка получает питания, следовательно растет, и тем меньше она будет отклоняться от направления родительской ветки.
\end{enumerate}

\subsubsection{Размещение деревьев по площадке}
\begin{enumerate}
	\item Вычислить $S$~---~площадь площадки.
	\item Вычислить
	\item 
	\begin{equation}
		S_{one} = \frac{S}{N},
	\end{equation}
	где $S_{one}$~---~площадь, заметаемая одним деревом.
	
	\item Вычислить 
	
	\begin{equation}
		R = \sqrt{\frac{S_{one}}{\pi}},
	\end{equation}
	
	где $R$~---~радиус окружности, заметаемой одним деревом.
	
	\item Создать пустой список деревьев $Trees$.
	\item Создать список всех доступных точек на площадке $Points$.
	\item Цикл $N$ раз:
	\begin{enumerate}
		\item Выбрать случайную точку $P$ из $Points$.
		\item Добавить в $Trees$ дерево, начинающееся с выбранной точки $P$.
		\item Удалить из $Points$ все точки, расстояние от которых до $P$ меньше или равно $R$.
	\end{enumerate}
	
	\item Вернуть $Trees$.
\end{enumerate}

\section{Обоснование типов и структур данных}

В рамках данной работы использованы следующие сущности.

\subsection{Материал}
Запись, содержащая следующие поля:

\begin{itemize}
	\item коэффициент фонового освещения~---~число с плавающей точкой;
	\item коэффициент диффузного отражения~---~число с плавающей точкой;
	\item цвет~---~вектор из трех значений целочисленного типа со значениями от $0$ до $255$ включительно.
\end{itemize}

\subsection{Полигон}
Запись, содержащая следующие поля:

\begin{itemize}
	\item материал~---~запись типа <<Материал>>;
	\item координаты первой точки полигона~---~вектор из трех значений типа числа с плавающей точкой;
	\item координаты второй точки полигона~---~вектор из трех значений типа числа с плавающей точкой;
	\item координаты третьей точки полигона~---~вектор из трех значений типа числа с плавающей точкой;
	\item коэффициенты уравнения плоскости, проходящей через полигон, четыре числа с плавающей точкой.
\end{itemize}

\subsection{Объект сцены}
Запись, содержащая следующие поля:

\begin{itemize}
	\item полигоны~---~массив, состоящий из элементов типа <<Полигон>>. Выбор массива обусловлен возможностью произвольного доступа к элементам массива за константное время;
	\item координаты центра сферической оболочки~---~вектор из трех значений типа числа с плавающей точкой;
	\item радиус сферической оболочки~---~число с плавающей точкой.
\end{itemize}

\subsection{Источник света}
Запись, содержащая следующие поля:

\begin{itemize}
	\item координаты источника света~---~вектор из трех значений типа числа с плавающей точкой;
	\item интенсивность~---~число с плавающей точкой.
\end{itemize}

\subsection{Сцена}
Запись, содержащая следующие поля:

\begin{itemize}
	\item объекты~---~массив, состоящий из элементов типа <<Объект сцены>>. Выбор массива обусловлен возможностью произвольного доступа к элементам массива за константное время;
	\item источники~---~массив, состоящий из элементов типа <<Источник света>>. Выбор массива обусловлен возможностью произвольного доступа к элементам массива за константное время;
	\item положение камеры~---~вектор из трех значений типа числа с плавающей точкой;
	\item высота и ширина результирующего изображения~---~целые числа.
	
\end{itemize}

\subsection{Дерево}
Запись, содержащая следующие поля:

\begin{itemize}
	\item длина ветки~---~число с плавающей точкой;
	\item направление~---~вектор из трех значений типа числа с плавающей точкой;
	\item потомки~---~два элемента типа <<Дерево>>;
	\item длина ветки, при которой произойдет разделение~---~число с плавающей точкой;
	\item соотношение количества питания между двумя ветками~---~число с плавающей точкой;
	\item коэффициент густоты кроны~---~число с плавающей точкой.
\end{itemize}

\subsection{Лес}
Запись, содержащая следующие поля:

\begin{itemize}
	\item деревья~---~массив, состоящий из элементов типа <<Дерево>>. Выбор массива обусловлен возможностью произвольного доступа к элементам массива за константное время.
\end{itemize}

\section{Выводы по конструкторской части}
В данном разделе были разработаны алгоритмы обратной трассировки лучей и генерации лесистой местности. Кроме того, были выбраны и обоснованы типы и структуры данных.

\newpage