\renewcommand{\bibname}{Список использованных источников}
\begin{thebibliography}{100}
\addcontentsline{toc}{chapter}{Список использованных источников}

\bibitem{bib:1}
Боресков А. В., Шикин Е. В. Компьютерная графика. – 2017.

\bibitem{bib:2}
Бутенков С. А., Семерий О. С. Аналитический подход к решению задач компьютерной графики //Искусственный интеллект. – 2000. – Т. 3. – С. 428-437.

\bibitem{bib:3}
ОШАРОВСКАЯ Е. В., СОЛОДКА В. И. Синтез трехмерных объектов с помощью полигональных сеток //Сборник научных трудов «Цифровые технологии». – 2012. – №. 12.

\bibitem{bib:4}
Прилепко М. А. Математические модели представления компьютерной графики //Динамика систем, механизмов и машин. – 2012. – №. 5. – С. 306-312.

\bibitem{bib:5}
Царькова Ю. Р., Внукова О. В. ПОЛИГОНАЛЬНЯ СЕТКА В КОМПЬЮТЕРНОЙ ГРАФИКЕ //Студенчество России: век XXI. – 2020. – С. 394-398.

\bibitem{bib:6}
Ульянов А. Ю., Котюжанский Л. А., Рыжкова Н. Г. Метод трассировки лучей как основная технология фотореалистичного рендеринга //Фундаментальные исследования. – 2015. – №. 11-6. – С. 1124-1128.

\bibitem{bib:7}
Куров А. В. Конспект курса лекций по компьютерной графике, 2022.

\bibitem{bib:8}
Роджерс Д. Алгоритмические основы машинной графики. – Москва: Издательство «Мир». Редакция литературы по математическим наукам, 2001.

\bibitem{bib:9}
Ульянов А. Ю., Котюжанский Л. А., Рыжкова Н. Г. Метод трассировки лучей как основная технология фотореалистичного рендеринга //Фундаментальные исследования. – 2015. – №. 11-6. – С. 1124-1128.

\bibitem{bib:10}
Ваншина Е. А., Ваншин В. В. Типы моделей и принципы моделирования в преподавании графических дисциплин. – 2017.

\bibitem{bib:11}
Цапко И. В., Цапко С. Г. Алгоритмы и методы обработки информации в задачах трехмерного сканирования объектов //Известия Томского политехнического университета. Инжиниринг георесурсов. – 2010. – Т. 317. – №. 5.

\bibitem{bib:12}
Романюк А. Н. и др. Алгоритмы построения теней. – 2000.

\bibitem{bib:13}
Тюкачев Н. А. Алгоритм определения принадлежности точки многоугольнику общего вида или многограннику с треугольными гранями //Вестник Тамбовского государственного технического университета. – 2009. – Т. 15. – №. 3. – С. 638-652.

\bibitem{bib:14}
John Miano, Compressed Image File Formats.

\bibitemweb{bib:15}{Yandex Compute Cloud}{https://cloud.yandex.ru/services/compute}{04.11.2022}

\bibitem{bib:16}
Xu L., Mould D. A procedural method for irregular tree models //Computers \& Graphics. – 2012. – Т. 36. – №. 8. – С. 1036-1047.

\bibitem{bib:17}
Yang Y., Wang R., Huo Y. Rule-based Procedural Tree Modeling Approach //arXiv preprint arXiv:2204.03237. – 2022.

\bibitem{bib:18}
Усольцев В. А. Биоэкологические аспекты таксации фитомассы деревьев. – 1997.

\bibitemweb{bib:19}{Date and time utilities}{https://en.cppreference.com/w/cpp/chrono}{06.11.2022}

\end{thebibliography}

\newpage
