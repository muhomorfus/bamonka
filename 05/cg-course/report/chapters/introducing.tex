
\chapter*{Введение}
\addcontentsline{toc}{chapter}{Введение} 

Компьютерная графика является важной частью современного мира. Визуальное отображение информации помогает показывать объекты, которых не существует в реальном мире. Данная наука находит свое применение в сфере компьютерных игр, фильмов, бизнеса, развлечений, медицины.
Благодаря достижениям компьютерной графики, сейчас становится возможным отображать объекты настолько реалистично, что становится трудно различить отрисованные компьютерами объекты от фотографий \cite{bib:1}.

Одной из важных задач современной компьютерной графики является задача генерации местности: ландшафта и растительности. 

\section*{Цель работы}
Разработать программу для построения трёхмерного изображения лесистой местности.

\section*{Задачи работы}
\begin{itemize}
	\item изучить алгоритмы генерации местности;
	\item изучить методы преобразования сгенерированной местности в модель для построения;
	\item изучить и сравнить алгоритмы построения реалистичных изображений;
	\item формализовать объекты сцены;
	\item выбрать средства разработки, позволяющие решить поставленные задачи;
	\item реализовать данные алгоритмы.
\end{itemize}


\newpage