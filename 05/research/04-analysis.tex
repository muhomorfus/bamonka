\chapter{Анализ предметной области}

В данной части рассматривается актуальность задачи и существующие методы достижения семантики строго однократной доставки в различных брокерах сообщений.

\section{Актуальность задачи}

В современных брокерах сообщений используется три основных семантики передачи сообщений \cite{bib:4}:

\begin{itemize}
	\item <<хотя бы один раз>>~---~сообщение будет передано хотя бы один раз. Гарантируется отправка сообщения, но не гарантируется отсутствие дубликатов. Вероятность появления дубликата в таком случае мала, но она есть. Например, если сообщение было сохранено брокером, но при этом до поставщика данных не дошла информация о принятии сообщения, он попытается отправить его еще один раз, из-за этого появится дубликат;
	\item <<не более одного раза>>~---~поставщик данных ни в каком случае не пытается повторно записывать данные в брокер. Такой подход не гарантирует доставку сообщения в случае ошибки;
	\item <<строго однократная доставка>>~---~сообщение гарантированно будет передано, причем строго один раз.
\end{itemize}

Несмотря на то, что последняя семантика более применима, так как гарантирует как доставку сообщения, так и отсутствие дубликатов, ее техническая реализация является наиболее сложной, в связи с такими нерешенными проблемами, как задача о двух генералах \cite{bib:3}.

Строго однократная доставка сообщений необходима в банковской сфере из-за недопустимости повторных финансовых операций, при синхронизации критично важных данных с использованием брокера сообщений, при сборе и отправке аналитических данных.

\section[Описание принципов работы брокеров сообщений]{Описание принципов работы брокеров \\ сообщений}
В большинстве брокеров сообщений хранение организовано в формате очереди. 
То есть когда поставщик данных отправляет сообщение брокеру, оно помещается в очередь. 
После принятия сообщения, брокер отправляет подтверждение принятия сообщения поставщику. 

Соответственно, от подтверждения зависит дальнейшее поведение поставщика, например, в случае отсутствия подтверждения может произойти повторная отправка сообщения. 
Брокер может предоставлять несколько стратегий подтверждения принятия сообщений \cite{bib:8}:

\begin{enumerate}
	\item отсутствие ожидания ответа от брокера. В таком случае, отправка сообщения сразу будет считаться успешной. При данной стратегии, при возникновении ошибок записи, поставщик данных все равно будет считать отправку сообщения успешной, соответственно, никаких действий принято не будет, и сообщение потеряется;
	\item ожидание ответа от ведущего узла. Если брокер сообщений состоит из нескольких узлов, то для подтверждения успешности операции достаточно получить подтверждение успешной записи только с одного узла;
	\item ожидание ответа от всех узлов. Для подтверждения успешной записи необходимо подтверждение от всех узлов.
\end{enumerate}

\section{Необходимые условия для достижения семантики строго однократной доставки}
\label{section:need_semantics}
\subsection{Сохранность данных}

Для гарантии доставки сообщения необходимо обеспечивать сохранность данных. 
Это означает, что сообщения должны сохраняться на диске в течение некоторого времени, в течение которого эти данные могут быть нужны.

\subsection{Надежность брокера сообщений}
Брокер должен состоять как минимум из нескольких реплик, чтобы при выходе узлов из строя, данные не были утеряны. 
Если условиться, что в некотором кластере брокера сообщений будет $n$ узлов, то допустим выход из строя до $n-1$ узлов при сохранении сообщений в системе \cite{bib:4}.

\section{Методы достижения семантики строго однократной доставки}
Не существует единого метода достижения семантики строго однократной доставки. Однако существуют методы, композиция которых позволяет сделать достижимой строго однократную доставку. Условно эти методы можно разделить на две группы: методы, обеспечивающие доставку не менее одного раза, и методы, обеспечивающие доставку не более одного раза. Совместное применение методов доставки не менее и не более одного раза дает строго однократную доставку. Кроме того, необходимо совмещать указанные методы, работающие как на стороне связки производитель-брокер, так и брокер-потребитель.

\subsection{Методы достижения семантики доставки не менее одного раза}
\subsubsection{Получение подтверждения от реплик брокера}
При отправке сообщения в брокер, производитель данных ожидает подтверждение получения сообщения от всех реплик брокера \cite{bib:8}. Только после подтверждения всеми экземплярами, сообщение считается успешно отправленным. Таким образом, метод позволяет гарантировать доставку сообщения в брокер.

\subsubsection{Метод двойного опроса}
В качестве дополнительного подтверждения получения брокером сообщения может быть использован метод двойного опроса \cite{bib:13}: производитель ожидает два подтверждения получения сообщения.
Если первое подверждения оказалось ошибочным, например из-за слишком долгого времени ответа, то производитель ожидает второй ответ. Если он оказывается успешным, то повторных попыток отправить сообщение не производится.

\subsubsection{Хранение информации о прочитанных данных на стороне потребителя}
\label{subsubsection:consumer}
Для выдачи нужного сообщения потребителям необходимо хранить индекс последнего прочитанного сообщения. 
Варианта расположения хранимого индекса два: можно хранить его на стороне сервера (брокера сообщений) \cite{bib:9}, можно на стороне клиента (потребителя) \cite{bib:10}. 
Возможность возникновения ошибок сети заметно усложняет реализацию семантики строго однократной доставки в брокерах, хранящих позицию прочитанного сообщения на стороне сервера, так как в таком случае позиция может быть неверно посчитана из-за сетевых ошибок при отправке потребителем подтверждения прочтения сообщения. 
Поэтому для обеспечения семантики строго однократной доставки необходимо хранить информацию о текущем сдвиге в очереди на стороне потребителя данных.

\subsection{Методы достижения семантики доставки не более одного раза}
\subsubsection{Идемпотентность сохранения сообщения на стороне брокера}
Идемпотентной операцией называют операцию, выполнение которой нес\-колько раз влечет тот же результат, что и однократное выполнение \cite{bib:5}. 
С помощью идемпотентности можно исключить дублирование записей в брокере сообщений. 
При повторной попытке записи того же сообщения, повторного сохранения не производится. 
Так как в соответствии с \ref{section:need_semantics} гарантируется доставка сообщения хотя бы один раз, то метод позволяет частично обеспечить семантику строго однократной доставки.

Реализуется такой подход следующим образом \cite{bib:7}: на стороне поставщика данных сообщению присваивается уникальный идентификатор. 
Брокер при получении сообщения сохраняет его идентификатор, и при последующем принятии данных, проверит наличие в хранилище этого идентификатора. 
Если он найден, значит это сообщение уже было принято и сохранено брокером, и необходимости в повторном сохранении этого сообщения нет. 
Иначе происходит сохранение нового сообщения с записью его идентификатора в хранилище.

\subsubsection{Обработка состояния прочтения на стороне потребителя}
Логику по обеспечению однократной доставки можно частично делегировать потребителю. 
При действительной логике <<хотя бы одной доставки>> со стороны сервера, на стороне потребителя можно контролировать отсутствие дубликатов следующим образом \cite{bib:2}: каждому сообщению присваивается уникальный идентификатор. 
При получении очередного сообщения, потребитель проверяет, сохранен ли идентификатор сообщения в базе данных (в качестве базы данных в данном случае можно использовать, например, Redis \cite{bib:10}). 
Если сохранен, то сообщение игнорируется как дубликат, если нет, то оно обрабатывается, а его идентификатор помещается в базу данных.

\subsubsection{Транзакции}
Транзакции позволяют работать с несколькими сообщениями атомарно, то есть, при попытке отправить несколько сообщений, они либо будут отправлены все, либо не будет отправлено ни одно. Поведение потребителей сообщений может быть настроено двумя способами:

\begin{enumerate}
	\item принимать сообщения в том формате, в котором они помещаются в брокер, в том числе принимать незавершенные транзакции;
	\item принимать завершенные транзакции. 
\end{enumerate}

При использовании метода транзакций приложению конечным пользователем назначается идентификатор.
Благодаря использованию идентификатора брокер может гарантировать \cite{bib:6}: 

\begin{itemize}
	\item идемпотентность поставки данных~---~при появлении в сети поставщика с существующим идентификатором, запущенные поставщики с тем же идентификатором прекращают свою работу. Таким образом, если существует несколько реплик приложения поставщика, в один момент времени может работать только одна;
	\item транзакционное восстановление между сессиями приложения-поставщика. Если экземпляр приложения завершается (например, завершается с ошибкой), то для следующего экземпляра приложения существует гарантия того, что все незавершенные транзакции будут либо выполнены успешно, либо прерваны.
\end{itemize}


Гарантии, получаемые при использовании транзакций, более характерны для поставщиков данных, для потребителей данных они не настолько сильны, так как нет гарантии того, что потребитель будет использовать все сообщения одной транзакции. Данное поведение возникает по следующим причинам:

\begin{itemize}
	\item при сжатии в целях экономии памяти часть сообщений могут заменяться более новыми версиями \cite{bib:6};
	\item потребители могут обращаться к конкретным сообщениям внутри транзакции;
	\item при записи нескольких сообщений в рамках одной транзакции сообщения могут быть записаны в разные очереди. Но потребитель может не иметь доступ ко всем очередям, поэтому не сможет прочитать все сообщения в рамках одной транзакции.
\end{itemize}

Транзакции полезны, когда данные о текущем прочитанном элементе передаются в брокер сообщений. При такой схеме, фактическое чтение сообщения и отправку брокеру пометки о его прочтении можно совершать атомарно, в рамках одной транзакции. Таким образом, исключается ситуация, когда сообщение было прочитано, при этом оно не было помечено как прочитанное.

\subsubsection{Хранение информации о прочитанных данных на стороне потребителя}
Метод, описанный в \ref{subsubsection:consumer}, также позволяет обеспечивать доставку не более одного раза на стороне потребителя данных, так как при успешном прочтении сообщение обязательно будет помечено как прочитанное, так как это действие делается на стороне потребителя. Таким образом исключается ситуация, когда сообщение было прочитано, при этом оно не было помечено как прочитанное.


\section{Области применения методов достижения семантики строго однократной доставки в брокерах сообщений}
В данной части рассмотрены наиболее подходящие области применения методов достижения семантики строго однократной доставки в брокерах сообщений.

Семантика строго одновременной доставки в брокерах сообщений является нужным инструментом во многих сферах. Ниже приведены некоторые из них.

\begin{itemize}
	\item \textbf{Банковская сфера.} Брокеры сообщений могут быть использованы для потоковой передачи данных о банковских операциях. Очевидно, что в такой системе недопустимы как потери данных, так и их дубликация, так как в первом случае это грозит потерей банковского перевода, а во втором~---~двойным списанием средств.
	\item \textbf{Аналитические данные.} Некоторые аналитические данные могут быть довольно важными и не терпеть потери части данных или появление <<мусорных>> данных, которые и возникают в ходе появления дубликатов.
	\item \textbf{Синхронизация данных.} Брокеры сообщений можно использовать для синхронизации данных между различными хранилищами и базами данных. Одной из таких систем является Kafka Connect \cite{bib:14}, работающий на базе Kafka. Для такой системы нежелательны потери данных, но при этом дубликация данных тоже нежелательна, так как она может привести к искажению данных в целевой базе данных.
	\item \textbf{Отправка уведомлений.} При использовании брокера сообщений в качестве шины уведомлений, например, для социальной сети, нежелательны потери или дубликации данных, так как в первом случае пользователю не придет уведомление о чем-то важном, а во втором случае ему придет одно и то же уведомление несколько раз.
\end{itemize}


\section{Вывод}
В данной части рассмотрены актуальность задачи и существующие методы достижения семантики строго однократной доставки в различных брокерах сообщений. 
Обеспечение этой семантики представляет собой комбинацию методов достижения семантики доставки не менее и не более одного раза. 
Кроме того, применяемые методы могут работать как со стороны брокера или производителя, так и со стороны потребителя. Например, в брокере сообщений может использоваться одновременно идемпотентность, получение подтверждения от всех реплик брокера и хранение информации о прочитанных сообщений на стороне потребителя.
Кроме того, были рассмотрены наиболее подходящие области применения методов достижения семантики строго однократной доставки в брокерах сообщений.