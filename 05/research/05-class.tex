\chapter[Классификация методов достижения семантики строго однократной доставки в брокерах сообщений]{Классификация методов достижения семантики строго однократной доставки в брокерах сообщений}

В данной части рассматривается критерии классификации методов достижения семантики строго однократной доставки в брокерах сообщений.

\section{Определение критериев классификации}

\subsection{По обеспечиваемой базовой семантике}
По обеспечиваемой семантике можно выделить методы, которые позволяют обеспечить доставку сообщений не более одного раза, и методы, которые позволяют обеспечить доставку сообщений хотя бы один раз.

\textbf{Не более одного раза}: идемпотентность, обработка состояния прочтения на стороне потребителя, транзакции, хранение информации о прочитанных данных на стороне потребителя.

\textbf{Хотя бы один раз}: получение подтверждения от реплик брокера, метод двойного опроса, хранение информации о прочитанных данных на стороне потребителя.

\subsection{По стороне обеспечения семантики}
По стороне обеспечения семантики строго однократной доставки можно выделить методы, которые обеспечиваются на стороне производителя данных, на стороне потребителя данных, и на стороне брокера сообщений.

\textbf{На стороне производителя:} получение подтверждения от реплик брокера, метод двойного опроса.

\textbf{На стороне потребителя:} транзакции, обработка состояния прочтения на стороне потребителя, хранение информации о прочитанных данных на стороне потребителя.

\textbf{На стороне брокера сообщений:} получение подтверждения от реплик брокера, идемпотентность, транзакции.

\subsection{По использованию внешних сервисов}
По использованию внешних сервисов можно выделить методы, которые можно реализовать исключительно с использованием средств брокера сообщений, и методы, для реализации которых потребуется использование внешних сервисов.

\textbf{Без использованию внешних сервисов:} получение подтверждения от реплик брокера, идемпотентность, транзакции, метод двойного опроса, хранение информации о прочитанных данных на стороне потребителя.

\textbf{С использованием внешних сервисов:} обработка состояния прочтения на стороне потребителя.

\section{Вывод}

В данной части были рассмотрены критерии классификации методов достижения семантики строго однократной доставки в брокерах сообщений  и выполнена классификация этих методов.