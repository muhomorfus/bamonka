\begin{definitions}
	\definition{Брокер сообщений}{приложение, которое выступает посредником между двумя приложениями~---~источником и приемником, выполняя преобразование сообщения по протоколу источника в сообщение по протоколу приемника.}
	\definition{Протокол}{набор определенных правил или соглашений, определяющий обмен данными между приложениями.}
	\definition{Потребитель}{приложение, которое осуществляет чтение сообщений из брокера сообщений, приемник данных.}
	\definition{Производитель}{приложение, которое осуществляет запись сообщений в брокера сообщений, источник данных.}
	\definition{Кластер}{группа компьютеров, связанных друг с другом, и воспринимаемых как единый ресурс.}
	\definition{Реплика}{один из экземпляров приложения. Приложение обычно запускают в нескольких экземплярах с целью обеспечения большей надежности.}
	\definition{Транзакция}{единая операция, возможно состоящая из нескольких операций, которая выполняется как единый неделимый блок, либо не выполняется вовсе.}
	\definition{Микросервис}{небольшой модуль, отвечающий за небольшой фрагмент бизнес-логики продукта. Минимально зависим от других модулей.}
\end{definitions}