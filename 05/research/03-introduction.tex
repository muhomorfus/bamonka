\chapter*{ВВЕДЕНИЕ}
\addcontentsline{toc}{chapter}{ВВЕДЕНИЕ}

С ростом сложности программных продуктов возникает необходимость задумываться о сложности управления приложением. 
Важным аспектом в решении данной задачи является архитектура приложения. 
В современном мире все более и более актуальным решением становится микросервисная архитектура, в которой бизнес-логика продукта разбита на отдельные части со своей кодовой базой~---~микросервисы \cite{bib:1}.

Однако, при использовании микросервисной архитектуры, возникает проблемы выбора технологии обмена данными между сервисами. 
Одной из распространенных технологий такого рода являются брокеры сообщений \cite{bib:2}.

Большинство брокеров сообщений гарантируют доставку сообщений <<хотя бы один раз>>, то есть не исключают вероятность появления дубликатов. Такое поведение допустимо для большого количества продуктов, но, например, для банковской сферы является недопустимым \cite{shegalov2002eos}. Для такого рода систем необходимо обеспечивать семантику строго однократной доставки~---~гарантию того, что сообщение дойдет до получателя строго один раз.

Целью данной научно-исследовательской работы является классификация методов достижения семантики строго однократной доставки в брокерах сообщений. Для достижения поставленной цели необходимо решить следующие задачи:

\begin{itemize}
\item провести анализ предметной области;
\item провести обзор существующих методов достижения семантики строго однократной доставки в брокерах сообщений;
\item сформулировать критерии сравнения методов достижения семантики строго однократной доставки в брокерах сообщений;
\item классифицировать методы достижения семантики строго однократной доставки.	
\end{itemize}