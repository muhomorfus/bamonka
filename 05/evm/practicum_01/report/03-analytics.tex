\section{Аналитическая часть}

\subsection{Структура вычислительного комплекса Тераграф}
Комплекс «Тераграф» предназначен для хранения и обработки графов сверхбольшой размерности и будет применяться для моделирования биологических систем, анализа финансовых потоков в режиме реального времени, для хранения знаний в системах искусственного интеллекта, создания интеллектуальных автопилотов с функциями анализа дорожной обстановки, и в других прикладных задачах. Он способен обрабатывать графы сверхбольшой размерности до 1012 (одного триллиона) вершин и 2·1012 ребер. Комплекс состоит из 3-х однотипных гетерогенных узлов, которые взаимодействуют между собой через высокоскоростные сетевые подключения 100Gb Ethernet. Каждый узел состоит из хост-подсистемы, подсистемы хранения графов, подсистемы коммутации узлов, а также подсистемы обработки графов. Структурная схема одного узла представлена на рисунке (~\ref{img:1.png}).

\subsection{Принципы взаимодействия микропроцессора Леонард Эйлер и хост-подсистемы}
Основу взаимодействия подсистем при обработке графов составляет передача блоков данных и коротких сообщений между GPC и хост-подсистемой. Для передачи сообщений для каждого GPC реализованы два аппаратных FIFO буфера на 512 записей: Host2GPC для передачи от хост-подсистемы к ядру, и GPC2Host для передачи в обратную сторону.

Обработка начинается с того, что собранное программное ядро (software kernel) загружается в локальное ОЗУ одного или нескольких CPE (микропроцессора riscv32im). Для этого используется механизм прямого доступа к памяти со стороны хост-подсистемы. В свою очередь, GPC (один или несколько) получают сигнал о готовности образа software kernel в Глобальной памяти, после чего вызывается загрузчик, хранимый в ПЗУ CPE. Загрузчик выполняет копирование программного ядра из Глобальной памяти в ОЗУ CPE и передает управление на начальный адрес программы обработки. Предусмотрен режим работы GPC, при котором во время обработки происходит обмен данными и сообщениями. Эти два варианта работы реализуется через буферы и очереди соответственно.
\par Если код программного ядра уже загружен в ОЗУ CPE, хост-подсистема может вызвать любой из содержащихся в нем обработчиков. Для этого в GPC передает оговоренный UID обработчика (handler), после чего передается сигнал запуска (сигнал START). В ответ CPE устанавливает состояние BUSY и начинает саму обработку. В ходе обработки ядро может обмениваться сообщениями с хост-подсистемой через очереди (команды mq\_send и mq\_receive). По завершении обработки устанавливается состояние IDLE и вырабатывается прерывание, которое перехватывается хост-подсистемой. Далее, пользовательское приложение хост-подсистемы уведомляется о завершении обработки и готовности результатов.

Если во время работы над кодом обработчика программному ядру software kernel требуется осуществить передачу больших блоков данным между CPE и хост-подсистемой, то может быть задействована Глобальная память и внешняя память большого размера (External Memory, до 16ГБ).
\img{0.45\textwidth}{1.png}{Схема одного узла Телеграф}

\subsection{Взаимодействие CPE(riscv32im) и SPE(lnh64)}
Микропроцессор lnh64 с набором команд дискретной математики (Discrete Mathematics Instruction Set Computer) является ассоциативным процессором, т.е. устройством, выполняющим операции обработки над данными, хранящимися в ассоциативной памяти (так называемой Локальной памяти структур). В качестве таковой выступает адресная память DDR4, причем для каждого ядра lnh64 доступны 2.5 ГБ адресного пространства в ней. Для организации ассоциативного способа доступа к адресному устройству микропроцессор lnh64 организует на аппаратном уровне структуру B+дерева. Причем 512МБ занимает древовидая структура от верхнего и до предпоследнего уровня, 2048МБ занимает последний уровень дерева, на котором и хранятся 64х разрядные ключи и значения. Каждый микропроцессор lnh64 может хранить и обрабатывать до 117 миллионов ключей и значений.

Исходя из этого, обработка множеств или графов представляется в DISC наборе команд, как работа со структурами ключей и значений (key-value). Однако, как было показано ранее при описании набора команд DISC, в отличие от общепринятых key-value хранилищ, доступны такие операции как ближайший больший (NGR), ближайший меньший (NSM), команды объединения множеств (OR) и ряд других. Это и позволяет использовать lnh64 в качестве устройства, хранящего большие множества (для графов это множества вершин и ребер).

Доступ к микропроцессору lnh64 (Structure Processing Element) осуществляется чтением и записью в пространство памяти микропроцессора riscv32im (Computing Processing Element) в диапазоне 0x60000000 - 0x60001000.
\newpage