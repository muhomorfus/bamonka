\chapter*{Заключение}
\addcontentsline{toc}{chapter}{Заключение} 

Для алгоритмов сортировки были рассмотрены лучшие и худшие случаи.

Для алгоритма блочной сортировки худшим случаем является случай, когда элементы разбросаны в небольшое количество сильно удаленных в плане величины групп таким образом, что в результате деления на блоки большинство элементов попадает в один блок. Лучшим случаем для алгоритма блочной сортировки является отсортированный по возрастанию массив.

Для алгоритма пирамидальной сортировки нет конкретного худшего случая, он работает примерно одинаковое время на практически любых наборах данных, кроме массивов, в которых все элементы равны. В этом случае, количество сравнений и перестановок минимально~---~это лучший случай. 

Для алгоритма сортировки подсчетом худшим случаем является случай, когда максимальный элемент массива очень большой. Лучшим случаем для алгоритма сортировки подсчетом является массив, состоящий их нулей.

Все дальнейшие выводы приведены исходя из анализов данных замеров для массива размером $10 000$, так как данная размерность позволяет получить достаточно показательные результаты для проведения сравнения.

По итогам замеров можно сказать, что в худшем случае по сравнению с лучшим алгоритмы сортировки заметно деградируют в плане производительности:

\begin{itemize}
	\item время работы алгоритма блочной сортировки в худшем случае в $7.24$ раз больше, чем в лучшем;
	\item время работы алгоритма пирамидальной сортировки в худшем случае в $9.09$ раз больше, чем в лучшем;
	\item время работы алгоритма сортировки подсчетом в худшем случае в $17.81$ раз больше, чем в лучшем.
\end{itemize}

В худшем случае, время работы алгоритма блочной сортировки больше времени работы алгоритма пирамидальной сортировки в $1.17$ раз. Время работы алгоритма блочной сортировки больше времени работы алгоритма сортировки подсчетом в $8.49$ раз. 

В лучшем случае, время работы алгоритма блочной сортировки больше времени работы алгоритма пирамидальной сортировки в $1.47$ раз. Время работы алгоритма блочной сортировки больше времени работы алгоритма сортировки подсчетом в $20.88$ раз. 

На наборе случайных элементов время работы алгоритма пирамидальной сортировки больше времени работы алгоритма блочной сортировки в $1.28$ раз. Время работы алгоритма пирамидальной сортировки больше времени работы алгоритма сортировки подсчетом в $81.69$ раз. 

С точки зрения потребления памяти, наименее эффективной является сортировка подсчетом. Объем потребляемой памяти сортировки подсчетом больше, чем у блочной сортировки в $1.03$ раз, и больше, чем у пирамидальной сортировки в $3.09$ раз.

Цель работы была достигнута: были изучены алгоритмы сортировки. Были выполнены все задачи:

\begin{itemize}
	\item изучены алгоритмы сортировки;
	\item проанализированы трудоемкости данных алгоритмов;
	\item кодированы данные алгоритмы;
	\item проведен замерный эксперимент для данных алгоритмов, с измерением времени работы и использования памяти; 
	\item проведен сравнительный анализа алгоритмов на основе полученных данных.
\end{itemize}


\newpage