\chapter{Технологическая часть}

В данном разделе представлены реализации алгоритмов сортировки. Кроме того, указаны требования к ПО и средства реализации.

\section{Требования к ПО}
\begin{itemize}
	\item программа позволяет вводить имя файла, содержащего информацию о массиве, с помощью аргументов командной строки;
	\item программа аварийно завершается в случае ошибок, выводя сообщение о соответствующей ошибке;
	\item программа выполняет замеры времени работы реализаций алгоритмов;
	\item программа строит зависимости времени работы реализаций алгоритмов от размеров входных данных;
	\item программа принимает на вход непустые массивы, состоящие из целых чисел.
\end{itemize}

\section{Средства реализации}
Для реализации данной работы выбран язык программирования Go, так как он содержит необходимые для тестирования библиотеки, а также обладает достаточными инструментами для реализации ПО, удовлетворяющего требованиям данной работы \cite{bib:3}.

\section{Реализации алгоритмов}
В листингах \ref{code:block1}~--~\ref{code:count} представлены реализации алгоритмов сортировки.

\newpage

\begin{code}
\caption{Исходный код реализации алгоритма блочной сортировки}
\label{code:block1}
\begin{minted}{go}
func (m *Bucket) getBucketFunc(max, min int) (func(int) int, int) {
	d := max - min + 1
	n := m.buckets
	if n > d {
		n = d
	}
	
	size := d / n
	if d%n != 0 {
		size++
	}
	
	return func(k int) int {
		k -= min
		bucket := k / size
		if bucket > n-1 {
			bucket = n - 1
		}
		return bucket
	}, n
}

func (m *Bucket) isSorted(a []int) bool {
	for i := 1; i < len(a); i++ {
		if a[i] < a[i-1] {
			return false
		}
	}

	return true
}
\end{minted}
\end{code}

\newpage

\begin{code}
\caption{Исходный код реализации алгоритма блочной сортировки (продолжение листинга \ref{code:block1})}
\label{code:block2}
\begin{minted}{go}
func (m *Bucket) sort(a []int) []int {
	if len(a) <= 1 {
		return a
	}

	f, n := m.getBucketFunc(utils.Max(a...), 
	utils.Min(a...))

	buckets := make([][]int, n)

	for _, e := range a {
		buckets[f(e)] = append(buckets[f(e)], e)
	}

	for i := range buckets {
		if !m.isSorted(buckets[i]) {
			buckets[i] = m.sort(buckets[i])
		}
	}

	res := make([]int, 0, len(a))
	for _, b := range buckets {
		for _, e := range b {
			res = append(res, e)
		}
	}

	return res
}
\end{minted}
\end{code}

\newpage

\begin{code}
\caption{Исходный код реализации алгоритма пирамидальной сортировки}
\label{code:heap1}
\begin{minted}{go}
func (m *Heap) heapify(a []int, i, heapSize int) []int {
	left := i<<1 + 1
	right := i<<1 + 2

	largest := i
	if left < heapSize && a[left] > a[i] {
		largest = left
	}

	if right < heapSize && a[right] > a[largest] {
		largest = right
	}

	if i != largest {
		tmp := a[i]
		a[i] = a[largest]
		a[largest] = tmp
		a = m.heapify(a, largest, heapSize)
	}

	return a
}

func (m *Heap) sort(a []int) []int {
	heapSize := len(a)

	for i := len(a)/2 - 1; i >= 0; i-- {
		a = m.heapify(a, i, heapSize)
	}
\end{minted}
\end{code}

\newpage

\begin{code}
\caption{Исходный код реализации алгоритма пирамидальной сортировки (продолжение листинга \ref{code:heap1})}
\label{code:heap2}
\begin{minted}{go}
    for i := len(a) - 1; i >= 1; i-- {
		tmp := a[0]
		a[0] = a[i]
		a[i] = tmp

		heapSize--a
		a = m.heapify(a, 0, heapSize)
	}

	return a
}
\end{minted}
\end{code}

\begin{code}
\caption{Исходный код реализации алгоритма сортировки подсчетом}
\label{code:count}
\begin{minted}{go}
func (m *Count) sort(a []int) []int {
	counts := make([]int, utils.Max(&m.stat, a...)+1)
	for _, e := range a {
		counts[e]++
	}

	res := make([]int, 0, len(a))
	for i, c := range counts {
		for j := 0; j < c; j++ {
			res = append(res, i)
		}
	}

	return res
}
\end{minted}
\end{code}

\section{Тестирование}
Тестирование проводилось по методологии чёрного ящика. \textbf{Тесты пройдены успешно.}

В таблице  представлены тестовые данные для реализаций алгоритмов сортировки. 

\begin{table}[H]
  \caption{\label{table:tests} Тестовые данные для алгоритмов сортировки}
  \begin{center}
    \begin{tabular}{|c|l|l|}
      \hline
      № & Массив & Результат \\ \hline
      1 
      & 
      [5, 6, 2, 6, 1, 7]
      &
      [1, 2, 5, 6, 6, 7]
      \\ \hline
      
      2
      & 
      [1, 2, 3, 4, 5, 6, 7, 8, 9, 10]
      &
      [1, 2, 3, 4, 5, 6, 7, 8, 9, 10]
      \\ \hline
      
      3 
      & 
      [10, 9, 8, 7, 6, 5, 4, 3, 2, 1]
      &
      [1, 2, 3, 4, 5, 6, 7, 8, 9, 10]
      \\ \hline
      
      4 
      & 
      [1, 1, 1, 1, 1, 1]
      &
      [1, 1, 1, 1, 1, 1]
      \\ \hline
      
      5 
      & 
      [1]
      &
      [1]
      \\ \hline
      
      6 
      & 
      []
      &
      []
      \\ \hline
    \end{tabular}
  \end{center}
\end{table}

\newpage