\setcounter{page}{3}
\chapter*{Введение}
\addcontentsline{toc}{chapter}{Введение} 

В современном мире важную роль выполняют задачи обработки данных, в частности, алгоритмы сортировки, которые используются во многих сферах, требующих работу с информацией. 
Сортировка~---~процесс установки последовательности сравнимых элементов в определенном порядке.
Задача сортировки не теряет своей актуальности и сейчас, так как еще не найден идеальный алгоритм сортировки, а объемы обрабатываемой информации постоянно растут \cite{bib:1}.

\section*{Цель работы}

Получение навыков кодирования программного продукта, тестирования и проведения замерного эксперимента работы программы на различных данных. Все это на примере решения задачи сортировки.

\section*{Задачи работы}

\begin{enumerate}[label={\arabic*)}]
	\item изучение алгоритмов сортировки:
	\begin{itemize}
		\item блочная сортировка;
		\item пирамидальная сортировка;
		\item сортировка подсчетом;
	\end{itemize}
	\item анализ трудоемкости данных алгоритмов;
	\item кодирование данных алгоритмов;
	\item проведение замерного эксперимента для данных алгоритмов, с измерением времени работы и использования памяти; 
	\item проведение сравнительного анализа алгоритмов на основе полученных данных.
\end{enumerate}

\newpage