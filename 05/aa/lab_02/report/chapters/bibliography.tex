\renewcommand{\bibname}{Список использованных источников}
\begin{thebibliography}{100}
\addcontentsline{toc}{chapter}{Список использованных источников}

\bibitem{bib:1}
Боревич З. И. Определители и матрицы. – Рипол Классик, 2013.

\bibitem{bib:2}
Зубков С. Assembler. Для DOS, Windows и Unix. – Litres, 2022.

\bibitemweb{bib:3}{Документация по языку программирования $Go$}{https://go.dev/doc}{07.10.2022}

\bibitemweb{bib:4}{Документация по пакетам языка программирования $Go$}{https://pkg.go.dev}{07.10.2022}

\bibitemweb{bib:5}{Техническая спецификация ноутбука $MacBook Air$}{https://support.apple.com/kb/SP869}{08.10.2022}

\bibitemweb{bib:6}{Документация по функции $clock\_gettime$}{https://man7.org/linux/man-pages/man3/clock\_gettime.3.html}{25.10.2022}

\bibitem{bib:7}
Ватутин, Э. И. Оценка реальной производительности современных видеокарт с поддержкой технологии CUDA в задаче умножения матриц / Э. И. Ватутин, И. А. Мартынов, В. С. Титов // Известия Юго-Западного государственного университета. Серия: Управление, вычислительная техника, информатика. Медицинское приборостроение. – 2014. – № 2. – С. 8-17. – EDN SKHQXF.

\bibitem{bib:8}
Анисимов Н.С., Строганов Ю.В. Реализация алгоритма умножения матриц по винограду на языке Haskell // Новые информационные технологии в автоматизированных системах. 2018. №21. URL: https://cyberleninka.ru/article/n/realizatsiya-algoritma-umnozheniya-matrits-po-vinogradu-na-yazyke-haskell (дата обращения: 29.11.2022).

\bibitem{bib:9}
Pidodnya A. About improving integer multiplication in processors. – 2020.



\end{thebibliography}

\newpage