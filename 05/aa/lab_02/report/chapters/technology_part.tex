\chapter{Технологическая часть}

В данном разделе представлены реализации алгоритмов умножения матриц. Кроме того, указаны требования к ПО и средства реализации.

\section{Требования к ПО}
\begin{itemize}
	\item программа позволяет вводить имена файлов, содержащих информацию о матрицах, с помощью аргументов командной строки;
	\item программа аварийно завершается в случае ошибок, выводя сообщение о соответствующей ошибке;
	\item программа выполняет замеры времени работы реализаций алгоритмов;
	\item программа строит зависимости времени работы реализаций алгоритмов от размеров входных данных;
	\item программа принимает на вход непустые матрицы, состоящие из целых чисел.
\end{itemize}

\section{Средства реализации}
Для реализации данной работы выбран язык программирования Go, так как он содержит необходимые для тестирования библиотеки, а также обладает достаточными инструментами для реализации ПО, удовлетворяющего требованиям данной работы \cite{bib:3}.

\section{Реализации алгоритмов}
В листингах \ref{code:usual}~--~\ref{code:winograd-improved-cols} представлены реализации алгоритмов нахождения расстояний Левенштейна и Дамерау-Левенштейна.

\newpage

\begin{code}
\caption{Исходный код реализации стандартного алгоритма умножения матриц}
\label{code:usual}
\begin{minted}{go}
func (m *Usual) Multiply(a, b [][]int) ([][]int, error) {
	if len(a[0]) != len(b) {
		return nil, algs.ErrInvalidArgsSize
	}

	res := make([][]int, len(a))
	for i := range res {
		res[i] = make([]int, len(b[0]))
	}

	for i := range res {
		for j := range res[0] {
			for k := range b {
				res[i][j] = res[i][j] + a[i][k]*b[k][j]
			}
		}
	}

	return res, nil
}
\end{minted}
\end{code}

\newpage

\begin{code}
\caption{Исходный код реализации алгоритма Винограда}
\begin{minted}{go}
func (m *Winograd) Multiply(a, b [][]int) ([][]int, error) {
	if len(a[0]) != len(b) {
		return nil, algs.ErrInvalidArgsSize
	}
	res := make([][]int, len(a))
	for i := range res {
		res[i] = make([]int, len(b[0]))
	}
	r := m.rowCoefficients(a)
	c := m.colCoefficients(b)
	for i := 0; i < len(res); i++ {
		for j := 0; j < len(res[0]); j++ {
			res[i][j] = res[i][j] - r[i] - c[j]
			for k := 0; k < len(a[0])/2; k++ {
				res[i][j] = res[i][j] + 
				(a[i][2*k]+b[2*k+1][j])*
				(a[i][2*k+1]+b[2*k][j])
			}
		}
	}
	if len(a[0])%2 != 0 {
		for i := 0; i < len(res); i++ {
			for j := 0; j < len(res[0]); j++ {
				res[i][j] = res[i][j] + 
				a[i][len(a[0])-1]*b[len(a[0])-1][j]
			}
		}
	}
	return res, nil
}
\end{minted}
\end{code}

\newpage

\begin{code}
\caption{Исходный код функции нахождения сумм произведений соседних элементов в строках}
\begin{minted}{go}
func (m *Winograd) rowCoefficients(a [][]int) []int {
	c := make([]int, len(a))
	for i := 0; i < len(a); i++ {
		for j := 0; j < len(a[0])/2; j++ {
			c[i] = c[i] + a[i][2*j]*a[i][2*j+1]
		}
	}
	return c
}
\end{minted}
\end{code}

\begin{code}
\caption{Исходный код функции нахождения сумм произведений соседних элементов в столбцах}
\begin{minted}{go}
func (m *Winograd) colCoefficients(a [][]int) []int {
	c := make([]int, len(a[0]))
	for i := 0; i < len(a[0]); i++ {
		for j := 0; j < len(a)/2; j++ {
			c[i] = c[i] + a[2*j][i]*a[2*j+1][i]
		}
	}
	return c
}
\end{minted}
\end{code}

\newpage

\begin{code}
\caption{Исходный код реализации оптимизированного алгоритма Винограда}
\begin{minted}{go}
func (m *WinogradImproved) Multiply(a, b [][]int) ([][]int, error) {
	if len(a[0]) != len(b) {
		return nil, algs.ErrInvalidArgsSize
	}
	res := make([][]int, len(a))
	for i := range res {
		res[i] = make([]int, len(b[0]))
	}
	r := m.rowCoefficients(a)
	c := m.colCoefficients(b)
	odd := len(a[0])%2 != 0
	len2 := len(res[0])
	for i := 0; i < len(res); i++ {
		for j := 0; j < len2; j++ {
			res[i][j] -= r[i] + c[j]
			for k := 0; k < len(a[0])-1; k += 2 {
				res[i][j] += (a[i][k]+b[k+1][j])*
				(a[i][k+1]+b[k][j])
			}
			if odd {
				res[i][j] += a[i][len(a[0])-1] *
				 b[len(a[0])-1][j]
			}
		}
	}
	return res, nil
}
\end{minted}
\end{code}

\newpage

\begin{code}
\caption{Исходный код оптимизированной функции нахождения сумм произведений соседних элементов в строках}
\begin{minted}{go}
unc (m *WinogradImproved) rowCoefficients(a [][]int) []int {
	c := make([]int, len(a))
	len2 := len(a[0]) >> 1
	for i := 0; i < len(a); i++ {
		for j := 0; j < len2; j++ {
			j2 := j << 1
			c[i] += a[i][j2] * a[i][j2+1]
		}
	}
	return c
}
\end{minted}
\end{code}


\begin{code}
\caption{сходный код оптимизированной функции нахождения сумм произведений соседних элементов в столбцах}
\label{code:winograd-improved-cols}
\begin{minted}{go}
func (m *WinogradImproved) colCoefficients(a [][]int) []int {
	c := make([]int, len(a[0]))
	len1 := len(a[0])
	len2 := len(a) >> 1
	for i := 0; i < len1; i++ {
		for j := 0; j < len2; j++ {
			j2 := j << 1
			c[i] += a[j2][i] * a[j2+1][i]
		}
	}
	return c
}
\end{minted}
\end{code}

\newpage

\section{Тестирование}
Тестирование проводилось по методологии чёрного ящика. \textbf{Тесты пройдены успешно.}

В таблице  представлены тестовые данные для реализаций алгоритмов умножения матриц. 

\begin{table}[H]
  \caption{\label{table:tests} Тестовые данные для алгоритмов умножения матриц}
  \begin{center}
    \begin{tabular}{|c|c|c|c|}
      \hline
      № & $A$ & $B$ & $C$ \\ \hline
      1 & 
      $\begin{pmatrix}
	  1
	  \end{pmatrix}$
      &
      $\begin{pmatrix}
	  1
	  \end{pmatrix}$
      &
      $\begin{pmatrix}
	  1
	  \end{pmatrix}$
      \\ \hline
      
      
      2 &
      $\begin{pmatrix}
	  1 & 2 \\ 3 & 4
	  \end{pmatrix}$
      &
      $\begin{pmatrix}
	  1 & 2 \\ 3 & 4
	  \end{pmatrix}$
      &
      $\begin{pmatrix}
	  7 & 10 \\ 15 & 22
	  \end{pmatrix}$
      \\ \hline
      
      3 &
      $\begin{pmatrix}
	  1 & 2 \\ 3 & 4
	  \end{pmatrix}$
      &
      $\begin{pmatrix}
	  1 & 0 \\ 0 & 1
	  \end{pmatrix}$
      &
      $\begin{pmatrix}
	  1 & 2 \\ 3 & 4
	  \end{pmatrix}$
      \\ \hline
      
      4 &
      $\begin{pmatrix}
	  1 & 2 & 3 \\ 4 & 5 & 6
	  \end{pmatrix}$
      &
      $\begin{pmatrix}
	  1 & 2 \\ 3 & 4 \\ 5 & 6
	  \end{pmatrix}$
      &
      $\begin{pmatrix}
	  38 & 32 \\ 101 & 86
	  \end{pmatrix}$
      \\ \hline
      
      5 &
      $\begin{pmatrix}
	  1 & 2 & 3 \\ 4 & 5 & 6 \\ 7 & 8 & 9
	  \end{pmatrix}$
      &
      $\begin{pmatrix}
	  10 \\ 10 \\ 10
	  \end{pmatrix}$
      &
      $\begin{pmatrix}
	  60 \\ 150 \\ 240
	  \end{pmatrix}$
      \\ \hline
      
      6 &
      $\begin{pmatrix}
	  1 & 2 & 3 \\ 4 & 5 & 6 \\ 7 & 8 & 9
	  \end{pmatrix}$
      &
      $\begin{pmatrix}
	  10
	  \end{pmatrix}$
      &
      ошибка
      \\ \hline
      
      7 &
      $\begin{pmatrix}
	  1 & \text{ап} & 3 \\ 4 & 5 & 6 \\ 7 & 8 & 9
	  \end{pmatrix}$
      &
      $\begin{pmatrix}
	  10
	  \end{pmatrix}$
      &
      ошибка
      \\ \hline
    \end{tabular}
  \end{center}
\end{table}

\newpage