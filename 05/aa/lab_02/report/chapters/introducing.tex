\setcounter{page}{3}
\chapter*{Введение}
\addcontentsline{toc}{chapter}{Введение} 

Матрица~---~это таблица чисел \cite{bib:1}. Размерность матрицы задается такими характеристиками, как количество строк и столбцов таблицы. Основными математическими операциями на матрицах являются:

\begin{itemize}
	\item умножение;
	\item сложения;
	\item вычисление определителя;
	\item умножение на скаляр.
\end{itemize}

Задача умножения матриц имеет важное значение для ряда научно-технических направлений, таких как:

\begin{itemize}
	\item томография;
	\item компьютерная графика;
	\item проектирование роботизированных систем;
	\item классификация бинарных отношений \cite{bib:7}.
\end{itemize}

В связи с этим, существует большое количество подходов, связанных с оптимизацией выполняемых операций.

\section*{Цель работы}

Получение навыков кодирования программного продукта, тестирования и проведения замерного эксперимента работы программы на различных данных. Все это на примере решения задачи умножения матриц.

\section*{Задачи работы}

\begin{enumerate}[label={\arabic*)}]
	\item изучение алгоритмов умножения матриц:
	\begin{itemize}
		\item стандартный алгоритм;
		\item Винограда;
		\item оптимизированного алгоритма Винограда;
	\end{itemize}
	\item анализ трудоемкости данных алгоритмов;
	\item кодирование данных алгоритмов;
	\item проведение замерного эксперимента для данных алгоритмов, с измерением времени работы и использования памяти; 
	\item проведение сравнительного анализа алгоритмов на основе полученных данных.
\end{enumerate}

\newpage