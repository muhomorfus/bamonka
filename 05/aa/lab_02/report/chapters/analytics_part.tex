\chapter{Аналитическая часть}

В данном разделе рассмотрены теоретические выкладки по алгоритмам умножения матриц, а также общие сведение о данной задаче.

Стоит отметить, что умножение матриц возможно только в том случае, когда количество столбцов первого операнда совпадает с количеством строк второго операнда.

\section{Стандартный алгоритм умножения матриц}

Пусть даны матрицы $A$ ($r_A \times c_A$) и $B$ ($r_B \times c_B$):

\begin{equation} \label{eqn:A}
	A = \begin{pmatrix}
		a_{11} & a_{12} & \ldots & a_{1c_A}\\
		a_{21} & a_{22} & \ldots & a_{2c_A}\\
		\vdots & \vdots & \ddots & \vdots\\
		a_{r_A1} & a_{r_A2} & \ldots & a_{r_Ac_A}
	\end{pmatrix},
\end{equation}

\begin{equation} \label{eqn:B}
	B = \begin{pmatrix}
		b_{11} & b_{12} & \ldots & b_{1c_B}\\
		b_{21} & b_{22} & \ldots & b_{2c_B}\\
		\vdots & \vdots & \ddots & \vdots\\
		b_{r_B1} & b_{r_B2} & \ldots & b_{r_Bc_B}
	\end{pmatrix}.
\end{equation}

Тогда произведением матриц $A$ и $B$ называется матрица $C$ ($r_C \times c_C$):

\begin{equation} \label{eqn:r-C}
	r_C = r_A,
\end{equation}

\begin{equation} \label{eqn:c-C}
	c_C = c_B,
\end{equation}

\begin{equation} \label{eqn:C}
	C = \begin{pmatrix}
		c_{11} & c_{12} & \ldots & c_{1c_C}\\
		c_{21} & c_{22} & \ldots & c_{2c_C}\\
		\vdots & \vdots & \ddots & \vdots\\
		c_{r_C1} & c_{r_C2} & \ldots & c_{r_Cc_C}
	\end{pmatrix}.
\end{equation}

При этом, каждый элемент матрицы $C$ может быть вычислен по формуле:

\begin{equation} \label{eqn:c-formula}
	c_{ij} = \sum_{k=1}^{r_B} a_{ik}b_{kj},\quad i = \overline{1,r_C},j = \overline{1,c_C}.
\end{equation}

\section{Алгоритм Винограда}

Алгоритм Винограда является одним из самых эффективных с точки зрения быстродействия алгоритмов \cite{bib:8}. Основной идеей алгоритма является сокращение количества трудоемких операций путем предварительного расчета.

Пусть даны два вектора $D = (d_1, d_2, d_3, d_4)$ и $E = (e_1, e_2, e_3, e_4)$. 

Тогда:

\begin{equation} \label{eqn:de}
	D \cdot E = d_1 \cdot e_1 + d_2 \cdot e_2 + d_3 \cdot e_3 + d_4 \cdot e_4.
\end{equation}

Путем алгебраических преобразований можно привести формулу \ref{eqn:de} к следующему виду:

\begin{equation} \label{eqn:de-good}
	D \cdot E = (d_1 + e_2) \cdot (d_2 + e_1) + (d_3 + e_4) \cdot (d_4 + e_3) - d_1 \cdot d_2 - d_3 \cdot d_4 - e_1 \cdot e_2 - e_3 \cdot e_4.
\end{equation}

Несмотря на большую сложность вычисления формулы \ref{eqn:de-good}, по сравнению с \ref{eqn:de}, преобразованный вариант допускает возможность предварительного расчета слагаемых $d_1 \cdot d_2$, $d_3 \cdot d_4$, $e_1 \cdot e_2$ и $e_3 \cdot e_4$. С учетом предварительно рассчитанных значений, для вычисления формулы \ref{eqn:de-good} необходимо совершить 7 операций сложения и 2 операции умножения, против 3 операция сложения и 4 операций умножения в формуле \ref{eqn:de}. В связи с тем, что операция умножения является более трудоемкой \cite{bib:9}, вариант с предварительным расчетом является менее тредоемким с точки зрения вычисления. 

\section{Оптимизированный алгоритм Винограда}

В данной работе рассмотрены следующий улучшения алгоритма Винограда:

\begin{itemize}
	\item замена операции $v = v + n$ на $v += n$;
	\item замена операции $v * 2$ на $v << 1$;
	\item замена операции $v / 2$ на $v >> 1$;
	\item предварительный расчет некоторых элементов, например половинной длины массива, используемой в цикле;
	\item объединение побочного цикла для крайнего случая внутрь основного цикла;
	\item замена счетчиков цикла для уменьшения количества выполняемых арифметических операций.
\end{itemize}

\newpage