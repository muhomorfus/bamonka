\chapter*{Заключение}
\addcontentsline{toc}{chapter}{Заключение} 

Все дальнейшие выводы приведены исходя из анализов данных замеров для матрицы размером $60 \times 60$, так как данная размерность позволяет получить достаточно показательные результаты для проведения сравнения.

По итогам замеров можно сказать, что алгоритм Винограда быстрее стандартного алгоритма умножения матриц в $1.98$ раз в лучшем случае, и в $1.95$ раз в худшем. Оптимизированный алгоритм Винограда быстрее версии без улучшений в $1.01$ раз в лучшем случае, и в $1.02$ раз в худшем.

Потребление памяти стандартного алгоритма умножения матриц меньше, чем у алгоритма Винограда в $1.02$ раз. Алгоритм Винограда без улучшений в $1.0005$ раз эффективнее оптимизированного алгоритма в плане потребляемой памяти~---~это обусловлено наличие дополнительных переменных, используемых для предварительного расчета некоторых параметров в варианте с улучшениями.

Цель работы была достигнута: были изучены алгоритмы умножения матриц. Были выполнены все задачи:

\begin{itemize}
	\item изучены алгоритмы умножения матриц:
	\item проанализированы трудоемкости данных алгоритмов;
	\item кодированы данные алгоритмы;
	\item проведен замерный эксперимент для данных алгоритмов, с измерением времени работы и использования памяти; 
	\item проведен сравнительный анализа алгоритмов на основе полученных данных.
\end{itemize}


\newpage