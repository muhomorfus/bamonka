\setcounter{page}{3}
\chapter*{Введение}
\addcontentsline{toc}{chapter}{Введение} 

Расстояние Левенштейна между двумя строками~---~это минимальное количество операций вставки одного символа, удаления одного символа и замены одного символа на другой, необходимых для превращения строки в другую \cite{bib:1}.


Рассматриваются следующие редакционные операции:
\begin{itemize}
	\item вставка одного символа;
	\item удаление одного символа;
	\item замена символа на другой.
\end{itemize}

Расстояние Дамерау-Левенштейна~---~это мера разницы двух строк символов, определяемая, как минимальное количество операций вставки, удаления, замены и транспозиции (перестановки двух соседних символов), необходимых для перевода одной строки в другую.

Редакционные расстояния находят свое применение в:
\begin{itemize}
	\item компьютерной лингвистике;
	\item биоинформатике (сравнение генов).
\end{itemize}

\section*{Цель работы}

Получение навыков кодирования программного продукта, тестирования и проведения замерного эксперимента работы программы на различных данных. Все это на примере решения задачи нахождения редакционного расстояния.

\section*{Задачи работы}

\begin{enumerate}[label={\arabic*)}]
	\item изучение алгоритмов вычисления редакционного расстояния:
	\begin{itemize}
		\item Левенштейна;
		\item Дамерау-Левенштейна;
		\item рекурсивного Дамерау-Левенштейна;
		\item рекурсивного Дамерау-Левенштейна с кешем;
	\end{itemize}
	\item кодирование данных алгоритмов;
	\item проведение замерного эксперимента для данных алгоритмов, с измерением времени работы и использования памяти; 
	\item проведение сравнительного анализа алгоритмов на основе полученных данных.
\end{enumerate}

\newpage