\chapter{Технологическая часть}

В данном разделе представлены реализации алгоритмов поиска редакционного расстояния. Кроме того, указаны требования к ПО и средства реализации.

\section{Требования к ПО}
\begin{itemize}
	\item программа позволяет вводить строки с использованием аргументов командной строки;
	\item программа аварийно завершается в случае ошибок, выводя сообщение о соответствующей ошибке;
	\item программа выполняет замеры времени работы реализаций алгоритмов;
	\item программа строит зависимости времени работы реализаций алгоритмов от размеров входных данных;
	\item программа принимает на вход строки из символов различных языков.
\end{itemize}

\section{Средства реализации}
Для реализации данной работы выбран язык программирования Go, так как он содержит необходимые для тестирования библиотеки, а также обладает достаточными инструментами для реализации ПО, удовлетворяющего требованиям данной работы \cite{bib:3}.

\section{Реализации алгоритмов}
В листингах \ref{code:levenshtein}~--~\ref{code:damerau_levenshtein_recursive_cache} представлены реализации алгоритмов нахождения расстояний Левенштейна и Дамерау-Левенштейна.

\newpage

\begin{code}
\caption{Исходный код реализации матричного алгоритма нахождения расстояния Левенштейна}
\label{code:levenshtein}
\begin{minted}{go}
func (l *Levenshtein) diff(s1, s2 []rune) int {
	m := make([][]int, len(s1)+1)
	for i := range m {
		m[i] = make([]int, len(s2)+1)
	}

	for i := range m[0] {
		m[0][i] = i
	}

	for i := range m {
		m[i][0] = i
	}

	for i := 1; i < len(m); i++ {
		for j := 1; j < len(m[i]); j++ {
			r := 0
			if s1[i-1] != s2[j-1] {
				r++
			}

			m[i][j] = utils.Min(
				m[i][j-1]+1,
				m[i-1][j]+1,
				m[i-1][j-1]+r,
			)
		}
	}

	return m[len(m)-1][len(m[0])-1]
}
\end{minted}
\end{code}

\newpage

\begin{code}
\caption{Исходный код реализации матричного алгоритма нахождения расстояния Дамерау-Левенштейна}
\begin{minted}{go}
func (l *DamerauLevenshtein) diff(s1, s2 []rune) int {
	m := make([][]int, len(s1)+1)
	for i := range m {
		m[i] = make([]int, len(s2)+1)
	}
	for i := range m[0] {
		m[0][i] = i
	}
	for i := range m {
		m[i][0] = i
	}
	for i := 1; i < len(m); i++ {
		for j := 1; j < len(m[i]); j++ {
			r := 0
			if s1[i-1] != s2[j-1] {
				r++
			}
			m[i][j] = utils.Min(m[i][j-1]+1,m[i-1][j]+1,
				m[i-1][j-1]+r,
			)
			if i > 1 && j > 1 {
				if s1[i-1] == s2[j-2] && s1[i-2] == 
				s2[j-1] {
					m[i][j] = utils.Min(m[i][j],
						m[i-2][j-2]+1,
					)
				}
			}
		}
	}
	return m[len(m)-1][len(m[0])-1]
}
\end{minted}
\end{code}

\newpage

\begin{code}
\caption{Исходный код реализации рекурсивного алгоритма нахождения расстояния Дамерау-Левенштейна}
\begin{minted}{go}
func (l *DamerauLevenshteinRecursive) diff(s1, s2 []rune) int {
	if len(s1) == 0 || len(s2) == 0 {
		return len(s1) + len(s2)
	}

	r := 0
	if s1[len(s1)-1] != s2[len(s2)-1] {
		r++
	}

	result := utils.Min(
		&l.stat,
		l.diff(s1, s2[:len(s2)-1])+1,
		l.diff(s1[:len(s1)-1], s2)+1,
		l.diff(s1[:len(s1)-1], s2[:len(s2)-1])+r,
	)

	if len(s1) > 1 && len(s2) > 1 {
		if s1[len(s1)-1] == s2[len(s2)-2] && 
		s1[len(s1)-2] == s2[len(s2)-1] {
			result = utils.Min(
				result,
				l.diff(s1[:len(s1)-2], s2[:len(s2)-2])+1,
			)
		}
	}

	return result
}
\end{minted}
\end{code}

\newpage

\begin{code}
\caption{Исходный код реализации рекурсивного алгоритма нахождения расстояния Дамерау-Левенштейна с кешем}
\label{code:damerau_levenshtein_recursive_cache}
\begin{minted}{go}
func (l *DamerauLevenshteinRecursiveCached) diff1(s1, s2 []rune,
 cache [][]int) int {
	if cache[len(s1)][len(s2)] != math.MaxInt {
		return cache[len(s1)][len(s2)]
	}
	if len(s1) == 0 || len(s2) == 0 {
		cache[len(s1)][len(s2)] = len(s1) + len(s2)
	}
	r := 0
	if s1[len(s1)-1] != s2[len(s2)-1] {
		r++
	}
	cache[len(s1)][len(s2)] = utils.Min(
		l.diff1(s1, s2[:len(s2)-1], cache)+1,
		l.diff1(s1[:len(s1)-1], s2, cache)+1,
		l.diff1(s1[:len(s1)-1], s2[:len(s2)-1], cache)+r,
	)
	if len(s1) > 1 && len(s2) > 1 {
		if s1[len(s1)-1] == s2[len(s2)-2] &&
		 s1[len(s1)-2] == s2[len(s2)-1] {
			cache[len(s1)][len(s2)] = utils.Min(
				cache[len(s1)][len(s2)],
				l.diff1(s1[:len(s1)-2], 
				s2[:len(s2)-2], cache)+1,
			)
		}
	}

	return cache[len(s1)][len(s2)]
}
\end{minted}
\end{code}

\section{Тестирование}
Тестирование проводилось по методологии чёрного ящика. Тесты пройдены успешно.

В таблице  представлены тестовые данные для реализаций алгоритмов нахождения расстояний Левенштейна и Дамерау-Левенштейна. 

\begin{table}[H]
  \caption{\label{table:tests} Тестовые данные для алгоритмов нахождения расстояний Левенштейна и Дамерау-Левенштейна}
  \begin{center}
    \begin{tabular}{|c|c|c|c|c|}
      \hline
      № & $s_1$ & $s_2$ & Левентшейн & Дамерау-Левенштейн \\ \hline
      1 & "{}" & "{}" & 0 & 0 \\ \hline
      2 & "{}" & "a" & 1 & 1 \\ \hline
      3 & "a" & "{}" & 1 & 1 \\ \hline
      4 & "bababa" & "bebebe" & 3 & 3 \\ \hline
      5 & "abc" & "cbc" & 1 & 1 \\ \hline
      6 & "опарышв" & "опарывш" & 2 & 1 \\ \hline
      7 & "ффф" & "енф" & 2 & 2 \\ \hline
    \end{tabular}
  \end{center}
\end{table}

\newpage