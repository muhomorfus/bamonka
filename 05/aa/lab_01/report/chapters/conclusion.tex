\chapter*{Заключение}
\addcontentsline{toc}{chapter}{Заключение} 

Все дальнейшие выводы приведены исходя из анализов данных замеров для длин строк равных десяти, так как данная длина строки позволяет получить достаточно показательные результаты для проведения сравнения.

По итогам замеров можно сказать, что рекурсивный алгоритм поиска расстояния Дамерау-Левенштейна в $139 371$ раз медленнее матричной реализации алгоритма.

Рекурсивная реализация с кешем медленнее матричной в $3$ раза, но объем необходимой памяти в реализации с кешем в $3.8$ раз больше, чем в матричной.

Матричный реализации алгоритма Левенштейна и Дамерау-Левенштейна не отличаются в плане использования памяти. Алгоритм Левенштейна быстрее алгоритма Дамерау-Левенштейна в $1.002$ раза, так как они почти идентичны, с одним отличием~---~в алгоритме Дамерау-Левенштейна есть дополнительная проверка.

Цель работы была достигнута: были изучены алгоритмы поиска редакционного расстояния. Были выполнены все задачи:

\begin{itemize}
	\item изучены алгоритмы вычисления редакционного расстояния;
	\item кодированы данные алгоритмы;
	\item проведен замерный эксперимент для данных алгоритмов, с измерением времени работы и использования памяти;
	\item проведен сравнительный анализ результатов полученных данных.
\end{itemize}


\newpage