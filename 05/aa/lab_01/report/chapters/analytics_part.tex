\chapter{Аналитическая часть}

В данном разделе рассмотрены теоретические выкладки по алгоритмам поиска редакционного расстояния, а также общие сведение о данной задаче.

Так как за редакционное расстояние принято минимальное количество редакционных операций над строкой, с помощью которых она может быть преобразована в другую строку. Для вычисление данной характеристики для каждой операции вводится условная цена \cite{bib:1}:

\begin{itemize}
	\item $w(a, a) = 0$~---~цена замены символа на него же;
	\item $w(a, b) = 1, a \ne b$~---~цена замены одного символа на другой;
	\item $w(\epsilon, a) = 0$~---~цена вставки символа $a$;
	\item $w(a, \epsilon) = 0$~---~цена удаления символа $a$.
\end{itemize}


Для описания алгоритма вычисления расстояния Левенштейна применяется рекуррентное соотношение

\begin{equation}
\label{eqn:levenshtein}
	D(s_1[1..i], s_2[1..j]) = \begin{cases}
      	0,& \text{если}\ i = 0, j = 0, \\ 
        i,& \text{если}\ i > 0, j = 0, \\ 
        j,& \text{если}\ i = 0, j > 0, \\
        D_{min},& \text{иначе}.
      \end{cases}
\end{equation}

При этом 

\begin{equation}
	D_{min} = \min \begin{cases}
		D(s_1[1..i], s_2[1..j - 1]) + 1, \\
		D(s_1[1..i - 1], s_2[1..j]) + 1, \\
		D(s_1[1..i - 1], s_2[1..j - 1]) + r, \\
	\end{cases}
\end{equation}

\begin{equation}
	r = \begin{cases}
      0, & \text{если}\ s_1[i] = s_2[j], \\
      1, & \text{иначе}. \\
      \end{cases}
\end{equation}

\newpage

Для алгоритма вычисления расстояния Дамерау-Левенштейна добавляется ещё один элемент:

\begin{equation}
\label{eqn:damerau_levenshtein}
	D(s_1[1..i], s_2[1..j]) = \begin{cases}
      	0,& \text{если}\ i = 0, j = 0, \\ 
        i,& \text{если}\ i > 0, j = 0, \\ 
        j,& \text{если}\ i = 0, j > 0, \\
        D_{min},& \text{иначе}.
      \end{cases} \\
\end{equation}

При этом 

\begin{equation}
	D_{min} = \min \begin{cases}
      	D(s_1[1..i], s_2[1..j - 1]) + 1, \\
      	D(s_1[1..i - 1], s_2[1..j]) + 1, \\
      	D(s_1[1..i - 1], s_2[1..j - 1]) + r, \\
      	D(s_1[1..i - 2], s_2[1..j - 2]) + k,\qquad \text{если}\ i > 1, j > 1, \\
    \end{cases} \\
\end{equation}

\begin{equation}
	r = \begin{cases}
      0, & \text{если}\ s_1[i] = s_2[j], \\
      1, & \text{иначе}, \\
      \end{cases}
\end{equation}

\begin{equation}
	k = \begin{cases}
      0, & \text{если}\ s_1[i] = s_2[j] \text{ и } s_1[i - 1] = s_2[j - 1], \\
      1, & \text{если}\ s_1[i] = s_2[j - 1] \text{ и } s_1[i - 1] = s_2[j], \\
      \infty, & \text{иначе}.
      \end{cases} \\
\end{equation} \\

Могут быть использованы два алгоритма: матричный и рекурсивный.

Под матричным подходом понимается реализация, при которой сначала вычисляются значения для элементарных случаев, а на основе них вычисляется значение для исходных строк. 

Под рекурсивным подходом понимается реализация, при которой сначала вычисляется редакционное расстояние от исходных строк, и все необходимые для этого данные вычисляются вложенно. Рекурсивный метод представляет собой перенос математического представления функции в код.

\section{Матричный алгоритм нахождения расстояния Левенштейна}
\label{section:levenshtein_matrix}
В матричном алгоритме нахождения расстояния Левенштейна используется формула (\ref{eqn:levenshtein}). Для того, чтобы не проводить повторные вычисления, промежуточные значения вычисления помещаются в матрицу $M$. Таким образом, значение в ячейке $M[i][j]$ соответствует расстоянию Левенштейна между строками $s_1[1..i]$ и $s_2[1..j]$.

\section{Матричный алгоритм нахождения расстояния \\Дамерау-Левенштейна}
\label{section:damerau_levenshtein_matrix}
В матричном алгоритме нахождения расстояния Дамерау-Левенштейна используется формула (\ref{eqn:damerau_levenshtein}). Для того, чтобы не проводить повторные вычисления, промежуточные значения вычисления помещаются в матрицу $M$. Таким образом, значение в ячейке $M[i][j]$ соответствует расстоянию Дамерау-Левенштейна между строками $s_1[1..i]$ и $s_2[1..j]$.

\section{Рекурсивный алгоритм нахождения расстояния \\Дамерау-Левенштейна}
\label{section:damerau_levenshtein_recursive}
В рекурсивном алгоритме нахождения расстояния Дамерау-Левенштейна используется формула (\ref{eqn:damerau_levenshtein}). При этом, в отличие от метода, описанного в \ref{section:damerau_levenshtein_matrix}, матрица промежуточных значения не используется, и значения расстояний Дамерау-Левенштейна вычисляются с использованием рекурсивных вызовов. 


Данный метод является неэффективным с точки зрения времени выполнения и потребления памяти, так как:

\begin{itemize}
	\item на рекурсивный вызов затрачивается дополнительное время \cite{bib:2};
	\item на рекурсивный вызов выделяется стек под вызванную функцию \cite{bib:2};
	\item производится много повторных вычислений \cite{bib:1}.
\end{itemize}

\section{Рекурсивный алгоритм нахождения расстояния \\ Дамерау-Левенштейна с кешированием}

Данный метод является оптимизацией алгоритма \ref{section:damerau_levenshtein_recursive}, в котором используется матрица промежуточных вычислений, аналогичная \ref{section:damerau_levenshtein_matrix}. Использование матрицы позволяет повысить эффективность работы алгоритма с точки зрения времени выполнения и потребления памяти путем исключения повторных вызовов.

Сначала все ячейки матрицы заполняются значением $\infty$. Затем, по мере вычисления промежуточных значений, заполняются соответствующие ячейки матрицы. И если при шаге рекурсии значение в ячейке матрицы не равно $\infty$, то вместо повторного вычисления текущего значения, результат берется из матрицы.


\newpage