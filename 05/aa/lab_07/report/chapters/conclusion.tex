\chapter*{Заключение}
\addcontentsline{toc}{chapter}{Заключение} 

На основе проведенных замеров можно сделать следующие выводы.

Муравьиный алгоритм работает медленнее полного перебора на небольших данных. При количестве вершин, равном $6$, полный перебор быстрее муравьиного алгоритма в $106$ раз. Но при увеличении количества вершин, муравьиный алгоритм становится производительнее полного перебора. На $10$ вершинах муравьиный алгоритм работает быстрее в $44$ раза.

Цель работы была достигнута: были изучены алгоритм полного перебора и муравьиный алгоритм на примере решения задачи комивояжера. Были выполнены все задачи:

\begin{itemize}
	\item изучены алгоритм полного перебора и муравьиный алгоритм;
	\item разработаны алгоритм полного перебора и муравьиный алгоритм;
	\item оценена трудоемкость данных алгоритмов;
	\item кодированы данные алгоритмы;
	\item проведена параметризация для муравьиного алгоритма;
	\item проведены замеры времени для данных алгоритмов, с измерением времени работы; 
	\item проведен сравнительный анализа алгоритмов на основе полученных данных.
\end{itemize}


\newpage