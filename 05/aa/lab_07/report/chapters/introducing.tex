\setcounter{page}{3}
\chapter*{Введение}
\addcontentsline{toc}{chapter}{Введение} 

Задача коммивояжера является одной из важнейших задач современной дискретной оптимизации \cite{bib:1}. Она имеет множество формулировок, но в рамках данной работы будет рассмотрена следующая формулировка: необходимо найти кратчайший путь, по которому можно обойти все морские порты. Порты в данном случае представлены вершинами графа, маршруты между ними~---~ребрами графа, причем ребра имеют вес~---~длину маршрута.

\subsection*{Цель работы}
Получение навыков кодирования программного продукта, тестирования и проведения замеров времени выполнения программы на различных данных. Все это на примере решения задачи коммивояжера с использованием алгоритма полного перебора и муравьиного алгоритма.

\subsection*{Задачи работы}

\begin{enumerate}[label={\arabic*)}]
	\item изучение алгоритма полного перебора и муравьиного алгоритма;
	\item разработка алгоритма полного перебора и муравьиного алгоритма;
	\item оценка трудоемкости алгоритмов;
	\item реализация данных алгоритмов;
	\item проведение параметризации для муравьиного алгоритма;
	\item проведение замеров времени работы данных алгоритмов на наилучшей комбинации параметров; 
	\item проведение сравнительного анализа алгоритмов на основе полученных данных.
\end{enumerate}

\newpage