\chapter{Аналитическая часть}

В данном разделе рассмотрены теоретические выкладки по задаче коммивояжера, алгоритму полного перебора и муравьиному алгоритму.

\section{Задача коммивояжера}
Рассматривается следующая формулировка задачи коммивояжера: необходимо найти кратчайший путь, по которому можно обойти все морские порты. Порты в данном случае представлены вершинами графа, маршруты между ними~---~ребрами графа, причем ребра имеют вес~---~длину маршрута. Используется неориентированный граф.

В рамках данной работы будут оцениваться маршруты между следующими странами:

\begin{itemize}
	\item Грузия;
	\item Греция;
	\item Германия;
	\item Россия;
	\item Нидерланды;
	\item Франция;
	\item Норвегия;
	\item и другие.
\end{itemize}



\section{Алгоритм полного перебора}
Основной идеей алгоритма полного перебора является перебор всех возможных маршрутов между вершинами графа, отсечение из полученных вариантов несуществующих маршрутов, поиск среди оставшихся маршрутов кратчайшего незамкнутого.

Особенностями данного алгоритма являются низкая производительность и точное решение задачи поиска кратчайшего пути.

\section{Муравьиный алгоритм}

Основной идеей муравьиного алгоритма является моделирование колонии муравьев, которые перемешаются между вершинами графа. Особенностью муравьев является выделение так называемых феромонов~---~веществ, на которые реагируют другие муравьи. Муравей с помощью феромона помечает свой маршрут, вследствии чего этот маршрут становится привлекательнее для других членов колонии. 

Работа алгоритма представляет собой моделирования некоторого количества циклов жизни колонии муравьев. В рамках одного цикла, муравьи, действуя независимо друг от друга, проходят пути, состоящие из всех вершин графа. В конце цикла происходит проверка пройденных путей на предмет кратчайшего расстояния. В процессе прохождения муравьи оставляют феромоны на маршрутах.

Муравей обладает следующими свойствами.

\begin{itemize}
	\item Зрение~---~муравей может оценить ребро по его длине, чем он длиннее, чет хуже муравей его видит, следовательно, тем меньше его привлекательность. Видимость вершины графа для муравья можно вычислить по формуле
	
	\begin{equation}
		\eta_{ij} = \frac{1}{D_{ij}}, \  \text{где} \  D_{ij}\text{~---~длина ребра между пунктами} \ i \  \text{и} \  j.
	\end{equation}
	\item Память~---~муравей способен запоминать посещенные вершины, чтобы не обходить их снова;
	\item Обоняние~---~муравей способен чувствовать запах феромона и определять его интенсивность.

\end{itemize}

Учитывая перечисленные компетенции можно вывести формулу для расчёта вероятности посещения $k$-ым муравьём, находящимся в вершине $i$, вершины  $j$:
\begin{equation} \label{eqn:1}
	P_{ijk} = \frac{\left[\eta_{ij}\right]^{\beta} \cdot \left[\tau_{ij}\right]^{\alpha}}{\sum_{q \notin J_k} \left[\eta_{iq}\right]^{\beta} \cdot \left[\tau_{iq}\right]^{\alpha}},
\end{equation}

где

\begin{itemize}
	\item $\alpha$~---~коэффициент стадности;
	\item $\beta$~---~коэффициент жадности;
	\item $\eta_{ij}$~---~видимость вершины $j$ из вершины $i$.
\end{itemize}

%В формуле (\ref{eqn:1}) $\alpha$~---~это коэффициент стадности, задающий вес концентрации феромона при вычислении вероятности, а $\beta$~---~коэффициент жадности, определяющий вес длины ребра, причём $\alpha,\beta \in (0,1)$. В случае, когда $\alpha = 0$, алгоритм вырождается в жадный, то есть всегда будет выбираться ближайший к текущей позиции город. В случае, когда $\beta = 0$, алгоритм становится стадным, что влечёт за собой преждевременный приход к одному субоптимальному решению.
%
%Если муравей сумел пройти маршрут, удовлетворяющий понятию Гамильтонова цикла, на пройденных им рёбрах должна возрасти концентрация феромона в соответствии со следующей формулой:

Для подходящего под условия задачи маршрута происходит повышения количества феромона на нем.

\begin{equation}
	\tau_{ij}(t + 1) = (1 - \rho) \cdot \tau_{ij}(t) + \sum_{k=1}^m \Delta\tau_{ijk}(t),
\end{equation}

где
\begin{itemize}
	\item $\rho$~---~коэффициент испарения феромона с течением времени;
	\item $m$~---~количество муравьёв в колонии;
	\item $\tau_{ij}(t)$~---~концентрация феромона на ребре между пунктами $i$ и $j$ в текущий цикл;
	\item $\tau_{ij}(t + 1)$~---~концентрация феромона на ребре между пунктами $i$ и $j$ в наступающий цикл;
	\item $\tau_{ijk}(t)$~---~концентрация феромона на ребре между пунктами $i$ и $j$ в текущий цикл для муравья $k$;
	\item $\Delta\tau_{ij,k}(t)$ определяется по формуле \begin{equation}
	\Delta\tau_{ijk}(t) = \frac{Q}{L_k(t)},
	\end{equation}
	
	где 
	
	\begin{itemize}
	\item $Q$~---~параметр, определяющий значение порядка длины искомого маршрута;
	\item $L_k(t)$~---~длина маршрута для муравья $k$.
	\end{itemize}
\end{itemize}

В результате работы алгоритма находится кратчайший маршрут среди найденных муравьями. Из особенностей алгоритма можно отметить полиномиальное время работы и приближенный результат \cite{bib:ant}.

\newpage