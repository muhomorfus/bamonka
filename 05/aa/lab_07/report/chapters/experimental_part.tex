\chapter{Экспериментальная часть}

В данном разделе описаны замерные эксперименты и представлены результаты исследования.

\section{Технические характеристики}
Технические характеристики устройства, на котором выполнялось исследование \cite{bib:5}:
\begin{itemize}
	\item 8 ГБ оперативной памяти;
	\item процессор Apple M2 (тактовая частота~---~до $3.5$ГГц);
    \item операционная система macOS Ventura 13.0.
\end{itemize}

\section{Параметризация муравьиного алгоритма}
Параметризация~---~подбор таких параметров алгоритма, при котором он будет выдавать наиболее точный и стабильный результат. 

В результате параметризации была получена таблица со следующими столбцами:

\begin{itemize}
	\item $\alpha$~---~коэффициент стадности;
	\item $\rho$~---~коэффициент испарения феромона;
	\item $T$~---~количество циклов жизни колонии;
	\item $ideal$~---~идеальное значение, полученное в результате работы полного перебора;
	\item $\Delta$~---~разница между идеальным и фактическим результатом.
\end{itemize}

\newpage

\subsection{Параметризация для графа №1}
Матрица смежности для графа №1 равна

\begin{equation}
	\begin{pmatrix}
	  -1 & 1797 & 2326 & 4022 \\ 
1797 & -1 & 3954 & 5650 \\ 
2326 & 3954 & -1 & 1905 \\ 
4022 & 5650 & 1905 & -1 \\
	  \end{pmatrix}
\end{equation}

В таблице \ref{t:t01} представлена выборка параметризации с коэффициентами, дающими наибольшую точность.

\begin{center}
	\captionsetup{justification=raggedright,singlelinecheck=off}
	\begin{longtable}[c]{|l|l|l|l|l|}
	\caption{Результаты параметризующего запуска для графа №1 из 4 вершин\label{t:t01}} \\ \hline
	\endfirsthead
	\captionsetup{labelformat=continued,labelsep=quad}%
	\caption{}\\
	\endhead
	$\alpha$ & $\rho$ & $T$ & $ideal$ & $\Delta$ 
	
	\csvreader{../data/param_best_4.txt}{}
	{\\ \hline \csvcoli & \csvcolii & \csvcoliii & \csvcoliv & \csvcolv}
	\\ \hline
		
	\end{longtable}	
\end{center}

\subsection{Параметризация для графа №2}
Матрица смежности для графа №2 равна

\begin{equation}
	\begin{pmatrix}
-1 & 1797 & 2326 & 4022 & 1051 & 2699 & 936 \\ 
1797 & -1 & 3954 & 5650 & 1008 & 4327 & 1637 \\ 
2326 & 3954 & -1 & 1905 & 3202 & 1217 & 3115 \\ 
4022 & 5650 & 1905 & -1 & 4898 & 1897 & 4811 \\ 
1051 & 1008 & 3202 & 4898 & -1 & 3575 & 891 \\ 
2699 & 4327 & 1217 & 1897 & 3575 & -1 & 3488 \\ 
936 & 1637 & 3115 & 4811 & 891 & 3488 & -1 \\
	  \end{pmatrix}
\end{equation}

В таблице \ref{t:t02} представлена выборка параметризации с коэффициентами, дающими наибольшую точность.

\begin{center}
	\captionsetup{justification=raggedright,singlelinecheck=off}
	\begin{longtable}[c]{|l|l|l|l|l|}
	\caption{Результаты параметризующего запуска для графа №2 из 7 вершин\label{t:t02}} \\ \hline
	\endfirsthead
	\captionsetup{labelformat=continued,labelsep=quad}%
	\caption{}\\
	\endhead
	$\alpha$ & $\rho$ & $T$ & $ideal$ & $\Delta$ 
	
	\csvreader{../data/param_best_7.txt}{}
	{\\ \hline \csvcoli & \csvcolii & \csvcoliii & \csvcoliv & \csvcolv}
	\\ \hline
		
	\end{longtable}	
\end{center}

\subsection{Параметризация для графа №3}
Матрица смежности для графа №3 равна

\begin{equation}
	\begin{pmatrix}
-1 & 1797 & 2326 & 4022 & 1051 & 2699 & 936 & 1428 & 2147 & 2712 \\ 
1797 & -1 & 3954 & 5650 & 1008 & 4327 & 1637 & 1425 & 3775 & 4340 \\ 
2326 & 3954 & -1 & 1905 & 3202 & 1217 & 3115 & 3573 & 227 & 589 \\ 
4022 & 5650 & 1905 & -1 & 4898 & 1897 & 4811 & 5269 & 1964 & 1333 \\ 
1051 & 1008 & 3202 & 4898 & -1 & 3575 & 891 & 417 & 3023 & 3588 \\ 
2699 & 4327 & 1217 & 1897 & 3575 & -1 & 3488 & 3946 & 1176 & 866 \\ 
936 & 1637 & 3115 & 4811 & 891 & 3488 & -1 & 1270 & 2936 & 3501 \\ 
1428 & 1425 & 3573 & 5269 & 417 & 3946 & 1270 & -1 & 3394 & 3959 \\ 
2147 & 3775 & 227 & 1964 & 3023 & 1176 & 2936 & 3394 & -1 & 654 \\ 
2712 & 4340 & 589 & 1333 & 3588 & 866 & 3501 & 3959 & 654 & -1 \\
	  \end{pmatrix}
\end{equation}

В таблице \ref{t:t03} представлена выборка параметризации с коэффициентами, дающими наибольшую точность.

\begin{center}
	\captionsetup{justification=raggedright,singlelinecheck=off}
	\begin{longtable}[c]{|l|l|l|l|l|}
	\caption{Результаты параметризующего запуска для графа №3 из 10 вершин\label{t:t03}} \\ \hline
	\endfirsthead
	\captionsetup{labelformat=continued,labelsep=quad}%
	\caption{}\\
	\endhead
	$\alpha$ & $\rho$ & $T$ & $ideal$ & $\Delta$ 
	
	\csvreader{../data/param_best_7.txt}{}
	{\\ \hline \csvcoli & \csvcolii & \csvcoliii & \csvcoliv & \csvcolv}
	\\ \hline
		
	\end{longtable}	
\end{center}


\section{Измерение процессорного времени выполнения реализаций алгоритмов}

Для измерения процессорного времени выполнения реализаций алгоритмов была использована функция языка $C$~---~$clock\_gettime$, которая позволяет получить текущее процессорное время в наносекундах \cite{bib:6}.

В таблице \ref{table:time} представлены результаты измерений процессорного времени выполнения в зависимости от размера графа для алгоритма полного перебора и муравьиного алгоритма. На рисунке \ref{img:time} представлена зависимость времени выполнения от количества вершин графва.

\begin{table}[h]
  \caption{\label{table:time} Результаты замеров процессорного времени (в нс)}
  \begin{center}
    \begin{tabular}{|r|r|r|r|}
      \hline
      Количество вершин & Полный перебор & Муравьиный алгоритм \\ \hline
2 & 160 & 415320 \\ \hline 
3 & 360 & 747320 \\ \hline 
4 & 1260 & 1320880 \\ \hline 
5 & 10680 & 2144000 \\ \hline 
6 & 28900 & 3083100 \\ \hline 
7 & 211940 & 4222370 \\ \hline 
8 & 4177960 & 5475500 \\ \hline 
9 & 32073460 & 6915110 \\ \hline 
10 & 381928610 & 8661820 \\ \hline 


    \end{tabular}
  \end{center}
\end{table}

\newpage

\noindent
\begin{figure}[h!]
	\centering
    \includegraphics[width=0.75\linewidth]{../data/time}
    \caption{Результаты замеров времени}
    \label{img:time}
\end{figure}

\newpage