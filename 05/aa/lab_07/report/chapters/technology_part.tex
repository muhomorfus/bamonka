\chapter{Технологическая часть}

В данном разделе представлены реализации алгоритмов поиска кратчайшего расстояния в графе. Кроме того, указаны требования к ПО и средства реализации.

\section{Требования к ПО}
\begin{itemize}
	\item программа позволяет вводить имя файла, содержащего информацию о графе, с помощью аргументов командной строки;
	\item программа аварийно завершается в случае ошибок, выводя сообщение о соответствующей ошибке;
	\item программа выполняет замеры времени работы реализаций алгоритмов;
	\item программа строит зависимости времени работы реализаций алгоритмов от размеров входных данных;
	\item программа позволяет вводить основные параметры муравьиного алгоритма, такие как $\alpha$, $\beta$ и другие с помощью аргументов командной строки.
\end{itemize}

\section{Средства реализации}
Для реализации данной работы выбран язык программирования Go, так как он содержит необходимые для тестирования библиотеки, а также обладает достаточными инструментами для реализации ПО, удовлетворяющего требованиям данной работы \cite{bib:3}.

\section{Реализации алгоритмов}
В листингах \ref{code:1}~--~\ref{code:6} представлены реализации алгоритма полного перебора и муравьиного алгоритма.

\newpage

\begin{code}
\caption{Исходный код реализации алгоритма полного перебора}
\label{code:1}
\begin{minted}{go}
func (m *Manager) MinPath(g *graph.Graph) ([]int, int, bool) {
	paths := allPaths(g.GetSize())
	minLen := math.MaxInt
	var minPath []int

	for _, path := range paths {
		l, ok := g.LenPath(path)
		if !ok {
			continue
		}

		if l < minLen {
			minLen = l
			minPath = path
		}
	}

	return minPath, minLen, minLen != math.MaxInt
}
\end{minted}
\end{code}

\begin{code}
\caption{Исходный код реализации алгоритма поиска следующей вершины в маршруте муравья}
\label{code:2}
\begin{minted}{go}
func (a *ant) trip(vertices []int, pheromones [][]float64, 
visibles [][]float64) (int, bool) {
	probabilities := make([]float64, 0)
	probability := 0.0

	for _, v := range vertices {
		if visibles[a.getPos()][v] != -1 {
			p := math.Pow(visibles[a.getPos()][v], a.beta) * 
			math.Pow(pheromones[a.getPos()][v], a.alpha)
\end{minted}
\end{code}

\newpage

\begin{code}
\caption{Исходный код реализации алгоритма поиска следующей вершины в маршруте муравья (продолжение листинга \ref{code:2})}
\label{code:3}
\begin{minted}{go}
			probabilities = append(probabilities, p)
			probability += p
		}
	}

	if probability <= 0 {
		return 0, false
	}

	maxProbability := 0.0
	idx := 0
	for i, v := range probabilities {
		v /= probability
		if v > maxProbability {
			maxProbability = v
			idx = vertices[i]
		}
	}

	if a.elite {
		return idx, true
	}

	return vertices[randVariant(probabilities, probability)-1], true
}
\end{minted}
\end{code}

\newpage

\begin{code}
\caption{Исходный код реализации алгоритма поиска маршрута для муравья}
\label{code:4}
\begin{minted}{go}
func (a *ant) findPath(g *graph.Graph, pheromones [][]float64, 
visibles [][]float64) bool {
	vertices := make([]int, 0, g.GetSize())
	for i := 0; i < g.GetSize(); i++ {
		if i != a.getPos() {
			vertices = append(vertices, i)
		}
	}
	for len(vertices) != 0 {
		next, ok := a.trip(vertices, pheromones, visibles)
		if !ok {
			return false
		}
		dst, _ := g.GetDistance(a.getPos(), next)
		a.move(next, dst)
		vertices = rm(vertices, next)
	}
	return true
}
\end{minted}
\end{code}

\begin{code}
\caption{Исходный код реализации муравьиного алгоритма}
\label{code:5}
\begin{minted}{go}
func (m *Manager) MinPath(g *graph.Graph) ([]int, int, bool) {
	minLen := math.MaxInt
	var minPath []int
	pheromones := newPheromones(g)
	visibles := newVisibles(g)
	for t := 0; t < m.n; t++ {
		c := newColony(g, m.alpha, m.beta, m.k)
		c.evaporate(pheromones)
\end{minted}
\end{code}

\newpage

\begin{code}
\caption{Исходный код реализации муравьиного алгоритма (продолжение листинга \ref{code:5}}
\label{code:6}
\begin{minted}{go}
		for _, v := range c.ants {
			ok := v.findPath(g, pheromones, visibles)
			if !ok {
				return nil, 0, false
			}
			if v.len < minLen && len(v.path) == g.GetSize() {
				minLen = v.len
				minPath = v.path
			}
		}
		c.correct(pheromones)
	}
	if len(minPath) == 0 {
		return nil, 0, false
	}
	return minPath, minLen, true
}
\end{minted}
\end{code}


\section{Тестирование}
Тестирование проводилось по методологии чёрного ящика. \textbf{Тесты пройдены успешно.}

В таблице \ref{table:tests}  представлены тестовые данные для реализаций алгоритма полного перебора и муравьиного алгоритма. 

\begin{table}[H]
  \caption{\label{table:tests} Тестовые данные для реализаций алгоритма полного перебора и муравьиного алгоритма}
  \begin{center}
    \begin{tabular}{|c|l|l|}
      \hline
      № & Граф & Результат \\ \hline
      1 
      & 
      $\begin{pmatrix}
	  -1 & 875 \\
875 & -1 \\
	  \end{pmatrix}$
      &
      875
      \\ \hline
      
      2 
      & 
      $\begin{pmatrix}
	  -1 &	1797 &	2326 \\
1797 &	-1 &	3954 \\
2326 &	3954 &	-1 \\
	  \end{pmatrix}$
      &
      4123
      \\ \hline
      
      3 
      & 
      $\begin{pmatrix}
	  -1 &	1797 &	2326 &	4022 \\
1797 &	-1 &	3954 &	5650 \\
2326 &	3954 &	-1 &	1905 \\
4022 &	5650 &	1905 &	-1 \\
	  \end{pmatrix}$
      &
      6028
      \\ \hline
    \end{tabular}
  \end{center}
\end{table}

\newpage