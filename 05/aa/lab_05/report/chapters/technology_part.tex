\chapter{Технологическая часть}

В данном разделе представлены реализации алгоритмов сортировки. Кроме того, указаны требования к ПО и средства реализации.

\section{Требования к ПО}
\begin{itemize}
	\item программа позволяет вводить имя файла, содержащего информацию о наборах слов, с помощью аргументов командной строки;
	\item программа аварийно завершается в случае ошибок, выводя сообщение о соответствующей ошибке;
	\item программа выполняет замеры времени работы реализаций алгоритмов;
	\item программа строит зависимости времени работы реализаций алгоритмов от размеров входных данных;
	\item программа принимает на вход файл в следующем формате: наборы слов (тексты) разделены последовательностью символов <<***>>, слова разделены символами переноса строки;
	\item программа принимает на вход набор массивов слов, нормализует его, удаляет дубликаты и рассчитывает частоту терма для каждого терма в тексте.
\end{itemize}

\section{Средства реализации}
Для реализации данной работы выбран язык программирования Go, так как он содержит необходимые для тестирования библиотеки, а также обладает достаточными инструментами для реализации ПО, удовлетворяющего требованиям данной работы \cite{bib:3}.

\section{Реализации алгоритмов}
В листингах \ref{code:1}~--~\ref{code:3} представлены реализации алгоритмов обработки текста.

В листингах \ref{code:4}~--~\ref{code:6} представлены реализации алгоритмов обработчиков конвейера.

В листингах \ref{code:7}~--~\ref{code:8} представлены реализации алгоритмов конвейерной обработки данных.

\newpage

\begin{code}
\caption{Реализация алгоритма нахождения начальных форм слов}
\label{code:1}
\begin{minted}{go}
func Normalize(words []string) []string {
	terms := make([]string, 0)
	
	for _, v := range words {
		_, normals, _ := morph.Parse(strings.ToLower(v))
		if len(normals) == 0 {
			panic("no normals")
		}
		
		terms = append(terms, normals[0])
	}
	
	return terms
}
\end{minted}
\end{code}

\begin{code}
\caption{Реализация алгоритма дедупликации}
\label{code:2}
\begin{minted}{go}
func Deduplicate(terms []string) map[string]int {
	termsMap := make(map[string]int)
	
	for _, v := range terms {
		termsMap[v] = termsMap[v] + 1
	}
	
	return termsMap
}
\end{minted}
\end{code}

\newpage

\begin{code}
\caption{Реализация алгоритма подсчета частоты терма}
\label{code:3}
\begin{minted}{go}
func CountTF(termsMap map[string]int) map[string]float64 {
	n := 0
	
	for _, v := range termsMap {
		n += v
	}
	
	tfs := make(map[string]float64)
	
	for k, v := range termsMap {
		tfs[k] = float64(v) / float64(n)
	}
	
	return tfs
}
\end{minted}
\end{code}

\begin{code}
\caption{Обработчик первой ленты конвейера~---~нормализация текста}
\label{code:4}
\begin{minted}{go}
func Normalize(in <-chan task.Task, out chan<- task.Task) {
	for t := range in {
		t.StartTime1 = time.Now()
		t.Terms = text.Normalize(t.Text)
		t.EndTime1 = time.Now()
		
		out <- t
	}
	
	close(out)
}
\end{minted}
\end{code}

\newpage

\begin{code}
\caption{Обработчик второй ленты конвейера~---~дедупликация текста}
\label{code:5}
\begin{minted}{go}
func Deduplicate(in <-chan task.Task, out chan<- task.Task) {
	for t := range in {
		t.StartTime2 = time.Now()
		t.Stat = text.Deduplicate(t.Terms)
		t.EndTime2 = time.Now()
		
		out <- t
	}
	
	close(out)
}
\end{minted}
\end{code}

\begin{code}
\caption{Обработчик третьей ленты конвейера~---~поиск частоты слова}
\label{code:6}
\begin{minted}{go}
func CountTF(in <-chan task.Task, out chan<- task.Task) {
	for t := range in {
		t.StartTime3 = time.Now()
		t.TFs = text.CountTF(t.Stat)
		t.EndTime3 = time.Now()
		
		out <- t
	}

	close(out)
}
\end{minted}
\end{code}

\newpage

\begin{code}
\caption{Реализация последовательного конвейера}
\label{code:7}
\begin{minted}{go}
func Pipeline(texts [][]string, log bool) []map[string]float64 {
	tfs := make([]map[string]float64, len(texts))
	resTasks := make([]task.Task, 0, len(texts))
	now := time.Now()
	
	for _, v := range texts {
		chanNormalize := make(chan task.Task, 1)
		chanDeduplicate := make(chan task.Task, 1)
		chanCountTF := make(chan task.Task, 1)
		chanResult := make(chan task.Task, 1)
		
		t := task.Task{
			Text: v,
			TimeStat: task.TimeStat{
				StartTime0: now,
			},
		}
		
		chanNormalize <- t
		close(chanNormalize)
		
		workers.Normalize(chanNormalize, chanDeduplicate)
		workers.Deduplicate(chanDeduplicate, chanCountTF)
		workers.CountTF(chanCountTF, chanResult)
		resTasks = append(resTasks, <-chanResult)
	}
	
	for i, v := range resTasks {
		tfs[i] = v.TFs
	}
	
	return tfs
}
\end{minted}
\end{code}

\newpage

\begin{code}
\caption{Реализация параллельного конвейера}
\label{code:8}
\begin{minted}{go}
func Pipeline(texts [][]string, log bool) []map[string]float64 {
	tfs := make([]map[string]float64, len(texts))
	resTasks := make([]task.Task, 0, len(texts))
	now := time.Now()
	chanNormalize := make(chan task.Task, len(texts))
	chanDeduplicate := make(chan task.Task, len(texts))
	chanCountTF := make(chan task.Task, len(texts))
	chanResult := make(chan task.Task, len(texts))
	go workers.Normalize(chanNormalize, chanDeduplicate)
	go workers.Deduplicate(chanDeduplicate, chanCountTF)
	go workers.CountTF(chanCountTF, chanResult)

	for _, v := range texts {
		t := task.Task{
			Text: v,
			TimeStat: task.TimeStat{
				StartTime0: now,
			},
		}

		chanNormalize <- t
	}
	close(chanNormalize)
	for t := range chanResult {
		resTasks = append(resTasks, t)
	}
	for i, v := range resTasks {
		tfs[i] = v.TFs
	}

	return tfs
}
\end{minted}
\end{code}

\newpage

\section{Тестирование}
Тестирование проводилось по методологии чёрного ящика. \textbf{Тесты пройдены успешно.}

В таблице \ref{table:tests} представлены тестовые данные для реализаций алгоритмов конвейерной обработки данных. 

\begin{table}[H]
  \caption{\label{table:tests} Тестовые данные для алгоритмов конвейерной обработки данных}
  \begin{center}
    \begin{tabular}{|c|l|l|}
      \hline
      № & Набор слов & Результат \\ \hline
      1 
      & 
      \makecell{
      <<Автомобили>> \\
	  <<ехали>> \\
	  <<и>> \\
	  <<едут>> \\
	  <<и>> \\
	  <<будут>> \\
	  <<ехать>> \\
      }
      &
      \makecell{
      <<и>>: $0.285714$ \\
        <<быть>>: $0.142857$ \\
        <<автомобиль>>: $0.142857$ \\
        <<ехать>>: $0.428571$
      }
      \\ \hline
      
      
      2
      & 
      \makecell{
      <<тридцать>> \\
	  <<девять>> \\
	  <<около>> \\
	  <<девяти>> 
      }
      &
      \makecell{
      <<тридцать>>: $0.250000$ \\
        <<девять>>: $0.500000$ \\
        <<около>>: $0.250000$
      }
      \\ \hline
      
      3
      & 
      \makecell{
      <<летел>> \\
	  <<лебедь>> \\
	  <<по>> \\
	  <<синему>> \\
	  <<морю>>
      }
      &
      \makecell{
      <<лететь>>: $0.200000$ \\
        <<лебедь>>: $0.200000$ \\
        <<по>>: $0.200000$ \\
        <<синий>>: $0.200000$ \\
        <<море>>: $0.200000$ \\
      }
      \\ \hline
    \end{tabular}
  \end{center}
\end{table}

\newpage