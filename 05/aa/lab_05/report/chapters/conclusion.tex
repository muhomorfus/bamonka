\chapter*{Заключение}
\addcontentsline{toc}{chapter}{Заключение} 

Параллельная конвейерная обработка данных дает преимущество в плане производительности при работе с наборами данных, требующих последовательной обработки. При количестве заявок, равном двадцати, параллельный конвейер работает в $1.03$ раз быстрее последовательного.

Среднее время простоя в первой очереди при параллельном конвейере в $1.03$ раза меньше, чем при последовательном. При этом времена простоя во второй и третьей очереди равны нулю. Такое поведения связано с тем, что первый обработчик конвейера работает медленнее, чем второй и третий. Поэтому заявки накапливаются перед первым обработчиком, а после выдачи из первого обработчика сразу принимаются вторым, так как второй обработчик данных простаивает в данном случае.

Цель работы была достигнута: были изучены алгоритмы конвейерной обработки данных в контексте решения задачи расчета частоты вхождения термов в набор слов. Были выполнены все задачи:

\begin{itemize}
	\item изучены этапы обработки текстов;
	\item изучены алгоритмы конвейерной обработки данных;
	\item разработаны алгоритмы параллельной и последовательной конвейерной обработки данных;
	\item кодированы данные алгоритмы;
	\item проведен замерный эксперимент для данных алгоритмов, с измерением времени работы; 
	\item проведен сравнительный анализа алгоритмов на основе полученных данных.
\end{itemize}


\newpage