\setcounter{page}{3}
\chapter*{Введение}
\addcontentsline{toc}{chapter}{Введение} 

В современном мире часто возникает задача обработки данных в несколько этапов. 
Так как в этом случаи этапы зависят друг от друга по данным, их нельзя распараллелить привычными способами. 
В этом случае часто применяют конвейерную обработку данных~\cite{bib:1}.

\section*{Цель работы}

Получение навыков кодирования программного продукта, тестирования и проведения замерного эксперимента работы программы на различных данных. Все это на примере решения задачи конвейерной обработки данных для расчета частоты вхождения термов в набор слов.

\section*{Задачи работы}

\begin{enumerate}[label={\arabic*)}]
	\item изучение этапов обработки текстов:
	\begin{itemize}
		\item приведение к начальной форме;
		\item дедупликация;
		\item расчет частоты вхождения терма в текст;
	\end{itemize}
	\item изучение алгоритмов конвейерной обработки данных;
	\item разработка алгоритмов параллельной и последовательной конвейерной обработки данных;
	\item кодирование данных алгоритмов;
	\item проведение замерного эксперимента для данных алгоритмов, с измерением времени работы; 
	\item проведение сравнительного анализа алгоритмов на основе полученных данных.
\end{enumerate}

\newpage