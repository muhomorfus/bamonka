\chapter{Экспериментальная часть}

В данном разделе описаны замерные эксперименты и представлены результаты исследования.

\section{Технические характеристики}
Технические характеристики устройства, на котором выполнялось исследование \cite{bib:5}:
\begin{itemize}
	\item 8 ГБ оперативной памяти;
	\item процессор Apple M2 (тактовая частота~---~до $3.5$ГГц);
    \item операционная система macOS Ventura 13.0.
\end{itemize}

\section{Измерение реального времени выполнения реализаций алгоритмов}

Для измерения реального времени выполнения реализаций алгоритмов была использована стандартная библиотека языка программирования Go, позволяющая замерять время работы реализаций алгоритмов в микросекундах \cite{bib:4}.

\subsection{Время работы реализаций алгоритмов}
В таблице \ref{table:time} представлены результаты измерений реального времени выполнения в зависимости от количества заявок для алгоритмов конвейеризации. На рисунке \ref{img:time} представлена зависимость времени выполнения от количества заявок.

\newpage

\begin{table}[h]
  \caption{\label{table:time} Результаты замеров реального времени (в мкс)}
  \begin{center}
    \begin{tabular}{|r|r|r|}
      \hline
      Кол-во заявок & Последовательный & Параллельный \\ \hline 
1 & 416 & 442 \\ \hline 
2 & 828 & 838 \\ \hline 
3 & 1244 & 1236 \\ \hline 
4 & 1649 & 1636 \\ \hline 
5 & 2075 & 2030 \\ \hline 
6 & 2475 & 2426 \\ \hline 
7 & 2888 & 2820 \\ \hline 
8 & 3345 & 3217 \\ \hline 
9 & 3747 & 3619 \\ \hline 
10 & 4126 & 4017 \\ \hline 
12 & 4953 & 4807 \\ \hline 
14 & 5780 & 5613 \\ \hline 
16 & 6671 & 6393 \\ \hline 
18 & 7489 & 7189 \\ \hline 
20 & 8259 & 7987 \\ \hline 
    \end{tabular}
  \end{center}
\end{table}

\noindent
\begin{figure}[h!]
	\centering
    \includegraphics[width=0.7\linewidth]{../data/time}
    \caption{Результаты замеров времени}
    \label{img:time}
\end{figure}

\newpage


\subsection{Журналирование}

В таблицах \ref{table:journal1} и \ref{table:journal2} представлен пример журналирования времени обработки заявок каждым обработчиком последовательного и параллельного конвейера для количества заявок, равной трем.

\begin{table}[h]
  \caption{\label{table:journal1} Пример журналирования на последовательном конвейере}
  \begin{center}
    \begin{tabular}{|r|r|r|r|}
      \hline
  № заявки &       Этап &     Начало &      Конец \\ \hline
         1 &          1 &          1 &        648 \\ \hline
         1 &          2 &        648 &        681 \\ \hline
         1 &          3 &        681 &        704 \\ \hline
         2 &          1 &        705 &       1098 \\ \hline
         2 &          2 &       1099 &       1124 \\ \hline
         2 &          3 &       1124 &       1154 \\ \hline
         3 &          1 &       1155 &       1549 \\ \hline
         3 &          2 &       1549 &       1576 \\ \hline
         3 &          3 &       1576 &       1600 \\ \hline
    \end{tabular}
  \end{center}
\end{table}

\begin{table}[h]
  \caption{\label{table:journal2} Пример журналирования на параллельном конвейере}
  \begin{center}
    \begin{tabular}{|r|r|r|r|}
      \hline
  № заявки &       Этап &     Начало &      Конец \\ \hline
         1 &          1 &         10 &        695 \\ \hline
         1 &          2 &        712 &        752 \\ \hline
         1 &          3 &        754 &        783 \\ \hline
         2 &          1 &        697 &       1099 \\ \hline
         2 &          2 &       1108 &       1138 \\ \hline
         2 &          3 &       1139 &       1163 \\ \hline
         3 &          1 &       1101 &       1493 \\ \hline
         3 &          2 &       1496 &       1525 \\ \hline
         3 &          3 &       1526 &       1552 \\ \hline
    \end{tabular}
  \end{center}
\end{table}

\newpage

\subsection{Анализ характеристик работы реализаций алгоритмов}

В таблице \ref{table:info} представлен анализ характеристик работы реализаций алгоритмов конвейерной обработки. Анализ характеристик производится для количества заявок, равному двадцати.

\begin{table}[H]
  \renewcommand{\arraystretch}{1.4}
  \caption{\label{table:info} Анализ характеристик работы реализаций алгоритмов конвейерной обработки}
  \begin{center}
  \begin{tabular}{|l|l|l|l|l|l|l|l|} \cline{1-8}
   \multicolumn{2}{|c|}{Характеристика} & \multicolumn{3}{|c|}{Последовательно, мкс} & \multicolumn{3}{|c|}{Параллельно, мкс} \\ \cline{1-8}
   \multicolumn{2}{|c|}{Линия} & \multicolumn{1}{c}{1} & \multicolumn{1}{|c|}{2} & \multicolumn{1}{|c|}{3} & \multicolumn{1}{|c|}{1} & \multicolumn{1}{|c|}{2} & \multicolumn{1}{|c|}{3} \\ \cline{1-8}
   \multirow{4}{*}{Простой очереди}
   & sum &  83232 &     0 &     0     & 80654 &   206 &    20 \\
   & min  &   1 &     0 &     0           &   12 &     2 &     0     \\
   & max  &  8065 &     0 &     0     & 7826 &    17 &     2     \\
   & avg  &  4161 &     0 &     0      &  4032 &    10 &     1     \\ \cline{1-8}
   \multirow{3}{*}{\begin{tabular}[c]{@{}l@{}}Время заявки\\ в системе\end{tabular}} 
   & min & \multicolumn{3}{c|}{683} & \multicolumn{3}{c|}{788} \\  
   & max & \multicolumn{3}{c|}{8469} & \multicolumn{3}{c|}{8289} \\ 
   & avg & \multicolumn{3}{c|}{4584} & \multicolumn{3}{c|}{4510} \\ \hline
  \end{tabular}
  \end{center}
 \end{table}

\newpage