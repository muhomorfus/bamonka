\chapter{Аналитическая часть}

В данном разделе рассмотрена формализация объекта и его признака, а также описан алгоритм поиска в словаре.

\section{Формализация объекта и его признака}
Словарь в данном случае представлен в виде списка объектов с признаками. Объектом является название пива. Признаком является крепость пива \cite{bib:1}.

При анализе признака в таком словаре осуществляется поиск диапазона допустимых значений для крепости пива.

\section{Двоичный поиск в словаре}
Двоичный поиск~---~алгоритм поиска, работающий на отсортированных данных, и позволяющий осуществлять поиск элемента с трудоемкостью $O(\log N)$. 

Идея алгоритма заключается в том, что исходный словарь разделяется на две половины, и проверяется серединный элемент, если он больше искомого, то в правой половине словаря поиск уже не имеет смысла, так как элементы в правой части больше серединного. Соответственно, имеет смысл дальнейшее рассмотрение только левой части.

Аналогичные выводы можно сделать для ситуации, когда серединный элемент меньше искомого. Тогда следует рассматривать только правую часть словаря.

\newpage