\chapter{Конструкторская часть}

В данном разделе представлены описание структуры данных словаря, и также алгоритм двоичного поиска в словаре.

\section{Описание структуры данных словаря}
Словарь представляет собой список элементов. Каждый элемент списка представляет собой запись, содержащую два поля:

\begin{itemize}
	\item $strength$~---~крепость пива, число с плавающей точкой;
	\item $data$~---~информация о пиве, например, название, фирма и так далее.
\end{itemize}

\section{Разработка алгоритма двоичного поиска в словаре}
Пусть

\begin{itemize}
	\item $D$~---~словарь;
	\item $n$~---~количество элементов в словаре;
	\item $v$~---~искомое значение;
	\item $flag$~---~если значение <<истина>>, то вычисляется индекс ближайшего элемента, большего данного, иначе~---~меньшего.
\end{itemize}

\begin{enumerate}
	\item Вычислить левую границу обрабатываемого диапазона:
	\begin{equation}
		l = 0.
	\end{equation}
	
	\item Вычислить правую границу обрабатываемого диапазона:
	\begin{equation}
		r = n.
	\end{equation}
	
	\item Пока $l \leq r$
	\begin{enumerate}
		\item Вычислить индекс середины диапазона:
		\begin{equation}
			m = \frac{l + r}{2}.
		\end{equation}
		
		\item Если $v < D_m$, то $r = m - 1$.
		\item Иначе, если $v > D_m$, то $l = m + 1$.
		\item Иначе вернуть $m$.
	\end{enumerate}
	
	\item Если $(D_l > D_r) = flag$, то вернуть $l$.
	\item Вернуть $r$.
\end{enumerate}


\newpage