\chapter*{Заключение}
\addcontentsline{toc}{chapter}{Заключение} 

Алгоритм двоичного поиска имеет асимптотику $O(\log N)$, то есть является более эффективным алгоритмом, чем, например линейный поиск. Но двоичный поиск требует отсортированности данных. Так как в рамках задачи предполагается упорядочивание данных, то данный алгоритм применим в данной задаче.

Цель работы была достигнута: был получен навык поиска по словарю при ограничении на значение признака, заданном при помощи логической переменной. Были выполнены все задачи:

\begin{itemize}
	\item формализован объект и его признак;
	\item составлена анкеты для заполнения респондентами;
	\item проведено анкетирование респондентов;
	\item простроена функция принадлежности термам числовых значений признака, описываемого лингвистической переменной, на основе статистической обработки мнений респондентов;
	\item описаны типовые вопросы на русском языке, имеющие целью запрос на поиск в словаре;
	\item описаны алгоритмы поиска в словаре объектов, удовлетворяющих ограничению, заданному в вопросе на ограниченном естественном языке;
	\item описана структура данных словаря, хранящего наименования объектов согласно варианту и числовое значения признака объекта;
	\item реализован данный алгоритм поиска в словаре;
	\item приведены примеры запросов пользователя и сформированный реализацией алгоритма поиска выборки объекта из словаря, используя составленные респондентами вопросы;
	\item дано заключение о применимости предложенного алгоритма и его ограничениях.
\end{itemize}


\newpage