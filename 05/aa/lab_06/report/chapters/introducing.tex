\setcounter{page}{3}
\chapter*{Введение}
\addcontentsline{toc}{chapter}{Введение} 

При работе с большими объемами текстовой информации часто возникает необходимость организации ее хранения, удобной для поиска. Одной из структур данных, используемой при работе с текстом, является словарь, который представляет собой набор элементов типа <<ключ-значение>>.

\section*{Цель работы}

Получение навыка поиска по словарю при ограничении на значение признака, заданном при помощи лингвистической переменной.

\section*{Задачи работы}

\begin{enumerate}[label={\arabic*)}]
	\item формализовать объект и его признак;
	\item составить анкету для её заполнения респондентом;
	\item провести анкетирование респондентов;
	\item построить функцию принадлежности термам числовых значений признака, описываемого лингвистической переменной, на основе статистической обработки мнений респондентов;
	\item описать 3~--~5 типовых вопросов на русском языке, имеющих целью запрос на поиск в словаре;
	\item описать алгоритм поиска в словаре объектов, удовлетворяющих ограничению, заданному в вопросе на ограниченном естественном языке;
	\item описать структуру данных словаря, хранящего наименования объектов согласно варианту и числовое значение признака объекта;
	\item реализовать описанный алгоритм поиска в словаре;
	\item привести примеры запросов пользователя и сформированной реализацией алгоритма поиска выборки объектов из словаря, используя составленные респондентами вопросы;
	\item дать заключение о применимости предложенного алгоритма и о его ограничениях.

\end{enumerate}

\newpage