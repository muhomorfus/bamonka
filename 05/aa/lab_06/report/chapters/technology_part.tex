\chapter{Технологическая часть}

В данном разделе представлены реализации структуры словаря и алгоритма двоичного поиска в словаре. Кроме того, указаны требования к ПО и средства реализации.

\section{Требования к ПО}
\begin{itemize}
	\item программа позволяет вводить имя файла, содержащего информацию о наборах слов, с помощью аргументов командной строки;
	\item программа аварийно завершается в случае ошибок, выводя сообщение о соответствующей ошибке;
	\item программа принимает на вход строку запроса в свободном формате вида: <<выдай очень крепкое пиво>>;
	\item программа предлагает исправление опечаток в запросе.
\end{itemize}

\section{Средства реализации}
Для реализации данной работы выбран язык программирования Go, так как он содержит необходимые для тестирования библиотеки, а также обладает достаточными инструментами для реализации ПО, удовлетворяющего требованиям данной работы \cite{bib:3}.

\section{Реализация структуры словаря}
В листингах \ref{code:1}~--~\ref{code:2} представлена реализация структуры данных <<словарь>>.

\begin{code}
\caption{Реализация структуры словаря}
\label{code:1}
\begin{minted}{go}
type Beer struct {
	Name string
	Strength float64
}
\end{minted}
\end{code}

\newpage

\begin{code}
\caption{Реализация структуры словаря (продолжение листинга \ref{code:1})}
\label{code:2}
\begin{minted}{go}
type Dictionary struct {
	Elements []Beer
}
\end{minted}
\end{code}

\section{Реализации алгоритмов}
В листинге \ref{code:3} представлена реализация алгоритма двоичного поиска по словарю.

\begin{code}
\caption{Реализация алгоритма}
\label{code:3}
\begin{minted}{go}
func (d *Dictionary) search(v float64, flag bool) int {
	if v > d.Elements[len(d.Elements) - 1].Strength {
		return len(d.Elements) - 1
	} else if v < d.Elements[0].Strength {
		return 0
	}
	l, r := 0, len(d.Elements) - 1
	for l <= r {
		m := (l + r) / 2
		if v > d.Elements[m].Strength {
			l = m + 1
		} else if v < d.Elements[m].Strength {
			r = m - 1
		} else {
			return m
		}
	}
	if d.Elements[l].Strength > d.Elements[r].Strength == flag {
		return l
	}
	return r
}
\end{minted}
\end{code}

\newpage

\section{Тестирование}
Тестирование проводилось по методологии чёрного ящика. \textbf{Тесты пройдены успешно.}

В таблице \ref{table:tests} представлены тестовые данные для реализаций выполнения запросов с использованием поиска по словарю. 

\begin{table}[H]
  \caption{\label{table:tests} Тестовые данные для реализаций выполнения запросов с использованием поиска по словарю}
  \begin{center}
    \begin{tabular}{|c|l|l|}
      \hline
      № & Запрос & Результат \\ \hline
      1 
      & 
      \makecell{
      <<какое пивко для детейй>>
      }
      &
      \makecell{
      Возможно, вы имели в виду «какое пивко для детей». \\
 		1. Балтика № 0 «Безалкогольное» - 0.5\% \\
 		2. Балтика № 7 «Безалкогольное» - 0.5\% \\
 		3. Пиво Primator Chipper - 2.0\% \\
 		4. Hoegaarden, Radler Lemon & Lime - 2.0\% \\
 		5. Балтика № 1 «Лёгкое» - 3.0\% \\
 		6. Балтика № 2 «Светлое» - 3.5\% \\
 		7. Жигулевское - 3.5\% \\
 		8. Балтика № 3 «Классическое» - 3.8\% 
      }
      \\ \hline
      
      
      2
      & 
      \makecell{
      <<крепкое пиво>>
      }
      &
      \makecell{
       1. Балтика № 9 «Крепкое» - 8.5\% \\
 2. Охота крепкое - 8.5\% 
      }
      \\ \hline
      
      3
      & 
      \makecell{
      <<квашеные гвозди>>
      }
      &
      \makecell{
      не понятно, о чем запрос
      }
      \\ \hline
      
      4
      & 
      \makecell{
      <<льмкльедлк>>
      }
      &
      \makecell{
      не найдено слово
      }
      \\ \hline
    \end{tabular}
  \end{center}
\end{table}

\newpage