\chapter{Аналитическая часть}

В данном разделе рассмотрены теоретические выкладки по алгоритму обратной трассировки лучей, а также по его многопоточной версии.

\section{Алгоритм обратной трассировки лучей}
Алгоритмы трассировки лучей, являются методами грубой силы, так как не учитывают специфику обрабатываемых объектов \cite{bib:rojers}. Идея алгоритмов заключается в отслеживании световых лучей между источниками света, объектами и наблюдателем. 

Алгоритм прямой трассировки лучей отслеживает лучи от источников света, с учетом их отражения от объектов сцены. Данный алгоритм является неэффективным \cite{bib:rojers}, так как множество лучей не доходят до наблюдателя, но все равно оказываются обработаны.

Альтернативой методу прямой трассировки является алгоритм обратной трассировки лучей, в котором происходит отслеживание световых лучей в обратном направлении, то есть от наблюдателя к объектам, от объектов к источникам света. Это позволяет обрабатывать только видимые наблюдателем лучи. Визуализация алгоритма приведена на рис.~\ref{img:raytracing}.

\noindent
\begin{figure}[h!]
	\centering
    \includegraphics[width=0.4\linewidth]{ray}
    \caption{Визуализация алгоритма обратной трассировки лучей}
    \label{img:raytracing}
\end{figure}

Алгоритмы позволяют учитывать тени и освещения, работают со сценами любой сложности. Кроме того, из-за присущей алгоритмам параллельности~---~лучи можно обрабатывать независимо друг от друга, алгоритм можно реализовывать с использованием методов параллельной обработки \cite{bib:parallel_ray}.

\section{Многопоточный алгоритм трассировки лучей}

Основной идеей алгоритма трассировки лучей является то, что к каждому пикселу экрана пускается луч, и проверяется его пересечение с объектами сцены. Соответственно, так как пиксели не зависят друг от друга, то алгоритм можно использованием с использованием многопоточной технологии.

Для использования алгоритма на нескольких потоках, пикселы экрана нумеруются от $0$ до $w \cdot h$, где $w$~---~ширина экрана в пикселах, $h$~---~высота экрана в пикселах.

Соответственно координаты точки экрана под номером $n$ можно вычислить по формулам:

\begin{equation}
	x = n \bmod w,
\end{equation}

\begin{equation}
	y = n \div w.
\end{equation}

Количество пикселов, обрабатываемых каждым потоком, можно вычислить по формуле:

\begin{equation}
	p = \frac{w \cdot h}{T},
\end{equation}

где \begin{itemize}
	\item $p$~---~количество пикселов, обрабатываемых каждый потоком;
	\item $T$~---~количество потоков.
\end{itemize}

Соответственно, каждый поток будет обрабатывать свой набор пикселов, где номер пиксела будет представлять собой элемент подмножества всех номеров пикселов длиной $p$.

\newpage