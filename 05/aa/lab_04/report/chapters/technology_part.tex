\chapter{Технологическая часть}

В данном разделе представлены реализации алгоритмов обратной трассировки лучей. Кроме того, указаны требования к ПО и средства реализации.

\section{Требования к ПО}
\begin{itemize}
	\item программа позволяет вводить имя файла, содержащего информацию о сцене, с помощью аргументов командной строки;
	\item программа аварийно завершается в случае ошибок, выводя сообщение о соответствующей ошибке;
	\item программа выполняет замеры времени работы реализаций алгоритмов;
	\item программа строит зависимости времени работы реализаций алгоритмов от размеров входных данных;
	\item программа предоставляет графический интерфейс.
\end{itemize}

\section{Средства реализации}
Для реализации данной работы выбран язык программирования C++, так как он содержит необходимые для тестирования библиотеки, а также обладает достаточными инструментами для реализации ПО, удовлетворяющего требованиям данной работы \cite{bib:3}.

В качестве набора библиотек был выбран \texttt{Qt}, так как он обладает достаточным инструменталом для работы с компьютерной графикой, а также имеет набор средств для создания кроссплатформенных графических приложений \cite{bib:qt}.

\section{Реализации алгоритмов}
В листингах \ref{code:start}~--~\ref{code:end} представлены реализации алгоритмов обратной трассировки лучей.

\newpage

\begin{code}
\caption{Реализация алгоритма обратной трассировки лучей}
\label{code:start}
\begin{minted}{C}
QColor Scene::traceRay(const Ray &ray) {
    TraceRayData data;
    size_t index = 0;

    bool first = true;

    for (size_t i = 0; i < polygons.size(); i++) {
        auto res = polygons[i].traceRay(ray);

        if (!res.ok) {
            continue;
        }

        if (first || res.t < data.t) {
            index = i;
            data = res;
            first = false;
        }
    }

    if (first) {
        return background;
    }

    auto col = getColor(data.point, ray, lights, polygons[index]);
    if (!col.haveColor) {
        return Qt::black;
    }

    return col.color;
}
\end{minted}
\end{code}

\newpage

\begin{code}
\caption{Реализация функции, выполняющей трассировку лучей в одном потоке}
\begin{minted}{C}
void Scene::paintThread(std::vector<std::vector<QColor>> &cbuf, int start,
int size)
{
    int maxSize = width * height;
    int end = start + size;
    if (end > maxSize) {
        end = maxSize;
    }

    for (int n = start; n < end; n++) {
        int y = n / width;
        int x = n % width;

        Ray ray = Ray(camera, (QVector3D(x, y, 0) - camera).normalized());

        cbuf[y][x] = traceRay(ray);
    }
}
\end{minted}
\end{code}

\begin{code}
\caption{Основная функция трассировки лучей}
\label{code:prev}
\begin{minted}{C}
std::shared_ptr<QImage> Scene::paint(int nthreads) {
    polygons = forest.toPolygons();
    auto image = std::make_shared<QImage>(width, height,
     QImage::Format_RGB32);
    image->fill(Qt::black);
    QPainter painter(image.get());
    std::vector<std::vector<QColor>> colorBuffer(height,
     std::vector<QColor>(width, Qt::black));
    if (nthreads == 0) {
        Ray ray = Ray(camera, QVector3D());
\end{minted}
\end{code}

\newpage

\begin{code}
\caption{Основная функция трассировки лучей (продолжение листинга \ref{code:prev})}
\label{code:end}
\begin{minted}{C}
        for (int y = 0; y < height; y++) {
            for (int x = 0; x < width; x++) {
                ray.direction = (QVector3D(x, y, 0) - camera).normalized();
                colorBuffer[y][x] = traceRay(ray);
            }
        }
    } else {
        int hw = height * width;
        int n =  hw / nthreads;
        if (hw % nthreads != 0) {
            n++;
        }
        int start = 0;
        std::vector<std::thread> threads(nthreads);
        for (int i = 0; i < nthreads; i++) {
            threads[i] = std::thread(&Scene::paintThread, this, 
            std::ref(colorBuffer), start, n);
            start += n;
        }
        for (int i = 0; i < nthreads; i++)
            threads[i].join();
    }
    for (int y = 0; y < height; y++) {
        for (int x = 0; x < width; x++) {
            painter.setPen(colorBuffer[y][x]);
            painter.drawPoint(x, y);
        }
    }
    return image;
}
\end{minted}
\end{code}

\newpage

\section{Тестирование}
Тестирование проводилось по методологии белого ящика, с помощью проверки корректности получаемых изображений. \textbf{Тесты пройдены успешно.}

Ни рисунках \ref{img:example1}~--~\ref{img:example4} представлены тесты для алгоритма обратной трассировки лучей.

\noindent
\begin{figure}[h!]
	\centering
    \includegraphics[width=0.7\linewidth]{example1.png}
    \caption{Обычное дерево}
    \label{img:example1}
\end{figure}

\noindent
\begin{figure}[h!]
	\centering
    \includegraphics[width=0.7\linewidth]{example2.png}
    \caption{Дерево в японском стиле}
    \label{img:example2}
\end{figure}

\newpage 

\noindent
\begin{figure}[h!]
	\centering
    \includegraphics[width=0.7\linewidth]{example3.png}
    \caption{Дерево в северном стиле}
    \label{img:example3}
\end{figure}

\noindent
\begin{figure}[h!]
	\centering
    \includegraphics[width=0.7\linewidth]{example4.png}
    \caption{Небольшой лес}
    \label{img:example4}
\end{figure}


\newpage