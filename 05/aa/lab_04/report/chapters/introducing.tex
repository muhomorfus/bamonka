\setcounter{page}{3}
\chapter*{Введение}
\addcontentsline{toc}{chapter}{Введение} 

Поток~---~наименьшая единица выполнения программы. В рамках процесса может параллельно выполняться несколько потоков \cite{bib:1}. Потоки разделяют адресное пространство процесса. Порядок выполнение потоков зависит от количества ядер процессора, так как параллельно может выполняться количество задач не большее количества ядер процессора. Если потоков больше, то они выполняются квазипараллельно, то есть потоки поочередно выполняются на ядрах процессора.

Использование потоков позволяет увеличить производительность задач, требующих большого объема вычислений, так как часть задач выполняется параллельно.

\section*{Цель работы}

Получение навыков кодирования программного продукта, тестирования и проведения замерного эксперимента работы программы на различных данных. Все это на примере решения задачи реализации алгоритма обратной трассировки лучей в виде однопоточной и многопоточной версий.

\section*{Задачи работы}

\begin{enumerate}[label={\arabic*)}]
	\item изучение алгоритма обратной трассировки лучей;
	\item разработка многопоточной версии алгоритма обратной трассировки лучей;
	\item кодирование данных алгоритмов;
	\item проведение замерного эксперимента для данных алгоритмов, с измерением времени работы; 
	\item проведение сравнительного анализа алгоритмов на основе полученных данных.
\end{enumerate}

\newpage