\chapter*{Заключение}
\addcontentsline{toc}{chapter}{Заключение} 

На основе исследования можно сделать следующие выводы. 
Скорость работы алгоритма обратной трассировки лучей повышается с увеличением количества потоков. 
Причем заметное увеличение скорости работы происходит при повышении количества потоков до $100$, что близко к количеству логических ядер процессора ($96$). 
При дальнейшем повышении количества потоков, время работы алгоритма почти не изменяется.

При количестве потоков, равному $100$, алгоритм работает быстрее однопоточной реализации в $26.7$ раз.

При количестве потоков, равному $320$, алгоритм начинает работать в $1.05$ раз медленнее, чем при $290$ потоках, так как начинают быть заметными затраты на диспетчеризацию большого числа потоков.

При создании единственного дополнительного потока, алгоритм работает дольше в $1.02$ раза, чем без создания потоков, так как распараллеливания алгоритма не происходит, при этом есть затраты на диспетчеризацию.

Цель работы была достигнута: была изучена многопоточная реализация алгоритма обратной трассировки лучей. Были выполнены все задачи:

\begin{itemize}
	\item изучены алгоритмы обратной трассировки лучей;
	\item разработана многопоточная версия алгоритма обратной трассировки лучей;
	\item кодированы данные алгоритмы;
	\item проведен замерный эксперимент для данных алгоритмов, с измерением времени работы; 
	\item проведен сравнительный анализа алгоритмов на основе полученных данных.
\end{itemize}


\newpage