\chapter*{ВВЕДЕНИЕ}

Методология – это учение о методах познания, теория метода. Поэтому методологию естественно трактовать как ту часть теории познания, в которой рассматриваются общие способы исследования бытия и различные методики добывания знания. Методология есть научное мировоззрение, то есть мировоззрение осознанное и осмысленное человеком, структурированное в рамках тех или иных парадигм согласно определенным правилам.

Методология науки (науки в целом) относится к философии науки, занимается методами научного познания. Методология конкретной области науки должна учитывать (наряду с вопросами общей методологии, скажем, законами диалектики) специфику и характер данной науки.

Методология математики изучает структуру, архитектуру и ведущие принципы математики, выясняет ее природу и статус, анализирует математические факты, квалифицирует основные понятия и идеи математики, классифицирует математические методы, исследует вопросы возникновения, развития, организации и функционирования математического знания, рассматривает формы математической деятельности и специфику математического познания.

Целью данной работы является обзор методологии математики. Для достижения поставленной цели необходимо решить следующие задачи.

\begin{enumerate}
	\item Определить основания математики.
	\item Проанализировать фундаментальные понятия и идеи математики.
	\item Рассмотреть основные методы математики.
\end{enumerate}

\chapter{Основная часть}

\section{Основания математики}

Основания математики – это важная часть самой математики, являющаяся ее обоснованием и надежной опорой. Основания математики теснейшим образом связаны с математической логикой~\cite{перминов2001философия}. Математика всегда имела тот или иной фундамент, включающий осмысление основополагающих понятий. До XIX века не возникало сомнений в непротиворечивости математики. Так, геометрия Евклида была наукой о реальном пространстве. Ее аксиомы – очевидные истины о пространстве, дедуктивные выводы – неопровержимые факты. Пространство существует, значит, не противоречива и описывающая ее геометрия (реальное пространство есть модель геометрии). Из существования природы следует непротиворечивость описывающей ее математики.

Но открытие неевклидовых геометрий поколебало безмятежность математиков. Появилась необходимость в анализе основ геометрии. В 1899 году Гильберт публикует знаменитый труд «Основания геометрии», в котором евклидова геометрия изложена строго аксиоматически, ее непротиворечивость сведена к непротиворечивости действительных чисел. Во второй половине XIX века была осуществлена перестройка основ математического анализа на базе строгой теории действительных чисел. Все это способствовало зарождению оснований математики. 

Как наука основания математики появились на свет чуть более 100 лет назад. С одной стороны, это было связано с кризисом в математике, вызванным появлением логических парадоксов. С другой стороны, в это время успешно развивался формально-логический язык, т. е. существовали предпосылки разрешения кризисной ситуации. Разные подходы к обоснованию математики, к избавлению от парадоксов и выходу из кризиса привели к следующим направлениям в основаниях математики – логицизму, формализму, интуиционизму и теоретико-множественному направлению. Охарактеризуем каждое из этих направлений.

\subsubsection{Логицизм}

Главный тезис логицизма – математика выводится из логики. Законы логики суть абсолютные истины. Значит, истинна и математика. Единственно верной считается классическая двузначная логика, которая должна быть формализована и аксиоматизирована. Создатели логицизма – Фреге, Пеано, Рассел, Уайтхед. Они стремились изложить логику строго аксиоматически и представить всю математику в логическом исчислении. Фреге опубликовал книги «Исчисление понятий» и «Основные законы арифметики». Пеано издал «Формуляр математики». Наконец, в 1913 году вышел из печати последний третий том грандиозного труда Рассела и Уайтхеда «Принципы математики», в основу которого положена очень сложная иерархическая теория типов. Однако им не удалось вывести из своих логических аксиом существование бесконечного множества. Огромной заслугой логицистов является создание начал математической логики. Недостатки программы логицизма: громоздкость изложения, нелогический характер некоторых аксиом (бесконечности, сводимости, выбора), отсутствие доказательства непротиворечивости логического исчисления (хотя конкретных противоречий не было обнаружено).

\subsubsection{Интуиционизм}

Это течение имеет ярко выраженную философско-психологическую часть. Так, математика рассматривается как определенного рода деятельность, а не как сумма математических знаний. В чисто математическом плане интуиционизм отождествляется с конструктивизмом. Основателем интуиционизма как отдельного направления в основаниях математики является Брауэр. Его последователь, голландский математик Аренд Гейтин, разработал интуиционистскую логику, одновременно с российскими математиками В. И. Гливенко и А. Н. Колмогоровым. Интуиционисты признают лишь конструктивные доказательства – доказательства, основанные на том или ином алгоритме. Для них нет теорем чистого существования. Существование – это «построяемость». По-своему трактуются и логические связки. Высказывание «Гипотеза A либо верна, либо неверна» не является истинным до тех пор, пока эту гипотезу не докажут или не опровергнут. Отрицание предложения истинно, если имеется алгоритм проверки ложности данного предложения. Поэтому не годится и метод от противного. Интуиционисты ввели понятие потенциальной бесконечности: каким бы большим ни было натуральное число, можно указать еще большее следующее за ним натуральное число. Но они не рассматривают натуральный ряд как актуальную (саму по себе) бесконечность. Таким образом, интуиционистская математика содержит всю конечную математику, но отвергает почти всю бесконечную теоретико-множественную математику. Конструктивисты внесли очень большой вклад в алгоритмическую математику. 

\subsubsection{Формализм}

Это направление возникло в работах Давида Гильберта первой трети XX века. Гильберт и его коллеги завершили создание математической логики как самостоятельной математической дисциплины. В ответ на нападки интуиционистов на канторовскую теорию множеств Гильберт выдвинул свою грандиозную программу обоснования математики. Голландский математик Лейтен Брауэр утверждал, что нельзя применять законы классической логики к бесконечным множествам, тем самым ставя под сомнение всю теорию множеств. Гильберт отвечал, что «никто не может изгнать нас из рая, созданного Кантором». Программа формализма Гильберта состояла в построении такой формальной аксиоматической системы, включающей аксиоматическую теорию множеств, чтобы финитными методами можно было доказать ее непротиворечивость и полноту. Рассуждения о формальной системе – это метаматематика. К метаматематике Гильберт предъявил требование финитизма: использовать только конструктивные методы, в частности оспариваемые интуиционистами законы логики (например, закон исключенного третьего) применять только к конечным множествам. Осуществление программы Гильберта означало бы создание универсального математического исчисления, о котором так мечтал Лейбниц. Однако теорема Геделя о неполноте (1931 год) развеяла надежды.

\subsubsection{Теоретико-множественное направление}

Теоретико-множественное направление в основаниях математики – это основания теории множеств. Созданная Кантором в последней трети XIX века теория множеств была аксиоматизирована Цермело в 1908 году, что позволило избежать противоречий «наивной» теории множеств. В дальнейшем появилось много различных аксиоматик теории множеств, содержащих или не содержащих аксиому выбора, гипотезу континуума, аксиому Улама (о существовании измеримых мощностей), аксиому детерминированности и т. п. Ту часть аксиоматической теории, которая содержится в любой аксиоматике теории множеств, можно назвать абсолютной теорией множеств. Непротиворечивость абсолютной теории множеств доказать невозможно (в рамках самой теории). Однако абсолютная теория множеств принимается всеми представителями данного направления. 

Можно строить математику на базе только абсолютной теории множеств. Но без аксиомы выбора мы потеряем большую часть классического математического анализа, геометрии, алгебры и т. д. А с аксиомой выбора можно получить результаты, плохо согласующиеся с нашей интуицией, например «парадокс» Тарского: шар можно разбить на четыре части так, что, перемещая эти части в пространстве, можно собрать из них два точно таких же шара. Возникает вопрос, не приводит ли присоединение аксиомы выбора к абсолютной теории множеств к противоречию. Как было показано Геделем и американским математиком Полом Коэном~\cite{ph281}, добавление аксиомы выбора или ее отрицания к абсолютной теории множеств не ведет к противоречию (при условии непротиворечивости самой абсолютной теории множеств). Они же доказали, что добавление гипотезы континуума либо ее отрицания к аксиомастике Цермело–Френкеля не приводит к противоречию. 

Со статьи Л. Заде 1965 года берет свое начало теория нечетких множеств~\cite{ph280}, получившая существенное развитие и имеющая разнообразные применения в дискретной математике, информатике и теории принятия решений. В теории нечетких множеств принадлежность элемента данному множеству определяется с вероятностью $p$, не обязательно равной 0 или 1, как это имеет место в обычной теории множеств.

\section{Фундаментальные понятия математики}

Каждая наука имеет свой понятийный аппарат, свои ключевые идеи, методы и конструкции. Исходные интуитивные понятия математики – число и геометрическая фигура. В конце XIX века они были аксиоматизированы. В настоящее время можно выделить два ведущих понятия: функция и доказательство. Понятие функции отражает в математике категорию движения, а доказательство – причинно-следственные связи. Функции выражают содержание математики, а доказательства – ее логику~\cite{ахмедов2021определение}.

По типу изучаемых функций можно провести классификацию математических дисциплин. Современная алгебра изучает алгебраические операции. Классический математический анализ исследует числовые функции и такие их важнейшие свойства, как непрерывность, дифференцируемость, интегрируемость, измеримость. Геометрия изучает геометрические преобразования. Топология изучает непрерывные отображения. Теория упорядоченных структур – изотонные (сохраняющие порядок) отображения. В теории категорий понятию функции соответствует обобщенное понятие морфизма.

В теоретико-множественной (классической) математике первичными неопределяемыми понятиями служат множество и отношение принадлежности элемента множеству. Через них определяется и понятие функции. С понятием функции тесно связаны следующие важные понятия: образ, прообраз, частичная функция, отображение, инъективность, сюръективность, биективность. Но основным является понятие композиции, или суперпозиции функций, - это последовательное выполнение функций. В терминах (частичной) бинарной операции композиции отображений можно выразить многие свойства данного отображения, в частности его инъективность, сюръективность и биективность.

Фундаментальное понятие доказательства выражает суть дедуктивного характера математики. По Бурбаки, «математика – это доказательство». Строгое определение доказательства дает математическая логика, существует специальный раздел математической логики – теория доказательств (у Аристотеля – система дедукции).

\textbf{Фундаментальное понятие группы} формировалось более 100 лет. Неформально группу подстановок корней многочленов рассматривал еще Лагранж в 1771 году при попытке решения проблемы о разрешимости алгебраических уравнений в радикалах. Далее эти исследования были развиты итальянским математиком Паоло Руффини, норвежцем Нильсом Абелем и французом Эваристом Галуа. Именно Галуа в 1830 году ввел термин «группа», хотя и не дал строгого определения.

Понятие группы формализует общенаучное понятие симметрии. С каждой фигурой обычного трехмерного пространства ассоциирована группа всех ее самосовмещений, т. е. движений пространства, рассматриваемых с операцией композиции, при которых фигура отображается сама на себя. Чем богаче группа самосовмещений фигуры, чем большую мощность она имеет, тем совершеннее, симметричнее сама фигура. Так, квадрат симметричнее отрезка, а окружность симметричнее любого правильного многоугольника. 

\textbf{Топологические пространства} тесно связаны с общим понятием предела и являются частью топологии, непрерывной математики. Центральной идей, формирующей топологию как математическую дисциплину, служит идея непрерывности.

Общее определение топологического пространства было дано в начале 20-х годов XX века российским математиком П. С. Александровым (через семейство открытых множеств) и поляком Казимиром Куратовским (через оператор замыкания). П. С. Александров и П. С. Урысон являются основателями всемирно известной российской топологической школы. К основным топологическим понятиям относятся компактность, связность, так называемые аксиомы отделимости и счетности, непрерывное отображение и гомеоморфизм, размерность и ряд других.

Переходим к \textbf{порядковой структуре}. В теории упорядоченных множеств воплощается в чистом виде идея сравнения объектов по величине. Понятия отношения порядка и упорядоченного множества формулируются столь же естественно и просто, как и метрика или топологическое пространство. Бинарное отношение на множестве называют порядком, если оно рефлексивно (сравнимость с самим собой), транзитивно («транзитом» через второй элемент) и антисимметрично (двойное неравенство есть равенство). Множество с заданным на нем порядком называется упорядоченным множеством.

Основными понятиями, связанными с порядком, являются: наибольший и наименьший элементы, максимальный и минимальный элементы, точные грани, сравнимость, линейность (цепь), сечение, вполне упорядоченное множество, решетка, полная решетка, изотонное отображение, дополнение, дистрибутивная решетка, булева алгебра, упорядоченные группа, кольцо и поле.

\textbf{Конечные упорядоченные множества}, как и конечные графы, играют важную роль в дискретной математике. Существует наглядное и продуктивное изображение конечных упорядоченных множеств диаграммами Хассе. Важно отметить, что имеются естественные взаимно однозначные соответствия между конечными упорядоченными множествами, конечными дистрибутивными решетками (их можно рассматривать чисто алгебраически) и конечными T0-пространствами. Это показывает, что в конечном случае имеется единство (взаимоопределяемость) порядковой, алгебраической и топологической структур.

В математике существует порядковый подход (идея упорядоченности), при котором акцентируется внимание на порядковой структуре изучаемых объектов, на возможности их упорядочивания, и свойства объектов выражаются в терминах отношения порядка. Отметим и теоретико-решеточный способ (метод) мышления, когда математический объект исследуется с помощью решетки его подобъектов.

Остановимся также на фундаментальном понятии \textbf{алгоритма}. На обычном языке алгоритмом называется четко прописанная процедура действий, приводящая к результату через конечное число шагов~\cite{дьяченко1972основные}. Это интуитивное описание алгоритма. Любой алгоритм применим к целому классу задач. Вспомним алгоритм Евклида нахождения НОД двух любых целых чисел или метод Гаусса решения произвольных систем линейных уравнений. Не имея строгого определения алгоритма, нельзя доказать алгоритмическую неразрешимость (отсутствие соответствующего алгоритма) математической проблемы. 

В современной математике часто встречается понятие \textbf{многообразия}, имеющего в разных дисциплинах различный смысл: многообразие универсальных алгебр; многообразие в теории моделей; топологическое, аналитическое и дифференцируемое многообразия. Многообразием универсальных алгебр называется класс всех алгебр одной и той же сигнатуры, удовлетворяющих некоторой фиксированной системе тождеств. Отметим многообразия полугрупп, групп, абелевых групп, колец, решеток.

\section{Основополагающие идеи математики}

\textbf{Координатизация} заключается в сопоставлении изучаемым реальным или математическим объектам чисел, систем чисел или «числоподобных» объектов (координат исследуемых объектов), над которыми можно выполнять некоторые операции, производить вычисления, позволяющие получать информацию об исходных объектах. В известной мере эта идея характеризует саму математику и место алгебры в ней. Примеры координатизации дают пересчет и счет предметов; различные величины – длина отрезка, площадь фигуры, объем и вес реального тела, величина угла; тригонометрические функции как функции углов; числовая прямая; метод координат в аналитической геометрии; полярные и сферические координаты точки; алгебраическая геометрия и алгебраическая топология; конечные геометрии; группа автоморфизмов (симметрии) математического объекта.

С идеей координатизации тесно связана \textbf{идея арифметизации} математики. Она состоит в сведении той или иной математической дисциплины или даже всей классической математики к числовым системам. Построение системы действительных чисел на базе рациональных чисел и ее аксиоматизация (это непрерывное линейно упорядоченное поле) сделали действительные числа сугубо арифметическим (алгебраическим) объектом. Евклидова геометрия сведена Гильбертом к действительным числам~\cite{вечтомов2013философия}.

\textbf{Идея линейности.} Эта идея находит свое пристанище в линейной алгебре: системы линейных уравнений, матрицы, векторные пространства и линейные многообразия, линейные отображения. В дифференциальном исчислении и в дифференциальной геометрии идея линейности предполагает спрямление, или линеаризацию, кривых и поверхностей в малом с целью упрощения ситуации и возможности применения аппарата линейной алгебры.

\textbf{Понятие и идея изоморфизма}. Изоморфизм (или изоморфность) – одно из основополагающих понятий современной математики. Два однотипных математических объекта (или структуры) называются изоморфными, если существует взаимно однозначное отображение одного из них на другой, такое, что оно и обратное к нему сохраняют строение объектов, т. е. элементы, находящиеся в некотором отношении, переводятся в элементы, находящиеся в соответствующем отношении~\cite{макаридина2022изучение}. Однотипность математических объектов означает, что они имеют одинаковую сигнатуру, или набор отношений. 

Изоморфные объекты могут иметь различную природу элементов и отношений между ними, но они совершенно одинаково абстрактно устроены, служат копиями друг друга. Изоморфизм представляет собой «абстрактное равенство» однотипных объектов. 

\textbf{Идея предельного перехода} витала в воздухе со времен Архимеда. Ньютон и Лейбниц положили ее в основу дифференциального и интегрального исчисления – в виде понятия бесконечно малой величины. Коши строил математический анализ на основе теории пределов. Однако строгое обоснование идея предельного перехода получила вместе со строгим определением предела функции (Больцано, Вейерштрасс). 

\textbf{Идея двойственности.} Двойственность (дуализм) – это эквивалентность или антиэквивалентность категорий соответствующих объектов. Существует много различных теорий двойственности в алгебраической геометрии.

\textbf{Идея аппроксимации.} Аппроксимация (приближение) означает замену одних математических объектов другими, более просто устроенными объектами, например приближение иррациональных чисел рациональными.

\textbf{Идея интерполяции.} Интерполяция (восстановление) означает точное или приближенное нахождение функции по отдельным известным ее значениям.

Кроме того, существуют фундаментальные идеи \textbf{интегрирования}, \textbf{симметрии}, \textbf{непрерывности} и \textbf{упорядоченности}.

Наряду с понятиями и идеями математика имеет свои методы установления истины~\cite{вечтомов2013философия}.

\section{Методы математики}

Методы исследования в математике можно разделить на два класса: логические и сугубо математические. К логическим методам относятся метод от противного, контрапозиция, перебор случаев (полная индукция), исключение случаев, правило силлогизма, метод минимального контрпримера, метод математической индукции и т. д. Рассмотрим некоторые важнейшие общематематические методы~\cite{александров1956общий}. Заметим, что зачастую между методами и идеями существует неразрывная связь. Например, идея координатизации в аналитической геометрии и метод координат или идея функционального представления и функциональный метод~\cite{вечтомов2013философия}.

\textbf{Функциональный метод.} Он состоит в представлении данного математического объекта в виде «функционального» объекта, его элементами служат функции, над которыми естественным образом выполняются некоторые операции~\cite{коллатц1969функциональный}. Особенно плодотворно функциональный метод применяется в современной алгебре и абстрактном функциональном анализе.

Вспомним представление Кэли групп группами подстановок, представление Стоуна булевых алгебр и булевых колец (данное американским математиком Маршаллом Стоуном) и преобразование Гельфанда для коммутативных банаховых алгебр. В первом из этих функциональных представлений групповая операция интерпретируется как композиция подстановок, а в двух других представлениях элементы алгебры «изображаются» непрерывными отображениями на пространстве максимальных идеалов этой алгебры со значениями, соответственно, в дискретной двухэлементной цепи или в поле комплексных чисел, над которыми алгебраические операции выполняются поточечно.

В случае абстрактных колец (или полуколец) возможны различные их функциональные представления как колец сечений соответствующих пучков, т. е. пучковые представления. В кольце сечений, сопоставляемом исследуемому кольцу, операции выполняются поточечно. В каждой точке базисного пространства сечения принимают значения в соответствующем слое – кольце. Слои, вообще говоря, не изоморфны между собой, но для успешных применений они должны быть устроены проще исходного кольца.

Функциональный подход позволяет более наглядно представлять элементы абстрактного объекта и операции над ними. Применительно, скажем, к кольцам он обосновывает необходимость изучения колец сечений пучков колец, а также их прообраза – колец непрерывных функций на топологических пространствах со значениями в том или ином топологическом кольце. 

\textbf{Метод координат.} Осуществляет идею координатизации путем введения координат на плоскости или в пространстве. При этом не только точки получают свои числовые координаты, но и кривые и поверхности приобретают алгебраическую форму – они описываются уравнениями. Приводя алгебраические уравнения к каноническому виду, математики классифицируют различные кривые и поверхности, получают новую информацию о них.

Рассмотрим пример задачи нахождения уравнения прямой в двумерной декартовой системе координат. Используем формулу прямой через точку и угол наклона.


Мы знаем точку, через которую проходит прямая: $A(2, 3)$, и угол наклона прямой к оси $x$: $\alpha = 45^\circ$. Требуется найти уравнение этой прямой.

\textbf{Решение}

1. Уравнение прямой в общем виде при известной точке и угле наклона записывается так:
\[
y - y_1 = k(x - x_1),
\]
где:
\begin{itemize}
    \item $k = \tan(\alpha)$ — коэффициент наклона,
    \item $(x_1, y_1)$ — координаты заданной точки.
\end{itemize}

2. Подставляем данные задачи:
\begin{itemize}
    \item $\tan(45^\circ) = 1$,
    \item $x_1 = 2$, $y_1 = 3$.
\end{itemize}

Уравнение преобразуется в:
\[
y - 3 = 1 \cdot (x - 2).
\]

3. Раскрываем скобки и приводим к стандартному виду:
\[
y = x - 2 + 3,
\]
что даёт:
\[
y = x + 1.
\]

График прямой $y = x + 1$ проходит через точку $(2, 3)$ и поднимается на $1$ единицу при увеличении $x$ на каждую единицу, так как угол наклона $k = 1$.

Прямая в декартовой системе координат задаётся уравнением:
\[
y = x + 1.
\]

Этот пример показывает, как метод координат упрощает описание геометрических объектов (прямых, окружностей, кривых) на плоскости.

В широком смысле метод координат заключается в построении фунтора A из данной категории геометрических или топологических объектов X в некоторую категорию алгебраических объектов A(X). Выше был указан функтор X→C(X). В алгебраической топологии основополагающее значение имеют группы гомотопий и гомологий топологических многообразий.


\textbf{Теоретико-модельный метод.}  Успешно применяется в различных областях математики, включая:

\begin{itemize}
    \item абстрактную алгебру,
    \item теорию числовых систем,
    \item упорядоченные структуры,
    \item основания геометрии,
    \item нестандартный анализ.
\end{itemize}

Применительно к современной алгебре теория моделей зачастую называется \textit{метаматематикой алгебры}. Этот метод позволяет исследовать математические структуры через построение формальных моделей и их интерпретаций.

Огромный вклад в создание и развитие теории моделей внесли следующие учёные:
\begin{itemize}
    \item А. Тарский,
    \item А. И. Мальцев,
    \item А. Робинсон,
    \item П. С. Новиков,
    \item Ю. Л. Ершов.
\end{itemize}

Особо стоит отметить основателя нестандартного анализа Абрахама Робинсона, который формализовал и обосновал лейбницевское понятие бесконечно малой величины. Он доказал существование неархимедовых расширений поля действительных чисел, что стало важным шагом в развитии математического анализа.

Теоретико-модельный метод играет ключевую роль в современных исследованиях. Его универсальность проявляется в способности формальных моделей описывать и анализировать сложные математические структуры, таким образом расширяя понимание фундаментальных концепций.

Основная черта \textbf{метаматематического подхода} состоит в следующем~\cite{матиясевич1975метаматематическом}. Пусть мы изучаем некоторое свойство R конечных дискретных объектов какого-либо типа. Первый этап метаматематического подхода состоит в формализации свойства R: мы должны попытаться найти некоторую дедуктивную систему R и связать с каждым объектом X обладающим свойством R, новый конечный дискретный объект - формальное доказательство того, что X обладает свойством R, проведенное в рамках системы. Дальнейшее изучение свойства R включает в себя рассмотрение наряду с X также и доказательства и анализ структур обоих объектов. Именно введение в рассмотрение формальных доказательств и является основной характеристической чертой метаматематического подхода. 

\textbf{Алгоритмические методы} основаны на применении того или иного алгоритма. \textbf{Вычислительные, или численные, методы} связаны с приближенными вычислениями: методы хорд и касательных, метод Штурма, метод половинного деления, итерация, методы аппроксимации и интерполяции и т. п. При решении геометрических задач весьма полезен \textbf{метод геометрических преобразований}. В геометрии и в топологии эффективно применяется \textbf{метод триангуляции} – разбиение поверхности на треугольники.

\textbf{Диагональный канторовский метод.} Применялся Кантором для доказательства несчетности числовых и функциональных множеств.

Покажем его работу на примере доказательства несчетности числового промежутка \([0, 1)\). Предположим, что это множество счетно, т. е. его элементы можно занумеровать натуральными числами: \( r_1, r_2, \ldots, r_n, \ldots \). Целая часть этих чисел равна 0. Пусть \( a_{ij} \) – \( j \)-й десятичный знак (после запятой) числа \( r_i \). Расположим десятичные записи чисел \( r_1, r_2, \ldots, r_n, \ldots \) друг под другом в виде бесконечной матрицы и возьмем ее диагональ \( a_{11}, a_{22}, \ldots, a_{nn}, \ldots \). Рассмотрим число 

\[
r = 0.a_1 a_2 \ldots a_n \ldots,
\] 

в котором для любого натурального \( n \) цифра \( a_n \) отлична от \( a_{nn} \) и \( a_1 \neq 9 \). Тогда \( 0 \leq r < 1 \) и \( r \) отлично от всех занумерованных чисел. Полученное противоречие и доказывает несчетность континуума. Заметим, что этим методом доказывается и существование алгоритмически невычислимой функции.


\textbf{Метод нумерации в математической логике.} Все слова логикоматематического языка нумеруются натуральными числами. Тем самым любое рассуждение о формальной математической теории (метарассуждение) становится утверждением о натуральных числах – номерах формул, содержащихся в рассуждении. Впервые метод арифметизации применил Курт Гедель в 1931 году для доказательства неполноты формальной арифметики.



%\chapter{Предмет и объект математики}

%Задача определения предмета математики активно обсуждается как математиками, так и философами, начиная с Античности (пифагорейцы, Платон, Аристотель). С созданием арифметики возникла проблема числа, впервые поставленная пифагорейцами. В изложении Гегеля она сформулирована следующим образом: «Где находятся числа? Обитают ли они отдельно в небе идей, отделенные от всего другого пространством? Они не суть непосредственно сами вещи, ибо вещь, субстанция, отнюдь не является числом; тело не имеет с ним никакого свойства»~\cite{gegel}. Здесь содержатся два различных, но взаимосвязанных вопроса: что такое число, и как оно существует. Представляет интерес парадоксальное высказывание о сущности математики Б. Рассела: «Математика может быть определена как доктрина, о которой мы никогда не знаем ни того, о чем мы говорим, ни того, верно ли то, что мы говорим»~\cite{rassel}. Правомерно поставить вопрос: существует ли нечто, о чем математика дает нам знание? Другими словами, что составляет объект математики? Если в классическом естествознании на эмпирическом уровне ответ на поставленный вопрос не вызывал затруднений, то в современном естествознании ситуация существенно иная. Формулируемые на теоретическом уровне гипотезы описывают не реальные предметы и процессы, а их идеальные модели. 
%
%В работах, посвященных философии математики, проблеме предмета математики уделяется много внимания, гораздо больше, чем подобной проблеме в других отраслях науки. Это обусловлено спецификой математики. В отличие от естественных наук, для которых можно с большей или меньшей точностью установить область изучаемых явлений, для математики эта задача трудно разрешима вследствие ее особого характера: высокой степени абстрактности ее понятий и не только. Главным обстоятельством является то, что математические объекты (число, фигура, функция и др.)~---~результат идеализации, мысленного конструирования. Если хотя бы некоторым понятиям естественных наук можно поставить в соответствие вещи, свойства или отношения действительного мира, то связь математических понятий с действительностью многократно опосредствована. Здесь встает со всей определенностью проблема: имеет ли математика специфический объект и предмет исследования или нет, является она отражением какого-либо фрагмента, свойства объективной реальности или же она представляет собой только знаковую систему, особый язык, удобный для выражения мыслей?
%
%%В отечественной литературе по философии математики проблема предмета математики интенсивно обсуждается [5]. Их авторы опираются на положение, содержащееся в книге Ф. Энгельса «Анти-Дюринг»: «...чистая математика имеет своим объектом пространственные формы и количественные отношения действительного мира, стало быть~---~весьма реальный материал. Тот факт, что этот материал принимает чрезвычайно абстрактную форму, может лишь слабо затушевать его происхождение из внешнего мира...» [6]. Данное высказывание различными авторами оценивается либо как классическое определение математики, либо как определение предмета математики или одной из его сторон. Итак, необходимо различать две вещи: объект математики и ее предмет. Под объектом математики, как и любой другой науки, в соответствии с категориальным смыслом этого философского термина (объект~---~это есть нечто, противостоящее субъекту деятельности) следует понимать те вещи, свойства и отношения реальности, которые находят в науке свое выражение. Для Платона объектом геометрии было пространство как особая реальность, расположенная между миром идей и миром вещей (пространство как «кормилица, восприемница идей»). В вышеприведенных словах Энгельса речь идет об определении объекта математики, т.~е. о выяснении свойств и отношений действительного мира, которые получили в ней теоретическое выражение.
%
%С расширением понятийного аппарата математики, ее познавательных средств соответственно расширяется и объект математики, т.~е. множество тех явлений, которые она может описывать. Например, с созданием тригонометрии стало возможным производить измерения таких расстояний (в геодезии и астрономии), которые средствами обычной геометрии были неосуществимы. Теория вероятностей дала метод количественной оценки случайных событий. Поэтому установить объект математики на каждом этапе ее развития~---~дело непростое. Задача определения объекта математики имеет два смысла: определение области действительности (вещи, процессы, свойства, отношения), послужившей прообразом для формирования новой теории, и определение области приложения данной теории. Первая задача~---~дело исторической науки, которая должна произвести реконструкцию форм культуры, обусловивших появление тех или иных математических понятий и методов. Ее решение зависит от наличия историко-культурного материала, его осмысления и систематизации. Здесь только не следует впадать в крайности эмпиризма, требующего для каждого математического понятия указывать его непосредственный аналог или прообраз. Дело в том, что новые понятия часто вводятся как обобщение уже существующих (например, рациональные, отрицательные, комплексные, гиперкомплексные числа), а новые теории или исчисления создаются на основе других математических теорий (например, теория групп как развитие классической алгебры или неевклидовы геометрии как результат замены пятого постулата евклидовой геометрии на другое суждение). Однако и в Новое время, а не только в глубокой древности, многие математические понятия, теории, исчисления первоначально возникают как прикладные для решения задач механики, физики и других наук, техники, экономики и лишь затем приобретают статус чисто математических (например, дифференциальное и интегральное исчисление, теория вероятностей, математическая статистика, теория информации и др.). Будучи абстрагированы из одной области явлений, они затем приобретают всеобщее значение, становятся универсальным аппаратом, техническим средством решения задач из многих других областей, как чисто теоретических, так и прикладных.
%
%Можно ли считать достаточным для определения науки указания на объект ее изучения?
%
%Если ограничиться определением математики как науки о пространственных формах и количественных отношениях действительного мира, то будет ли такая её характеристика определением её сущности? Можно ли на основе такого определения отличить математику от других наук? Что изучают астрономия, физика, химия и другие естественные науки? Всякая естественная наука, достигшая зрелости, изучает то же самое. Наряду с утверждениями качественного, содержательного характера они включают различные формулы, уравнения, схемы. Так, химия дает знания о количественном составе веществ, о пространственных конфигурациях молекул; механика~---~о пространственных свойствах движения (кинематика), о количественных отношениях между силами, массами, ускорениями и т.~п. (динамика), кристаллография изучает типы пространственной симметрии кристаллов. Примеры можно приводить неограниченно. Но что из них следует? Все названные науки являются разделами математики? Вряд ли этот вывод правомерен и найдёт сторонников за пределами сообщества математиков, да и среди математиков таких встретится немного. Например, Дж.~Кемени утверждал: «Я хочу доказать, что каждая наука есть прикладная математика»~\cite{object7}. Но тогда возможен противоположный вывод: математики как особой науки вообще не существует. Поскольку объект её познания оказался занятым другими науками, она, подобно королю Лиру, разделившему королевство между дочерьми, оказывается лишней. Этот вывод оказывается ещё более экстравагантным на фоне величественного здания математики и огромной армии математиков, которые в таком случае оказываются как бы лишними. Следовательно, истина находится глубже. Значит, науки характеризуются не только объектом познания, не только тем, что они изучают, но и тем, как они подходят к объекту познания, каким способом они его осваивают. Это различие как раз и выражается в предмете науки. Что такое предмет математики?
%
%%Прежде чем говорить о предмете математики по существу, посмотрим, как он трактуется в литературе. Академик А.~Д.~Александров дал такое его определение: «В общем, в предмет математики могут входить любые формы и отношения действительности, которые объективно обладают такой степенью независимости от содержания, что могут быть от него полностью отвлечены. ...непосредственным предметом математики оказываются: числа, а не совокупности предметов, геометрические фигуры, а не реальные тела и т. п.»~\cite{object8}. В этих высказываниях авторитетного математика один и тот же термин~---~предмет математики~---~употребляется в различных значениях: для обозначения форм и отношений объективной действительности и для обозначения продуктов абстрагирующей деятельности людей (числа, фигуры и другие математические понятия). «В качестве же своих объектов она рассматривает пространственные формы и количественные отношения действительности, точнее, идеализированные объекты, начиная с натурального чисда и фигуры и кончая абстрактными фигурами»~\cite{object9}. Так что такое, с точки зрения автора этих строк Н.~И.~Жукова, математический объект: объективные отношения или абстракции этих отношений? Примеры подобных рассуждений можно найти и в других работах. Причина неясных рассуждений о предмете математики видится в том, что, сознавая отличие математики (в этом пункте) от естественных наук, некоторые авторы не нашли адекватной логической формы для выражения этого отличия. Проблема решается введением понятий предмета и объекта математики.
%
%Чтобы дать ответ на вопрос о том, что составляет предмет математики, необходимо рассмотреть предметы её отдельных частей: арифметики, геометрии, алгебры, математического анализа и др. Обобщая полученный материал, предмет математики можно представить в виде иерархии систем абстрактных идеализированных объектов, выступающих в единстве их идеального содержания и материальной, знаковой формы (числовые системы, системы функций, геометрических фигур и пр.). Они образуют своего рода «математическую реальность», с которой имеет дело математик в своей работе. Первые этажи этой иерархии являются абстракцией и идеализацией некоторых реальных свойств и отношений действительности, вышележащие~---~оказываются абстракциями второго, третьего и т.~д. порядков.
%
%Введение понятий объекта и предмета математики позволяет провести чёткое разграничение между формами и отношениями объективной реальности, которые потенциально могут получить математическое описание, и абстракциями и идеализациями этих форм и отношений, для которых математика выработала специальные понятийные и языковые средства выражения. Это делает возможным видеть как сходства «математической реальности» и объективной реальности, так и их принципиальное различие.
%
%К чему ведёт отождествление понятий объекта и предмета математики, т.~е. употребление терминов «объект математики» и «предмет математики» как синонимов? В логическом отношении некорректно использовать один термин для выражения двух понятий. С философской точки зрения, такое неразличение объекта и предмета науки приводило к наивно-созерцательному взгляду на математику, и, как следствие, к догматизму, к кризисам в её развитии. 

%Что тормозило развитие понятия числа в истории математики? Почему математики различных эпох с удивительным упорством повторяли одну и ту же ошибку, не желая признавать равноправия с другими вначале дробных, позднее~---~отрицательных, иррациональных, мнимых чисел? Это происходило потому, что символические выражения этих чисел не имели смысла в системе тех понятий, которые использовались в соответствующую эпоху. Для древнего грека, не являвшегося математиком, числом было только натуральное число, ибо оно количествен-но характеризовало некоторую совокупность вещей. Обыкновенная дробь вида $\frac{1}{\pi}$~---~это не число, а оперативный символ, указывающий на то действие, которое надлежало произвести с некоторой величиной, т.~е. разделить ее на $\pi$ частей. Даже Платон говорил, что математики умножают там, где надо делить. «Отрицательное число» стало числом, когда ему нашли истолкование в торговых расчётах как меры долга, убытка и т.~п. С «мнимыми числами» мирились почти триста лет только потому, что они, встречаясь в промежуточных выкладках при решении кубических уравнений, не входили в конечный результат. Чисто прагматический подход к употреблению таких чисел можно видеть в словах французского математика Л.~Карно: «В алгебре вводят в вычисления чисто мнимые понятия, фиктивные сущности, которые не могут существовать, ни даже быть понятыми и которые не теряют, однако, от этого своей полезности. Их употребляют вспомогательным образом, как термины сравнения для облегчения сопоставления истинных количеств... и затем их исключают посредством преобразований, представляющих, так сказать, чисто механическую работу»~\cite{object10}. В этом высказывании характерно противопоставление «истинных количеств» «фиктивным сущностям» и «мнимым понятиям» (каковыми являются отрицательные и мнимые числа, бесконечно малые и бесконечно большие величины). Деление математических объектов на два сорта (истинные и фиктивные) является результатом эмпирического взгляда на математику, согласно которому «истинный» математический объект должен иметь содержательное истолкование, понятный смысл, т.~е., в конечном итоге, он должен выражать некоторое свойство реальности, иметь непосредственный аналог. Понадобились столетия, чтобы в сознание подавляющей части математиков вошло новое понимание предмета математики как науки, изучающей особую реальность, сконструированную самими учёными. Объекты этой реальности образуют множество классов с особыми свойствами, однако здесь уже нет деления их на истинные, т.~е. настоящие, полноценные, и фиктивные понятия. Фиктивность некоторых величин оказалась временной; они получили интерпретации в других понятиях, уже имеющих полноценный статус в математике, и поэтому стали вполне равноправными. Например, комплексным числам была дана геометрическая интерпретация; проблема бесконечно малой и бесконечно большой разрешилась пониманием того, что это не постоянные, а переменные величины, имеющие пределом нуль и бесконечность соответственно. Единственное, что их отличает,~---~это их происхождение: одни являются абстракциями и идеализациями низшего, а другие~---~высшего порядка, одни, с точки зрения их происхождения, стоят ближе к объективному миру, другие~---~дальше от него.
%
%\chapter{Понятие методологии математики}





\chapter*{ВЫВОД}

В рамках данной работы был проведен анализ методологии математики. Рассмотрены основания математики, выделены основные направления оснований математики, такие как:

\begin{itemize}
	\item логицизм;
	\item интуиционизм;
	\item формализм;
	\item теоретики-множественное направление.
\end{itemize}

Были рассмотрены основные понятия математики:

\begin{itemize}
	\item группа;
	\item топологическое пространство;
	\item порядковая структура;
	\item конечное упорядоченное множество;
	\item алгоритм;
	\item многообразие.
\end{itemize}

Проанализированы основные идеи и методы математики, показано, что существует прямая связь между некоторыми методами и идеями математики, например, идеи координатизации и метода координат.
