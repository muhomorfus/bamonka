\chapter{Конструкторский раздел}

Для проектирования метода ограничения дискового ввода-вывода в системе оркестрации Kubernetes необходимо определить следующее:

\begin{itemize}
	\item способ непосредственно применения ограничений дискового ввода-вывода;
	\item способ хранения данных об ограничения дискового ввода-вывода для приложения с использованием абстракций Kubernetes;
	\item алгоритм обработки структур данных, хранящих ограничения дискового ввода-вывода.
\end{itemize}

\section{Способ применения ограничений дискового ввода-вывода}

Для применения ограничений разделяемых ресурсов на уровне процессов в ядре Linux существует механизм cgroup v2~\cite{cgroupv2}. Процессы добавляются в так называемые «контрольные группы», для которых можно задавать ограничения на процессорное время, объем оперативной памяти, дискового ввода-вывода и другие. Для ограничения дискового ввода-вывода в cgroup v2 работает отдельный контроллер ввода-вывода.

Для взаимодействия с пользователем системы, ядро предоставляет интерфейс для cgroup v2 в формате виртуальной файловой системы~\cite{tabatabai2023fbmm}. В дереве ФС для каждой контрольной группы создается каталог, в котором находятся интерфейсные файлы контроллеров. Пример содержания каталога для контрольной группы представлен в Приложении~А.

Файлы с окончанием \texttt{.max} предназначены для указания лимитов разделяемых ресурсов для заданной контрольной группы. В файле \texttt{io.max} указываются ограничения дискового ввода-вывода, строки файла имеют вид, представленный в листинге~\ref{lst:io_max}.

\begin{lstlisting}[label=lst:io_max, caption={Строка файла \texttt{io.max}, в которой описаны ограничения дискового ввода-вывода}]
8:16 rbps=2097152 wbps=max riops=max wiops=120
\end{lstlisting}

Для каждого создаваемого оркестратором Kubernetes контейнера создается отдельная контрольная группа. Пусть к интерфейсной папке контрольной группы контейнера в Kubernetes имеет вид, представленный в листинге~\ref{lst:cgroup_container}.

\begin{lstlisting}[label=lst:cgroup_container, caption={Пример пути к интерфейсной папке контрольной группы контейнера}]
/sys/fs/cgroup/kubepods.slice/kubepods-pod<POD UID>.slice/cri-containerd-<CONTAINER ID>.scope
\end{lstlisting}

В указанном примере, \texttt{<POD~UID>}~---~идентификатор пода, \texttt{<CONTAINER~ID>}~---~идентификатор контейнера.

Таким образом, для ограничения дискового ввода-вывода для контейнера, запущенного с использованием Kubernetes, необходимо записать ограничения в файл \texttt{io.max} внутри интерфейсной папки контрольной группы контейнера.

\section{Выбор способа хранения данных об ограничениях пропускной способности дискового ввода-вывода}

В Kubernetes есть несколько вариантов, в которых можно было бы разместить информацию об ограничениях дискового ввода-вывода.

\begin{itemize}
	\item \textbf{Ограничения на разделяемые ресурсы.} Kubernetes позволяет указывать запросы и ограничения на ресурсы в описании пода~\cite{resource_management}. Изначально можно задавать ограничения на процессорное время и память, но можно определять и другие типы разделяемых ресурсов. Пример описания пода с ограничениями по дисковому вводу-выводу представлен в листингах~\ref{lst:pod_limits1}--\ref{lst:pod_limits2}.

\begin{minipage}[H]{\textwidth-1.25cm}
\begin{lstlisting}[label=lst:pod_limits1, caption={Указание ограничений в описании объекта Pod}]
spec:
    containers:
    - name: cool-storage
        resources:
            request:
                cpu: 2000m
                memory: 2G
\end{lstlisting}
\end{minipage}

\begin{minipage}[H]{\textwidth-1.25cm}
\begin{lstlisting}[label=lst:pod_limits2, caption={Указание ограничений в описании объекта Pod (продолжение листинга~\ref{lst:pod_limits1})}]
                disk_read_iops: 100
                disk_write_iops: 100
                disk_read_bandwidth_mbps: 100
                disk_write_bandwidth_mbps: 100
            limits:
                disk_read_iops: 100
                disk_write_iops: 100
                disk_read_bandwidth_mbps: 100
                disk_write_bandwidth_mbps: 100
\end{lstlisting}
\end{minipage}

Касательно определений ограничения для разделяемых ресурсов, отличных от \texttt{cpu} и \texttt{memory}, Kubernetes запрещает указывать отличные друг от друга запросы и ограничения на ресурсы.

\item \textbf{Аннотации.} В метаданных каждого объекта Kubernetes имеются аннотации~---~дополнительная информация об объекте в формате ключ-значение~\cite{annotations}. Оркестратор позволяет указывать в аннотации любые пользовательские данные. Соотвественно, в аннотации пода можно указывать ограничения дискового ввода-вывода. Пример описания пода с указанными в аннотациях ограничениями приведен в листинге~\ref{lst:annotations}.

\begin{minipage}[H]{\textwidth-1.25cm}
\begin{lstlisting}[label=lst:annotations, caption={Указание ограничений в аннотациях пода}]
metadata:
    annotations:
        limit_disk_read_iops: 100
        limit_disk_write_iops: 100
        limit_disk_read_bandwidth_mbps: 100
        limit_disk_write_bandwidth_mbps: 100
\end{lstlisting}
\end{minipage}

\item \textbf{Custom Resource.} Помимо основных ресурсов Kubernetes, таких как Pod, Deployment, StatefulSet и других, разработчики могут определять свои типы ресурсов, со специфичным для ресурса набором параметров~\cite{crd}. Соответственно, ограничения дискового ввода-вывода можно указывать в Custom Resource. 
\end{itemize}

Для сравнения возможных способов хранения информации об ограничениях дискового ввода-вывода использовались особенности, которыми должен обладать разрабатываемый метод.

\begin{enumerate}
	\item Задание различных ограничений дискового ввода-вывода для доступа к томам из разных контейнеров.
	\item Изменение ограничений дискового ввода-вывода с помощью изменения существующих ресурсов.
	\item Независимость метода от конкретных деталей реализации CSI.
	\item Возможность разделять запросы и ограничения на ресурсы.
\end{enumerate}

В таблице~\ref{tab:compare_info} представлено сравнение методов.


	


\begin{table}[H]
\caption{Сравнение методов хранения данных об ограничения дискового ввода-вывода}
\label{tab:compare_info}
\begin{center}
\begin{tabular}{|l|l|l|l|}
\hline
                                                                                                                                                   & \multicolumn{1}{c|}{\textbf{\begin{tabular}[c]{@{}c@{}}Ограничения \\ на разделяемые \\ ресурсы\end{tabular}}} & \multicolumn{1}{c|}{\textbf{Аннотации}} & \multicolumn{1}{c|}{\textbf{\begin{tabular}[c]{@{}c@{}}Custom \\ Resource\end{tabular}}} \\ \hline
\begin{tabular}[c]{@{}l@{}}Задание различных \\ ограничений дискового \\ ввода-вывода для доступа \\ к томам из разных \\ контейнеров\end{tabular} & $-$                                                                                                            & $-$                                     & $+$                                                                                      \\ \hline
\begin{tabular}[c]{@{}l@{}}Изменение ограничений \\ дискового ввода-вывода \\ с помощью изменения \\ существующих ресурсов\end{tabular}            & $+$                                                                                                            & $+$                                     & $+$                                                                                      \\ \hline
\begin{tabular}[c]{@{}l@{}}Независимость метода \\ от конкретных \\ деталей реализации CSI\end{tabular}                                            & $+$                                                                                                            & $+$                                     & $+$                                                                                      \\ \hline
\begin{tabular}[c]{@{}l@{}}Возможность разделять \\ запросы и ограничения \\ на ресурсы\end{tabular}                                               & $-$                                                                                                            & $+$                                     & $+$                                                                                      \\ \hline
\end{tabular}
\end{center}
\end{table}


На основании сравнения принято решение использовать Custom Resource для хранения ограничений дискового ввода-вывода.

\section{Проектирование способа хранения данных об ограничениях пропускной способности дискового ввода-вывода}

Поды в Kubernetes предназначены для запуска группы контейнеров, составляющих единое приложение~\cite{pods}. При этом, один под характеризует один экземпляр приложения. Часто с целью сохранения отказоустойчивости или повышения уровня принимаемой нагрузки, приложения запускаются в формате нескольких копий-реплик~\cite{alelyani2024optimizing}. Для таких целей в Kubernetes есть более высокоуровневые абстракции такие как ReplicaSet~\cite{rs}, Deployment~\cite{deploy} и StatefulSet~\cite{sts}, позволяющие запускать несколько подов одного приложения с использованием общих настроек. Реплики представляют собой одинаковые копии приложений, соответственно для того, чтобы настройки для набора реплик были одинаковые, имеет смысл создать общий Custom Resource, содержащий ограничения дискового ввода-вывода для одного приложения. Ресурс представляет собой \textbf{запись}, в котором должны содержаться следующие данные.

\begin{itemize}
	\item \textbf{Название ресурса (строка)}~---~название ресурса, задающего ограничения дискового ввода-вывода, должно быть уникальным.
	\item \textbf{Идентифицирующие метки (строки)}~---~метки, с помощью которых можно идентифицировать поды, относящиеся к конкретному набору реплик приложения.
	\item \textbf{Ограничения дискового ввода-вывода для контейнеров (массив записей)}~---~массив, каждая запись которого содержит информацию об ограничениях дискового ввода-вывода для контейнера, запущенного в рамках пода. Запись содержит следующие поля.
	\begin{itemize}
		\item \textbf{Имя контейнера (строка)}~---~название контейнера, к которому относятся ограничения, описанные в записи.
		\item \textbf{Ограничения дискового ввода-вывода для томов (массив записей)}~---~массив, каждая запись которого содержит информацию об ограничениях дискового ввода-вывода для подключаемого к контейнеру тома дискового хранилища. Каждая запись содержит следующие поля:
		
		\begin{itemize}
		\item \textbf{имя тома (строка)};
		\item \textbf{ограничение широты полосы пропускания на чтение (целое)};
		\item \textbf{ограничение широты полосы пропускания на запись (целое)};
		\item \textbf{ограничение количества операций ввода-вывода в секунду на чтение (целое)};
		\item \textbf{ограничение количества операций ввода-вывода в секунду на запись (целое)}.
		\end{itemize}
	\end{itemize}
\end{itemize}

Описанные ресурс является высокоуровневым, и не содержит информации, специфичной для подов. Для применения ограничения дискового ввода-вывода в cgroup v2 необходимо знать о поде следующую информацию:

\begin{itemize}
	\item идентификатор пода;
	\item идентификатор конкретного контейнера;
	\item номера устройства монтируемого тома.
\end{itemize}

Для хранения низкоуровневой информации необходим еще один Custom Resource, описывающий информацию об ограничениях дискового ввода-вывода для конкретного экземпляра приложения. Ресурс представляет собой запись со следующими полями.

\begin{itemize}
	\item \textbf{Название ресурса (строка)}~---~название ресурса, задающего ограничения дискового ввода-вывода для экземпляра приложения, должно быть уникальным.
	\item \textbf{Имя пода (строка)}~---~название объекта Pod, к которому относятся описываемые ограничения.
	\item \textbf{Идентификатор пода (строка)}~---~идентификатор объекта Pod, к которому относятся описываемые ограничения.
	\item \textbf{Имя узла кластера (строка)}~---~название узла кластера, на котором запущен Pod.
	\item \textbf{Ограничения (массив записей)}~---~массив, каждая запись которого содержит информацию об ограничениях дискового ввода-вывода для каждой пары контейнер-примонтированный том. Запись содержит следующие поля.
	
	\begin{itemize}
		\item \textbf{Имя контейнера (строка)}~---~название контейнера, к которому относятся описываемые ограничения.
		\item \textbf{Идентификатор контейнера (строка)}~---~идентификатор контейнера, к которому относятся описываемые ограничения.
		\item \textbf{Имя тома (строка)}~---~название примонтированного тома дискового хранилища, к которому относятся описываемые ограничения.
		\item \textbf{Номера устройства (строка)}~---~номера устройства, к которому относится примонтированный том.
		\item \textbf{Ограничение широты полосы пропускания на чтение (целое)}.
		\item \textbf{Ограничение широты полосы пропускания на запись (целое)}.
		\item \textbf{Ограничение количества операций ввода-вывода в секунду на чтение (целое)}.
		\item \textbf{Ограничение количества операций ввода-вывода в секунду на запись (целое)}.
	\end{itemize}
\end{itemize}


\section{Cхемы алгоритмов работы системы}

\textbf{Алгоритм применения ограничений дискового ввода-вывода на уровне группы экземпляров приложения}

На рисунках~\ref{img:iolimit}--\ref{img:podiolimit_generation} представлена схема алгоритма применения ограничений дискового ввода-вывода на уровне группы экземпляров приложения.

\newpage

\begin{figure}[h!]
    \centering
    \includegraphics[width=\textwidth]{assets/iolimit.pdf}
    \caption{Схема алгоритма применения ограничений дискового ввода-вывода на уровне группы экземпляров приложения}
    \label{img:iolimit}
\end{figure}

\newpage

\begin{figure}[h!]
    \centering
    \includegraphics[width=\textwidth]{assets/podiolimit_generation.pdf}
    \caption{Схема алгоритма создания ресурса, описывающего ограничения ввода-вывода для экземпляра}
    \label{img:podiolimit_generation}
\end{figure}

\newpage

Системная информация, необходимая для применения ограничений, включает в себя следующие данные:

\begin{itemize}
	\item идентификатор контейнера;
	\item идентификатор пода;
	\item номера устройства, на доступ к которому применяются ограничения.
\end{itemize}

Идентификаторы контейнера и пода можно узнать средствами Kubernetes, они содержатся в объекте Pod. Номера устройств не содержатся в каких-либо абстракциях Kubernetes, их можно получить из файловой системы proc~\cite{proc}. В файловой системе для заданного процесса, есть файл \texttt{/proc/PID/mountinfo}, где PID~---~идентификатор основного процесса контейнера. PID процесса также нельзя узнать, используя средства Kubernetes, поэтому для его получения можно использовать Container Runtime Interface~\cite{cri}. Интерфейс представляет собой спецификацию в формате GRPC API, которую могут реализовать сторонние разработчики и Kubernetes использует ее для запуска контейнеров. С использованием запроса в API CRI можно узнать такую информацию в контейнере, как PID основного процесса.

\newpage

На рисунке~\ref{img:fill} представлена диаграмма последовательностей получения системной информации, необходимой для применения ограничений.

\begin{figure}[h!]
    \centering
    \includegraphics[width=\textwidth]{assets/podiolimit_fill.png}
    \caption{Диаграмма последовательностей получения системной информации, необходимой для применения ограничений}
    \label{img:fill}
\end{figure}

\newpage

\textbf{Диаграмма последовательностей применения ограничений дискового ввода-вывода на уровне одного экземпляра приложения}

На рисунке~\ref{img:podiolimit} представлена диаграмма последовательностей применения ограничений дискового ввода-вывода на уровне одного экземпляра приложения.

\begin{figure}[h!]
    \centering
    \includegraphics[width=\textwidth]{assets/apply.png}
    \caption{Диаграмма последовательностей применения ограничений дискового ввода-вывода на уровне одного экземпляра приложения}
    \label{img:podiolimit}
\end{figure}

\newpage

\textbf{Схема алгоритма обработки подов приложения, созданных после применения лимитов}

На рисунке~\ref{img:pod} представлена схема алгоритма обработки подов приложения, созданных после применения лимитов.

\begin{figure}[h!]
    \centering
    \includegraphics[width=\textwidth]{assets/pod.pdf}
    \caption{Схема алгоритма обработки подов приложения, созданных после применения лимитов}
    \label{img:pod}
\end{figure}


