\chapter{Технологический раздел}

В качестве языка программирования для реализации разработанного метода был выбран Go~\cite{donovan2015go}, так как с помощью него можно решить поставленную задачу, а также он имеет достаточный набор библиотек для взаимодействия с Kubernetes. Наличие библиотек также вызвано официальной поддержкой языка Go авторами оркестратора, так как сама система оркестрации реализована с использованием этого языка~\cite{kubernetes}.

Для реализации разрабатываемого ПО был использован паттерн <<Оператор>>, так как он является рекомендуемым для разработки расширений для Kubernetes~\cite{operator}.

Для реализации Kubernetes-оператора использовался Operator SDK, так как он предоставляет достаточный набор функций для создания операторов, в том числе~\cite{operator_sdk}:

\begin{itemize}
	\item генерация шаблона проекта;
	\item генерация обработчиков обновления Custom Resources;
	\item генерация средств развертывания приложения в Kubernetes;
	\item набор библиотек для взаимодействия с Kubernetes.
\end{itemize}

Таким образом, Kubernetes-оператор представляет собой единое приложение, запускающее в себе набор циклов-контроллеров, реагирующих на изменение Custom Resource.

\newpage

На рисунке~\ref{img:po} представлена схема взаимодействия оператора с компонентами Kubernetes и ядра операционной системы Linux.

\begin{figure}[h!]
    \centering
    \includegraphics[width=\textwidth]{assets/po}
    \caption{Схема взаимодействия оператора с компонентами Kubernetes и ядра операционной системы Linux}
    \label{img:po}
\end{figure}

\section{Входные данные}

Входными данными являются сущности Kubernetes. При этом используются три возможных варианта сущностей: Pod, IOLimit, PodIOLimit. При этом Pod является стандартным ресурсом Kubernetes, а IOLimit и PodIOLimit~---~Custom Resource.

\textbf{IOLimit}

Сущность IOLimit хранит информацию об ограничениях дискового ввода-вывода на уровне группы реплик приложения. Манифест этого ресурса имеет вид, представленный в листингах~\ref{lst:iolimit1}--\ref{lst:iolimit2}.

\begin{lstlisting}[label=lst:iolimit1, caption={Пример манифеста IOLimit}]
apiVersion: ioba.dbaas.avito.ru/v1alpha1
\end{lstlisting}

\begin{lstlisting}[label=lst:iolimit2, caption={Пример манифеста IOLimit (продолжение листинга~\ref{lst:iolimit1})}]
kind: IOLimit
metadata:
  name: test-limit
spec:
  storageName: db1
  replicaSetName: rs001
  containers:
    - name: postgresql
      volumes:
        - name: data
          reads:
            iops: 300
            bandwidth: 100MBps
          writes:
            iops: 300
            bandwidth: 20MBps
\end{lstlisting}

При создании или изменении сущности IOLimit, оператор дозаполнит необходимую низкоуровневую информацию и создаст ресурсы PodIOLimit для каждого экземпляра Pod.      

\textbf{Pod}

Так как может быть такая ситуация, что некоторые поды, относящиеся к приложению будут созданы после применения ограничений на существующие поды хранилища, то необходимо отслеживать создание новых подов, и в случае несуществования для них подходящих ресурсов PodIOLimit создавать их.

\textbf{PodIOLimit}

Сущность PodIOLimit хранит информацию об ограничениях дискового ввода-вывода на уровне одной реплики приложения. Манифест этого ресурса имеет вид, представленный в листингах~\ref{lst:podiolimit1}--\ref{lst:podiolimit2}.

\begin{lstlisting}[label=lst:podiolimit1, caption={Пример манифеста PodIOLimit}]
apiVersion: ioba.dbaas.avito.ru/v1alpha1
kind: PodIOLimit
metadata:
  name: test-limit-ioba-test-0
spec:
\end{lstlisting}

\begin{lstlisting}[label=lst:podiolimit2, caption={Пример манифеста PodIOLimit (продолжение листинга~\ref{lst:podiolimit1})}]
  limits:
    - containerID: bb6deeaa978e99a
      containerName: postgresql
      deviceNumbers: "254:1"
      readBandwidth: "100000000"
      readIOPS: "300"
      volumeName: data
      writeBandwidth: "20000000"
      writeIOPS: "300"
  nodeName: worker2
  podName: test-0
  podUID: 9b06e0ab-fd1f-49b9-8581-d4eed47b282a
  replicaSetName: rs001
  storageName: db1
\end{lstlisting}

\section{Создание PodIOLimit на основе IOLimit}

Кроме информации непосредственно из IOLimit, PodIOLimit для каждого контейнера имеет набор системной информации, такой как идентификатора пода и контейнера, и номера примонтированного устройства. Соответственно, для генерации PodIOLimit необходимо получить эту низкоуровневую информацию.

Идентификатор пода содержится в его метаданных. Листинги~\ref{lst:pod_uid1}--\ref{lst:pod_uid2} содержат фрагмент кода, получающий идентификатор пода.

\begin{lstlisting}[language=Go,label=lst:pod_uid1, caption={Получение идентификатора пода}]
podLimit := &dbaasv1alpha1.PodIOLimit{
	ObjectMeta: metav1.ObjectMeta{
		Name:       fmt.Sprintf("%s-%s", iolimit.Name, pod.Name),
		Namespace:  pod.Namespace,
		Finalizers: []string{dbaasv1alpha1.CleanupFinalizer},
		Labels: map[string]string{
			dbaasv1alpha1.NodeNameLabel: g.nodeName,
		},
	},
	Spec: dbaasv1alpha1.PodIOLimitSpec{
\end{lstlisting}

\begin{lstlisting}[language=Go,label=lst:pod_uid2, caption={Получение идентификатора пода (продолжение листинга~\ref{lst:pod_uid1})}]
		StorageName:    iolimit.Spec.StorageName,
		ReplicaSetName: iolimit.Spec.ReplicaSetName,
		Limits:         limits,
		NodeName:       g.nodeName,
		PodName:        pod.Name,
		PodUID:         string(pod.UID),
		},
}
\end{lstlisting}

Идентификатор контейнера получается из информации о состоянии пода. Листинги~\ref{lst:container_id1}--\ref{lst:container_id2} содержат фрагмент кода получения идентификатора контейнера.

\begin{lstlisting}[language=Go,label=lst:container_id1, caption={Получение идентификатора контейнера}]
func (g *Generator) prepareLimits(pod *v1.Pod, iolimit *dbaasv1alpha1.IOLimit) ([]models.LimitBase, error) {
	volumeToPVC := make(map[string]string, len(pod.Spec.Volumes))
	for _, volume := range pod.Spec.Volumes {
		if volume.PersistentVolumeClaim == nil {
			continue
		}

		volumeToPVC[volume.Name] = volume.PersistentVolumeClaim.ClaimName
	}

	containerNameToID := make(map[string]string, len(pod.Spec.Containers))
	for _, status := range pod.Status.ContainerStatuses {
		containerNameToID[status.Name] = g.trimCRIPrefix(status.ContainerID)
	}

	limits := make([]models.LimitBase, 0, len(iolimit.Spec.Containers))
	for _, container := range iolimit.Spec.Containers {
		for _, volume := range container.Volumes {
\end{lstlisting}

\begin{lstlisting}[language=Go,label=lst:container_id2, caption={Получение идентификатора контейнера (продолжение листинга~\ref{lst:container_id1})}]
			pvc, ok := volumeToPVC[volume.Name]
			if !ok {
				return nil, fmt.Errorf("get pvc by volume: %w", models.ErrPVCNotFound)
			}

			containerID, ok := containerNameToID[container.Name]
			if !ok || containerID == "" {
				return nil, fmt.Errorf("get container id by name: %w", models.ErrContainerNotFound)
			}

			limits = append(limits, models.LimitBase{
				Namespace:     iolimit.Namespace,
				PodUID:        string(pod.UID),
				ContainerName: container.Name,
				ContainerID:   containerID,
				VolumeName:    volume.Name,
				PVCName:       pvc,
				Reads:         volume.Reads,
				Writes:        volume.Writes,
			})
		}
	}

	return limits, nil
}
\end{lstlisting}


Фрагмент получения номеров устройства из файловой системы proc представлен в листингах~\ref{lst:proc_parse1}--\ref{lst:proc_parse2}.

\begin{lstlisting}[language=Go,label=lst:proc_parse1, caption={Получение номеров устройства}]
func (e *Extractor) DeviceNumbers(
    ctx context.Context, 
    containerPID int, 
    volumeHandle string,
\end{lstlisting}

\newpage

\begin{lstlisting}[language=Go,label=lst:proc_parse2, caption={Получение номеров устройства (продолжение листинга~\ref{lst:proc_parse1})}]
) (string, error) {
	file, err := e.reader.Read(filepath.Join(strconv.Itoa(containerPID), "mountinfo"))
	if err != nil {
		return "", fmt.Errorf("read proc file: %w", err)
	}
	deviceNumbers, err := e.deviceNumberFromMountInfo(file, volumeHandle)
	if err != nil {
		return "", fmt.Errorf("extract device numbers: %w", err)
	}
	
	return deviceNumbers, nil
}

func (e *Extractor) deviceNumberFromMountInfo(mountInfo, volumeHandle string) (string, error) {
	for _, line := range strings.Split(mountInfo, "\n") {
		if !strings.Contains(line, volumeHandle) {
			continue
		}
		splitted := strings.Split(line, " ")
		if len(splitted) < deviceNumbersColumn {
			return "", fmt.Errorf("line must contain more than 3 columns: %w (%q)", models.ErrInvalidProcFile, line)
		}
		return splitted[deviceNumbersColumn-1], nil
	}

	return "", fmt.Errorf("mount not found for volumehandle %s: %w", volumeHandle, models.ErrNoVolumeEntry)
}
\end{lstlisting}

Фрагмент кода получения PID основного процесса контейнера представлен в листинге~\ref{lst:pid}.

\newpage

\begin{lstlisting}[language=Go,label=lst:pid, caption={Получение PID основного процесса контейнера}]
func (e *Extractor) ContainerPID(ctx context.Context, containerID string) (int, error) {
	resp, err := e.cri.ContainerStatus(ctx, containerID, true)
	if err != nil {
		return 0, fmt.Errorf("call containerStatus: %w", err)
	}
	infoMarshalled, ok := resp.Info["info"]
	if !ok {
		return 0, fmt.Errorf("container info response must contain info block: %w", models.ErrInvalidContainerInfo)
	}

	var info containerInfo
	if err := json.Unmarshal([]byte(infoMarshalled), &info); err != nil {
		return 0, fmt.Errorf("unmarshal container info: %w", err)
	}

	if info.PID == 0 {
		return 0, fmt.Errorf("extract container pid: %w", models.ErrInvalidContainerInfo)
	}

	return info.PID, nil
}

type containerInfo struct {
	PID int `json:"pid"`
}
\end{lstlisting}

На основе всех полученных данных может быть сгенерирован PodIOLimit. Фрагмент кода по генерации PodIOLimit представлен в Приложении~Б.

\section{Применение ограничений дискового ввода-вывода с использованием cgroup v2}

Применение ограничений дискового ввода-вывода представляет собой запись в интерфейсный файл \texttt{io.max} контрольной группы контейнера. Фрагмент кода с записью ограничений в файл представлен в листинге~\ref{lst:write_limit}.

\begin{lstlisting}[language=Go,label=lst:write_limit, caption={Запись ограничений в cgroup v2}]
func (w *LimitWriter) WriteLimit(ctx context.Context, podUID string, limit dbaasv1alpha1.Limit) error {
	fileName := w.path.Format(podUID, limit, ioMaxFileName)
	line := w.toCGroupLine(limit)

	if err := w.writer.Write(fileName, line); err != nil {
		return fmt.Errorf("write limit to file: %w", err)
	}

	return nil
}

func (w *LimitWriter) toCGroupLine(limit dbaasv1alpha1.Limit) string {
	return fmt.Sprintf("%s rbps=%s wbps=%s riops=%s wiops=%s",
		w.deviceNumbersGenerator.Generate(limit.DeviceNumbers),
		limit.ReadBandwidth,
		limit.WriteBandwidth,
		limit.ReadIOPS,
		limit.WriteIOPS,
	)
}
\end{lstlisting}

\section{Тестирование}

Для тестирования ПО были реализованы полностью автоматизированные модульные и интеграционные тесты, и частично автоматизированные функциональные тесты. Суммарное тестовое покрытие составляет 89\%. Кроме того, была измерена оценка по мутационному тестированию с использованием утилиты \texttt{go-mutesting}~\cite{mutesting}, полученная оценка равна 86\%.

Пример модульного теста представлен в Приложении~В.

Пример интеграционного теста представлен в приложении~Г.
