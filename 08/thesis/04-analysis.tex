\chapter{Аналитический раздел}

% НАЧАТЬ ПРО ТО ЧТО ВООБЩЕ ТАКОЕ ДИСКОВЫЙ ВВОД-ВЫВОД
% ПРО iops и bandwidth
% про контейнеризацию
% про оркестрацию
% про кубер и то что там нет ограничений


Дисковый ввод-вывод~---~операции записи и чтения с диска. Существуют разные виды дисков, однако они не могут обеспечить параллельность всех операций ввода-вывода, из-за этого данные операции осуществляются конкурентно, соответственно, дисковые операции имеют ограниченную пропускную способность. Для того, чтобы увеличить пропускную способность, существуют такие аппаратные инструменнты, как RAID~\cite{chen1994raid}. Таким образом, дисковый ввод-вывод представляет собой такой же разделяемый ресурс, как и процессорное время~\cite{кузьминский2012контрольные}.

Так как процессы, запущенные на одном узле конкурируют за разделяемые ресурсы, имеет смысл ограничивать дисковый ввод-вывод. При этом, приложения имеют разных характер доступа к данным: одни пишут и читают с диска большими  фрагментами, но с небольшой частотой, другие~---~небольшими фрагментами с большой частой. Таким образом, ограничивается разделяемый ресурс дискового комбинированным способом: задаются ограничения на широту полосы пропускания (байт в секунду), то есть непосредственно скорость доступа к диску и ограничения на количество операций в секунду.

С 2020 года в сообществе Kubernetes ведется активная дискуссия на тему добавления в оркестратор механизма, позволяющего разграничивать дисковый ввод-вывод~\cite{k8s_iops_limit_github}. На текущий момент обсуждение еще ведется, и в последних версиях ведется работа над добавлением данного механизма в систему.

\section{Kubernetes}

Kubernetes~---~система для запуска контейнеризированных приложений (оркестратор) с открытым исходным кодом, позволяет управлять жизненным циклом работы контейнеров: запускать, удалять, перезапускать в случае возникновения ошибок~\cite{kubernetes}. При этом пользователь абстрагирован от внутренней логики выполнения приложений, такой как расположение контейнера на каком-то из физических узлов кластера.

Основным понятием в Kubernetes является ресурс. Ресурс~---~описание некоторого типа объекта в Kubernetes. Все понятия, с которыми оперирует оркестратор, представлены в виде объектов, и хранятся в базе данных~\cite{kubernetesresources}.

Kubernetes оперирует со следующими основными типами ресурсов.

\begin{itemize}
	\item \textbf{Pod (под)} описывает экземпляр приложения. В рамках одного пода может быть запущено несколько связанных контейнеров. Является минимальной единицей планирования в Kubernetes, то есть все контейнеры одного пода обязательно будут запущены на одном узле кластера~\cite{pods}.
	\item \textbf{ReplicaSet (набор реплик)} описывает набор реплик запускаемого приложения. На основе него Kubernetes создает поды в количестве, указанном в свойствах объекта~\cite{rs}.
	\item \textbf{Deployment} описывает приложение без сохранения состояния. На основе него Kubernetes создает набор реплик и регулирует процесс обновления этого набора~\cite{deploy}.
	\item \textbf{PersistentVolume (том)} описывает том постоянной области данных, это может быть область диска узла кластера, сетевое хранилище, и так далее~\cite{pv}. Так как том постоянно области данных может быть реализован по-разному, в Kubernetes существует спецификация Container Storage Interface, через которую оркестратор создает тома. Производители оборудования и разработчики могут создавать собственные реализации CSI~\cite{csi}.
	\item \textbf{PersistentVolumeClaim (запрос на том)} описывает пользовательский запрос на том постоянного хранилища данных. Пользователь кластера, если ему необходимо постоянно хранить данные не взаимодействует напрямую с PersistentVolume, а создает запрос на том~\cite{pv}.
	\item \textbf{StatefulSet} описывает приложение с сохранением состояния. На основе него Kubernetes создает запросы на тома и поды в количестве, указанном в спецификации объекта. Каждый под будет использовать уникальный запрос на том~\cite{sts}.
\end{itemize}

\newpage

Для описания и отображения объектов пользователю используются манифесты. Манифест~---~текстовое описание объекта в Kubernetes в формате YAML~\cite{kubernetesresources}. Пример манифеста пода представлен в листинге~\ref{lst:manifest}.

\begin{lstlisting}[label=lst:manifest, caption={Пример манифеста пода}]
apiVersion: v1
kind: Pod
metadata:
  name: nginx
spec:
  containers:
  - name: nginx
    image: nginx:1.14.2
\end{lstlisting}


\section{Обзор существующего решения}

В Kubernetes версии 1.29 был представлен новый тип ресурса VolumeAttributeClass~\cite{volumeattributeclass}. С помощью этого ресурса можно объявлять классы томов. Пример объявления ресурса VolumeAttributeClass приведен в листинге~\ref{lst:volumeattributeclass}.

\begin{lstlisting}[label=lst:volumeattributeclass, caption={Пример объявления ресурса VolumeAttributesClass}]
apiVersion: storage.k8s.io/v1alpha1
kind: VolumeAttributesClass
metadata:
  name: silver
driverName: pd.csi.storage.gke.io
parameters:
  provisioned-iops: "3000"
  provisioned-throughput: "50" 
\end{lstlisting}

После объявления можно выделять тома данных с использованием созданного класса. При этом конкретная реализация Container Storage Interface~\cite{csi} при выделении тома может использовать описанные параметры класса. Такими параметрами могут быть, как в примере, ограничения на пропускную способность обращений к диску и на количество операций ввода-вывода в секунду.

Данное решение имеет следующие ограничения.

\begin{enumerate}
	\item VolumeAttributeClass не позволяет задавать различные ограничения на дисковый ввод-вывод для доступа к одному тому из разных контейнеров.
	\item Параметры класса томов являются неизменяемыми, то есть в случае необходимости изменить ограничения, необходимо создавать новый класс и изменять класс выделенного тома.
	\item API VolumeAttributeClass находится в нестабильной версии.
	\item Ограничения дискового ввода-вывода с использованием данного API должны быть реализованы на уровне конкретных реализаций CSI.
	\item Невозможность разделять запросы и ограничения на разделяемые ресурсы.
\end{enumerate}
 
\section{Постановка задачи}

\textbf{Целью} данной работы является создания метода ограничения пропускной способности дискового ввода-вывода в системе оркестрации Kubernetes. При этом, данный метод должен иметь следующие особенности.

\begin{enumerate}
	\item Задание различных ограничений дискового ввода-вывода для доступа к томам из разных контейнеров.
	\item Изменение ограничений дискового ввода-вывода с помощью изменения существующих ресурсов.
	\item Независимость метода от конкретных деталей реализации CSI.
	\item Возможность разделять запросы и ограничения на ресурсы.
\end{enumerate}

Для выполнения цели работы необходимо решить \textbf{задачи}, приведенные ниже.

\begin{enumerate}
	\item Спроектировать способ хранения данных об ограничениях пропускной способности дискового ввода-вывода для приложения в системе оркестрации приложений Kubernetes.
	\item Сформулировать и описать основные шаги, связанные с применением ограничений пропускной способности дискового ввода-вывода.
	\item Разработать программное обеспечение, реализующее метод ограничения пропускной способности дискового ввода-вывода.
	\item Выполнить тестирование разработанного программного обеспечения.
	\item Провести нагрузочное тестирования известного приложения, использующего дисковый ввод-вывод. Исследовать зависимость полученных результатов от параметров разработанного программного обеспечения.
\end{enumerate}

На рисунке \ref{img:idef0} представлена диаграмма разрабатываемого метода в нотации IDEF0 нулевого уровня.

\begin{figure}[h!]
    \centering
    \includegraphics[width=\textwidth]{assets/idef0.pdf}
    \caption{Диаграмма разрабатываемого метода в нотации IDEF0 нулевого уровня}
    \label{img:idef0}
\end{figure}

На рисунке \ref{img:idef0-1} представлена диаграмма разрабатываемого метода в нотации IDEF0 первого уровня.

\newpage

\begin{figure}[h!]
    \centering
    \includegraphics[width=\textwidth]{assets/idef0-1.pdf}
    \caption{Диаграмма разрабатываемого метода в нотации IDEF0 первого уровня}
    \label{img:idef0-1}
\end{figure}