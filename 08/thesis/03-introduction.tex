\chapter*{ВВЕДЕНИЕ}
\addcontentsline{toc}{chapter}{ВВЕДЕНИЕ}

Контейнеризация~---~технология, позволяющая изолировать выполнение процессов между собой. В отличие от виртуальных машин, контейнеры используют ядро операционной системы, на базе которой они запущены, и не требуют запуска отдельного экземпляра ядра~\cite{luksa2017kubernetes}. При этом, контейнеризация так же, как и виртуализация, предоставляет ограничение разделяемых ресурсов системы. Разделяемыми ресурсами являются процессорное время, память, дисковое пространство, полоса пропускания дискового и сетевого ввода-вывода~\cite{cgroups}\cite{кузьминский2012контрольные}.

Kubernetes~---~система для запуска контейнеризированных приложений с открытым исходным кодом~\cite{kubernetes}. Оркестратор позволяет управлять жизненным циклом работы контейнеров: запускать, удалять, перезапускать в случае возникновения ошибок. При этом пользователь абстрагирован от внутренней логики выполнения приложений, такой как расположение контейнера на каком-то из физических узлов кластера.

Система оркестрации контейнеров Kubernetes позволяет запускать разные виды приложений: как без сохранения состояния, так и с сохранением состояния. Примерами приложений с сохранением состояния являются, например, базы данных~\cite{vayghan2021kubernetes}. 

Так как в системах оркестрации подразумевается, что на одном физическом сервере может быть запущено несколько контейнеров, Kubernetes поддерживает ограничение разделяемых ресурсов, чтобы не столкнуться с так называемой <<проблемой шумных соседей>>~\cite{makroo2016systematic}, когда усиленное потребление какого-либо разделяемого ресурса сервера контейнером влияет на работу других контейнеров. Актуальные версии Kubernetes~\cite{kubernetes} поддерживают ограничение таких ресурсов, как использование процессора и объем доступной контейнеру оперативной памяти. При этом, приложения могут разделять и другие ресурсы, в частности дисковый ввод-вывод. Отсутствие разграничений дискового ввода-вывода в Kubernetes может затруднять эксплуатацию баз данных~\cite{avito_database_meetup}\cite{phdays}.
