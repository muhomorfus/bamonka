\documentclass[14pt]{extarticle}
\usepackage[T1]{fontenc}
\usepackage[utf8]{inputenc}
\usepackage{amsmath,amssymb}
\usepackage[russian]{babel}
\usepackage{geometry}
\geometry{a4paper,
total={170mm,257mm},left=2cm,right=1cm,
top=1.5cm,bottom=1.5cm}

\usepackage{tabularx}
\usepackage{tikz}

\newcommand{\podpis}[2]{
    \parbox[b]{6cm}{#1}
    \hspace{1cm}
    \tikz[baseline=2pt]{\draw(0,0) to node[below=-2pt]{\scriptsize подпись}(5cm,0);}
    \hspace{1cm}
    \tikz[baseline=2pt]{
        \def\familywidth{\textwidth-6cm-1cm-5cm-1cm-20pt}
        \draw(0,0) to node[below=-2pt]{\scriptsize дата}(\familywidth,0);
        \node[anchor=west](f) at (25pt,8pt){#2};
    }
}

\begin{document}

\pagestyle{empty}

\begin{center}
\section*{РЕЦЕНЗИЯ}
на выпускную квалификационную работу бакалавра

Карповой Екатерины Олеговны

<<Метод автоматического сегментирования баз данных на основе системы управления базами данных PostgreSQL>>
\end{center}	

В настоящее время во многих крупных компаниях возникает необходимость горизонтального масштабирования баз данных. PostgreSQL изначально не имеет достаточных возможностей для такого масштабирования, поэтому разработка средств для обеспечения масштабируемости данной СУБД является актуальной задачей. 

В аналитической части квалификационной работы проведен анализ предметной области и обзор существующих решений.

В конструкторской части представлена концепция метода шардирования базы данных на основе СУБД PostgreSQL, описаны компоненты системы и их взаимодействие.

В технологической части произведен выбор средств реализации, показано разработанное программное обеспечение, его архитектура и интерфейс. 

В исследовательской части проведено исследование разработанного метода, показана его работа на объемах данных порядка десяти миллионов строк. Произведено измерение времени работы метода в зависимости от объема хранимых данных.

К достоинствам разработанного ПО можно отнести автоматическую перебалансировку при высокой нагруженности шардов, распределение данных на существующие менее нагруженные шарды, а также автоматизированное создание новых экземпляров СУБД при недостатке ресурсов внутри кластера. 

К недостаткам метода можно отнести отсутствие поддержки внешних ключей и кроссшард-запросов. Однако, кросс-шард запросы не считаются хорошей практикой в использовании шардированных СУБД, так как существующие решения не обеспечивают приемлемую производительность запросов такого вида, поэтому данный недостаток можно считать несущественным.

Считаю, что выпускная квалификационная работа Карповой~Е.~О. <<Метод автоматического сегментирования баз данных на основе системы управления базами данных PostgreSQL>> соответствует квалификационным требованиям, предъявляемым к выпускной квалификационной работе бакалавра, заслуживает оценки <<отлично>>, а Карпова~Е.~О.~---~присвоения степени бакалавра по направлению подготовки 09.03.04 <<Программная инженерия>>.

\vspace{1.5cm}

\setlength{\parindent}{0pt}

\podpis{Рецензент:\\Старший инженер\\ООО <<Авито Тех>>\\Жига Никита Викторович}{29.05.2024}\\

\end{document}