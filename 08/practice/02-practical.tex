\chapter{Основная часть}

\section{Выбор средств реализации}

В качестве языка программирования для реализации разработанного метода был выбран Go, так как с помощью него можно решить поставленную задачу, а также он имеет достаточный набор библиотек для взаимодействия с Kubernetes. Наличие библиотек также вызвано официальной поддержкой языка Go авторами оркестратора, так как сама система оркестрации реализована с использованием этого языка.

Для реализации разрабатываемого ПО был использован паттерн <<Оператор>>, так как он является рекомендуемым для разработки расширений для Kubernetes.

Для реализации Kubernetes-оператора использовался Operator SDK, так как он предоставляет достаточный набор функций для создания операторов, в том числе:

\begin{itemize}
	\item генерация шаблона проекта;
	\item генерация обработчиков обновления Custom Resources;
	\item генерация средств развертывания приложения в Kubernetes;
	\item набор библиотек для взаимодействия с Kubernetes.
\end{itemize}

Таким образом, Kubernetes-оператор представляет собой единое приложение, запускающее в себе набор циклов-контроллеров, реагирующих на изменение Custom Resource.

\section{Реализация программного обеспечения метода}

\subsection{Входные данные}

Входными данными являются сущности Kubernetes. При этом используются три возможных варианта сущностей: Pod, IOLimit, PodIOLimit. При этом Pod является стандартным ресурсом Kubernetes, а IOLimit и PodIOLimit~---~Custom Resource.

\subsubsection{IOLimit}

Сущность IOLimit хранит информацию об ограничениях дискового ввода-вывода на уровне группы реплик приложения. Описание этого ресурса в формате YAML имеет вид, представленный в листинге~\ref{lst:iolimit}.

\begin{lstlisting}[label=lst:iolimit, caption={Пример описания сущности IOLimit в формате YAML}]
apiVersion: ioba.dbaas.avito.ru/v1alpha1
kind: IOLimit
metadata:
  name: test-limit
spec:
  storageName: db1
  replicaSetName: rs001
  containers:
    - name: postgresql
      volumes:
        - name: data
          reads:
            iops: 300
            bandwidth: 100MBps
          writes:
            iops: 300
            bandwidth: 20MBps
\end{lstlisting}

При создании или изменении сущности IOLimit, оператор дозаполнит необходимую низкоуровневую информацию и создаст ресурсы PodIOLimit для каждого экземпляра Pod.      

\subsubsection{Pod}

Так как может быть такая ситуация, что некоторые поды, относящиеся к приложению будут созданы после применения ограничений на существующие поды хранилища, то необходимо отслеживать создание новых подов, и в случае несуществования для них подходящих ресурсов PodIOLimit создавать их.

\subsubsection{PodIOLimit}

Сущность PodIOLimit хранит информацию об ограничениях дискового ввода-вывода на уровне одной реплики приложения. Описание этого ресурса в формате YAML имеет вид, представленный в листинге~\ref{lst:podiolimit}.

\begin{lstlisting}[label=lst:podiolimit, caption={Пример описания сущности PodIOLimit в формате YAML}]
apiVersion: ioba.dbaas.avito.ru/v1alpha1
kind: PodIOLimit
metadata:
  name: test-limit-ioba-test-0
spec:
  limits:
    - containerID: bb6deeaa978e99a
      containerName: postgresql
      deviceNumbers: "254:1"
      readBandwidth: "100000000"
      readIOPS: "300"
      volumeName: data
      writeBandwidth: "20000000"
      writeIOPS: "300"
  nodeName: worker2
  podName: test-0
  podUID: 9b06e0ab-fd1f-49b9-8581-d4eed47b282a
  replicaSetName: rs001
  storageName: db1
\end{lstlisting}

\subsection{Создание PodIOLimit на основе IOLimit}

Кроме информации непосредственно из IOLimit, PodIOLimit для каждого контейнера имеет набор более низкоуровневой информации, такой как идентификатора пода и контейнера, и номера примонтированного устройства. Соответственно, для генерации PodIOLimit необходимо получить эту низкоуровневую информацию.

Идентификатор пода содержится в его метаданных. Листинг~\ref{lst:pod_uid} содержит фрагмент кода, получающий идентификатор пода.

\begin{lstlisting}[language=Go,label=lst:pod_uid, caption={Получение идентификатора пода}]
podLimit := &dbaasv1alpha1.PodIOLimit{
	ObjectMeta: metav1.ObjectMeta{
		Name:       fmt.Sprintf("%s-%s", iolimit.Name, pod.Name),
		Namespace:  pod.Namespace,
		Finalizers: []string{dbaasv1alpha1.CleanupFinalizer},
		Labels: map[string]string{
			dbaasv1alpha1.NodeNameLabel: g.nodeName,
		},
	},
	Spec: dbaasv1alpha1.PodIOLimitSpec{
		StorageName:    iolimit.Spec.StorageName,
		ReplicaSetName: iolimit.Spec.ReplicaSetName,
		Limits:         limits,
		NodeName:       g.nodeName,
		PodName:        pod.Name,
		PodUID:         string(pod.UID),
		},
}
\end{lstlisting}

Идентификатор контейнера получается из информации о состоянии пода. Листинг~\ref{lst:container_id} содержит фрагмент кода получения идентификатора контейнера.

\begin{lstlisting}[language=Go,label=lst:container_id, caption={Получение идентификатора контейнера}]
func (g *Generator) prepareLimits(pod *v1.Pod, iolimit *dbaasv1alpha1.IOLimit) ([]models.LimitBase, error) {
	volumeToPVC := make(map[string]string, len(pod.Spec.Volumes))
	for _, volume := range pod.Spec.Volumes {
		if volume.PersistentVolumeClaim == nil {
			continue
		}

		volumeToPVC[volume.Name] = volume.PersistentVolumeClaim.ClaimName
	}

	containerNameToID := make(map[string]string, len(pod.Spec.Containers))
	for _, status := range pod.Status.ContainerStatuses {
		containerNameToID[status.Name] = g.trimCRIPrefix(status.ContainerID)
	}

	limits := make([]models.LimitBase, 0, len(iolimit.Spec.Containers))
	for _, container := range iolimit.Spec.Containers {
		for _, volume := range container.Volumes {
			pvc, ok := volumeToPVC[volume.Name]
			if !ok {
				return nil, fmt.Errorf("get pvc by volume: %w", models.ErrPVCNotFound)
			}

			containerID, ok := containerNameToID[container.Name]
			if !ok || containerID == "" {
				return nil, fmt.Errorf("get container id by name: %w", models.ErrContainerNotFound)
			}

			limits = append(limits, models.LimitBase{
				Namespace:     iolimit.Namespace,
				PodUID:        string(pod.UID),
				ContainerName: container.Name,
				ContainerID:   containerID,
				VolumeName:    volume.Name,
				PVCName:       pvc,
				Reads:         volume.Reads,
				Writes:        volume.Writes,
			})
		}
	}

	return limits, nil
}
\end{lstlisting}

Номера устройств не содержатся в каких-либо абстракциях Kubernetes, поэтому они получаются из файловой системы proc~\cite{proc}. В файловой системе для заданного процесса, есть файл \texttt{/proc/PID/mountinfo}, где PID~---~идентификатор основного процесса контейнера. Фрагмент получения номеров устройства из файловой системы proc представлен в листинге~\ref{lst:proc_parse}.

\begin{lstlisting}[language=Go,label=lst:proc_parse, caption={Получение номеров устройства}]
func (e *Extractor) DeviceNumbers(ctx context.Context, containerPID int, volumeHandle string) (string, error) {
	file, err := e.reader.Read(filepath.Join(strconv.Itoa(containerPID), "mountinfo"))
	if err != nil {
		return "", fmt.Errorf("read proc file: %w", err)
	}

	deviceNumbers, err := e.deviceNumberFromMountInfo(file, volumeHandle)
	if err != nil {
		return "", fmt.Errorf("extract device numbers: %w", err)
	}

	return deviceNumbers, nil
}

func (e *Extractor) deviceNumberFromMountInfo(mountInfo, volumeHandle string) (string, error) {
	for _, line := range strings.Split(mountInfo, "\n") {
		if !strings.Contains(line, volumeHandle) {
			continue
		}

		splitted := strings.Split(line, " ")
		if len(splitted) < deviceNumbersColumn {
			return "", fmt.Errorf("line must contain more than 3 columns: %w (%q)", models.ErrInvalidProcFile, line)
		}

		return splitted[deviceNumbersColumn-1], nil
	}

	return "", fmt.Errorf("mount not found for volumehandle %s: %w", volumeHandle, models.ErrNoVolumeEntry)
}
\end{lstlisting}

PID процесса также нельзя узнать, используя средства Kubernetes, поэтому для его получения используется Container Runtime Interface~\cite{cri}. Интерфейс представляет собой спецификацию в формате GRPC API, которую могут реализовать сторонние разработчики и Kubernetes использует ее для запуска контейнеров. С использованием запроса в API CRI можно узнать такую информацию в контейнере, как PID основного процесса. Фрагмент кода получения PID основного процесса контейнера представлен в листинге~\ref{lst:pid}.

\begin{lstlisting}[language=Go,label=lst:pid, caption={Получение PID основного процесса контейнера}]
func (e *Extractor) ContainerPID(ctx context.Context, containerID string) (int, error) {
	resp, err := e.cri.ContainerStatus(ctx, containerID, true)
	if err != nil {
		return 0, fmt.Errorf("call containerStatus: %w", err)
	}

	infoMarshalled, ok := resp.Info["info"]
	if !ok {
		return 0, fmt.Errorf("container info response must contain info block: %w", models.ErrInvalidContainerInfo)
	}

	var info containerInfo
	if err := json.Unmarshal([]byte(infoMarshalled), &info); err != nil {
		return 0, fmt.Errorf("unmarshal container info: %w", err)
	}

	if info.PID == 0 {
		return 0, fmt.Errorf("extract container pid: %w", models.ErrInvalidContainerInfo)
	}

	return info.PID, nil
}

type containerInfo struct {
	PID int `json:"pid"`
}
\end{lstlisting}

На основе всех полученных данных может быть сгенерирован PodIOLimit. Фрагмент кода по генерации PodIOLimit представлен в листинге~\ref{lst:podiolimit_gen}.

\begin{lstlisting}[language=Go,label=lst:podiolimit_gen, caption={Генерация PodIOLimit}]
func (g *Generator) Generate(
	ctx context.Context,
	pod *v1.Pod,
	iolimit *dbaasv1alpha1.IOLimit,
) (*dbaasv1alpha1.PodIOLimit, error) {
	if !g.isPodReady(pod) {
		return nil, fmt.Errorf("pod contains empty fields: %w", models.ErrPodIsNotReady)
	}

	baseLimits, err := g.prepareLimits(pod, iolimit)
	if err != nil {
		return nil, fmt.Errorf("extract base limits: %w", err)
	}

	limits := make([]dbaasv1alpha1.Limit, len(baseLimits))
	for i, baseLimit := range baseLimits {
		constructed, err := g.limitConstructor.Construct(ctx, baseLimit)
		if err != nil {
			return nil, fmt.Errorf("construct limit: %w", err)
		}

		limits[i] = *constructed
	}

	podLimit := &dbaasv1alpha1.PodIOLimit{
		ObjectMeta: metav1.ObjectMeta{
			Name:       fmt.Sprintf("%s-%s", iolimit.Name, pod.Name),
			Namespace:  pod.Namespace,
			Finalizers: []string{dbaasv1alpha1.CleanupFinalizer},
			Labels: map[string]string{
				dbaasv1alpha1.NodeNameLabel: g.nodeName,
			},
		},
		Spec: dbaasv1alpha1.PodIOLimitSpec{
			StorageName:    iolimit.Spec.StorageName,
			ReplicaSetName: iolimit.Spec.ReplicaSetName,
			Limits:         limits,
			NodeName:       g.nodeName,
			PodName:        pod.Name,
			PodUID:         string(pod.UID),
		},
	}

	if err := g.ownerReferencer.SetOwnerReference(iolimit, podLimit); err != nil {
		return nil, fmt.Errorf("set controller reference for podiolimit from iolimit: %w", err)
	}

	return podLimit, nil
}

func (d *Constructor) Construct(ctx context.Context, base models.LimitBase) (*dbaasv1alpha1.Limit, error) {
	pv, provisioner, err := d.persistentVolume.PersistentVolume(ctx, base.Namespace, base.PVCName)
	if err != nil {
		return nil, fmt.Errorf("enrich persistent volume: %w", err)
	}

	if !slices.Contains(d.allowedProvisioners, provisioner) {
		return nil, fmt.Errorf("provisioner %s is not allowed: %w", provisioner, models.ErrUnsupportedProvisioner)
	}

	volumeHandle, err := d.volumeHandle.VolumeHandle(ctx, base.Namespace, pv)
	if err != nil {
		return nil, fmt.Errorf("enrich volume handle: %w", err)
	}

	containerPID, err := d.container.ContainerPID(ctx, base.ContainerID)
	if err != nil {
		return nil, fmt.Errorf("enrich container: %w", err)
	}

	deviceNumbers, err := d.deviceNumbers.DeviceNumbers(ctx, containerPID, volumeHandle)
	if err != nil {
		return nil, fmt.Errorf("enrich device numbers: %w", err)
	}

	return &dbaasv1alpha1.Limit{
		ContainerName:  base.ContainerName,
		ContainerID:    base.ContainerID,
		VolumeName:     base.VolumeName,
		DeviceNumbers:  deviceNumbers,
		ReadIOPS:       base.Reads.GetIOPS(),
		ReadBandwidth:  base.Reads.GetBandwidthBytesPerSecond(),
		WriteIOPS:      base.Writes.GetIOPS(),
		WriteBandwidth: base.Writes.GetBandwidthBytesPerSecond(),
	}, nil
}
\end{lstlisting}

\subsection{Применение ограничений дискового ввода-вывода с использованием cgroup v2}

Применение ограничений дискового ввода-вывода представляет собой запись в интерфейсный файл io.max контрольной группы контейнера. Фрагмент кода по записи ограничений в файл представлен в листинге~\ref{lst:write_limit}.

\begin{lstlisting}[language=Go,label=lst:write_limit, caption={Запись ограничений в cgroup v2}]
func (w *LimitWriter) WriteLimit(ctx context.Context, podUID string, limit dbaasv1alpha1.Limit) error {
	fileName := w.path.Format(podUID, limit, ioMaxFileName)
	line := w.toCGroupLine(limit)

	if err := w.writer.Write(fileName, line); err != nil {
		return fmt.Errorf("write limit to file: %w", err)
	}

	return nil
}

func (w *LimitWriter) toCGroupLine(limit dbaasv1alpha1.Limit) string {
	return fmt.Sprintf("%s rbps=%s wbps=%s riops=%s wiops=%s",
		w.deviceNumbersGenerator.Generate(limit.DeviceNumbers),
		limit.ReadBandwidth,
		limit.WriteBandwidth,
		limit.ReadIOPS,
		limit.WriteIOPS,
	)
}
\end{lstlisting}

\section{Исследование характеристик разработанного программного обеспечения}

Был проведен бенчмарк базы PostgreSQL с различными ограничениями дискового ввода-вывода. 

Характеристики контейнера, на котором производилось тестирование следующие.

\begin{itemize}
	\item Процессорное время: 1000 единиц.
	\item Память: 4Гб.
\end{itemize}

Для теста производительности использовался pgbench. Для тестирования вызывался pgbench с параметрами, показанными в листинге~\ref{lst:pgbench}.

\begin{lstlisting}[language=Go,label=lst:pgbench, caption={Параметры pgbench}]
POOL=10
DURATION=120
SIZE_GB=4
SCALE_FACTOR=$(($SIZE_GB * 67))

pgbench -s ${SCALE_FACTOR}
pgbench  -c ${POOL} -j 2 -T ${DURATION}
\end{lstlisting}

Изначально было измерена производительность базы данных в стандартном тесте без каких-либо ограничений дискового ввода-вывода. Затем было выделено три группы ограничений по IOPS и Bandwidth, отличающиеся между собой в 2 раза.

Группы по IOPS:

\begin{enumerate}
	\item IOPS на чтение: 500 операций в секунду, IOPS на запись: 250 операций в секунду;
	\item IOPS на чтение: 1000 операций в секунду, IOPS на запись: 500 операций в секунду;
	\item IOPS на чтение: 2000 операций в секунду, IOPS на запись: 1000 операций в секунду.
\end{enumerate}

Группы по Bandwidth:

\begin{enumerate}
	\item полоса пропускания на чтение: 50 мегабайт в секунду, полоса пропускания на запись: 25 мегабайт в секунду;
	\item полоса пропускания на чтение: 100 мегабайт в секунду, полоса пропускания на запись: 50 мегабайт в секунду;
	\item полоса пропускания на чтение: 200 мегабайт в секунду, полоса пропускания на запись: 100 мегабайт в секунду.
\end{enumerate}

Для каждой комбинации групп по IOPS и Bandwidth было проведено исследование производительности базы данных PostgreSQL.

\subsection{Результаты}

Без каких-либо ограничений дискового ввода-вывода результат теста производительности равен 3782 транзакций в секунду.

Результаты теста для комбинаций групп представлены в таблице~\ref{tab:results}. Строки~---~группы по полосе пропускания, столбцы~---~по количеству операций в секунду. Единица измерения~---~количество транзакций в секунду.

\begin{table}[H]
\caption{Результаты теста pgbench}
\label{tab:results}
\begin{tabular}{|l|l|l|l|}
\hline
  & 1   & 2    & 3    \\ \hline
1 & 796 & 1227 & 1686 \\ \hline
2 & 823 & 1759 & 2162 \\ \hline
3 & 940 & 1926 & 2546 \\ \hline
\end{tabular}
\end{table}

На основании проведенного теста можно сказать, что ограничения работают и влияют на производительности СУБД PostgreSQL.
    При этом, результаты бенчмарков коррелируют с примененными ограничениями, что может говорить о том, что ограничения применяются правильно. 