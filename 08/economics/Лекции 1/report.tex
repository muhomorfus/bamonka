\documentclass[14pt]{extarticle}
\usepackage[T1]{fontenc}
\usepackage[utf8]{inputenc}
\usepackage{amsmath,amssymb}
\usepackage[russian]{babel}

\begin{document}
\title{\text{Лекции}}
\maketitle

\section{Лекция 1}

Предприниматель~---~ФЛ, несущее полную ответственность за результаты деятельности.

Предпринимательство~---~вид деятельности, предлагают что-то новое, ищут идеи, основная цель~---~прибыль.

Бизнес~---~не одно и то же (здесь повторение, отсутствие изменений).

Предприятия:

\begin{itemize}
	\item Коммерческие
	\begin{itemize}
		\item Хоз. товарищество (полное, коммандитное)
		\item Хоз. общество (ООО, ПАО (ОАО)/НАО (ЗАО), ИП, производственный кооператив)
	\end{itemize}
	\item Некоммерческие
\end{itemize}

Коммерческое предприятие~---~направлено на получение прибыли. 

Некоммерческая организация тоже может заниматься коммерческой деятельностью, но ее прибыль может быть направлена только на ее прямую деятельность. 

Кооператив~---~предприятия на основе частной собственности.

Виды собственности:

\begin{itemize}
	\item Государственная (федеральная, муниципальная)
	\item Частная
	\item Иностранная
\end{itemize}

Предпринимательская деятельность:

\begin{itemize}
	\item Индивидуальная (не ЮЛ)
	\item Коллективная (иное)
\end{itemize}

ОО, АО, ПК могут быть созданы гражданином от 18 лет. В ООО могут участвовать как ФЛ, так и ЮЛ.

ОАО (открытое)~---~можно акции продавать на рынке (все могут купить, но обязан предоставить формы финансовой отчетности в открытом виде), число участников не ограничено, владелец акций~---~полностью принимает все решения, основной документ~---~устав.

ЗАО (закрытое)~---~до 50 участников, ФЛ и ЮЛ, нельзя продавать акции вне компании (есть исключения в уставе, по решению состава акционеров).

ИП~---~может осуществлять предпринимательскую деятельность только после государственной регистрации.

Бухгалтерский баланс:

\begin{itemize}
	\item Активы
	\begin{itemize}
		\item Внеоборотные (нематериальные, основные средства)
		\item Оборотные (запасы, денежные средства, дебетовая задолженность)
	\end{itemize}
	\item Пассивы
	\begin{itemize}
		\item Собственные
		\item Заемные (долгосрочные обязательства, краткосрочные обязательства)
	\end{itemize}
\end{itemize}

Пассивы~---~источники капитала. Активы~---~имущество предприятия. (> А = П)

Внеоборотные~---~используются предприятием > 12 месяцев.

Нематериальные~---~интеллектуальная собственность, патенты, деловая репутация, товврный знак, знак обслуживания.

Основные~---~технологии, делятся на производственные~---~в процессе производства (станки, измерительные), и непроизводственные.

Оборотные~---~до 12 месяцев, быстро используем (запасы~---~материальные ресурсы/сырье)

Дебетовская задолженность~---~мы кредитуем кого-то, нам должны.

Эффективность управления:

\begin{itemize}
	\item Учет
	\item Исследование и анализ (сделать экономическую оценку, рассчитать показатели, +- от 5 лет)
	\item Планирование
	\item Реализация
\end{itemize}

Показатели:

\begin{enumerate}
	\item Чистые активы~---~величина, равная вычитанию из активов, принимаемых к расчету, суммы обязательств (> 0, если нет, то банкрот; в АО и ОО, если более 2-ух фин. лет ЧА < уставного капитала, то ликвидация) (> ЧА = А~---~ЗК)
	\item Собственный оборотный капитал (СКО, рабочий капитал)~---~часть оборотных активов, покрывается СК и долгосрочными обязательствами (> 0, если нет, то КО используется не по назначению, а д.б. для оборотных активов) (> СКО = ОА~---~КО)
	\item Инвестированный капитал~---~для собственной деятельности (> ИК = СК + ДО)
\end{enumerate}

\section{Лекция 2}

Амортизация~---~процесс переноса стаимости основных фондов на себестоимость продукции.

Ей подлежат:

\begin{itemize}
	\item Основные средства (с даты принятия бухучета, прекращение~---~с даты списания, ФСБУ 6/2020)
	\item Нематериальные активы
\end{itemize}

Особенности основных средств (здания, станки):

\begin{itemize}
	\item Используются в процессе производства
	\item Срок > 12 месяцев
	\item В дальнейшем перепродажа
	\item Приносят доход
\end{itemize}

Срок полезного использования~---~параметр для расчета амортизации.

2 варианта:

\begin{itemize}
	\item Во времени, когда приносит доход
	\item В количестве продукции, полученной от использования (натуральное)
\end{itemize}

Первоначальная стоимость основных средств~---~сумма фактических затрат организации на приобретение, сооружение и изготовление, кроме НДС.

Остаточная стоимость основных средств~---~разница между первоначальной стоимостью объекта и его амортизацией в период эксплуатации.

Ликвидационная стоимость~---~вероятная цена, по которой можно продать на открытом рынке.

Способы начисления амортизации:

\begin{itemize}
	\item Линейный~---~исходя из первоначальной стоимости объекта основных средств на начало отчетного периода и норма амортизации (зависит от СПИ).
	
        А\_год = (ПС  Норма) / 100
 
        Норма = 1 / T  100\% => А\_год = ПС  1 / T

        Пример:
        
        Если факт затрат на ПОС = 15 т.р., T = 4 г: Норма = 25\%, А\_год = 3750, А\_мес = 312.5
        
    \item Уменьшаемого остатка~---~начисление исходя из остаточной стоимости основного средства на начало отчетного периода и нормы амортизации.
    
        А\_год = ОС\_ост  НА  Коэф.ускорения / 100:
        
        Пример:
        
        Из предыдущего + КУ = 2
        
        1год: ОС = 15 000 А\_год = 7500 НА = 7500
        
        2год: ОС = 7500 А\_год = 3750 НА = 11 250
        
        3год: ОС = 3750 А\_год = 1875 НА = 13 125
        
        4год: ОС = 1875 А\_год = 1875 НА = 15 000
        
    \item Способ списания стоимости пропорционально объему продукции~---~начисляется из натурального показателя объема продукции в отчетном периоде и предполагаемого отчета продукции за весь срок использвания; первоначальная стоимость делится на предполагаемый объем продукции.
        
        Пример:
        
        Из предыдущего + V = 1000 т., А\_год = 15 руб/т
\end{itemize}

Задача: ПС = 160 000, T = 3 г., СПИ = 10 лет, Сумма А за T = ?

Линейный:

Норма = 10%

А\_год = 16 000

Сумма за T = 48000

Уменьшаемого остатка:

1год: ОС = 160 000 А\_год = 16 000 НА = 16 000

2год: ОС = 144 000 А\_год = 14 400 НА = 30 400

3год: ОС = 129 600 А\_год = 12 960 НА = 43 360


\end{document}