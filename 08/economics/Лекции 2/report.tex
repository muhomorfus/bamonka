\documentclass[14pt]{extarticle}
\usepackage[T1]{fontenc}
\usepackage[utf8]{inputenc}
\usepackage{amsmath,amssymb}
\usepackage[russian]{babel}

\begin{document}
\title{\text{Лекции}}
\maketitle

\section{Лекция 3}

Финансовое состояние предприятия~---~сложная экономическая категория, отражающая способность субъекта хозяйствования финансировать свою деятельность и вовремя платить по обязательствам на определенный момент.

Обязательства:

\begin{enumerate}
	\item платить ЗП
	\item оплачивать поставки сырья
	\item расчет с кредиторами
	\item обязательные платежи + налоги
\end{enumerate}

ФСП: устойчивое, неустойчивое, предкризисное, кризисное.

Фин. устойчивость~---~внутреннее состояние предприятия. Платежеспособность~---~внешний показатель финансовой устойчивости.

Чем обеспечивается устойчивость? Баланс активов и пассивов, баланс доходов и расходов и денежные потоки.

Уровни управления предприятием:

\begin{enumerate}
	\item стратегический~---~работа на перспективу
	\item тактический~---~связывает различные уровни политики предприятия 
	\item оперативный~---~проектирование, изготовление, реализация продукции
\end{enumerate}

Факторы финансовой устойчивости:

\begin{enumerate}
	\item положение на рынке
	\item степень охвата рынка
	\item производство и выпуск дешевой и качественной продукции
	\item степень зависимости от внешних кредиторов и инвесторов наличие платежеспособных дебеторов
	\item эффективность хозяйских фин. операций
\end{enumerate}

Платежеспособность~---~способность предприятия своевременно полностью выполнять платежные обязательства, вытекающие из торговых, кредитных и иных операций платежа.

Виды плат-ти: случайная, временная, длительная, хроническая.

Причины снижения:

\begin{enumerate}
	\item недостаточная обеспеченность финансовыми ресурсами
	\item неоп. структура оборотных средств
	\item несвоевременное поступление платежей по контрактам
\end{enumerate}

Высшая форма устойчивости~---~возможность развиваться.

Ликвидность~---~характеристика отдельных видов активов предприятия по их способности к быстрому превращению в денежную форму без потери балансовой стоимости с целью обеспечения необходимого уровня платежеспособности.

Что характеризует ФУ и Л:

\begin{enumerate}
	\item М-СК~---~показатель обратной фин. зависимости, в какой степени предприятие зависит от фнешних источников фин. (рост~---~негативная тенденция, недостаточность собственного капитала) К-АВТ~---~обратный М-СК (д.б. > 0.5, но если около 1~---~то сдерживание развития)
	\item К-ФИН~---~рекомендуется < 0.7, если < 0.5~---~тоже нормально~---~снижение зависимости от внешних источников
	\item К-МАН~---~степень мобильности использования СК, д.б. около 0.3-0.6, если снижается~---~то вложение средств в трудноликвидный актив и формирование ОА за счет займов, показывает независимость
	\item К-СОК~---~д.б. ~0.3, характеризует степень обеспеченности оборотных активов собственными оборотными средствами, позволяет отследить снижение фин. устойчивости
	\item Л-АБС~---~д.б. 0.2-0.3~---~показывает, какую часть своих КО можно покрыть за счет имеющихся ср-в, если выше~---~то избыток средств~---~мало прибыли
	\item Л-СРОЧ~---~предел 0.7-0.8~---~показывает, какую часть КО может заплатить без продажи запасов, если около 1~---~плохо, тк рост дебетовой задолженности
	\item Л-ТЕК~---~предел 2-2.5~---~если больше, то ОА много и они неэффективно используются~---~если ниже, то есть высокий финансовый риск, трудности в сбыте, плохая организация
\end{enumerate}


\section{Лекция 4}

Доходы~---~увеличение экономической выгоды в результате поступления активов или погашения обязательств, приводит к увеличению капитала, за исключением вкладов собственников.

Расходы~---~уменьшение экономических выгод в результате убыточных активов или возникновения обязательств, приводящих к уменьшению капитала за исключением вкладов участников.

Статьи доходов:

\begin{enumerate}
	\item по обычным видам деятельности аренда
	\item лицензионные платежи
\end{enumerate}

Статьи доходов/расходов:

\begin{enumerate}
	\item операционные
	\item внереализационные
	\item прибыль прошлых лет
\end{enumerate}

Денежный поток (CF)~---~отражает распределение по времени денежных поступлений и платежей предприятия, определяемые для всего расчетного периода (привязка к определенному периоду времени).

Денежный поток характеризуется:

\begin{enumerate}
	\item притоком
	\item оттоком
	\item сальдо~---~разница между притоком и оттоком
\end{enumerate}


Выгрузка~---~все, что произвело предприятие, умноженное на цену.

Денежные потоки:

\begin{enumerate}
	\item операционные (CFO) $CFO = B - C + A_{m} - P_{comm} - P_{upr} - M_{p}$
	\item инвестиционные (CFI) $CFI = -delta(BA) - delta(COK)$
	\item финансовые (CFF) $CFF = delta(DO) = P_{pr}$
\end{enumerate}

\section{Лекция 5}

Получение прибыли~---~залог успеха бизнеса. Необходимое, но не достаточное условие для развития предпринимательской деятельности. Нужно распределить прибыль по заранее определенным целям. 

Виды:

\begin{enumerate}
	\item выручка~---~объем * цена/ед (В) 
	\item валовая прибыль (ПВал) 
	\item прибыль от продаж (Ппрод) 
	\item прибыль до налогообложения 
	\item чистая прибыль
\end{enumerate}

Чистая прибыль => распределение прибыли: реинвестирование в предприятие, дивиденды.

Повышение рыночной стоимости предприятия: повышение привлекательности для инвесторов, кредиторов или/и для последующей продажи.

Раньше 50\% ЧП отправлялось на реинвестирование, а сейчас предприятия самостоятельно определяют распределение ЧП.

Степени свободы:

\begin{enumerate}
	\item выбор номенклатуры и ассортимента
	\item формирование объема продукции
	\item формирование затрат
	\item установка цен
\end{enumerate}

\end{document}