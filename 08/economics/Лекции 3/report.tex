\documentclass[14pt]{extarticle}
\usepackage[T1]{fontenc}
\usepackage[utf8]{inputenc}
\usepackage{amsmath,amssymb}
\usepackage[russian]{babel}

\begin{document}
\title{\text{Лекции}}
\maketitle

\section*{Лекция 6}
Ресурсы:
\begin{itemize}
  \item Материальные
  \item Амортизация
  \item Затраты живого труда (З/П)
\end{itemize}

Основная задача — изучение динамики показателей, выявление тенденций и развития, составление плана.

\subsection{Перспективные прогнозные показатели ресурсов}
Объем выпуска растет — амортизация растет. Неправильный подход: амортизация растет, а выручка падает.

Фондоотдача: f = В/ОС, В — выручка, ОС — основные средства. Надо смотреть за несколько лет.

Факторы:
\begin{itemize}
  \item Интенсивные (улучшение качества ресурсов, коэффициент сложности).
  \item Экстенсивные (увеличение объема ресурсов).
  \item Социальные (повышение квалификации).
  \item Территориальные.
\end{itemize}

Материалоотдача: $\mu$ = В/М.

Материалоемкость — количество материала на единицу продукции.

Построение факторной модели зависит от уровня расходов на 1 рубль продукции. С = С/В = (М + З + Сотр + А + Р)/В, М — материалы, З — затраты, А — амортизация, Р — расходы.

Производительность труда растет — зарплата растет.

Рабочие:
\begin{itemize}
  \item Основные
  \item Вспомогательные
\end{itemize}

\section*{Лекция 7}
Операционный и финансовый циклы тесно взаимосвязаны. Предприятие заинтересовано в снижении длительности циклов. Операционный и финансовый циклы рассматриваются в динамике. Снижение = эффективность.

Период оборота кредиторской задолженности должен быть больше периода оборота дебиторской задолженности. 

Операционный цикл — характеризует производственно-технологическую деятельность предприятия. Это период времени между покупкой запасов и получением денег. В состав операционного цикла сходит финансовый цикл. 

Финансовый цикл — это период между отгрузкой продукции и оплатой. Это период времени между оплатой кредиторской задолженности и моментом получения денег за готовую продукцию.

Ц{фин} = Ц{опец} - Т{общ}

Положительная тенденция — снижения периода оборота кредиторской и дебиторской задолженности. 

Предприятие может быть и кредитором и дебитором. 

Запасы: сырье, материалы.

Положительная тенденция — снижение T{обзап}. 

Финансовый цикл может быть как положительным, так и отрицательным. Если он отрицательный, то это значит, что у предприятия достаточно своих средств для обеспечения своих собственных производственных интересов. 

Причины снижения/увеличения продолжительности финансового цикла:
\begin{itemize}
  \item Ускорение производственных процессов.
  \item Ускорение оборачиваемости дебиторской задолженности.
  \item Замедление сроков погашения дебиторской задолженности.
\end{itemize}

\section*{Лекция 8}
Рентабельность позволяет выявить тенденции развития предприятия (развивается или нет). Рентабельность — это относительный показатель уровня доходности предприятия. Характеризует эффективность работы предприятия в целом и доходность отдельных направлений деятельности (производственная, коммерческая, инвестиционная). Рентабельность более полно отражает окончательный результат хозяйствования, так как показывает соотношение эффекта чистой прибыли.

Виды:
Рентабельность продаж ROS = П{чист}/В. Характеризует чистую прибыль на 1 рубль выручки.

Рентабельность активов ROA = П{чист}/{А}. Характеризует чистую прибыль на 1 рубль средств, вложенных в предприятие.

Рентабельность собственного капитала ROE = П{чист}/{СК}. Характеризует чистую прибыль на 1 рубль собственного капитала. Показывает, насколько эффективно используется собственный капитал.

Рентабельность инвестиционного капитала ROI = П{чист}/{ИК}

\subsection{Двухфакторная модель рентабельности}
Отражает взаимосвязь между источниками финансирования. Факторы:

1. Эффективная ценовая политика

2. Контроль за расходами

3. Повышение эффективности активов

4. Контроль за использованием кредитов

\subsection{Оценка эффективности проекта}
Ключевой показатель для оценки денежных потоков — NPV — чистая приведенная стоимость. Шаги для расчета:

1. Продумать первоначальные инвестиции в проект
2. Спланировать денежные поступления от проекта
3. Расcчитать стоимость СК в форме ставки дисконтирования
4. Суммировать дисконтированные денежные потоки

$$
NPV = -IC + \sum{t = 1}^{T}{\frac{CFt}{(1+i)^t}}
$$

IC — первоначальные инвестиции, i — ставка дисконтирования.

\subsection{Пример}
Предприятие планирует вложиться в проект А. Первоначальные инвестиции IC = 200 млн рублей. Ежегодный доход — 30 процентов от суммы капиталовложений. Срок реализации проекта 6 лет. Издержки 10 процентов. Расcчитать NPV.

\begin{center}
\begin{tabular}{|l|c|c|c|c|c|c|}
\hline
Показатели          & 0    & 1   & 2   & 3   & 4   & 5   \\
\hline
Поступления (приток) & 0    & 60  & 60  & 60  & 60  & 60  \\
Платежи (отток)     & 200  & 0   & 0   & 0   & 0   & 0   \\
Итого (денежный поток) & -200 & 60  & 60  & 60  & 60  & 60  \\
Коэффициент дисконтирования & 1    & 0.91 & 0.83 & 0.75 & 0.68 & 0.62 \\
Дисконтированный поток & -200 & 54.6  & 49.8  & 45   & 40.8 & 37.2 \\
Д. поток нараст. ит. & -200 & -145.4 & -95.6 & -50.6 & -9.8 & 27.4 \\
\hline
\end{tabular}
\end{center}

NPV = 27.4 > 0 — Проект экономически эффективен.

\end{document}