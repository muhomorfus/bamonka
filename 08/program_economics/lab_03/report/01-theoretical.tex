\chapter{Основное задание}

\section{Содержание проекта}

Команда разработчиков из 16 человек занимается созданием карты города на основе собственного модуля отображения. Проект должен быть завершен в течение 6 месяцев. Бюджет проекта: 50 000 рублей.

\section{Задание 1: Выравнивание загрузки ресурсов в проекте}

Затраты проекта составляют 48 286 руб. при трудозатратах в 9473 часа.  Визуальный оптимизатор показал перегрузки для: 

\begin{enumerate}

\item Системного аналитика~---~он одновременно задействован на <<Анализ и построение структуры базы объектов>> и <<Анализ и проектирование ядра>>.

\begin{figure}[H]
    \begin{center}
    \includegraphics[width=1\linewidth]{assets/load_sys.png}
    \caption{Перегрузка системного аналитика}
    \label{fig:2}
    \end{center}
\end{figure}

\item Художника-дизайнера~---~он одновременно задействован на <<Разработка дизайна руководства>> и <<Разработка дизайна сайта>>.

\begin{figure}[H]
    \begin{center}
    \includegraphics[width=1\linewidth]{assets/load_des.png}
    \caption{Перегрузка дизайнера}
    \label{fig:2}
    \end{center}
\end{figure}

\item Технического писателя~---~он одновременно задействован на <<Написание руководства пользователя>> и <<Создание справочной системы>>.

\begin{figure}[H]
    \begin{center}
    \includegraphics[width=1\linewidth]{assets/load_wrt.png}
    \caption{Перегрузка технического писателя}
    \label{fig:2}
    \end{center}
\end{figure}

\end{enumerate}

Для устранения перегрузок принято решение выполнить автоматическое выравнивание. Для этого в панели «Ресурс» был выбран пункт «Параметры выравнивания». Настройки:

\begin{figure}[H]
    \begin{center}
    \includegraphics[width=1\linewidth]{assets/nastr_1.png}
    \caption{Настройки выравнивания}
    \label{fig:2}
    \end{center}
\end{figure}

«Поиск превышений доступности» выставленный в формате «по неделям» позволяет запланировать рабочие дни с длительностью более 8 часов, оставив при этом полностью свободными другие дни в качестве компенсации. Такой вариант не подходит, так как работы более 8 часов в день в данном случае недопустимы. Было выбрано выравнивание по часам.

Чтобы наглядно посмотреть изменения, произошедшие в результате выравнивания, лучше всего выбрать диаграмму Ганта с выравниванием.

\begin{figure}[H]
    \begin{center}
    \includegraphics[width=0.8\linewidth]{assets/gant_1.png}
    \caption{Результат выравнивания}
    \label{fig:2}
    \end{center}
\end{figure}

Статистика по проекту после выравнивания:

\begin{figure}[H]
    \begin{center}
    \includegraphics[width=0.8\linewidth]{assets/stat_1.png}
    \caption{Результат выравнивания}
    \label{fig:2}
    \end{center}
\end{figure}

После применения на диаграмме Ганта с выравниванием видно, что часть задач сдвинулась, дата окончания проекта стала 23.09. Затраты проекта уменьшились на 144 рубля и составляют теперь 48 142 руб., трудозатраты~---~9416 часов. 

Microsoft Project при выравнивании стремится не затрагивать критический путь. Но в конце проекта есть две параллельные задачи на дизайнере, лежащие на критическом пути. Из-за сдвига этих задач увеличился срок проекта.

\section{Задание 2: Учет периодических задач в плане проекта}

Создана повторяющаяся задача через меню <<Задача>> $\rightarrow$ <<Повторяющаяся задача>>. При этом выбран режим переноса задачи на ближайший день в случае выпадения на выходной.

\begin{figure}[H]
    \begin{center}
    \includegraphics[width=0.8\linewidth]{assets/repeat.png}
    \caption{Настройка повторяющейся задачи}
    \label{fig:2}
    \end{center}
\end{figure}

Совещание появилось в листе задач.

\begin{figure}[H]
    \begin{center}
    \includegraphics[width=0.85\linewidth]{assets/repeated.png}
    \caption{Лист задач}
    \label{fig:2}
    \end{center}
\end{figure}

\newpage

С предварительной фиксацией по длительности, на него были назначены сотрудники.

\begin{figure}[H]
    \begin{center}
    \includegraphics[width=0.85\linewidth]{assets/repeat_naznach.png}
    \caption{Назначение на повторяющуюся задачу}
    \label{fig:2}
    \end{center}
\end{figure}

Затраты только на совещания получаются 20 039 руб., а затраты на весь проект~---~68 171 руб. Кроме этого, происходит перегрузка, так как совещания происходят во время рабочего дня. Затраты выходят большими, так как неверно вычисляются фиксированные затраты~---~на каждое совещание нет смысла совершать фиксированные затраты. Это можно поменять, изменив план затрат.

Создадим план B c нулевыми фиксированными затратами на трудовые ресурсы.

\begin{figure}[H]
    \begin{center}
    \includegraphics[width=0.5\linewidth]{assets/planb.png}
    \caption{План B}
    \label{fig:2}
    \end{center}
\end{figure}

Во вкладке <<Использование задач>> нужно проставить в столбец <<Таблица норм затрат>> план B для совещаний, чтобы исключить фиксированные трудозатраты.

\begin{figure}[H]
    \begin{center}
    \includegraphics[width=0.7\linewidth]{assets/task_usage.png}
    \caption{Применение плана B}
    \label{fig:2}
    \end{center}
\end{figure}

Теперь затраты проекта~---~49 901 руб., что помещается в бюджет.

\begin{figure}[H]
    \begin{center}
    \includegraphics[width=0.7\linewidth]{assets/after_sov.png}
    \caption{Результат применения плана B}
    \label{fig:2}
    \end{center}
\end{figure}

\section{Задание 3: Оптимизация критического пути}

В окне <<Диаграмма Ганта с отслеживанием>> в режиме <<Вид>> выполняется фильтрация и сортировка по критическим задачам, чтобы определить, какие из них занимают больше времени.

\begin{figure}[H]
    \begin{center}
    \includegraphics[width=0.85\linewidth]{assets/monit.png}
    \caption{Критический путь}
    \label{fig:2}
    \end{center}
\end{figure}

В визуальном оптимизаторе ресурсов можно заметить, что программисты используются не равномерно. Их можно перепланировать вручную.

Для оптимизации критического пути и сроков выполнения проекта были внесены следующие изменения.

\begin{enumerate}
	\item На создание рабочей версии ядра добавили программистов 3 и 4.
	\item На анализ и проектирование ядра добавили программиста 1, 2, 3, 4.
	\item На создание модели ядра добавили программистов 2, 3, 4.
	\item На тестирование сайта добавили программистов 3 и 4.
	\item На программирование средств обработки базы объектов добавили программистов 1, 2 и ведущего.
	\item На тестирование модели ядра добавили программистов 2, 3, 4 и ведущего.
\end{enumerate}

% , на программирование средств обработки базы объектов, и тестирование рабочей версии поставили всех программистов.

В результате удалось сократить сроки проекта.

\begin{figure}[H]
    \begin{center}
    \includegraphics[width=0.85\linewidth]{assets/final_res.png}
    \caption{Финальная статистика}
    \label{fig:2}
    \end{center}
\end{figure}

Соотношения «Трудозатраты — Затраты» по сравнению с прошлой лабораторной работой изменилось слабо.

На рисунке \ref{fig:zatraty} представлена круговая диаграмма затрат по группам ресурсов.

\begin{figure}[H]
    \begin{center}
    \includegraphics[width=0.5\linewidth]{assets/zatr.png}
    \caption{График затрат}
    \label{fig:zatraty}
    \end{center}
\end{figure}

На рисунке \ref{fig:trud} представлена круговая диаграмма трудозатрат по группам ресурсов.

\begin{figure}[H]
    \begin{center}
    \includegraphics[width=0.5\linewidth]{assets/trud.png}
    \caption{График трудозатрат}
    \label{fig:trud}
    \end{center}
\end{figure}

