\chapter{Тренировочное задание}

\section{Задание (вариант 1)}

Осуществить планирование проекта со следующими временными характеристиками:

\begin{figure}[H]
    \begin{center}
    \includegraphics[width=0.5\linewidth]{assets/prepare_task.png}
    \caption{Тренировочное задание}
    \label{fig:1}
    \end{center}
\end{figure}

Дата начала проекта~---~1-й рабочий день марта текущего года. Провести планирование работ проекта, учитывая следующие связи между задачами:

\begin{enumerate}
	\item Предусмотреть, что D – исходная работа проекта;
	\item Работы С, E и F начинаются сразу по окончании работы D;
	\item Работы A и J следуют за C, а работа G – за F;
	\item Работа I следует за A, а работа B – за G;
	\item Работа H начинается после завершения E, но не может начаться, пока не завершены I и B.
\end{enumerate}

\newpage

\section{Выполнение задания}

В качестве типа задач по умолчанию был выбран фиксированный объем ресурсов. 
Это означает, что при увеличении трудозатрат будет увеличена длительность работ.

На рисунке \ref{fig:prepare_result} приведены результаты выполнения тренировочного задания. Новые задачи назначаются на дату начала проекта, длительность выражается в днях.

\begin{figure}[H]
    \begin{center}
    \includegraphics[width=1\linewidth]{assets/prepare_result.png}
    \caption{Результаты выполнения тренировочного задания}
    \label{fig:prepare_result}
    \end{center}
\end{figure}

\chapter{Основное задание}

\section{Содержание проекта}

Команда разработчиков из 16 человек занимается созданием карты города на основе собственного модуля отображения. Проект должен быть завершен в течение 6 месяцев. Бюджет проекта: 50 000 рублей.

\section{Задание 1: Настройка рабочей среды проекта}

Установлена дата начала проекта --- 1 марта 2024 года и стандартный календарь рабочего времени. 

\begin{figure}[H]
    \begin{center}
    \includegraphics[width=1\linewidth]{assets/start.png}
    \caption{Настройка календаря проекта}
    \label{fig:2}
    \end{center}
\end{figure}

Установлены длительность работы в неделях, объем работ в часах, а тип работ по умолчанию --- с фиксированными трудозатратами, количество рабочих часов в день равным 8, количество рабочих часов в неделю равным 40. Задано начало рабочей недели в понедельник, а финансового года --- в январе, продолжительность рабочего дня с 9 до 18 часов. 

\begin{figure}[H]
    \begin{center}
    \includegraphics[width=1\linewidth]{assets/calendar.png}
    \caption{Настройка календаря проекта}
    \label{fig:2}
    \end{center}
\end{figure}

Отмечены выходные и праздничные дни на 2024 год. 

\begin{figure}[H]
    \begin{center}
    \includegraphics[width=1\linewidth]{assets/holidays.png}
    \caption{Настройка выходных дней}
    \label{fig:2}
    \end{center}
\end{figure}

Выведена на экран суммарная задача проекта и заполнена вкладка заметок информацией об основных параметрах проекта (его длительности, бюджете и количественном составе команды). 

\begin{figure}[H]
    \begin{center}
    \includegraphics[width=1\linewidth]{assets/main.png}
    \caption{Вывод суммарной задачи}
    \label{fig:2}
    \end{center}
\end{figure}

\newpage

\section{Задание 2: Создание списка задач}

Был введен набор задач.

\begin{figure}[H]
    \begin{center}
    \includegraphics[width=1\linewidth]{assets/length.png}
    \caption{Задачи}
    \label{fig:2}
    \end{center}
\end{figure}

\section{Задание 3: Структурирование списка задач}

Список был структурирован с помощью изменения иерархии. 

\begin{figure}[H]
    \begin{center}
    \includegraphics[width=1\linewidth]{assets/nesting.png}
    \caption{Структурированный список задач}
    \label{fig:2}
    \end{center}
\end{figure}

Задачи 1 и 27 являются задачами вехами, поэтому они имеют нулевую продолжительность. Задачи 2, 3, 8, 12, 19 и 22 преобразованы в фазы проекта~---~их длительность стала соответствовать длительности наиболее трудоёмкой задачи в этой группе.

\newpage

\section{Задание 4: Установление связей между задачами}

Для задач были установлены связи.

\begin{figure}[H]
    \begin{center}
    \includegraphics[width=0.9\linewidth]{assets/full.png}
    \caption{Список задач с установленными связями}
    \label{fig:2}
    \end{center}
\end{figure}

После установки связи между задачами изменилась длительность суммарных задач, она стала равняться 27.6 недель. Окончанием проекта в этом случае стала дата 19 сентября 2024 года.