\documentclass[14pt]{extarticle}
\usepackage[T1]{fontenc}
\usepackage[utf8]{inputenc}
\usepackage{amsmath,amssymb}
\usepackage[russian]{babel}
\usepackage{geometry}
\geometry{a4paper,
total={170mm,257mm},left=2cm,right=1cm,
top=1.5cm,bottom=1.5cm}

\usepackage{tabularx}
\usepackage{tikz}

\newcommand{\podpis}[2]{
    \parbox[b]{6cm}{#1}
    \hspace{1cm}
    \tikz[baseline=2pt]{\draw(0,0) to node[below=-2pt]{\scriptsize подпись}(5cm,0);}
    \hspace{1cm}
    \tikz[baseline=2pt]{
        \def\familywidth{\textwidth-6cm-1cm-5cm-1cm-20pt}
        \draw(0,0) to node[below=-2pt]{\scriptsize дата}(\familywidth,0);
        \node[anchor=west](f) at (25pt,8pt){#2};
    }
}

\begin{document}

\pagestyle{empty}

\begin{center}
\section*{РЕЦЕНЗИЯ}
на выпускную квалификационную работу бакалавра

Княжева Алексея Викторовича

<<Метод ограничения пропускной способности дискового ввода-вывода в системе оркестрации приложений Kubernetes>>
\end{center}	

Kubernetes может быть использован для эксплуатации баз данных. При этом, для для поддержки промышленных СУБД, хранящих данные на дисковых устройствах с различной скоростью доступа к данным необходимо ограничивать разделяемые ресурсы, к которым относится и дисковый ввод-вывод. Работа актуальна, так как сейчас не существует готовых стабильных решений, позволяющих вводить такие ограничения в промышленных масштабах.

В аналитической части квалификационной работы проведен обзор существующих решений и постановка задачи.

В конструкторской части произведен выбор метода хранения информации об ограничениях дискового ввода-вывода, а также приведен алгоритм применения ограничений.

В технологической части произведен выбор средств реализации, показано разработанное программное обеспечение и методы его тестирования. 

В исследовательской части проведено нагрузочное тестирования СУБД \\ PostgreSQL с расличными ограничениями дискового ввода-вывода. По данным исследования видно, что результаты коррелируют с установленными ограничениями.

К достоинствам разработанного ПО можно отнести возможность указания различных лимитов для разных контейнеров и томов в рамках одного экземпляра приложения, а также возможность изменения лимитов без перезапуска приложения.

К недостаткам метода можно отнести указание ограничений в отдельности от описания приложения. Но этот недостаток является следствием ограничений Kubernetes, поэтому он является несущественным.

Считаю, что выпускная квалификационная работа Княжева~А.~В. <<Метод ограничения пропускной способности дискового ввода-вывода в системе оркестрации приложений Kubernetes>> соответствует квалификационным требованиям, предъявляемым к выпускной квалификационной работе бакалавра, заслуживает оценки <<отлично>>, а Княжев~А.~В.~---~присвоения степени бакалавра по направлению подготовки 09.03.04 <<Программная инженерия>>.

\vspace{1.5cm}

\setlength{\parindent}{0pt}

\podpis{Рецензент:\\Старший инженер\\ООО <<Авито Тех>>\\Жига Никита Викторович}{29.05.2024}\\

\end{document}