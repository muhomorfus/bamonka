\chapter{Проведение конкурса}

\section{Оценка по результатам очной проверки}
Для дальнейшей оценки студенческого жюри необходимо внимательно изучить работы участников конкурса и заполнить таблицы с баллами. Для этого все студенты-организаторы разбиваются на группы по 2 человека, чтобы оценить всех абитуриентов.

Сначала происходит ознакомление с докладом автора, опрос по теме выступления. Далее необходимо оценить в работе следующее:

\begin{enumerate}
	\item структуру и оформление работы;
	\item актуальность тематики работы;
	\item полноту раскрытия темы;
	\item логику изложения, оригинальность мышления;
	\item используемые методы и обоснование их использования;
	\item наличие в тексте работы заимствований из источников, в том числе из ресурсов сети Интернет;
	\item наличие предложений по практическому использованию программы;
	\item вклад автора в выбранную тему.
\end{enumerate}

Работая в группе с Екатериной Карповой, мы просмотрели и оценили 6 работ абитуриентов.

Была составлена результирующая таблица с оценками участников, со стороны как студенческого, так и преподавательского жюри. После подсчета набранных участниками конкурса баллов, были определены три победителя этапа олимпиады «Шаг в будущее» и один участник, выбранный студенческим жюри. Призерам была вручена научная литература, выпущенная в издательстве МГТУ им. Н.Э. Баумана.


\section{Сведения об участнике и работе}

Студенческому жюри были представлены работы участников конкурса, после чего была произведена их оценка. Далее будет представлен анализ одной из работ.

Рассматриваемая далее работа была выполнена Решетниковым Федором Владимировичем, учеником ГБОУ Инженерная Школа 1581. Тема работы, представленная им на конкурсе: «Создание сервиса для определения оптимального времени для проведения события».

По словам участника, цель работы~---~<<сделать приложение для оптимизации выбора группой времени для встречи или событий, с учётом желаний потенциальных пользователей.>>

Участник утверждает: <<Во время карантина 2021 года компании и команды стали работать удалённо и виделись через камеру. После огромное количество команд осталось работать удалённо, не требует затрат на офис, да и все успели привыкнуть. В таких условиях еще больше участились случаи когда нужно в онлайн договориться, когда провести созвон или любое другое событие.  Например команде стартапа нужно решить, когда провести созвон, вне расписания, что бывает довольно часто. Или работникам компании с плотным графиком, нужно разобраться, когда лучше провести совещание. Также это может быть нужно даже вне работы например сбор бывших одноклассников на встречу или друзей на день рождения. Чисто переписываться муторно и долго, т.к. о времени все могут писать по-разному, а считать всё нужно в ручную.>> Орфография и пунктуация автора сохранены.

Заявлена следующая работа ПО: совместное планирование событий, создание ссылок на события.

\begin{table}[h!]
  \centering
  \begin{tabular}{p{1\linewidth}}
    \centering
    \includegraphics[width=1\linewidth]{./images/example.pdf}
    \captionof{figure}{Пример работы ПО}
    \label{img:func}
  \end{tabular}
\end{table}

\newpage

\section{Результаты очной проверки}

Большая часть работы не была выполнена самим участником (CI/CD, бекенд были сделаны не участником). Актуальность работы не обоснована, не рассмотрены такие очевидные аналоги, как Google Calendar. На защите участник не ориентировался в коде. РПЗ содержит огромное количество орфографических, грамматических, лексических и пунктуационных ошибок, не выдержан стиль речи.

\newpage