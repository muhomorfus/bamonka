\setcounter{page}{3}

\chapter{Практическая часть}

\section{Задание 1}
\subsection*{Задание}
Напишите функцию, которая уменьшает на 10 все числа из списка-аргумента этой
функции, проходя по верхнему уровню списковых ячеек. (* Список смешанный структурированный)

\subsection*{Решение}
\begin{code}
\begin{minted}{lisp}
(defun f1 (x)
	(mapcar #'(lambda (el) (if (numberp el) (- el 10) el)) x)
)
\end{minted}
\end{code}

\section{Задание 2}
\subsection*{Задание}
Написать функцию которая получает как аргумент список чисел, а возвращает список квадратов этих чисел в том же порядке.

\subsection*{Решение}
\begin{code}
\begin{minted}{lisp}
(defun f2 (x)
	(mapcar #'(lambda (el) (* el el)) x)
)
\end{minted}
\end{code}

\section{Задание 3}
\subsection*{Задание}
Напишите функцию, которая умножает на заданное число-аргумент все числа из заданного списка-аргумента, когда:
\begin{enumerate}
	\item все элементы списка~---~числа;
	\item элементы списка~---~любые объекты.
\end{enumerate}

\subsection*{Решение}
Приведен вариант 2. Для первого варианта убрать проверку на число.

\begin{code}
\begin{minted}{lisp}
(defun f3 (x n)
	(mapcar #'(lambda (el) (if (numberp el) (* el n) el)) x)
)
\end{minted}
\end{code}

\section{Задание 4}
\subsection*{Задание}
Написать функцию, которая по своему списку-аргументу lst определяет является ли он палиндромом (то есть равны ли lst и (reverse lst)), для одноуровнего смешанного списка.

\subsection*{Решение}
\begin{code}
\begin{minted}{lisp}
(defun my_and (x y) (and x y))
(defun f4 (lst)
        (reduce #'my_and (mapcar #'equalp lst (reverse lst)))
)
\end{minted}
\end{code}

\section{Задание 5}
\subsection*{Задание}
Используя функционалы, написать предикат set-equal, который возвращает t, если два его множества-аргумента (одноуровневые списки) содержат одни и те же элементы, порядок которых не имеет значения.

\subsection*{Решение}
\begin{code}
\begin{minted}{lisp}
(defun in_element (e x)
	(reduce
		#'(lambda (res elem) (or res (equal elem e)))
		x
		:initial-value NIL
	)
)

(defun in (x y)
	(reduce
		#'my_and
		(mapcar #'(lambda (elem) (in_element elem y)) x)
		:initial-value T
	)
)

(defun set-equal (x y)
	(and (in x y) (in y x))
)
\end{minted}
\end{code}

\section{Задание 6}
\subsection*{Задание}
Напишите функцию, select-between, которая из списка-аргумента, содержащего только числа, выбирает только те, которые расположены между двумя указанными числами - границами-аргументами и возвращает их в виде списка (упорядоченного по возрастанию (+ 2 балла)).

\subsection*{Решение}
\begin{code}
\begin{minted}{lisp}
(defun f5 (x a b)
	(reduce
		#'(lambda (res elem) 
			(if (< a elem b) (append res (cons elem nil)) res)
		)
		x
		:initial-value ()
	)
)
\end{minted}
\end{code}

\section{Задание 7}
\subsection*{Задание}
Написать функцию, вычисляющую декартово произведение двух своих списков- аргументов. (Напомним, что А х В это множество всевозможных пар (a b), где а принадлежит А, b принадлежит В).

\subsection*{Решение}
\begin{code}
\begin{minted}{lisp}
(defun f6 (x y)
    (reduce
        #'append
        (mapcar
            #'(lambda (elem1)
                (mapcar #'(lambda (elem2) (cons elem1 elem2)) y)
            )
            x
        )
        :initial-value ()
    )
)
\end{minted}
\end{code}

\section{Задание 8}
\subsection*{Задание}
Почему так реализовано reduce, в чем причина? 

\begin{code}
\begin{minted}{lisp}
(reduce #'+ ()) -> 0
\end{minted}
\end{code}

\begin{code}
\begin{minted}{lisp}
(reduce #'* ()) -> 1
\end{minted}
\end{code}

\subsection*{Решение}
В данном случае, \texttt{reduce} применяет функцию-аргумент к нулевому количеству аргументов. Так как при отсутствии аргументов, функции \texttt{+} и \texttt{*} возвращают нейтральные элементы, то \texttt{reduce} возвращает соответствующее значение.

\section{Задание 9}
\subsection*{Задание}
* Пусть list-of-list список, состоящий из списков. Написать функцию, которая вычисляет сумму длин всех элементов list-of-list (количество атомов), т.е. например для аргумента
((1 2) (3 4)) -> 4.

\subsection*{Решение}
\begin{code}
\begin{minted}{lisp}
(defun f (l)
	(reduce
		(lambda (res elem) (+ res (length elem)))
		l
		:initial-value 0
	)
)
\end{minted}
\end{code}

\newpage