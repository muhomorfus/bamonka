\setcounter{page}{3}

\chapter{Практическая часть}
\section{Задание}
Используя хвостовую рекурсию, разработать (комментируя назначение
аргументов) эффективную программу , позволяющую:
\begin{enumerate}
	\item Найти длину списка (по верхнему уровню);
	\item Найти сумму элементов числового списка;
	\item Найти сумму элементов числового списка, стоящих на нечетных позициях исходного списка (нумерация от 0);
	\item Сформировать список из элементов числового списка, больших заданного значения;
	\item Удалить заданный элемент из списка (один или все вхождения);
	\item Объединить два списка.
\end{enumerate}

Убедиться в правильности результатов.
Для одного из вариантов ВОПРОСА уметь составить таблицу, отражающую конкретный порядок работы системы
  
\begin{lstlisting}[label=div,caption=Реализация программы на языке Prolog]
domains 
 list = integer*.

predicates
 nondeterm len1(list, integer, integer).
 nondeterm len(list, integer).
 
 nondeterm sum1(list, integer, integer).
 nondeterm sum(list, integer).
 
 nondeterm sum_odd_pos1(list, integer, integer).
 nondeterm sum_odd_pos(list, integer).
 
 nondeterm list_bigger(list, integer, list).
 
 nondeterm rm_all(list, integer, list).
 nondeterm rm_one(list, integer, lLent).
 
 nondeterm union(list, list, list).
 
clauses
 len1([], Acc, Acc) :- !.
 len1([_ | T], Acc, Res) :- Acc1 = Acc + 1, len1(T, Acc1, Res).
 len(L, Res) :- len1(L, 0, Res).
 
 sum1([], Acc, Acc) :- !.
 sum1([H | T], Acc, Res) :- Acc1 = Acc + H, sum1(T, Acc1, Res).
 sum(L, Res) :- sum1(L, 0, Res).
 
 sum_odd_pos1([], Acc, Res) :- Res = Acc, !.
 sum_odd_pos1([_, H | T], Acc, Res) :- Acc1 = Acc + H, sum_odd_pos1(T, Acc1, Res).
 sum_odd_pos(L, Res) :- sum_odd_pos1(L, 0, Res).
 
 list_bigger([], _, []) :- !.
 list_bigger([H | T], N, [H | ResT]) :- H > N, !, list_bigger(T, N, ResT).
 list_bigger([_ | T], N, Res) :- list_bigger(T, N, Res).
 
 rm_all([], _, []) :- !.
 rm_all([H | T], N, Res) :- H = N, !, rm_one(T, N, Res).
 rm_all([H | T], N, [H | ResT]) :- rm_one(T, N, ResT).
 
 rm_one([], _, []) :- !.
 rm_one([H | T], N, T) :- H = N, !.
 rm_one([H | T], N, [H | ResT]) :- rm_one(T, N, ResT).
 
 union([], L, L) :- !.
 union([H | T], L, [H | ResT]) :- !, union(T, L, ResT).

goal
 % len([1, 2, 3], Len).
 % sum([1, 2, 3], Sum).
 % sum_odd_pos([1, 2, 3, 4, 5], Sum).
 % list_bigger([1, 2, 3, 4, 5], 3, List).
 % rm_all([1, 2, 2, 3, 4, 2, 4], 2, List).
 % rm_one([1, 2, 2, 3, 4, 2, 4], 2, List).
 % union([1, 2, 3, 4], [5, 6, 7], List).
\end{lstlisting}

\newpage

\newcommand{\specialcell}[2][c]{%
  \begin{tabular}[#1]{@{}l@{}}#2\end{tabular}}
  
\begin{table}[]
\resizebox{\textwidth}{!}{
\begin{tabular}{|l|l|l|l|}
\hline
\specialcell{№ шага} & \specialcell{Резольвента} & \specialcell{Сравниваемые термы; \\ результат; подстановки}                                                                                                                                                                                                                                                                                                                                                                                                                     & \specialcell{Дальнейшие \\ действия} \\ \hline

1   
& \specialcell{len([1, 2], Len)} 
& \specialcell{len([1, 2], Len) = \\ len1([\_ | T], Acc, Res) \\ 
\textbf{Нет} \\ 
Подстановка пуста} 
& \specialcell{Прямой ход} \\ \hline

2   
& \specialcell{...} 
&  
&  \\ \hline

3   
& \specialcell{! \\ len1([1, 2], 0, Res)} 
& \specialcell{len([1, 2], Len) = \\ len(L, Res) \\ 
\textbf{Успех} \\ 
\textbf{L = [1, 2]} \\ 
\textbf{Len и Res - сцепленные}} 
& \specialcell{Прямой ход} \\ \hline

4   
& \specialcell{!\\Acc1 = 0 + 1\\len1([2], Acc1, Res)} 
& \specialcell{len1([1, 2], 0, Res) = \\ len1([\_ | T], Acc, Res) \\ 
\textbf{Успех} \\ 
\textbf{Acc = 0} \\
\textbf{T = [2]} \\ 
\textbf{L = [1, 2]} \\ 
\textbf{Len, Res и Res - сцепленные}}
& \specialcell{Прямой ход} \\ \hline

5  
& \specialcell{len1([2], 1, Res)} 
& \specialcell{Acc1 = 0 + 1 \\ 
\textbf{Успех} \\ 
\textbf{Acc1 = 1} \\
\textbf{Acc = 0} \\
\textbf{T = [2]} \\ 
\textbf{L = [1, 2]} \\ 
\textbf{Len, Res и Res - сцепленные}}
& \specialcell{Прямой ход} \\ \hline

\end{tabular}
}
\end{table}

\newpage

\begin{table}[]
\resizebox{\textwidth}{!}{%
\begin{tabular}{|l|l|l|l|}
\hline
\specialcell{№ шага} & \specialcell{Резольвента} & \specialcell{Сравниваемые термы; \\ результат; подстановки}                                                                                                                                                                                                                                                                                                                                                                                                                     & \specialcell{Дальнейшие \\ действия} \\ \hline

6  
& \specialcell{!\\Acc1 = 1 + 1\\len1([], Acc1, Res)} 
& \specialcell{len1([2], Acc1, Res) = \\ len1([\_ | T], Acc, Res) \\ 
\textbf{Успех} \\ 
\textbf{Acc = 1} \\
\textbf{T = []} \\ 
\textbf{Acc1 = 1} \\
\textbf{Acc = 0} \\
\textbf{T = [2]} \\ 
\textbf{L = [1, 2]} \\ 
\textbf{Len, Res, Res и Res - сцепленные}}
& \specialcell{Прямой ход} \\ \hline

7  
& \specialcell{len1([], 2, Res)} 
& \specialcell{Acc1 = 1 + 1 \\ 
\textbf{Успех} \\ 
\textbf{Acc1 = 2} \\
\textbf{Acc = 1} \\
\textbf{T = []} \\ 
\textbf{Acc1 = 1} \\
\textbf{Acc = 0} \\
\textbf{T = [2]} \\ 
\textbf{L = [1, 2]} \\ 
\textbf{Len, Res, Res и Res - сцепленные}}
& \specialcell{Прямой ход} \\ \hline

8  
& \specialcell{len1([], 2, Res)} 
& \specialcell{len1([], 2, Res) = \\ len1([\_ | T], Acc, Res) \\ 
\textbf{Нет} \\ 
\textbf{Acc1 = 2} \\
\textbf{Acc = 1} \\
\textbf{T = []} \\ 
\textbf{Acc1 = 1} \\
\textbf{Acc = 0} \\
\textbf{T = [2]} \\ 
\textbf{L = [1, 2]} \\ 
\textbf{Len, Res, Res и Res - сцепленные}}
& \specialcell{Прямой ход} \\ \hline

\end{tabular}
}
\end{table}

\newpage


\begin{table}[]
\resizebox{\textwidth}{!}{
\begin{tabular}{|l|l|l|l|}
\hline
\specialcell{№ шага} & \specialcell{Резольвента} & \specialcell{Сравниваемые термы; \\ результат; подстановки}                                                                                                                                                                                                                                                                                                                                                                                                                     & \specialcell{Дальнейшие \\ действия} \\ \hline

9  
& \specialcell{!} 
& \specialcell{len1([], 2, Res) = \\ len1([], Acc, Acc) \\ 
\textbf{Успех} \\ 
\textbf{Res = Acc = 2} \\
\textbf{Acc = 2} \\
\textbf{Acc1 = 2} \\
\textbf{Acc = 1} \\
\textbf{T = []} \\ 
\textbf{Acc1 = 1} \\
\textbf{Acc = 0} \\
\textbf{T = [2]} \\ 
\textbf{L = [1, 2]} \\ 
\textbf{Len, Res, Res и Res - сцепленные}}
& \specialcell{Откат к пункту 6} \\ \hline

10  
& \specialcell{Acc1 = 1 + 1\\len1([], Acc1, Res)} 
& \specialcell{! (\textbf{ложь}) \\ 
\textbf{Нет} \\  
\textbf{Acc1 = 1} \\
\textbf{Acc = 0} \\
\textbf{T = [2]} \\ 
\textbf{L = [1, 2]} \\ 
\textbf{Len, Res, Res и Res - сцепленные}}
& \specialcell{Откат к пункту 4} \\ \hline

11 
& \specialcell{Acc1 = 0 + 1\\len1([2], Acc1, Res)} 
& \specialcell{! (\textbf{ложь}) \\ 
\textbf{Нет} \\  
\textbf{L = [1, 2]} \\ 
\textbf{Len, Res и Res - сцепленные}}
& \specialcell{Откат к пункту 3} \\ \hline

12 
& \specialcell{len1([1, 2], 0, Res)} 
& \specialcell{! (\textbf{ложь}) \\ 
\textbf{Нет}\\
Подстановка пуста} 
& \specialcell{Завершение работы} \\ \hline

13   
& \specialcell{Резольвента пуста} 
& \specialcell{Подстановка пуста} 
& \specialcell{Завершение работы} \\ \hline

\end{tabular}
}
\end{table}
