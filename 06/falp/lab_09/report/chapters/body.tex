\setcounter{page}{3}

\chapter{Практическая часть}
\section{Задание №1}
Создать базу знаний: <<ПРЕДКИ>>, позволяющую наиболее эффективным способом (за меньшее количество шагов, что обеспечивается меньшим количеством предложений БЗ – правил), и используя разные варианты (примеры) одного вопроса, определить (указать: какой вопрос для какого варианта):

\begin{enumerate}
	\item По имени субъекта определить всех его бабушек (предки 2-го колена);
	\item По имени субъекта определить всех его дедушек (предки 2-го колена);
	\item По имени субъекта определить всех его бабушек и дедушек (предки 2-го колена);
	\item По имени субъекта определить его бабушку по материнской линии (предки 2-го колена);
	\item По имени субъекта определить его бабушку и дедушку по материнской линии (предки 2-го колена).
\end{enumerate}

Минимизировать количество правил и количество вариантов вопросов. Использовать конъюнктивные правила и простой вопрос.

Для одного из вариантов ВОПРОСА задания 1 составить таблицу, отражающую конкретный порядок работы системы.


\section{Задание №2}
В одной программе написать правила, позволяющие найти:

\begin{enumerate}
	\item Максимум из двух чисел:
	\begin{itemize}
		\item Без использования отсечения;
		\item С использованием отсечения;
	\end{itemize}
	\item Максимум из трех чисел:
	\begin{itemize}
		\item Без использования отсечения;
		\item С использованием отсечения.
	\end{itemize}
\end{enumerate}

Для каждого случая пункта 2 обосновать необходимость всех условий тела. Для одного из вариантов ВОПРОСА и каждого варианта задания 2 составить таблицу, отражающую конкретный порядок работы системы.
  
\begin{code}
\caption{Задание №1-2}
\label{code:bf1}
\begin{minted}{lisp}
domains
	name = symbol.
	gender = symbol.
	human = human(name, gender).
	
predicates
	nondeterm parent(human, human).
	
	nondeterm grandparent_gender(name, name, gender).
	
	nondeterm grandparent(name, name).
	nondeterm grandfather(name, name).
	nondeterm grandparent(name, name).
	
	nondeterm m_grandparent_gender(name, name, gender).
	nondeterm m_grandparent(name, name).
	nondeterm m_grandparent(name, name).
	
	nondeterm max2(integer, integer, integer).
	nondeterm c_max2(integer, integer, integer).
	
	nondeterm max3(integer, integer, integer, integer).
	nondeterm c_max3(integer, integer, integer, integer).
\end{minted}
\end{code}

\newpage

\begin{code}
\caption{Задание №1-2}
\label{code:bf2}
\begin{minted}{lisp}		
clauses
	parent(human(vasya, m), human(katya, f)).
	parent(human(masha, f), human(katya, f)).
	parent(human(vasya, m), human(galya, f)).
	parent(human(masha, f), human(galya, f)).
	parent(human(vasya, m), human(lesha, m)).
	parent(human(masha, f), human(lesha, m)).		  
	parent(human(sonya, f), human(masha, f)).
	parent(human(gosha, m), human(vasya, m)).
	parent(human(ira, f), human(vasya, m)).
	parent(human(tolya, m), human(gosha, m)).
	parent(human(nina, f), human(ira, f)).
	
	grandparent_gender(Grand, Name, Gender) :- parent(human(Name,
	Gender), Parent), parent(Parent, human(Grand, _)).
	
	grandparent(Grand, Name) :- grandparent_gender(Grand, Name,
	f).
	grandfather(Grand, Name) :- grandparent_gender(Grand, Name, 
	m).
	grandparent(Grand, Name) :- grandparent_gender(Grand, Name, 
	_).
	
	m_grandparent_gender(Grand, Name, Gender) :- 
	parent(human(Name, Gender), human(ParentName, f)), 
	parent(human(ParentName, f), human(Grand, _)).
	
	m_grandparent(Grand, Name) :- 
	m_grandparent_gender(Grand, Name, f).
	m_grandparent(Grand, Name) :- 
	m_grandparent_gender(Grand, Name, _).
\end{minted}
\end{code}

\newpage

\begin{code}
\caption{Задание №1-2}
\label{code:bf3}
\begin{minted}{lisp}
	max2(A, B, A) :- A > B.
	max2(A, B, B) :- B >= A.
	
	max3(A, B, C, A) :- A >= B, A >= C.
	max3(A, B, C, B) :- B > A, B >= C.
	max3(A, B, C, C) :- C > A, C > B.
	
	c_max2(A, B, A) :- A > B, !.
	c_max2(_, B, B) :- !.

	c_max3(A, B, C, A) :- A >= B, A >= C, !.
	c_max3(_, B, C, B) :- B >= C, !.
	c_max3(_, _, C, C) :- !.
goal
	% grandparent(vasya, Result).
	% grandparent(lesha, Result).
	% m_grandparent(lesha, Result).				  
	%max2(2, 4, Result).
	%max3(3, 2, 3, Result).
	
	%c_max2(2, 4, Result).
	c_max3(1, 2, 3, Result).
\end{minted}
\end{code}

%\begin{table}[]
%\begin{tabular}{|l|l|l|}
%\hline
%№ шага & \begin{tabular}[c]{@{}l@{}}Сравниваемые термы; \\ результат; подстановки\end{tabular}                                                                                                                                                                                                                                                                                                                                                                                                                      & \begin{tabular}[c]{@{}l@{}}Дальнейшие \\ действия\end{tabular}           \\ \hline
%1      & \begin{tabular}[c]{@{}l@{}}all\_properties\_prices("lesha", Name, Price)=\\ phone\_note("katya", "111111", address("New-York", "One", 1, 2))\\ не унифицируется, тк разные функторы\end{tabular}                                                                                                                                                                                                                                                                                                          & \begin{tabular}[c]{@{}l@{}}переход на \\ след. терм\end{tabular}         \\ \hline
%2      & аналогичные действия до совпадения функтора                                                                                                                                                                                                                                                                                                                                                                                                                                                                &          \\ \hline
%3      & \begin{tabular}[c]{@{}l@{}}all\_properties\_prices("lesha", Name, Price)=\\ all\_properties\_prices(Surname, Name, Price)\\ Константа "lesha" унифицируется с \\ несвязанной переменной Surname, Surname \\ принимает значение "lesha"; остальные \\ несвязанные переменные становятся сцепленными\end{tabular}                                                                                                                                                                                         & \begin{tabular}[c]{@{}l@{}}переход к \\ проверке \\ условий\end{tabular} \\ \hline
%4      & \begin{tabular}[c]{@{}l@{}}ownership(Surname, car(Name, \_, \_, Price))=\\ phone\_note("katya", "111111", address("New-York", "One", 1, 2))\\ не унифицируется, тк разные функторы\end{tabular}                                                                                                                                                                                                                                                                                                            & \begin{tabular}[c]{@{}l@{}}переход на \\ след. терм\end{tabular}         \\ \hline
%5      & аналогичные действия до совпадения функтора                                                                                                                                                                                                                                                                                                                                                                                                                                                                &          \\ \hline
%6      & \begin{tabular}[c]{@{}l@{}}ownership(Surname, car(Name, \_, \_, Price))=\\ ownership("katya", car("BMW", "black", "AAA111", 1000))\\ не унифицируется, так как не совпадает константа \\ "katya" и значение связанной переменной Surname\end{tabular}                                                                                                                                                                                                                                                      & \begin{tabular}[c]{@{}l@{}}переход на \\ след. терм\end{tabular}         \\ \hline
%7      & аналогичные действия до совпадения функтора                                                                                                                                                                                                                                                                                                                                                                                                                                                                &          \\ \hline
%8      & \begin{tabular}[c]{@{}l@{}}ownership(Surname, car(Name, \_, \_, Price))=\\ ownership("lesha", car("Ford", "yellow", "BBB222", 1500))\\ успешная унификация\\ Константа "lesha" унифицируется со\\ связанной переменной Surname; константа "Ford"\\ унифицируется с несвязанной переменной Name, Name\\ принимает значение "Ford"; анонимные переменные не \\ связываются со значениями; константа 1500 унифицируется \\ с несвязанной переменной Price, Price пр.знач. 1500\end{tabular} & \begin{tabular}[c]{@{}l@{}}переход на \\ след. терм\end{tabular}         \\ \hline
%9      & попытки унификации до конца БЗ                                                                                                                                                                                                                                                                                                                                                                                                                                                                             & \begin{tabular}[c]{@{}l@{}}переход на \\ след. терм\end{tabular}         \\ \hline
%10     & аналогичные действия для остальных функторов условий                                                                                                                                                                                                                                                                                                                                                                                                                                                       &          \\ \hline
%\end{tabular}
%\end{table}


\newcommand{\specialcell}[2][c]{%
  \begin{tabular}[#1]{@{}l@{}}#2\end{tabular}}
  
\begin{table}[]
\resizebox{\textwidth}{!}{
\begin{tabular}{|l|l|l|l|}
\hline
\specialcell{№ шага} & \specialcell{Резольвента} & \specialcell{Сравниваемые термы; \\ результат; подстановки}                                                                                                                                                                                                                                                                                                                                                                                                                     & \specialcell{Дальнейшие \\ действия} \\ \hline

1   
& \specialcell{grandparent(vasya, Result)} 
& \specialcell{grandparent(vasya, Result) = \\ parent(human(vasya, m), \\ human(katya, f)) \\ 
\textbf{Нет} \\ 
Подстановка пуста} 
& \specialcell{Прямой ход} \\ \hline

2-10   
& \specialcell{...} 
&  
&  \\ \hline

11   
& \specialcell{grandparent\_gender(vasya, Name, f)} 
& \specialcell{grandparent(vasya, Result) = \\ grandparent(Grand, Name \\ 
\textbf{Успех} \\ 
\textbf{Grand = vasya} \\ 
\textbf{Остальные - сцепленные}} 
& \specialcell{Прямой ход} \\ \hline

12   
& \specialcell{grandparent\_gender(vasya, Name, f)} 
& \specialcell{grandparent\_gender(vasya, Name, f) = \\ parent(human(vasya, m), \\ human(katya, f)) \\ 
\textbf{Нет} \\ 
\textbf{Grand = vasya}} 
& \specialcell{Прямой ход} \\ \hline

13-22   
& \specialcell{...} 
&  
&  \\ \hline

23   
& \specialcell{parent(human(Name, f), Parent) \\ parent(Parent, human(vasya, \_))} 
& \specialcell{grandparent\_gender(vasya, Name, f) = \\ grandparent\_gender(Grand, Name, Gender) \\ 
\textbf{Успех} \\ 
\textbf{Grand = vasya} \\ 
\textbf{Gender = f} \\ 
\textbf{Остальные - сцепленные}} 
& \specialcell{Прямой ход} \\ \hline

24   
& \specialcell{parent(human(Name, f), Parent) \\ parent(Parent, human(vasya, \_))} 
& \specialcell{parent(human(Name, f), Parent) = \\ parent(human(vasya, m), \\ human(katya, f)) \\ 
\textbf{Нет} \\ 
\textbf{Grand = vasya} \\ 
\textbf{Gender = f} \\ 
\textbf{Остальные - сцепленные}} 
& \specialcell{Прямой ход} \\ \hline

25   
& \specialcell{parent(human(katya, f), human(vasya, \_))} 
& \specialcell{parent(human(Name, f), Parent) = \\ parent(human(masha, f), human(katya, f)) \\ 
\textbf{Успех} \\ 
\textbf{Grand = vasya} \\ 
\textbf{Gender = f} \\ 
\textbf{Name = masha} \\ 
\textbf{Parent =  human(katya, f)}} 
& \specialcell{Прямой ход} \\ \hline

26   
& \specialcell{parent(human(katya, f), human(vasya, \_))} 
& \specialcell{parent(human(katya, f), human(vasya, \_)) = \\ parent(human(vasya, m), human(katya, f)) \\ 
\textbf{Нет} \\
\textbf{Grand = vasya} \\ 
\textbf{Gender = f} \\ 
\textbf{Name = masha} \\ 
\textbf{Parent =  human(katya, f)}} 
& \specialcell{Прямой ход} \\ \hline

\end{tabular}
}
\end{table}

\newpage

\begin{table}[]
\resizebox{\textwidth}{!}{%
\begin{tabular}{|l|l|l|l|}
\hline
\specialcell{№ шага} & \specialcell{Резольвента} & \specialcell{Сравниваемые термы; \\ результат; подстановки}                                                                                                                                                                                                                                                                                                                                                                                                                     & \specialcell{Дальнейшие \\ действия} \\ \hline

27-52   
& \specialcell{...} 
&  
&  \\ \hline

53   
& \specialcell{parent(human(katya, f), human(vasya, \_))} 
& \specialcell{Конец базы знаний \\
\textbf{Grand = vasya} \\ 
\textbf{Gender = f}} 
& \specialcell{Откат к пункту 25} \\ \hline

54-246   
& \specialcell{...} 
&  
&  \\ \hline

247   
& \specialcell{parent(human(ira, f), human(vasya, \_))} 
& \specialcell{parent(human(Name, f), Parent) = \\ parent(human(nina, f), human(ira, f)) \\ 
\textbf{Успех} \\ 
\textbf{Grand = vasya} \\ 
\textbf{Gender = f} \\
\textbf{Name = nina} \\ 
\textbf{Parent = human(ira, f)}} 
& \specialcell{Прямой ход} \\ \hline

248   
& \specialcell{parent(human(ira, f), human(vasya, \_))} 
& \specialcell{parent(human(ira, f), human(vasya, \_)) = \\ parent(human(vasya, m), human(katya, f)) \\ 
\textbf{Нет} \\
\textbf{Grand = vasya} \\ 
\textbf{Gender = f} \\
\textbf{Name = nina} \\ 
\textbf{Parent = human(ira, f)}} 
& \specialcell{Прямой ход} \\ \hline

248-253   
& \specialcell{...} 
&  
&  \\ \hline

254   
& \specialcell{Пусто} 
& \specialcell{parent(human(ira, f), human(vasya, \_)) = \\ parent(human(ira, f), human(vasya, m)) \\ \textbf{Успех}
\textbf{Grand = vasya} \\ 
\textbf{Gender = f} \\
\textbf{Name = nina} \\ 
\textbf{Parent = human(ira, f)}}
& \specialcell{Откат к пункту 247} \\ \hline

255   
& \specialcell{parent(human(Name, f), Parent) \\ parent(Parent, human(vasya, \_))} 
& \specialcell{parent(human(Name, f), Parent) = \\ grandparent\_gender(Grand, Name, Gender) \\ 
\textbf{Нет} \\
\textbf{Grand = vasya} \\ 
\textbf{Gender = f}}  
& \specialcell{Прямой ход} \\ \hline

255-271   
& \specialcell{...} 
&  
&  \\ \hline

272   
& \specialcell{parent(human(Name, f), Parent) \\ parent(Parent, human(vasya, _))} 
& \specialcell{Конец базы знаний} 
& \specialcell{Откат к пункту 23} \\ \hline

273   
& \specialcell{grandparent\_gender(vasya, Name, f)} 
& \specialcell{grandparent\_gender(vasya, Name, f) = \\ grandparent(Grand, Name) \\ \textbf{Нет} \\ Подстановка пуста} 
& \specialcell{Прямой ход} \\ \hline

274-288   
& \specialcell{...} 
&  
&  \\ \hline

289   
& \specialcell{grandparent\_gender(vasya, Name, f)} 
& \specialcell{Конец базы знаний \\ Подстановка пуста} 
& \specialcell{Откат к пункту 11} \\ \hline

290-305   
& \specialcell{...} 
&  
&  \\ \hline

306   
& \specialcell{Пусто} 
& \specialcell{Конец базы знаний. \\ Подстановка пуста} 
& \specialcell{Завершение работы} \\ \hline



\end{tabular}
}
\end{table}

\newpage


\begin{table}[]
\resizebox{\textwidth}{!}{
\begin{tabular}{|l|l|l|l|}
\hline
\specialcell{№ шага} & \specialcell{Резольвента} & \specialcell{Сравниваемые термы; \\ результат; подстановки}                                                                                                                                                                                                                                                                                                                                                                                                                     & \specialcell{Дальнейшие \\ действия} \\ \hline

1   
& \specialcell{max3(1, 2, 3, Result).} 
& \specialcell{max3(1, 2, 3, Result = \\ parent(human(vasya, m), \\ human(katya, f)) \\ 
\textbf{Нет} \\ 
Подстановка пуста} 
& \specialcell{Прямой ход} \\ \hline

2-21
& \specialcell{max3(1, 2, 3, Result).} 
& ...
& \specialcell{Прямой ход} \\ \hline

22  
& \specialcell{Пусто} 
& \specialcell{max3(1, 2, 3, Result = \\ max3(A, B, C, A) \\ 
\textbf{Успех} \\ 
\textbf{A = 1} \\ 
\textbf{B = 2} \\ 
\textbf{C = 3} \\ 
\textbf{Res, A - сцепленные} \\ 
} 
& \specialcell{Прямой ход} \\ \hline

23
& \specialcell{A >= B \\ A >= C} 
& \specialcell{A >= B \\ 
\textbf{Нет} \\ 
Подстановка пуста
} 
& \specialcell{Откат к пункту 21} \\ \hline

24
& \specialcell{B >= C \\ B >= A} 
& \specialcell{max3(1, 2, 3, Result = \\ max3(A, B, C, B) \\  
\textbf{Успех} \\ 
\textbf{A = 1} \\ 
\textbf{B = 2} \\ 
\textbf{C = 3} \\ 
\textbf{Res, B - сцепленные} \\ 
} 
& \specialcell{Прямой ход} \\ \hline

25
& \specialcell{B >= C \\ B >= A} 
& \specialcell{B >= C \\ 
\textbf{Нет} \\ 
Подстановка пуста
} 
& \specialcell{Откат к пункту 21} \\ \hline

\end{tabular}
}
\end{table}

\newpage 

\begin{table}[]
\resizebox{\textwidth}{!}{
\begin{tabular}{|l|l|l|l|}
\hline
\specialcell{№ шага} & \specialcell{Резольвента} & \specialcell{Сравниваемые термы; \\ результат; подстановки}                                                                                                                                                                                                                                                                                                                                                                                                                     & \specialcell{Дальнейшие \\ действия} \\ \hline

26
& \specialcell{C > A \\ C > B} 
& \specialcell{max3(1, 2, 3, Result = \\ max3(A, B, C, C) \\  
\textbf{Успех} \\ 
\textbf{A = 1} \\ 
\textbf{B = 2} \\ 
\textbf{C = 3} \\ 
\textbf{Res, C - сцепленные} \\ 
} 
& \specialcell{Прямой ход} \\ \hline

27
& \specialcell{C > B} 
& \specialcell{C > A \\  
\textbf{Успех} \\ 
\textbf{A = 1} \\ 
\textbf{B = 2} \\ 
\textbf{C = 3} \\ 
\textbf{Res, C - сцепленные} \\ 
} 
& \specialcell{Прямой ход} \\ \hline

28
& \specialcell{Пусто} 
& \specialcell{C > B \\  
\textbf{Успех} \\ 
\textbf{A = 1} \\ 
\textbf{B = 2} \\ 
\textbf{C = 3} \\ 
\textbf{Res, C - сцепленные} \\ 
} 
& \specialcell{Откат к пункту 21} \\ \hline

29-34
& \specialcell{max3(1, 2, 3, Result).} 
& ...
& \specialcell{Прямой ход} \\ \hline

35   
& \specialcell{Пусто} 
& \specialcell{Конец базы знаний. \\ Подстановка пуста} 
& \specialcell{Завершение работы} \\ \hline

\end{tabular}
}
\end{table}

\newpage


\begin{table}[]
\resizebox{\textwidth}{!}{
\begin{tabular}{|l|l|l|l|}
\hline
\specialcell{№ шага} & \specialcell{Резольвента} & \specialcell{Сравниваемые термы; \\ результат; подстановки}                                                                                                                                                                                                                                                                                                                                                                                                                     & \specialcell{Дальнейшие \\ действия} \\ \hline

1   
& \specialcell{c\_max3(1, 2, 3, Result).} 
& \specialcell{c\_max3(1, 2, 3, Result = \\ parent(human(vasya, m), \\ human(katya, f)) \\ 
\textbf{Нет} \\ 
Подстановка пуста} 
& \specialcell{Прямой ход} \\ \hline

2-21
& \specialcell{c\_max3(1, 2, 3, Result).} 
& ...
& \specialcell{Прямой ход} \\ \hline

22  
& \specialcell{Пусто} 
& \specialcell{c\_max3(1, 2, 3, Result = \\ c\_max3(A, B, C, A) \\ 
\textbf{Успех} \\ 
\textbf{A = 1} \\ 
\textbf{B = 2} \\ 
\textbf{C = 3} \\ 
\textbf{Res, A - сцепленные} \\ 
} 
& \specialcell{Прямой ход} \\ \hline

23
& \specialcell{A >= B \\ A >= C} 
& \specialcell{A >= B \\ 
\textbf{Нет} \\ 
Подстановка пуста
} 
& \specialcell{Откат к пункту 21} \\ \hline

24
& \specialcell{B >= C} 
& \specialcell{c\_max3(1, 2, 3, Result = \\ c\_max3(\_, B, C, B) \\  
\textbf{Успех} \\ 
\textbf{A = 1} \\ 
\textbf{B = 2} \\ 
\textbf{C = 3} \\ 
\textbf{Res, B - сцепленные} \\ 
} 
& \specialcell{Прямой ход} \\ \hline

25
& \specialcell{Пусто} 
& \specialcell{B >= C \\ 
\textbf{Нет} \\ 
Подстановка пуста
} 
& \specialcell{Откат к пункту 21} \\ \hline

\end{tabular}
}
\end{table}

\newpage 

\begin{table}[]
\resizebox{\textwidth}{!}{
\begin{tabular}{|l|l|l|l|}
\hline
\specialcell{№ шага} & \specialcell{Резольвента} & \specialcell{Сравниваемые термы; \\ результат; подстановки}                                                                                                                                                                                                                                                                                                                                                                                                                     & \specialcell{Дальнейшие \\ действия} \\ \hline

26
& \specialcell{Пусто} 
& \specialcell{c\_max3(1, 2, 3, Result = \\ c\_max3(\_, \_, C, C) \\  
\textbf{Успех} \\ 
\textbf{A = 1} \\ 
\textbf{B = 2} \\ 
\textbf{C = 3} \\ 
\textbf{Res, C - сцепленные} \\ 
} 
& \specialcell{Откат к пункту 21} \\ \hline

27-32
& \specialcell{c\_max3(1, 2, 3, Result).} 
& ...
& \specialcell{Прямой ход} \\ \hline

33 
& \specialcell{Пусто} 
& \specialcell{Конец базы знаний. \\ Подстановка пуста} 
& \specialcell{Завершение работы} \\ \hline

\end{tabular}
}
\end{table}
