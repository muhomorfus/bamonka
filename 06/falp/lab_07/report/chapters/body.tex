\setcounter{page}{3}

\chapter{Теоретическая часть}
Программа на Prolog состоит из разделов. Каждый раздел начинается со своего заголовка. Структура программы:

\begin{itemize}
	\item директивы компилятора — зарезервиорванные символьные константы;
	\item CONSTANTS — раздел описания констант;
	\item DOMAINS — раздел описания доменов;
	\item DATABASE — раздел описания предикатов внутренней базы данных;
	\item PREDICATES — раздел описания предикатов;
	\item CLAUSES — раздел описания предложений базы знаний;
	\item GOAL — раздел описания внутренней цели (вопроса).
\end{itemize}

\newpage

\chapter{Практическая часть}

\section{Задание 1}
\subsection*{Задание}
Запустить среду Visual Prolog 5.2. Настроить утилиту TestGoal.
Запустить тестовую программу, проанализировать реакцию системы и множество ответов.
Разработать свою программу~---~«Телефонный справочник». Абоненты могут иметь несколько телефонов. Протестировать работу программы, используя разные вопросы.

\begin{itemize}
  \item «Телефонный справочник»: фамилия, номер телефона, адрес~---~структура (город, улица, номер дома, номер квартиры).
  \item «Автомобили»: фамилия владельца, марка, цвет, стоимость, номер.
\end{itemize}

Владелец может иметь несколько телефонов, автомобилей (Факты). В разных городах есть однофамильцы, в одном городе~---~фамилия уникальна.
Используя конъюнктивное правило и простой вопрос, обеспечить возможность поиска:

\begin{itemize}
  \item По Марке и Цвету автомобиля найти Фамилию, Город, Телефон. Лишней информации не находить и не передавать!!!
\end{itemize}

\subsection*{Решение}
\begin{code}
\begin{minted}{prolog}
domains
  name, phone = symbol.
  city, street = symbol.
  model, color, number = symbol.
  building, appartment = integer.
  price = integer.
  address = address(city, street, building, appartment).

predicates
  nondeterm phone_book(name, phone, address).
  nondeterm car(name, model, color, price, number).
  nondeterm info_by_car(model, color, name, city, phone).

clauses
  phone_book("Tomas", "123-456", address("London", "Temza", 3, 1)).
  phone_book("Tomas", "123-450", address("Paris", "Blabla", 2, 55)).
  phone_book("John", "123-111", address("London", "Temza", 333, 61)).
  phone_book("Artur", "123-000", address("Moskow", "Tverskaya", 15, 112)).
  
  car("Tomas", "Bentley", "black", 1200, "AE222K").
  car("Tomas", "Rolce Royce", "pink", 1500, "AE341C").
  car("Artur", "Bentley", "black", 1000, "AT111K").
  car("Vasya", "Zhigulie", "white", 3000, "EK228X").
  
  info_by_car(Model, Color, Name, City, Phone) :- 
  car(Name, Model, Color, _, _), 
  phone_book(Name, Phone, address(City, _, _, _)).
  
goal
  %car("Tomas", Model, Color, Price, Number).
  %car(Name, "Bentley", "black", _, Number).
  %phone_book(Name, Phone, address("London", Street, House, Room)).
  info_by_car("Bentley", "black", Name, City, Phone).
\end{minted}
\end{code}

\newpage