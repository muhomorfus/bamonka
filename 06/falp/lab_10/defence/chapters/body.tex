\setcounter{page}{3}

\chapter{Теоретическая часть}



\section{Библиотека clpfd. Логическое программирование в конечных доменах}

2 основных варианта использования clpfd:

\begin{enumerate}
	\item декларативная арифметика целых чисел;
	\item решение комбинаторных задач, таких как планирование, диспетчеризация и распределение задач.
\end{enumerate}

Предикаты этой библиотеки могут классифицироваться как:

\begin{enumerate}
	\item арифметические ограничения \texttt{\#=}, \texttt{\#>}, \texttt{\#\textbackslash=};
	\item ограничения членства \texttt{in} и \texttt{ins};
	\item предикаты перечисления \texttt{indomain};
	\item комбинаторные ограничения \texttt{all\_distinct};
	\item комбинаторные ограничения \texttt{#<==>};
	\item комбинаторные ограничения \texttt{fd\_dom}.
\end{enumerate}

Для подключения библиотеки нужно добавить строчку:

\begin{lstlisting}[label=div,caption=Подключение]
:- use_module(library(clpfd)).
\end{lstlisting}


\section{Арифметические ограничения}
\subsection{Эквивалентность}
Одним из наиболее важных ограничений является ограничение эквивалетности \texttt{\#=}. Оно ведет себя как \texttt{is} и \texttt{=:=} над целыми числами, кроме того, работает в обоих направлениях.

\begin{lstlisting}[label=div,caption=Пример эквивалетности]
?- X #= 1+2.
X = 3.

?- 3 #= Y+2.
Y = 1.

?- 3 is Y+2.
ERROR: is/2: Arguments are not sufficiently instantiated

?- 3 =:= Y+2.
ERROR: =:=/2: Arguments are not sufficiently instantiated
\end{lstlisting}

В примере ниже с использованием \texttt{\#=} реализован факториал. Оно позволяет производить вычисления факториала в обе стороны:

\begin{lstlisting}[label=div,caption=Реализация факториала]
fact1(0, Acc, Acc) :- !.
fact1(N, Acc, Res) :- N #> 0, Acc #=< Res, Res #> 0, Acc1 #= N * Acc, N1 #= N - 1, fact1(N1, Acc1, Res).

fact(N, Res) :- fact1(N, 1, Res).
\end{lstlisting}

\begin{lstlisting}[label=div,caption=Примеры запросов]
?- fact(5, N).
N = 120.

?- fact(N, 120).
N = 5.

?- fact(N, 11).
false.
\end{lstlisting}

\subsection{Другие ограничения}
\begin{enumerate}
	\item \texttt{Expr1 \#= Expr2}~---~Expr1 равно Expr;
	\item \texttt{Expr1 \#\textbackslash= Expr2}~---~Expr1 не равно Expr;
	\item \texttt{Expr1 \#>= Expr2}~---~Expr1 больше или равно Expr2;
	\item \texttt{Expr1 \#=< Expr2}~---~Expr1 меньше или равно Expr2;
	\item \texttt{Expr1 \#> Expr2}~---~Expr1 больше Expr2;
	\item \texttt{Expr1 \#< Expr2}~---~Expr1 меньше Expr2.
\end{enumerate}

Expr1 и Expr2~---~арифметические выражения, которые могут быть либо целыми числами, либо результатом арифметических операций, таких как \texttt{+, -, *, \^ \,, min, max, abs, mod, rem, //, div}, совершенных над арифметическими выражениями.

\section{Комбинаторные ограничения}
\begin{enumerate}
	\item \texttt{all\_distinct}~---~верно, если все элементы аргумента попарно различны;
	\item \texttt{global\_cardinality}~---~верно, если для каждого ключа второго аргумента-словаря, значение равно количеству вхождений этого ключа в первый аргумент;
	\item \texttt{cumulative}~---~описывает расписание фиксированного ресурса.
	\end{enumerate}
