\setcounter{page}{3}

\chapter{Практическая часть}

\section{Задание 1}
\subsection*{Задание}
Написать хвостовую рекурсивную функцию \texttt{my-reverse}, которая развернет верхний уровень своего списка-аргумента \texttt{lst}.

\subsection*{Решение}
\begin{code}
\begin{minted}{lisp}
(defun my-reverse1 (lst res)
    (cond
        ((null lst) res)
        (t (my-reverse1 (cdr lst) (cons (car lst) res)))
    )
)

(defun my-reverse (lst) (my-reverse1 lst NIL))
\end{minted}
\end{code}


\section{Задание 2}
\subsection*{Задание}
Написать функцию, которая возвращает первый элемент списка~---~аргумента, который сам является непустым списком.

\subsection*{Решение}
\begin{code}
\begin{minted}{lisp}
(defun get_atom (x)
    (cond
        ((atom x) x)
        (t (get_atom (car x)))
    )
)

(defun f1 (x)
    (cond
        ((null x) NIL)
        ((atom (car x)) (f1 (cdr x)))
        (t (get_atom (car x)))
    )
)
\end{minted}
\end{code}


\section{Задание 3}
\subsection*{Задание}
Напишите рекурсивную функцию, которая умножает на заданное число-аргумент все числа из заданного списка-аргумента, когда

\begin{enumerate}
    \item все элементы списка~---~числа;
    \item элементы списка~---~любые объекты.
\end{enumerate}

\subsection*{Решение}
\begin{code}
\begin{minted}{lisp}
(defun f21 (x n res)
    (cond 
        ((null x) res)
        (t (f21 (cdr x) n (cons (* (car x) n) res)))
    )
)

(defun f2 (x n)
    (f21 x n NIL)
)
\end{minted}
\end{code}

\begin{code}
\begin{minted}{lisp}
(defun f31 (x n res)
    (cond 
        ((null x) res)
        ((numberp (car x)) (f31 (cdr x) n (cons (* (car x) n) res)))
        ((atom (car x)) (f31 (cdr x) n (cons (car x) res)))
        (t (f31 (cdr x) n (cons (f31 (car x) n NIL) res)))
    )
)

(defun f3 (x n) 
	(f31 x n NIL)
)
\end{minted}
\end{code}


\section{Задание 4}
\subsection*{Задание}
Напишите функцию, \texttt{select-between}, которая из списка-аргумента, содержащего только числа, выбирает только те, которые расположены между двумя указанными числами~---~границами-аргументами и возвращает их в виде списка.

\subsection*{Решение}
\begin{code}
\begin{minted}{lisp}
(defun select-between1 (x a b)
    (cond 
        ((null x) NIL)
        ((< a (car x) b) (cons (car x) (select-between1 (cdr x) a b)))
        (t (select-between1 (cdr x) a b))
    )
)

(defun select-between (x a b)
    (cond 
        ((< a b) (select-between1 x a b))
        (t (select-between1 x b a))
    )
)
\end{minted}
\end{code}


\section{Задание 5}
\subsection*{Задание}
Написать рекурсивную версию (с именем \texttt{rec-add}) вычисления суммы чисел заданного списка:

\begin{enumerate}
    \item одноуровнего смешанного;
    \item структурированного.
\end{enumerate}

\subsection*{Решение}
\begin{code}
\begin{minted}{lisp}
(defun rec-add1 (x res)
    (cond
        ((null x) res)
        ((numberp (car x)) (rec-add (cdr x) (+ (car x) res)))
        (t (rec-add (cdr x) res))
    )
)

(defun rec-add (x)
	(rec-add1 x NIL)
)
\end{minted}
\end{code}

\begin{code}
\begin{minted}{lisp}
(defun rec-add (x)
    (cond
        ((null x) 0)
        ((numberp (car x)) (+ (car x) (rec-add (cdr x))))
        ((atom (car x)) (rec-add (cdr x)))
        (t (+ (rec-add (car x)) (rec-add (cdr x))))
    )
)
\end{minted}
\end{code}

\newpage