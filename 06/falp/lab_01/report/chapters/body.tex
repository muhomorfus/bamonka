\setcounter{page}{3}
\chapter{Теоретическая часть}
\section{Элементы языка: определение, синтаксис, представление в памяти}
Вся информация (данные и программы) в Lisp представляется в виде символьных выражений~---~S-выражений. По определению 

\begin{center}
	\text{S-выражение ::= <атом> | <точечная пара>}.
\end{center}

Элементарные значения структур данных: 
\begin{itemize}
	\item Атомы
	\begin{itemize}
		\item символы (идентификаторы)~---~синтаксически~---~набор литер (букв и цифр), начинающихся с буквы;
		\item специальные символы~---~\texttt{{Т, Nil}} (используются для обозначения логических констант);
		\item самоопределимые атомы~---~натуральные числа, дробные числа (например 2/3), вещественные числа, строки~---~последовательность символов, заключенных в двойные апострофы (например “abc”);
	\end{itemize}
	
	\item Точечные пары
	\begin{center}
	Точечные пары ::= (<атом>.<атом>) | (<атом>.<точечная пара>) |
(<точечная пара>.<атом>) | (<точечная пара>.<точечная пара>).
	\end{center}
	
	\item Списки
	\begin{center}
		Список ::= <пустой список> | <непустой список>,
		
		<пустой список> ::= ( ) | Nil,
		
<непустой список>::= (<первый элемент> . <хвост>),

 <первый элемент> ::= <S-выражение>,
<хвост> ::= <список>.
	\end{center}
\end{itemize}

Более сложные данные~---~списки и точечные пары (структуры) строятся из унифицированных структур~---~блоков памяти~---~бинарных узлов.

Синтаксически любая структура (точечная пара или список) заключается в круглые скобки \texttt{(A . B)}~---~точечная пара, \texttt{(А)}~---~список из одного элемента,
пустой список изображается как \texttt{Nil} или \texttt{()};
непустой список по определению может быть изображен:
\texttt{(A . (B . (C . (D ()))))}, допустимо изображение списка последовательностью атомов, разделенных пробелами~---~\texttt{(A B C D)}.
Элементы списка могут, в свою очередь, быть списками (любой список заключается в круглые скобки), например~---~\texttt{(A (B C) (D (E)))}. 

Таким образом, синтаксически наличие скобок является признаком структуры~---~списка или точечной пары.
Любая непустая структура Lisp в памяти представляется списковой ячейкой, хранящей два указателя: на голову (первый элемент) и хвост~---~все остальное.

\section{Особенности языка Lisp. Структура программы. Символ апостроф}
Важной особенностью языка Lisp является то, что программы, написанные на Lisp, представляются в виде его же структур данных. Это позволяет выдать программы за данные и изменять их.

Программа на языке Lisp представляет собой последовательность вычислимых выражений, которые являются атомами или списками.

Символ \texttt{'} (апостроф) эквивалентен функции \texttt{quote}~---~блокирует вычисление выражения, и оно воспринимается интерпретатором как данные.

\section{Базис языка Lisp. Ядро языка}
Базис~---~это минимальный набор действий в языке программирования. Базис языка Lisp представлен атомами, структурами и функциями. Некоторые базисные функции: \texttt{car}, \texttt{cdr}, \texttt{cons}, \texttt{list}, \texttt{lambda}, \texttt{quote}, \texttt{eq}.

Ядром языка называется базис языка и функции стандартной библиотеки (часто используемые функции, созданные на основе базиса).

\newpage

\chapter{Практическая часть}
\section{Задание 1}
\subsection{Задание}
Представить следующие списки в виде списочных ячеек:

\begin{itemize}
	\item \texttt{'(open close halph)}
	\item \texttt{'((open1) (close2) (halph3))} 
	\item \texttt{'((one) for all (and (me (for you))))};
	\item \texttt{'((TOOL) (call))};
	\item \texttt{'((TOOL1) ((call2)) ((sell)))}
	\item \texttt{'(((TOOL) (call)) ((sell)))}.
\end{itemize}


\subsection{Решение}
Приложено на отдельном листе.

\section{Задание 2}
\subsection{Задание}
Используя только функции \texttt{CAR} и \texttt{CDR}, написать выражения, возвращающие

\begin{enumerate}
	\item второй
	\item третий
	\item четвертый
\end{enumerate}

элементы заданного списка.

\subsection{Решение}
\begin{enumerate}
	\item \texttt{(car (cdr `(1 2 3 4 5 6)))}
	\item \texttt{(car (cdr (cdr `(1 2 3 4 5 6))))}
	\item \texttt{(car (cdr (cdr(cdr `(1 2 3 4 5 6)))))}
\end{enumerate}

\section{Задание 3}
\subsection{Задание}
Что будет в результате вычисления выражений?
\begin{enumerate}
	\item \texttt{(CAADR ' ((blue cube) (red pyramid)))}
	\item \texttt{(CDAR '((abc) (def) (ghi)))}
	\item \texttt{(CADR ' ((abc) (def) (ghi)))}
	\item \texttt{(CADDR ' ((abc) (def) (ghi)))}
\end{enumerate}

\subsection{Решение}
\begin{enumerate}
	\item \texttt{RED}
	\item \texttt{NIL}
	\item \texttt{(DEF)}
	\item \texttt{(GHI)}
\end{enumerate}

\section{Задание 4}
\subsection{Задание}
Напишите результат вычисления выражений и объясните как он получен:
\begin{enumerate}
	\item \texttt{(list 'Fred 'and 'Wilma)}
	\item \texttt{(list 'Fred '(and Wilma))}
	\item \texttt{(cons Nil Nil)}
	\item \texttt{(cons T Nil)}
	\item \texttt{(cons Nil T)}
	\item \texttt{(list Nil)}
	\item \texttt{(cons ' (T) Nil)}
	\item \texttt{(list ' (one two) ' (free temp))}
	\item \texttt{(cons 'Fred '(and Wilma))}
	\item \texttt{(cons 'Fred '(Wilma))}
	\item \texttt{(list Nil Nil)}
	\item \texttt{(list T Nil)}
	\item \texttt{(list Nil T)}
	\item \texttt{(cons T (list Nil))}
	\item \texttt{(list '(T) Nil)}
	\item \texttt{(cons '(one two) '(free temp))}
\end{enumerate}
\subsection{Решение}
Функция \texttt{cons} создает списочную ячейку. Принимает два аргумента. Если второй аргумент~---~список, то результатом вызова функции будет список, если атом~---~результатом будет точечная пара. Если второй аргумент~---~\texttt{Nil}, то результатом будет список из одного элемента.

Функция \texttt{list} создает столько списочных ячеек, сколько ей было передано аргументов. Она не является частью базиса.

\begin{enumerate}
	\item \texttt{(FRED AND WILMA)}
	\item \texttt{(FRED (AND WILMA))}
	\item \texttt{(NIL)}
	\item \texttt{(T)}
	\item \texttt{(NIL . T)}
	\item \texttt{(NIL)}
	\item \texttt{((T))}
	\item \texttt{((ONE TWO) (FREE TEMP))}
	\item \texttt{(FRED AND WILMA)}
	\item \texttt{(FRED WILMA)}
	\item \texttt{(NIL NIL)}
	\item \texttt{(T NIL)}
	\item \texttt{(NIL T)}
	\item \texttt{(T NIL) }
	\item \texttt{((T) NIL)}
	\item \texttt{((ONE TWO) FREE TEMP)}
\end{enumerate}

\section{Задание 5}
\subsection{Задание}
Написать лямбда-выражение и соответствующую функцию:
\begin{enumerate}
	\item Написать функцию \texttt{(f arl ar2 ar3 ar4)}, возвращающую список: \texttt{((arl ar2) (ar3 ar4))}.
	\item Написать функцию \texttt{(f arl ar2)}, возвращающую \texttt{((arl) (ar2))}.
	\item Написать функцию \texttt{(f arl)}, возвращающую \texttt{(((arl)))}.
	\item Представить результаты в виде списочных ячеек.
\end{enumerate}
\subsection{Решение}
\begin{enumerate}
	\item \texttt{((lambda (ar1 ar2 ar3 ar4) (list (list ar1 ar2) (list ar3 ar4))) 1 2 3 4)}
	
	\texttt{(defun f1 (ar1 ar2 ar3 ar4) (list (list ar1 ar2) (list ar3 ar4)))}
	
	\item \texttt{((lambda (ar1 ar2) (list (list ar1) (list ar2))) 1 2)}
	
	\texttt{(defun f2 (ar1 ar2) (list (list ar1) (list ar2)))}
	
	\item \texttt{((lambda (ar1) (list (list (list ar1)))) 1)}
	
	\texttt{(defun f3 (ar1) (list (list (list ar1))))}
	
	\item Приложено на отдельном листе.
\end{enumerate}

\newpage