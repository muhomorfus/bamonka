\setcounter{page}{3}

\chapter{Практическая часть}
Создать базу знаний «Собственники», дополнив (и минимально изменив) базу
знаний, хранящую знания:
\begin{itemize}
	\item «Телефонный справочник»: Фамилия, Noтел, Адрес – структура (Город, Улица, Noдома, Noкв),
	\item «Автомобили»: Фамилия владельца, Марка, Цвет, Стоимость, и др.,
	\item «Вкладчики банков»: Фамилия, Банк, счет, сумма, др.
\end{itemize}
знаниями о дополнительной собственности владельца. Преобразовать знания об автомобиле к форме знаний о собственности.
Вид собственности (кроме автомобиля): строение, стоимость и другие его характеристики; участок, стоимость и другие его характеристики; водный транспорт, стоимость и другие его характеристики.
Описать и использовать вариантный домен: собственность. Владелец может иметь, но только один объект каждого вида собственности (это касается и автомобиля), или не иметь некоторых видов собственности.
Используя конъюнктивное правило и разные формы задания одного вопроса (пояснять для какого задания – какой вопрос), обеспечить возможность поиска:
\begin{itemize}
	\item Названий всех объектов собственности заданного субъекта,
	\item Названий и стоимости всех объектов собственности заданного субъекта,
	\item Разработать правило, позволяющее найти суммарную стоимость всех объектов собственности заданного субъекта.
\end{itemize}

Для 2-го пункт и одной фамилии составить таблицу, отражающую конкретный порядок работы системы, с объяснениями порядка работы и особенностей использования доменов (указать конкретные Т1 и Т2 и полную подстановку на каждом шаге)
  
\begin{lstlisting}
domains
    surname, name, phone = symbol.
    city, street = symbol.
    color, number = symbol.
    building, appartment = integer.
    floors, size = integer.
    price = integer.
    sum = integer.
    bank_name = symbol.
    account = symbol.
    address = address(city, street, building, appartment).

    property = car(name, color, number, price);
               building(name, floors, price);
               land(name, size, price);
               water(name, color, number, price).
 
predicates
    nondeterm contact(surname, phone, address).
    nondeterm owner(surname, property).
    nondeterm bank(surname, bank_name, account, sum).

    nondeterm prop_names(surname, name).
    nondeterm prop_prices(surname, name, price).

    nondeterm car_price(surname, price).
    nondeterm building_price(surname, price).
    nondeterm land_price(surname, price).
    nondeterm water_price(surname, price).
    nondeterm sum_price(surname, price).

clauses
    contact("Thomas", "000-001", address("Erevan", "Manhatan", 1, 2)).
    contact("Artur", "000-002", address("London", "Kr. Armii", 3, 4)).
    contact("Thomas", "000-003", address("Paris", "Piva", 5, 6)).
    contact("John", "000-004", address("Moscow", "Mira", 7, 8)).
    contact("Mike", "000-005", address("Minsk", "Pobody", 9, 10)).
 
    owner("Thomas", car("Lada", "black", "pbd001", 5000)).
    owner("John", car("Bentley", "yellow", "pbd002", 10000)).
    owner("John", water("Yacht", "brown", "pbd003", 4000)).
    owner("Mike", building("House", 57, 5000)).
    owner("Artur", land("Hotel", 500, 4000)).
    owner("Artur", building("Bar", 4, 5000)).
    owner("Artur", car("Bentley", "brown", "GGG777", 10000)).

    bank("Thomas", "Sber", "debit", 5000).
    bank("John", "Tinkoff", "credit", 10000).
    bank("John", "Sovkombank", "credit", 4000).
    bank("Mike", "Sber", "debit", 5000).
    bank("Artur", "Sovkombank", "credit", 4000).
 
    prop_names(Surname, Name) :- owner(Surname, car(Name, _, _, _)); 
    				             owner(Surname, building(Name, _, _)); 
    				             owner(Surname, land(Name, _, _)); 
    				             owner(Surname, water(Name, _, _, _)).
 
    prop_prices(Surname, Name, Price) :- owner(Surname, car(Name, _, _, Price)); 
    				                     owner(Surname, building(Name, _, Price)); 
    				                     owner(Surname, land(Name, _, Price));
    				                     owner(Surname, water(Name, _, _, Price)).
 				  
    car_price(Surname, Price) :- owner(Surname, car(_, _, _, Price)), !; Price = 0.
    building_price(Surname, Price) :- owner(Surname, building(_, _, Price)), !; Price = 0.
    land_price(Surname, Price) :- owner(Surname, land(_, _, Price)), !; Price = 0.
    water_price(Surname, Price) :- owner(Surname, water(_, _, _, Price)), !; Price = 0.
 				  
    sum_price(Surname, S) :- car_price(Surname, Pc),
    				         building_price(Surname, Pb),
    				         land_price(Surname, Pl),
    				         water_price(Surname, Pw),
    				         S = Pc + Pb + Pl + Pw.
 
goal
    %prop_names("Artur", Name).
    %prop_prices("John", Name, Price).
    sum_price("John", SumPrice).
\end{lstlisting}

%\begin{table}[]
%\begin{tabular}{|l|l|l|}
%\hline
%№ шага & \begin{tabular}[c]{@{}l@{}}Сравниваемые термы; \\ результат; подстановки\end{tabular}                                                                                                                                                                                                                                                                                                                                                                                                                      & \begin{tabular}[c]{@{}l@{}}Дальнейшие \\ действия\end{tabular}           \\ \hline
%1      & \begin{tabular}[c]{@{}l@{}}prop\_prices("John", Name, Price)=\\ contact("Thomas", "000-001", address("Erevan", "Manhatan", 1, 2))\\ не унифицируется, тк разные функторы\end{tabular}                                                                                                                                                                                                                                                                                                          & \begin{tabular}[c]{@{}l@{}}переход на \\ след. терм\end{tabular}         \\ \hline
%2      & аналогичные действия до совпадения функтора                                                                                                                                                                                                                                                                                                                                                                                                                                                                &          \\ \hline
%3      & \begin{tabular}[c]{@{}l@{}}prop\_prices("John", Name, Price)=\\ prop\_prices(Surname, Name, Price)\\ Константа "John" унифицируется с \\ несвязанной переменной Surname, Surname \\ принимает значение "John"; остальные \\ несвязанные переменные становятся сцепленными\end{tabular}                                                                                                                                                                                         & \begin{tabular}[c]{@{}l@{}}переход к \\ проверке \\ условий\end{tabular} \\ \hline
%4      & \begin{tabular}[c]{@{}l@{}}owner(Surname, car(Name, \_, \_, Price))=\\ contact("Thomas", "000-001", address("Erevan", "Manhatan", 1, 2))\\ не унифицируется, тк разные функторы\end{tabular}                                                                                                                                                                                                                                                                                                            & \begin{tabular}[c]{@{}l@{}}переход на \\ след. терм\end{tabular}         \\ \hline
%5      & аналогичные действия до совпадения функтора                                                                                                                                                                                                                                                                                                                                                                                                                                                                &          \\ \hline
%6      & \begin{tabular}[c]{@{}l@{}}owner(Surname, car(Name, \_, \_, Price))=\\ owner("Thomas", car("Lada", "black", "pbd001", 5000))\\ не унифицируется, так как не совпадает константа \\ "Thomas" и значение связанной переменной Surname\end{tabular}                                                                                                                                                                                                                                                      & \begin{tabular}[c]{@{}l@{}}переход на \\ след. терм\end{tabular}         \\ \hline
%7      & аналогичные действия до совпадения функтора                                                                                                                                                                                                                                                                                                                                                                                                                                                                &          \\ \hline
%8      & \begin{tabular}[c]{@{}l@{}}owner(Surname, car(Name, \_, \_, Price))=\\ owner("John", car("Bentley", "yellow", "pbd002", 10000))\\ успешная унификация\\ Константа "John" унифицируется со\\ связанной переменной Surname; константа "Bentley"\\ унифицируется с несвязанной переменной Name, Name\\ принимает значение "Bentley"; анонимные переменные не \\ связываются со значениями; константа 10000 унифицируется \\ с несвязанной переменной Price, Price пр.знач. 10000\end{tabular} & \begin{tabular}[c]{@{}l@{}}переход на \\ след. терм\end{tabular}         \\ \hline
%9      & попытки унификации до конца БЗ                                                                                                                                                                                                                                                                                                                                                                                                                                                                             & \begin{tabular}[c]{@{}l@{}}переход на \\ след. терм\end{tabular}         \\ \hline
%10     & аналогичные действия для остальных функторов условий                                                                                                                                                                                                                                                                                                                                                                                                                                                       &          \\ \hline
%\end{tabular}
%\end{table}


\newcommand{\specialcell}[2][c]{%
  \begin{tabular}[#1]{@{}l@{}}#2\end{tabular}}
  
\begin{table}[]
\begin{tabular}{|l|l|l|}
\hline
\specialcell{№ шага} & \specialcell{Сравниваемые термы; \\ результат; подстановки}                                                                                                                                                                                                                                                                                                                                                                                                                     & \specialcell{Дальнейшие \\ действия} \\ \hline
1   & \specialcell{prop\_prices(<<John>>, Name, Price)=\\ contact(<<Thomas>>, <<000-001>>, address(<<New-York>>, \\ <<One>>, 1, 2)) \\ \textbf{Нет}} & \specialcell{Прямой ход} \\ \hline
2-18   & \specialcell{...} & \specialcell{} \\ \hline
19   & \specialcell{prop\_prices(<<John>>, Name, Price)=\\prop\_prices(Surname, Name, Price) \\ \textbf{Успех} \\ \textbf{Surname = <<John>>} \\ \textbf{Остальные~---~сцепленные} } & \specialcell{Прямой ход \\ Унификация \\ owner(Surname, car)} \\ \hline
20   & \specialcell{owner(Surname, car(Name, \_, \_, Price))=\\ contact(<<Thomas>>, <<000-001>>, address(<<New-York>>, \\ <<One>>, 1, 2))  \\ \textbf{Нет}} & \specialcell{Прямой ход} \\ \hline
21-29   & \specialcell{...} & \specialcell{} \\ \hline
30   & \specialcell{owner(Surname, car(Name, \_, \_, Price))=\\ owner(<<Thomas>>, car(<<Lada>>, <<black>>, <<pbd001>>, \\ 5000)) \\ \textbf{Нет}} & \specialcell{Прямой ход} \\ \hline
31-33   & \specialcell{...} & \specialcell{} \\ \hline
34   & \specialcell{owner(Surname, car(Name, \_, \_, Price))=\\ owner(<<Artur>>, land(<<Hotel>>, 500, 4000)) \\ \textbf{Нет}} & \specialcell{Прямой ход} \\ \hline
35   & \specialcell{...} & \specialcell{} \\ \hline
36  & \specialcell{owner(Surname, car(Name, \_, \_, Price))=\\ owner(<<Artur>>, car(<<Bentley>>, <<brown>>, <<GGG777>>, \\ 10000)) \\ \textbf{Успех} \\ \textbf{Surname = <<Artur>>} \\ \textbf{Name = <<Bentley>>} \\ \textbf{Price = 10000}} & \specialcell{Решение найдено \\ Откат} \\ \hline

\end{tabular}
\end{table}

\newpage

\begin{table}[]
\begin{tabular}{|l|l|l|}
\hline
\specialcell{№ шага} & \specialcell{Сравниваемые термы; \\ результат; подстановки}                                                                                                                                                                                                                                                                                                                                                                                                                     & \specialcell{Дальнейшие \\ действия} \\ \hline

37  & \specialcell{owner(Surname, car(Name, \_, \_, Price))=\\ prop\_names(Surname, Name) \\ \textbf{Нет}} & \specialcell{\specialcell{Прямой ход}} \\ \hline
38-42  & \specialcell{...} & \specialcell{} \\ \hline
43  & \specialcell{owner(Surname, car(Name, \_, \_, Price))=\\ sum\_price(Surname, S) \\ \textbf{Нет}} & \specialcell{Откат к шагу 19 \\ Унификация \\ owner(Surname, building)} \\ \hline
44-58  & \specialcell{...} & \specialcell{} \\ \hline
59  & \specialcell{owner(Surname, building(Name, \_, Price)) = \\ owner(<<Artur>>, building(<<Bar>>, 4, \\ 5000)) \\ \textbf{Успех} \\ \textbf{Surname = <<Artur>>} \\ \textbf{Name = <<Bar>>} \\ \textbf{Price = 5000}} & \specialcell{Решение найдено \\ Откат} \\ \hline
60-66  & \specialcell{...} & \specialcell{} \\ \hline
67  & \specialcell{owner(Surname, building(Name, \_, Price)) = \\ sum\_price(Surname, S)} & \specialcell{Откат к шагу 19 \\ Унификация \\ owner(Surname, land)} \\ \hline
68-81  & \specialcell{...} & \specialcell{} \\ \hline
82  & \specialcell{owner(Surname, land(Name, \_, Price)) = \\ owner(<<Artur>>, land(<<Hotel>>, 500, 4000)) \\ \textbf{Успех} \\ \textbf{Surname = <<Artur>>} \\ \textbf{Name = <<Hotel>>} \\ \textbf{Price = 4000}} & \specialcell{Решение найдено \\ Откат} \\ \hline
83-90  & \specialcell{...} & \specialcell{} \\ \hline
91  & \specialcell{owner(Surname, land(Name, \_, Price)) = \\ sum\_price(Surname, S) \\ \textbf{Нет}} & \specialcell{Откат к шагу 19 \\ Унификация \\ owner(Surname, water)} \\ \hline
92-114  & \specialcell{...} & \specialcell{} \\ \hline

\end{tabular}
\end{table}

\newpage
\noindent

\begin{table}[t!]
\begin{tabular}{|l|l|l|}
\hline
\specialcell{№ шага} & \specialcell{Сравниваемые термы; \\ результат; подстановки}                                                                                                                                                                                                                                                                                                                                                                                                                     & \specialcell{Дальнейшие \\ действия} \\ \hline

115  & \specialcell{owner(Surname, water(Name, \\ \_, \_, Price)) = sum\_price(Surname, S) \\ \textbf{Нет}} & \specialcell{Откат к шагу 19 \\ Унификация \\ prop\_prices \\ (<<John>>, Name, Price)} \\ \hline
116-119  & \specialcell{...} & \specialcell{} \\ \hline
120  & \specialcell{prop\_prices(<<John>>, Name, Price)=\\sum\_price(Surname, S) \\ \textbf{Нет}} & \specialcell{Прямой ход} \\ \hline
121 & Конец базы знаний & \\ \hline

\end{tabular}
\end{table}
