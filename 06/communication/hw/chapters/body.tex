\setcounter{page}{3}

\chapter{Об игре}

\section{Название}
Змейка.

\section{Цель игры}
Развитие навыков общения, группового взаимодействия. Адаптация в новом круге лиц, в том числе разного возраста. Развитие словарного запаса человека и расширение кругозора.

\section{Средства}
Игра не требует каких-либо дополнительных средств, кроме как самих игроков.

\section{Количество и возраст игроков}
Количество игроков для комфортной игры составляет от 2 до 5 человек.

\section{Время игры}
Зависит от кругозора игроков, начинается от нескольких минут, заканчивается тупиковой ситуацией, или <<пока не надоест>>.

\chapter{Ход игры}
Первым делом игроки выбирают тематику игры. Например, в качестве тематики могут выступать: города, страны, химические элементы, персонажи в играх, модели автомобилей, фирмы пива, и так далее.

На основе выбранной тематики первый игрок (выбирается случайным образом) произносит слово, относящееся к данной тематике.

Следующий за ним игрок (порядок игроков может быть любым, главное --- не меняться во время игры), должен произнести слово, относящееся к выбранной тематике, которое начинается на букву, с которой заканчивалось слово первого игрока.

Если игрок не может найти нужное слово, то он может передать ход другому игроку. Если никто не может найти подходящее слово, то поиск слова начинается снова, с учетом того, что первая буква должна совпадать с предпоследней буквой предыдущего слова.

Игра заканчивается, когда игроки не могут подобрать какие-либо слова по теме, которые начинаются с какой-либо буквы предыдущего слова, или когда игрокам надоест играть.

\chapter{Обсуждение}
Во время игры участники знакомятся между собой, активно обсуждают выбранную тему. Могут возникать споры по поводу существования произнесенного слова. По итогам этих споров игроки узнают много нового касаемо выбранной предметной области.

\newpage