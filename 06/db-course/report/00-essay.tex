\begin{essay}{}

\setcounter{page}{3}

\textbf{Ключевые слова:} база данных, интранет, социальная сесть, кеширование, Redis, PostgreSQL, сериализация.

В данной работе разработана база данных внутреннего портала для сотрудников предприятия. Для доступа к базе данных было реализовано приложение, работающее по модели клиент-сервер. Серверная часть реализована на языке Go, клиентская~---~на языке JavaScript с использованием фреймворка Vue.

Проведено исследование влияния кеширования на скорость поиска, а также влияние метода сериализации структур на скорость поиска и объем потребляемой закешированными данными памяти.

\end{essay}