\documentclass[a4paper,12pt]{extarticle}
%Gummi|065|=)
\usepackage[T1]{fontenc}
\usepackage[utf8]{inputenc}
\usepackage[russian]{babel}
\usepackage[left=3cm,right=1.5cm,
    top=2cm,bottom=2cm,bindingoffset=0cm]{geometry}
\usepackage{amsmath}
\usepackage{caption}
% \usepackage{color}
\usepackage{graphicx}
\usepackage{enumitem}
\usepackage{verbatimbox}
\usepackage{verbatim}
\usepackage{xcolor}
\usepackage{fancyvrb}
\usepackage{indentfirst}
\usepackage{alltt}


\linespread{1.3}
\begin{document}

% НАЧАЛО ТИТУЛЬНОГО ЛИСТА

    \begin{table}[h!]
     \begin{center}
     \begin{tabular}{  c  p{0.8\textwidth}  }
     % \hline
     \raisebox{-\totalheight}{\includegraphics[width=0.15\textwidth, height=30mm]{bmstu.png}}
      & 
      \begin{center}
          Министерство образования и науки Российской Федерации \\
      Федеральное государственное бюджетное образовательное учреждение высшего образования  \\
    
      «Московский государственный технический университет \\
      имени Н.Э. Баумана» \\
    
      (МГТУ им. Н.Э. Баумана)
      \end{center}
      \\
      \hline
      \end{tabular}
      \end{center}
\end{table}

\textbf{ }
\\
\textbf{ФАКУЛЬТЕТ: } \underline{Информатика и системы управления}
\\
\textbf{КАФЕДРА: } \underline{Программное обеспечение ЭВМ и информационные технологии}
\\
\textbf{ДИСЦИПЛИНА: } \underline{Типы и структуры данных}
\\
\textbf{ТЕМА: } \underline{Записи с вариантами. Обработка таблиц}
\\
\textbf{ВАРИАНТ: } \underline{9}

\vspace{1cm}

\begin{center}
    \large{\textbf{\underline{ОТЧЕТ ПО ЛАБОРАТОРНОЙ РАБОТЕ №2}}}
\end{center}

\textbf{ }
\vspace{1cm}
\\
\textbf{Студент: } \hspace{5.2cm} \underset{\text{(подпись, дата)}}{\underline{\hspace{0.3\textwidth}}} \textit{ Княжев А. В. }
\vspace{0.2cm}
\\
\textbf{Преподаватель: } \hspace{4cm} \underset{\text{(подпись, дата)}}{\underline{\hspace{0.3\textwidth}}} \textit{ Силантьева А. В. }

\vspace{7cm}
\begin{center} \textit{2021 г.} \end{center}
\thispagestyle{empty}
\newpage

\tableofcontents
\newpage

\section{Условие задачи}
Ввести  список  квартир,  содержащий  адрес,  общую  площадь, 
количество  комнат, стоимость  квадратного  метра,  первичное 
жилье или нет (первичное --- с отделкой или без нее;  вторичное ---
время  постройки,  количество  предыдущих  собственников, 
количество  последних  жильцов,  были  ли  животные).  Найти  все 
вторичное двухкомнатное комнатное жилье в указанном ценовом диапазоне 
без животных. 

\newpage

\section{Техническое задание}
Создать таблицу, содержащую не менее 40 записей с вариантной частью. Произвести 
поиск  информации  по  вариантному  полю.  Упорядочить  таблицу,  по  возрастанию  ключей 
(где  ключ --- любое  невариантное  поле  по  выбору  программиста),  используя:  
\begin{itemize}
    \item[(а)] исходную таблицу;
    \item[(б)] массив  ключей. 
\end{itemize}

Использовать  два  разных  алгоритма сортировки  (простой, 
ускоренный). Оценить  эффективность  этих  алгоритмов  (по  времени  и  по  используемому 
объему памяти) при различной реализации программы, то есть, в случаях (а) и (б). Обосновать 
выбор алгоритмов сортировки. Оценка эффективности должна быть относительной (в процентах).

Программа должна выводить меню с возможностью выбирать варианты. При вводе нуля на запрос варианта меню, происходит выход из программы.

\subsection{Входные данные}
\begin{itemize}
    \item[$*$] имя файла, содержащего информацию о таблице;
    \item[$*$] номер выбранного пункта меню.
\end{itemize}

\subsection{Выходные данные}
Выходные данные определяются выбранным пунктом меню.

\subsection{Описание пунктов меню}
\subsubsection{Вывод таблицы c информацией обо всех квартирах}
\paragraph{Выходные данные}
\begin{itemize}
    \item[$*$] таблица, содержащая исходные записи.
\end{itemize}

\subsubsection{Добавление записи в конец таблицы}
Выводит диалог, позволяющий ввести запись. Диалог выглядит как набор запросов ввода одного поля записи.
\paragraph{Входные данные}
\begin{itemize}
    \item[$*$] адрес квартиры --- строка, длиной до 256 символов;
    \item[$*$] площадь --- вещественное число;
    \item[$*$] количество комнат --- целое число;
    \item[$*$] цена за квадратный метр --- целое число;
    \item[$*$] статус --- первичная или вторичная (строка Primary или Secondary);
    \item[$*$] есть ли отделка (если первичная) --- выбор из пунктов подменю (1 или 2);
    \item[$*$] год постройки (если вторичная) --- целое число;
    \item[$*$] количество предыдущих владельцев (если вторичная) --- целое число;
    \item[$*$] количество жильцов (если вторичная) --- целое число;
    \item[$*$] жили ли животные (если вторичная) --- выбор из пунктов подменю (1 или 2).
\end{itemize}

\subsubsection{Удаление записи по полю}
Выводит диалог, в котором пользователь указывает поле, по которому задается критерий удаления.

\paragraph{Входные данные}
\begin{itemize}
    \item[$*$] номер поля, по которому выставляется значение критерия удаления;
    \item[$*$] значение указанного поля (формат как в пункте про добавление записей).
\end{itemize}

\subsubsection{Фильтрация}
Выводит информацию о вторичных двухкомнатных квартирах в указанном ценовом диапазоне без животных.

\paragraph{Входные данные}
\begin{itemize}
    \item[$*$] минимальное значение цены квартиры --- целое число;
    \item[$*$] максимальное значение цены квартиры --- целое число.
\end{itemize}

\paragraph{Выходные данные}
\begin{itemize}
    \item[$*$] таблица с информацией о записях, удовлетворяющих критериям фильтрации.
\end{itemize}

\subsubsection{Сортировка таблицы ключей методом быстрой сортировки}
Сортирует методом быстрой сортировки и выводит таблицу ключей.

Здесь и далее, \textbf{ключ сортировки --- количество комнат}.

\paragraph{Выходные данные}
\begin{itemize}
    \item[$*$] отсортированная таблица с информацией о ключах.
\end{itemize}

\subsubsection{Сортировка таблицы ключей методом сортировки вставками}
Сортирует методом сортировки вставками и выводит таблицу ключей.

\paragraph{Выходные данные}
\begin{itemize}
    \item[$*$] отсортированная таблица с информацией о ключах.
\end{itemize}

\subsubsection{Сортировка таблицы методом быстрой сортировки}
Сортирует методом быстрой сортировки и выводит таблицу записей.

\paragraph{Выходные данные}
\begin{itemize}
    \item[$*$] отсортированная таблица с информацией о записях.
\end{itemize}

\subsubsection{Сортировка таблицы методом сортировки вставками}
Сортирует методом сортировки вставками и выводит таблицу записей.

\paragraph{Выходные данные}
\begin{itemize}
    \item[$*$] отсортированная таблица с информацией о записях.
\end{itemize}

\subsubsection{Сортировка таблицы методом быстрой сортировки с помощью таблицы ключей}
Сортирует таблицу ключей методом быстрой сортировки и выводит отсортированную таблицу записей с помощью таблицы ключей. При этом, таблица записей не изменяется. 

\paragraph{Выходные данные}
\begin{itemize}
    \item[$*$] отсортированная таблица с информацией о записях.
\end{itemize}

\subsubsection{Сортировка таблицы методом сортировки вставками с помощью таблицы ключей}
Сортирует таблицу ключей методом сортировки вставками и выводит отсортированную таблицу записей с помощью таблицы ключей. При этом, таблица записей не изменяется. 

\paragraph{Выходные данные}
\begin{itemize}
    \item[$*$] отсортированная таблица с информацией о записях.
\end{itemize}

\subsubsection{Возврат таблицы к исходному состоянию}
Перепрочитывает исходную таблицу из файла, тем самым отменяя все изменения, совершенные программой.

\subsubsection{Сравнение эффективности сортировки таблицы с помощью таблицы ключей и без}
Сравнивает эффективность сортировки таблицы с помощью таблицы ключей и без нее.

\paragraph{Выходные данные}
\begin{itemize}
    \item[$*$] таблица с информацией о среднем времени выполнения сортировки без таблицы ключей для быстрой сортировки и сортировки вставками;
    \item[$*$] таблица с информацией о среднем времени выполнения сортировки с таблицей ключей для быстрой сортировки и сортировки вставками;
\end{itemize}

\subsubsection{Сравнение эффективности сортировки таблицы с помощью различных методов сортировки}
Сравнивает эффективность сортировки таблицы записей и таблицы ключей с помощью быстрой сортировки и сортировки вставками.

\paragraph{Выходные данные}
\begin{itemize}
    \item[$*$] таблица с информацией о среднем времени выполнения сортировки для быстрой сортировки и сортировки вставками;
    \item[$*$] таблица с информацией о среднем времени выполнения сортировки таблицы ключей для быстрой сортировки и сортировки вставками;
\end{itemize}

\subsection{Действие программы}
Программа осуществляет работу с таблицей, содержащей информацию о квартирах.

\subsection{Обращение к программе}
Программа может быть запущена из командных оболочек \texttt{sh/bash/zsh/fish}, а также от IDE, способных работать с языком Си. Программа не принимает никаких аргументов.

\subsection{Аварийные ситуации}
\subsubsection{Общие аварийные ситуации}
\begin{enumerate}
    \item строка имени файла с таблицей слишком длинная;
    \item невалидный номер пункта меню (строка или пустая строка);
    \item неверный номер пункта меню (такого пунта меню не существует);
    \item ошибка считывания таблицы из файла: есть невалидные данные;
    \item файл с таблицей не существует.
\end{enumerate}

\subsubsection{Пункт меню №1}
\begin{enumerate}
    \item таблица пуста.
\end{enumerate}

\subsubsection{Пункт меню №2}
\begin{enumerate}
    \item строка, содержащая адрес, слишком длинная;
    \item площадь не является валидным вещественным числом;
    \item площадь меньше или равна нулю;
    \item количество комнат не является валидным целым числом;
    \item количество комнат меньше или равно нулю;
    \item цена за квадратный метр не является валидным целым числом;
    \item цена за квадратный метр меньше или равна нулю;
    \item номер пункта статуса не является валидным целым числом;
    \item номер пункта статуса не существует;
    \item номер пункта наличия отделки не является валидным целым числом;
    \item номер пункта наличия отделки не существует;
    \item год постройки не является валидным целым числом;
    \item год постройки меньше или равен нулю;
    \item количество владельцев не является валидным целым числом;
    \item количество владельцев меньше или равно нулю;
    \item количество жильцов не является валидным целым числом;
    \item количество жильцов меньше или равно нулю;
    \item номер пункта проживания животных не является валидным целым числом;
    \item номер пункта проживания животных не существует;
    \item таблица уже заполнена.
\end{enumerate}

\subsubsection{Пункт меню №3}
\begin{enumerate}
    \item номер поля, задающего критерий удаления не является валидным целым числом;
    \item номер поля, задающего критерий удаления не существует;
    \item для ввода значения поля ограничения такие же, как для пункта меню №2.
\end{enumerate}

\subsubsection{Пункт меню №4}
\begin{enumerate}
    \item нижняя граница цены не является валидным целым числом;
    \item верхняя граница цены не ялвется валидным целым числом;
    \item ни одна квартира не удовлетворяет критерию поиска.
\end{enumerate}

\subsubsection{Пункт меню №5}
\begin{enumerate}
    \item таблица пуста.
\end{enumerate}

\subsubsection{Пункт меню №6}
\begin{enumerate}
    \item таблица пуста.
\end{enumerate}

\subsubsection{Пункт меню №7}
\begin{enumerate}
    \item таблица пуста.
\end{enumerate}

\subsubsection{Пункт меню №8}
\begin{enumerate}
    \item таблица пуста.
\end{enumerate}

\subsubsection{Пункт меню №9}
\begin{enumerate}
    \item таблица пуста.
\end{enumerate}

\subsubsection{Пункт меню №10}
\begin{enumerate}
    \item таблица пуста.
\end{enumerate}

\newpage

\section{Структуры данных}
В данной работе используется запись со статическими полями с вариативной частью --- с помощью нее хранятся записи о квартирах. 

\subsection{Основные константы}
\begin{verbbox}

\end{verbbox}
\setlength{\fboxsep}{5pt}
\setlength{\fboxsep}{10pt}
\fbox{\theverbbox}

\subsection{Модуль для работы с ошибками}
\subsubsection{Структуры данных}
\begin{verbbox}

\end{verbbox}
\setlength{\fboxsep}{5pt}
\setlength{\fboxsep}{10pt}
\fbox{\theverbbox}




\newpage
\section{Описания алгоритмов}
В случае неуспешности каких-либо операций, происходит вывод сообщения об ошибке и завершение работы программы с ненулевым кодом возврата.

\begin{enumerate}
    \item считываются входные данные;
    \item делимое делится на делитель;
    \item результат преобразуется в строку;
    \item результирующая строка выводится.
\end{enumerate}

\subsection{Считывание входных данных}
\begin{enumerate}
    \item считывание целого числа в виде строки (\texttt{f\_read\_line});
    \item проверка строки на переполнение;
    \item считывание вещественного числа в виде строки;
    \item проверка строки на переполнение;
    \item перевод строки с целым числом в большое целое (\texttt{bi\_from\_str});
    \item перевод большого целого в большое число с плавающей точкой (\texttt{bf\_from\_bigint});
    \item перевод строки с вещественным в большое число с плавающей точкой (\texttt{bf\_from\_str}).
\end{enumerate}

\subsection{Сложение (вычитание) больших целых чисел (\texttt{bi\_add, bi\_diff})}
Используется метод сложения <<в столбик>>. Большие целые числа хранятся в виде знака (целое число), и массива цифр, причем массив хранится в перевернутом виде.

\begin{enumerate}
    \item если операнд отрицательный, происходит инверсия знаков всех его цифр, то есть если есть число $-42$, то оно записывается при сложении, как \texttt{[-2 -4]};
    \item происходит проход числа <<справа налево>> (если смотреть в прямой записи), как только встречаются ненулевые разряды, проверяется знак итоговой суммы;
    \item при сложении цифр проверяется размер получившейся суммы, а также его знак, он должен соответствовать итоговому знаку суммы, определенному ранее;
    \item если цифра не соответствует (в данному случа, по модулю больше 10 или имеет знак, отличный от итогового), то оно приводится к нормальному виду с помощью прибавления/вычитания 10 и переноса единицы со знаком <<$+$>> или <<$-$>> на следующий разряд;
    \item происходит обратное приведение цифр к нормальному виду, то есть если они были отрицательные, они становятся положительными.
\end{enumerate}

\subsection{Деление больших целых (\texttt{bi\_div})}
\begin{enumerate}
    \item сравнивается количество цифр делимого и делителя;
    \item если количество цифр делителя больше, то возвращается нуль;
    \item происходит определение порядка $n$ такого, чтобы длина делимого и делителя умноженного на $10^n$ совпадали;
    \item пока возможно, вычитается домноженный делитель из делимого. Подсчитывается количество таких вычитаний;
    \item это количество вычитаний ставится на $n$-ую позицию результата;
    \item домноженный делитель делится на 10;
    \item значение $n$ уменьшается на единицу;
    \item все это происходит, пока $n \geq 0$.
\end{enumerate}

\subsection{Деление больших чисел с плавающей точкой (\texttt{bf\_div})}
\begin{enumerate}
    \item к мантиссе дописывается $30$ нулей (умножение мантиссы на $10^{30}$);
    \item происходит целочисленное деление домноженной мантиссы делимого на мантиссу делителя;
    \item порядок делителя вычитается из порядка делимого;
    \item происходит нормализация числа.
\end{enumerate}

\subsection{Нормализация большого числа с плавающей точкой \\ (\texttt{\_\_to\_normal})}
\begin{enumerate}
    \item происходит поиск первого вхождения ненулевой цифры мантиссы (в прямом порядке записи первой слева);
    \item происходит сдвиг вправо с округлением таким образом, чтобы эта цифра стояла на $29$ позиции;
    \item порядок изменяется в соответствие со сдвигом.
\end{enumerate}

\subsection{Сдвиг вправо с округлением (\texttt{bi\_rshift\_rounded})}
\begin{enumerate}
    \item если сдвигаем на отрицательное количество позиций, то вызывается сдвиг влево без округления;
    \item происходит проверка ближайшего элемента, который обрежется при округлении;
    \item происходит сдвиг числа вправо;
    \item если обрезанный элемент больше пяти, то к результату прибавляется единица.
\end{enumerate}

\subsection{Перевод строки в большое число с плавающей точкой \\ (\texttt{bf\_from\_str})}
\begin{enumerate}
    \item строка разбивается на мантиссу и порядок. Порядка может не быть, а вот мантисса быть обязана;
    \item мантисса переводится в большое целое с соответствующим положению точки изменению порядка:
    \begin{enumerate}
        \item если в начале мантиссы <<$-$>>, то сразу меняем ее знак;
        \item проходим по строке, пока не встретим точку или конце строки;
        \item если встречаем цифру, то записываем ее в соответсвующий элемент массива, содержащего цифры мантиссы. Запись начинается в обратном порядке с 29 по 0 элементы (так как в мантиссе не более 30 знаков);
        \item если встречаем что-то другое, выходим с ошибкой;
        \item прибавляем к порядку количество значащих цифр до точки;
        \item проходим по строке после точки до конца;
        \item если видим цифру, то дописываем ее к мантиссе;
        \item если еще не встретили ни одной значащей цифры, то вычитаем из порядка единицу при каждой встретившийся цифре;
        \item если встречаем не цифру, выходим с ошибкой.
    \end{enumerate}
    \item проверяем мантиссу на переполнение;
    \item порядок переводится в целое число:
    \begin{enumerate}
        \item проходим по строке;
        \item если встречаем знак <<$-$>>, то сохраняем его;
        \item проходим по всем цифрам, если встречаем не цифру, то выходим в ошибкой;
        \item если встречаем цифру, то умножаем порядок на $10$ и прибавляем цифру;
        \item если был минус в начале, то домножаем порядок на $-1$.
    \end{enumerate}
    \item проверяем порядок на переполнение;
    \item складываем изменение порядка от мантиссы и считанный порядок.
\end{enumerate}

\subsection{Перевод большого числа с плавающей точкой в строку    \\ (\texttt{bf\_to\_str})}
\begin{enumerate}
    \item записывается знак мантиссы и \texttt{0.};
    \item записывается сама мантисса (значащие разряды);
    \item записывается \texttt{E};
    \item записывается порядок со знаком.
\end{enumerate}

\subsection{Перевод строки в большое целое число (\texttt{bf\_from\_bigint})}
\begin{enumerate}
    \item проверяем строку на переполнение;
    \item определяется знак числа (первый символ);
    \item происходит проход строки от конца к началу;
    \item если встречаем не цифру, выходим с ошибкой;
    \item ставим цифру на соотвествующую позицию.
\end{enumerate}

\subsection{Перевод большого целого в большое число с плавающей точкой}
\begin{enumerate}
    \item присвоение мантиссе значения большого целого;
    \item изменение порядка числа, увеличение его на 30 разрядов;
    \item нормализация числа.
\end{enumerate}

\newpage

\section{Тестирование}
Для проверко корректности работы программы было проведено функциональное тестирование. Таблица с тестовыми данными для <<позитивных>> и <<негативных>> случаев приведена ниже.

\subsection{<<Негативные>> тесты}
\begin{tabular}{ |p{4cm}|p{5cm}|p{5cm}| }
\hline
\textbf{Название} & \textbf{Входные данные} & \textbf{Выходные данные} \\ \hline
введена пустая строка &  & Строка пуста (f\_read\_line). \\ \hline
некорректное целое число & 1a \textbackslash n 1 & Некорректное число (bi\_from\_str). \\ \hline
в целом числе введен только знак & - \textbackslash n 1 & Целое число пустое (bi\_from\_str). \\ 
\hline
в мантиссе второго числа больше двух точек & 1 \textbackslash n 1.2.3 & Некорретная дробная часть (\_\_parse\_significant). \\
\hline
введена строка из пробелов &      & Строка пуста (f\_read\_line). \\ \hline
введен неправильный знак в целом числе & *34 \textbackslash n 28 & Некорректное число (bi\_from\_str). \\ \hline
введен неправильный знак в мантиссе вещественного & 34 \textbackslash n *28.22 & Некорретная целая часть (\_\_parse\_significant). \\ \hline
введен неправильный знак в порядке & 10 \textbackslash n 1e*10 & Некорретная степень (\_\_parse\_exponent). \\ \hline
введены не цифры в целой части мантиссы & 10 \textbackslash n -1a.10e5 & Некорретная целая часть (\_\_parse\_significant). \\ \hline
введены не цифры в дробной части мантиссы & 10 \textbackslash n -10.1ae5 & Некорретная дробная часть (\_\_parse\_significant). \\ \hline
введены не цифры в порядке & 10 \textbackslash n 10.4e17y & Некорретная степень (\_\_parse\_exponent). \\ \hline
деление на нуль & 10 \textbackslash n 0 & Деление на нуль (bi\_div). \\ \hline
переполнение при делении & 10000000000000 \textbackslash n 1e-99999 & Переполнение экспоненты (bf\_div). \\ \hline
\end{tabular}

\begin{tabular}{ |p{4cm}|p{5cm}|p{5cm}| }
\hline
\textbf{Название} & \textbf{Входные данные} & \textbf{Выходные данные} \\ \hline
слишком большое целое & 10000000000000000000000 00000000 \textbackslash n 1 & Слишком большое целое число. (bf\_from\_bigint). \\ \hline
слишком большая мантисса & 1 \textbackslash n 999.999999999999999999 999999999e-10 & Мантисса слишком большая (\_\_parse\_significant). \\ \hline
слишком большой порядок & 1 \textbackslash n 1e100000 & Переполнение экспоненты (bf\_from\_str). \\ \hline
слишком маленький порядок & 1 \textbackslash n 1e-100000 & Переполнение экспоненты (bf\_from\_str). \\ \hline
\end{tabular}

\subsection{<<Позитивные>> тесты}
\begin{tabular}{ |p{4cm}|p{5cm}|p{5cm}| }
\hline
\textbf{Название} & \textbf{Входные данные} & \textbf{Выходные данные} \\ \hline
деление числа само на себя & 12 \textbackslash n 12 & +0.1E+1 \\ \hline
обычный тест & 1 \textbackslash n 2 & +0.5E+0 \\ \hline
длинный результат & 1 \textbackslash n 3 & +0.33333333333333333333 3333333333E+0 \\ \hline
обычный тест & 4444444444444444444444 44444444 \textbackslash n 2.2 & +0.202020202020202020202 02020202E+30 \\ \hline

целое очень большое & 999999999999999999999 999999999 \textbackslash n 1 & +0.999999999999999999999 999999999E+30 \\ \hline

деление большого числа самого на себя & 999999999999999999999 999999999 \textbackslash n 999999999999999999999 999999999 & +0.1E+1 \\ \hline

крайний порядок перед переполнением & 1 \textbackslash n 1e-99998 & +0.1E+99999 \\ \hline

округление разрядов & 999999999999999999999 999999999 \textbackslash n 4 & +0.25E+30 \\ \hline
деление на единицу & 2929291 \textbackslash n 1000e-3 & +0.2929291E+7 \\ \hline
деление нуля на число & 0 \textbackslash n 20e9999 & +0.0E+0 \\ \hline
порядок близок к граничному & 1 \textbackslash n 100e-99999 & +0.1E+99998 \\ \hline
\end{tabular}
\newpage

\section{Контрольные вопросы}
\subsection{Каков возможный диапазон чисел, представляемых в ПК?}
Каждый тип данных характеризуется определенным диапазоном  значений  чисел,  который,  в  свою  очередь,  зависит  от  размера области памяти, выделяемой под хранение переменной этого типа, от наличия знака в числе и от типа представления числа (целое или вещественное). 

Для 64-битного типа данных, например беззнакового, диапазон значений будет $[0, 2^{64}-1]$, то есть $[0, 18 446 744 073 709 551 615]$.

\subsection{Какова возможная точность представления чисел, чем она определяется?}
Для целых чисел количество разрядов определяет точность представления чисел. Для вещественных точность определяется длиной мантиссы.

В современных 64-битных компьютерах максимальный размер чисел (как вещественных, так и целых) обычно ограничивается 64 разрядами. То есть целое можно представить с точностью в 64 двоичных разряда. В то же время, из 64 разрядов под мантиссу числа с плавающей точкой выделяется 52 разряда, под представление порядка --- 11. 52 двоичных разряда соответствует около 15 десятичных разрядов, поэтому, если нам нужна точность более 15 знаков, мы прибегаем к необходимости реализовывать свои типы для хранения и обработки таких чисел.

\subsection{Какие стандартные операции возможны над числами?}
Сложение, вычитание, умножение, деление, инкремент, декремент, изменение знака, сдвиги.

\subsection{Какой тип данных может выбрать программист, если обрабатываемые числа превышают возможный диапазон представления чисел в ПК?}
Можно использовать массив цифр или символов, также число можно разбивать по кусочкам и хранить массив, состоящий из этих кусочков. 

Также хорошим методом для хранения таких чисел является запись, содержащая информацию о цифрах числа и его знаке, в случае чисел с плавающей запятой --- мантиссе и порядке.

\subsection{Как  можно  осуществить  операции  над  числами,  выходящими  за рамки машинного представления?}
Так как соотвествующих типов под такие числа нет, и их создаем мы, то соответствующие операции тоже реализовывать должны мы. Тогда, например, для сложения, вычитания и умножения можно использовать стандартные алгоритмы <<в столбик>>, для деления <<уголок>>.


\newpage

\section{Вывод}
В данной работе я ознакомился с числами с плавающей точкой большой точности. Не всегда нам подходят числовые типы данных, предоставляемые языком (например из-за отсутствия требуемой точности), так что в данном случае, логику работы числел с плавающей запятой пришлось реализовывывать самому. 

В качестве метода сложения/вычитания чисел был использовать метод наподобие <<столбика>>, для деления метод, похожий на школьный <<уголок>>. Данные методы довольно просты и удобны для реализации и восприятия.

\newpage
\begin{thebibliography}{2}
\addcontentsline{toc}{section}{Список литературы}
\bibitem{method}
Методические рекомендации по лабораторной работе №1 (\emph{http://wwwcdl.bmstu.ru/}) 
\end{thebibliography}

\end{document}
