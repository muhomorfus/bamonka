\chapter{Теоретическая часть}

\textbf{Цель работы:} приобретение практических навыков реализации наиболее важных (но не всех) видов преобразований грамматик, чтобы удовлетворить требованиям алгоритмов синтаксического разбора.

\textbf{Задачи работы:}

\begin{enumerate}
 \item Принять к сведению соглашения об обозначениях, принятые в литературе по теории формальных языков и грамматик и кратко описанные в приложении. 
 \item Познакомиться с основными понятиями и определениями теории формальных языков и грамматик.
 \item Разработать, тестировать и отладить программу распознавания цепочек регулярного или праволинейного языка в соответствии с предложенным вариантом грамматики.
 \item Детально разобраться в алгоритме устранения левой рекурсии.
 \item Разработать, тестировать и отладить программу устранения левой рекурсии.
 \item Разработать, тестировать и отладить программу преобразования грамматики в соответствии с предложенным вариантом.
\end{enumerate}

\section{Задание}

\begin{enumerate}
 \item Постройте программу, которая в качестве входа принимает приведенную КС-грамматику $G = (N, \Sigma, P, S)$ и преобразует ее в эквивалентную КС-грамматику $G'$ без левой рекурсии.
 \item Постройте программу, которая в качестве входа принимает приведенную КС-грамматику $G = (N, \Sigma, P, S)$ без правил вида $S \rightarrow \epsilon$ и преобразует ее в эквивалентную КС-грамматику $G'$ в нормальной форме Грейбах.
\end{enumerate}

   
