\chapter{Теоретическая часть}

\textbf{Цель работы:} приобретение практических навыков реализации синтаксически управляемого перевода.

\textbf{Задачи работы:}

\begin{enumerate}
	\item Разработать, тестировать и отладить программу синтаксического анализа
          в соответствии с предложенным вариантом грамматики.
	\item Включить в программу синтаксического анализ семантические действия
          для реализации синтаксически управляемого перевода инфиксного
          выражения в обратную польскую нотацию.
\end{enumerate}

\section{Задание}

Реализовать синтаксически управляемый перевод инфиксного выражения
в обратную польскую нотацию для грамматики выражений из лабораторной
работы №3. Для построения дерева разбора использовать синтаксический
анализатор для данной грамматики разработанный в лабораторной работы №3.

Грамматика по варианту:

\includelistingpretty{grammar}{}{Грамматика по варианту}

Грамматика по варианту после удаления левой рекурсии и добавления блока:

\includelistingpretty{without_left}{}{Грамматика по варианту после удаления левой рекурсии}


