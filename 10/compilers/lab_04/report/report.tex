\documentclass{bmstu}

\newcommand{\specialcell}[2][c]{%
	\begin{tabular}[#1]{@{}c@{}}#2\end{tabular}}
	
\renewcommand\labelitemi{---}
\renewcommand\labelitemii{---}

\usepackage{MnSymbol}

\usepackage{amsmath}
\DeclareMathOperator{\equaldot}{\mathrel{=\hskip-0.64em\cdot\hskip0.15em}}

\usepackage{multirow}

\newcommand{\img}[3] {
	\begin{figure}[H]
		\center{\includegraphics[width=#1]{inc/img/#2}}
		\caption{#3}
		\label{img:#2}
	\end{figure}
	}

\begin{document}

\makereporttitle{Информатика и системы управления}{Программное обеспечение ЭВМ и информационные технологии}{лабораторной работе №4}{Конструирование компиляторов}{Синтаксически управляемый перевод}{3}{А.~В.~Княжев/ИУ7-22М}{А.~А.~Ступников}

\chapter{Теоретическая часть}

\section{Задание}

\textbf{Цель работы:} изучение венгерского метода решения задачи о назначениях.

\textbf{Содержание работы:}

\begin{enumerate}
	\item реализовать венгерский метод решения задачи о назначениях в виде программы на ЭВМ;
	\item провести решение задачи с матрицей стоимостей, заданной в индивидуальном варианте, рассмотрев два случая:
	\begin{enumerate}
		\item задача о назначениях является задачей минимизации,
		\item задача о назначениях является задачей максимизации.
	\end{enumerate}
\end{enumerate}

\textbf{Индивидуальное задание:}

\begin{equation}
	C = 
	\begin{bmatrix}
		7 & 4 & 3 & 8 & 2 \\
		4 & 5 & 1 & 6 & 3 \\
		8 & 4 & 5 & 7 & 2 \\
		1 & 2 & 4 & 7 & 2 \\
		3 & 9 & 9 & 2 & 5
	\end{bmatrix}
\end{equation}

\section{Содержательная постановка задачи}

В распоряжении $n$ работ и $n$ исполнителей. Стоимость выполнения работы $c_{ij} \geq 0$ единиц, где $i$~---~номер работы, $j$~---~номер исполнителя. Нужно распределить работы по исполнителям так, чтобы:

\begin{enumerate}
	\item каждый исполнитель выполнял только одну работу;
	\item стоимость выполнения работ была минимальной.
\end{enumerate}

Запишем $c_{ij}$ в матрицу $C$:

\begin{equation}
	C = (c_{ij})_{i,j = \overline{1, n}} \text{~---~матрица стоимостей}.
\end{equation}

Для математического описания распределения работ используем булеву матрицу:

\begin{equation}
	X = (x_{ij})_{i,j = \overline{1; n}}  \text{~---~матрица назначений}.
\end{equation}

При этом: 

\begin{align}
	x_{ij} & =
	\begin{cases}
		1, \: \text{если $i$ работу делает $j$ исполнитель}, \\
		0, \: \text{иначе}.
	\end{cases} \nonumber
\end{align}

Тогда

\begin{enumerate}
	\item общая стоимость выполнения работ исполнителями:
	\begin{equation}
		f = \sum_{i=1}^n \sum_{j=1}^n c_{ij} x_{ij};
	\end{equation}
	
	\item условие того, что $j$ исполнитель делает только $i$ работу:
	\begin{equation}
		\sum_{i=1}^n x_{ij} = 1, \: j = \overline{1; n}.
	\end{equation}
	
	\item условие, что $i$ работу делает только $j$ исполнитель:
	\begin{equation}
		\sum_{j=1}^n x_{ij} = 1, \: i = \overline{1; n}.
	\end{equation}
\end{enumerate}

Таким образом, каждую работу делает только один исполнитель.

\section{Математическая постановка задачи}

\begin{equation}
	\begin{cases}
		f = \sum_{i=1}^n \sum_{j=1}^n c_{ij} x_{ij} \rightarrow min, \\
		\sum_{j=1}^n x_{ij} = 1, \: i = \overline{1, n}, \\
		\sum_{i=1}^n x_{ij} = 1, \: j = \overline{1, n}, \\
		x_{ij} \in \{0, 1\}, \: i,j = \overline{1, n}.
	\end{cases}
\end{equation}

\newpage

\section{Схема алгоритма}

\includeimage
    {general}
    {f}
    {h}
    {0.9\textwidth}
    {Общая схема алгоритма решения задачи о назначениях}
    
\newpage

\includeimage
    {improve}
    {f}
    {h}
    {1\textwidth}
    {Схема алгоритма улучшения СНН}

\chapter*{Вывод}

\textbf{Начало работ:} 01.03.2024. 

\textbf{Окончание работ:} 19.09.2024.

\textbf{Затраты:} 48 286 руб.

\textbf{Затраты на аренду сервера:} 6 210 руб.

\textbf{Трудозатраты:} 9 473 ч.

Программисты выполняют 29\% работы, затраты на них составляют 50\%~---~их труд стоит дорого, как и труд аналитиков (2\% работ~---~10\% затрат). Группа ввода данных выполняет 25\% задач, но затраты на нее составляют 11\%~---~то есть их труд обходится дешевле. Арендованный сервер требует 14\% бюджета, что является значительной долей. 

Перегружены оказались системный аналитик, художник-дизайнер и технический писатель. 

Проект укладывается в бюджет 50000 руб.


\chapter{Контрольные вопросы}

\section{Как может быть определён формальный язык?}

Формальный язык может быть определён, например:
\begin{enumerate}
 \item простым перечислением слов, входящих в данный язык. Этот способ, в основном, применим для определения конечных языков и языков простой структуры;
 \item словами, порождёнными некоторой формальной грамматикой;
 \item словами, порождёнными регулярным выражением;
 \item словами, распознаваемыми некоторым конечным автоматом;
 \item словами, порождёнными БНФ-конструкцией.
\end{enumerate}

\section{Какими характеристиками определяется грамматика?}

Грамматика определяется следующими характеристиками:
\begin{enumerate}
 \item $\Sigma$~---~набор (алфавит) терминальных символов;
 \item $N$~---~набор (алфавит) нетерминальных символов;
 \item $P$~---~набор правил вида: «левая часть» $\rightarrow$ «правая часть», где:
  \begin{itemize}
   \item «левая часть»~---~непустая последовательность терминалов и нетерминалов, содержащая хотя бы один нетерминал;
   \item «правая часть»~---~любая последовательность терминалов и нетерминалов;
  \end{itemize}
 \item $S$~---~стартовый (или начальный) символ грамматики из набора нетерминалов.
\end{enumerate}

\section{Дайте описания грамматик по иерархии Хомского.}

Грамматика с фразовой структурой $G$~---~это алгебраическая структура, упорядоченная четвёрка $(V_T, V_N, P, S)$, где:
\begin{itemize}
 \item $V_T$~---~алфавит (множество) терминальных символов;
 \item $V_N$~---~алфавит (множество) нетерминальных символов;
 \item $V = V_T \cup V_N$~---~словарь $G$, причём $V_T \cap V_N = \varnothing$;
 \item $P$~---~конечное множество продукций (правил) грамматики, $P \subseteq V^+ \times V^*$;
 \item $S$~---~начальный символ (источник).
\end{itemize}

Здесь $V^{*}$~---~множество всех строк над алфавитом $V$, а $V^{+}$~---~множество непустых строк над алфавитом $V$.

По иерархии Хомского, грамматики делятся на 4 типа, каждый последующий является более ограниченным подмножеством предыдущего (но и легче поддающимся анализу).
\begin{enumerate}
 \item неограниченные грамматики — возможны любые правила;
 \item контекстно-зависимые грамматики — левая часть может содержать один нетерминал, окруженный «контекстом» (последовательности символов, в том же виде присутствующие в правой части); сам нетерминал заменяется непустой последовательностью символов в правой части;
 \item контекстно-свободные грамматики — левая часть состоит из одного нетерминала;
 \item регулярные грамматики — более простые, эквивалентны конечным автоматам.
\end{enumerate}

\subsection{Неограниченные грамматики}
Это все без исключения формальные грамматики. Правила можно записать в виде: $\alpha \rightarrow \beta$, 
где $\alpha \in V^{+}$~---~любая непустая цепочка, содержащая хотя бы один нетерминальный символ, 
а $\beta \in V^{*}$~---~любая цепочка символов из алфавита.

\subsection{Контекстно-зависимые грамматики}
К этому типу относятся контекстно-зависимые (КЗ) грамматики и неукорачивающие грамматики. Для грамматики 
$G(V_{T},V_{N},P,S)$, $V=V_{T}\cup V_{N}$ все правила имеют вид:
\begin{itemize}
 \item $\alpha A\beta \rightarrow \alpha \gamma \beta$, где $\alpha ,\beta \in V^{*}$, $\gamma \in V^{+}$, $A\in V_{N}$. Такие грамматики относят к контекстно-зависимым.
 \item $\alpha \rightarrow \beta$, где $\alpha ,\beta \in V^{+}$, $1\leq |\alpha |\leq |\beta |$. Такие грамматики относят к неукорачивающим.
\end{itemize}

\subsection{Контекстно-свободные грамматики}
Для грамматики $G(V_{T},V_{N},P,S)$, $V=V_{T}\cup V_{N}$ все правила имеют вид: $A\rightarrow \beta$, где $\beta \in V^{+}$ (для неукорачивающих КС-грамматик) или $\beta \in V^{*}$ (для укорачивающих), 
$A\in V_{N}$. То есть грамматика допускает появление в левой части правила только нетерминального символа.

\subsection{Регулярные грамматики}
К третьему типу относятся регулярные грамматики (автоматные)~---~самые простые из формальных грамматик. Они являются контекстно-свободными, но с ограниченными возможностями.

Все регулярные грамматики могут быть разделены на два эквивалентных класса, которые для грамматики вида III будут иметь правила следующего вида:
\begin{itemize}
 \item $A\rightarrow B\gamma$ или $A\rightarrow \gamma$, где $\gamma \in V_{T}^{*}$, $A,B\in V_{N}$ (для леволинейных грамматик).
 \item $A\rightarrow \gamma B$ или $A\rightarrow \gamma$, где $\gamma \in V_{T}^{*}$, $A,B\in V_{N}$ (для праволинейных грамматик).
\end{itemize}

\section{Какие абстрактные устройства используются для разбора грамматик?}
\begin{enumerate}
 \item Для разбора слов из регулярных языков подходят формальные автоматы самого простого устройства, т. н. конечные автоматы. Их функция перехода задаёт только смену состояний и, возможно, сдвиг (чтение) входного символа.
 \item Для разбора слова из контекстно-свободных языков в автомат приходится добавлять «магазинную ленту» или «стек», в который при каждом переходе записывается цепочка на основе соответствующего алфавита магазина. Такие автоматы называют «магазинные автоматы».
 \item Для контекстно-зависимых языков разработаны ещё более сложные линейно-ограниченные автоматы, а для языков общего вида — машина Тьюринга.
\end{enumerate}

\section{Оцените временную и емкостную сложность предложенного вам алгоритма.}

Временная сложность~---~$O(|N|^2 \cdot |P|)$.

Ёмкостная сложность~---~$O(|N| \cdot |P|)$.

\end{document}