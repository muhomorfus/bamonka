\chapter{Теоретическая часть}

\textbf{Цель работы:} приобретение практических навыков реализации алгоритма рекурсивного спуска для разбора грамматики и построения синтаксического дерева.

\textbf{Задачи работы:}

\begin{enumerate}
	\item Познакомиться с методом рекурсивного спуска для синтаксического анализа.
	\item Разработать, тестировать и отладить программу построения синтаксического дерева методом рекурсивного спуска в соответствии с предложенным вариантом грамматики.
\end{enumerate}

\section{Задание}

\begin{enumerate}
	\item Дополнить грамматику по варианту блоком, состоящим из последовательности операторов присваивания (выбран стиль Алгол-Паскаль).
	\item Для модифицированной грамматики написать программу нисходящего синтаксического анализа с использованием метода рекурсивного спуска.
\end{enumerate}

Грамматика по варианту:

\includelistingpretty{grammar}{}{Грамматика по варианту}

Грамматика по варианту после удаления левой рекурсии и добавления блока:

\includelistingpretty{without_left}{}{Грамматика по варианту после удаления левой рекурсии и добавления блока}


