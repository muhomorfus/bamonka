\chapter{Теоретическая часть}

\section{Задание}

\textbf{Цель работы:} изучение венгерского метода решения задачи о назначениях.

\textbf{Содержание работы:}

\begin{enumerate}
	\item реализовать венгерский метод решения задачи о назначениях в виде программы на ЭВМ;
	\item провести решение задачи с матрицей стоимостей, заданной в индивидуальном варианте, рассмотрев два случая:
	\begin{enumerate}
		\item задача о назначениях является задачей минимизации,
		\item задача о назначениях является задачей максимизации.
	\end{enumerate}
\end{enumerate}

\textbf{Индивидуальное задание:}

\begin{equation}
	C = 
	\begin{bmatrix}
		7 & 4 & 3 & 8 & 2 \\
		4 & 5 & 1 & 6 & 3 \\
		8 & 4 & 5 & 7 & 2 \\
		1 & 2 & 4 & 7 & 2 \\
		3 & 9 & 9 & 2 & 5
	\end{bmatrix}
\end{equation}

\section{Содержательная постановка задачи}

В распоряжении $n$ работ и $n$ исполнителей. Стоимость выполнения работы $c_{ij} \geq 0$ единиц, где $i$~---~номер работы, $j$~---~номер исполнителя. Нужно распределить работы по исполнителям так, чтобы:

\begin{enumerate}
	\item каждый исполнитель выполнял только одну работу;
	\item стоимость выполнения работ была минимальной.
\end{enumerate}

Запишем $c_{ij}$ в матрицу $C$:

\begin{equation}
	C = (c_{ij})_{i,j = \overline{1, n}} \text{~---~матрица стоимостей}.
\end{equation}

Для математического описания распределения работ используем булеву матрицу:

\begin{equation}
	X = (x_{ij})_{i,j = \overline{1; n}}  \text{~---~матрица назначений}.
\end{equation}

При этом: 

\begin{align}
	x_{ij} & =
	\begin{cases}
		1, \: \text{если $i$ работу делает $j$ исполнитель}, \\
		0, \: \text{иначе}.
	\end{cases} \nonumber
\end{align}

Тогда

\begin{enumerate}
	\item общая стоимость выполнения работ исполнителями:
	\begin{equation}
		f = \sum_{i=1}^n \sum_{j=1}^n c_{ij} x_{ij};
	\end{equation}
	
	\item условие того, что $j$ исполнитель делает только $i$ работу:
	\begin{equation}
		\sum_{i=1}^n x_{ij} = 1, \: j = \overline{1; n}.
	\end{equation}
	
	\item условие, что $i$ работу делает только $j$ исполнитель:
	\begin{equation}
		\sum_{j=1}^n x_{ij} = 1, \: i = \overline{1; n}.
	\end{equation}
\end{enumerate}

Таким образом, каждую работу делает только один исполнитель.

\section{Математическая постановка задачи}

\begin{equation}
	\begin{cases}
		f = \sum_{i=1}^n \sum_{j=1}^n c_{ij} x_{ij} \rightarrow min, \\
		\sum_{j=1}^n x_{ij} = 1, \: i = \overline{1, n}, \\
		\sum_{i=1}^n x_{ij} = 1, \: j = \overline{1, n}, \\
		x_{ij} \in \{0, 1\}, \: i,j = \overline{1, n}.
	\end{cases}
\end{equation}

\newpage

\section{Схема алгоритма}

\includeimage
    {general}
    {f}
    {h}
    {0.9\textwidth}
    {Общая схема алгоритма решения задачи о назначениях}
    
\newpage

\includeimage
    {improve}
    {f}
    {h}
    {1\textwidth}
    {Схема алгоритма улучшения СНН}
