\documentclass[14pt]{extarticle}

% настройка шрифтов 
\usepackage[T2A]{fontenc}
\usepackage[utf8]{inputenc}
\usepackage{amsmath,amssymb}
\usepackage[russian]{babel}

% настройка геометрии листа
\usepackage{geometry}
\geometry{a4paper,
left=3cm,right=1.5cm,
top=2cm,bottom=2cm}

% все что связано с изображениями
\usepackage{graphicx} % вставка рисунков
\usepackage{float}

\newcommand{\img}[3] {
 \begin{figure}[H]
  \center{\includegraphics[width=#1]{img/#2}}
  \caption{#3}
  \label{img:#2}
 \end{figure}
}

% подписи к вставкам
\usepackage{caption} % Подпись картинок и таблиц
\captionsetup{labelsep=endash} % Разделитель между номером и текстом краткое тире и пробел
\captionsetup[figure]{name={Рисунок}} % Изменяет имя для всех фигур на "Рисунок"
%\captionsetup[figure]{justification=centering} % выравнивание подписи по центу для картинок
%\usepackage[justification=centering]{caption} % Настройка подписей всех float объектов
\captionsetup[table]{justification=centering, singlelinecheck=false} % центрирование многострочных подписей для таблиц
\captionsetup[lstlisting]{justification=raggedright, singlelinecheck=false} % выравнивание многострочных подписей по правому краю для листингов


\usepackage{tabularx} % расширенные возможности таблиц
\usepackage{tikz} % рисование графиков через сам латех

\usepackage{indentfirst} % красная строка
\usepackage{setspace}
\onehalfspacing % Полуторный интервал

% настройка листингов
\usepackage{listings}
\lstdefinestyle{mystyle}{ % Опеределение стиля 
	% language=C++,
	backgroundcolor=\color{white},
	basicstyle=\footnotesize\ttfamily,
	keywordstyle=\bfseries,
	stringstyle=\itshape,
	commentstyle=\color{gray}\slshape,
	numbers=left,
	numberstyle=\tiny,
	stepnumber=1,
	numbersep=5pt,
	frame=single,
	tabsize=4,
	captionpos=t,
	breaklines=true,
	breakatwhitespace=true,
	xleftmargin=10pt
}
\lstset{extendedchars=true, texcl=true}
\lstset{style=mystyle}

% Настройка оформления списков
\usepackage{enumitem} 
\setlist[enumerate,1]{label={\arabic*)}}
\setlist[itemize]{left=0.49cm}
\setlist[enumerate]{left=0.49cm}
\renewcommand{\labelitemi}{---}
\renewcommand{\labelitemii}{---}


% настройка титульника
\usepackage{wrapfig} % обтекание текста вокруг картинок и таблиц
% Cоздание полей
\RequirePackage[normalem]{ulem}
\RequirePackage{stackengine}
\newcommand{\longunderline}[1]
{
	\uline{#1\hfill\mbox{}}
}
\newcommand{\fixunderline}[3]
{
	$\underset{\text{#3}}{\text{\uline{\stackengine{0pt}{\hspace{#2}}{\text{#1}}{O}{c}{F}{F}{L}}}}$
}

% Создание горизонтальной линии
\makeatletter
\newcommand{\vhrulefill}[1]
{
	\leavevmode\leaders\hrule\@height#1\hfill \kern\z@
}
\makeatother

% Создание шапки титульной страницы
\newcommand{\titlepageheader}[2]
{
	\begin{wrapfigure}[7]{l}{0.14\linewidth}
		\vspace{3.4mm}
		\hspace{-8mm}
		\includegraphics[width=0.89\linewidth]{img/bmstu-logo}
	\end{wrapfigure}

	{
		\singlespacing \small
		Министерство науки и высшего образования Российской Федерации \\
		Федеральное государственное автономное образовательное учреждение \\
		высшего образования \\
		<<Московский государственный технический университет \\
		имени Н.~Э.~Баумана \\
		(национальный исследовательский университет)>> \\
		(МГТУ им. Н.~Э.~Баумана) \\
	}

	\vspace{-4.2mm}
	\vhrulefill{0.9mm} \\
	\vspace{-7mm}
	\vhrulefill{0.2mm} \\
	\vspace{2.8mm}

	{
		\small
		ФАКУЛЬТЕТ \longunderline{<<#1>>} \\
		\vspace{3.3mm}
		КАФЕДРА \longunderline{<<#2>>} \\
	}
}

% Создание поля студента
\RequirePackage{pgffor}

\newcommand*\titlepagestudentscontent{}

\newcommand{\maketitlepagestudent}[1]
{
	\foreach \s/\g in {#1} {
		\gappto\titlepagestudentscontent{Студент \fixunderline}
		\xappto\titlepagestudentscontent{{\g}}
		\gappto\titlepagestudentscontent{{25mm}{(Группа)} &}
		\gappto\titlepagestudentscontent{\fixunderline{}{40mm}{(Подпись, дата)} \vspace{1.3mm} &}
		\gappto\titlepagestudentscontent{\fixunderline}
		\xappto\titlepagestudentscontent{{\s}}
		\gappto\titlepagestudentscontent{{40mm}{(И.~О.~Фамилия)} \\}
	}
}

% Создание прочих полей
\newcommand*\titlepageotherscontent{}

\newcommand{\maketitlepageothers}[2]
{
	\foreach \c in {#2} {
		\gappto\titlepagestudentscontent{#1 &}
		\gappto\titlepagestudentscontent{\fixunderline{}{40mm}{(Подпись, дата)} \vspace{1.3mm} &}
		\gappto\titlepagestudentscontent{\fixunderline}
		\xappto\titlepagestudentscontent{{\c}}
		\gappto\titlepagestudentscontent{{40mm}{(И.~О.~Фамилия)} \\}
	}
}

% Установка исполнителей работы
\newcommand{\titlepageauthors}[3]
{
	{
		\renewcommand{\titlepagestudentscontent}{}
		\maketitlepagestudent{#1}
		
		\renewcommand{\titlepageotherscontent}{}
		\maketitlepageothers{#2}{#3}


		\small
		\begin{tabularx}{\textwidth}{@{}>{\hsize=.5\hsize}X>{\hsize=.25\hsize}X>{\hsize=.25\hsize}X@{}}
			\titlepagestudentscontent

			\titlepageotherscontent
		\end{tabularx}
	}
}

% Создание титульной страницы ТЗ
\newcommand{\maketzttitle}[9]
{
	\begin{titlepage}
		\centering

		\titlepageheader{#1}{#2}
		
		\vspace{15.8mm}
		
		#3
		
		\vspace{10mm}
		
		\textbf{Техническое задание}
		
		\vspace{55mm}
		\titlepageauthors{#4}{}{}
		\titlepageauthors{#5}{}{}
		\titlepageauthors{#6}{}{}
		\titlepageauthors{#7}{}{}
		\titlepageauthors{}{Преподаватель}{Т.~Н. Романова}
		\vspace{10mm}

		\textit{{\the\year} г.}
	\end{titlepage}

	\setcounter{page}{2}
}


% настройка приложений
\usepackage[titletoc, title]{appendix} %добавление приложений в оглавление

% настройка заголовков
\usepackage{titlesec}

\usepackage[unicode,pdftex]{hyperref} % Ссылки в pdf
\hypersetup{hidelinks}