\subsection*{Глоссарий}

\begin{enumerate}
	\item Репозиторий~---~файловое хранилище, позволяющее хранить файлы в виде древовидной структуры. Формат файлов может быть любым, например .docx, .py, .go, .pdf, .html, .epced и другие.
	\item Система контроля версий~---~система, записывающая изменения в файл или набор файлов в течение времени и позволяющая вернуться позже к определённой версии.
	\item Продукт, платформа, система~---~В данной работе термины «продукт», «платформа» и «система» взаимозаменяемы.
	\item Gitlab~---~монолитное веб-приложение жизненного цикла DevOps с открытым исходным кодом, представляющий систему управления репозиториями кода для Git с собственной вики, системой отслеживания ошибок, CI/CD пайплайном.
	\item Redmine~---~монолитное серверное веб-приложение с открытым исходным кодом для управления проектами и задачами, в том числе для отслеживания ошибок.
	%\item Экземпляр Redmine~---~приложение управления задачами Redmine с открытым исходным кодом, которое уже готово к использованию. 
	\item CAS~---~система авторизации МГТУ им.~Н.~Э. Баумана, которая позволяет авторизовываться во внешних приложениях с использованием университетской учетной записи.
\end{enumerate}

\subsection*{Введение}

Данное техническое задание составляется для проектирования платформы развертывания приложений для управления проектами Redmine и приложений системы контроля версий Git (далее~---~\textbf{Платформа}).

 Техническое задание выполняется в соответствии со стандартом ГОСТ 19.201-78 «Единая система программной документации. Техническое задание. Требования к содержанию и оформлению».


%В условиях современного образовательного процесса кафедры <<Программное обеспечение ЭВМ и информационные технологии>> МГТУ им. Н.Э. Баумана преподаватели сталкиваются с необходимостью эффективного управления учебными заданиями, такими как лабораторные работы, курсовые проекты и рубежные контроли, для студентов различных групп. 
%
%На некоторых курсах кафедры для выдачи заданий используется приложение для управления проектами Redmine, а выполненные задания сохраняются на сервере системы контроля версий Git GitLab. В настоящее время процесс развертывания и администрирования системы осуществляется вручную, что не только неэффективно, но и создает дополнительные сложности для преподавателей кафедры. Отсутствие автоматизированной системы управления заданиями и группами студентов приводит к снижению качества образовательного процесса и увеличению нагрузки на преподавателей. Кроме того, в текущей реализации системы не поддерживается шифрование трафика, что повышает риски безопасности.
%
%В связи с этим возникает необходимость разработки современной административной системы, которая позволит автоматизировать процессы развертывания приложений для управления проектами, серверов системы контроля версий Git, распределения заданий, управления группами и обеспечения конфиденциальности данных. Такая система должна быть удобной в использовании, легко разворачиваемой и адаптируемой под текущие нужды кафедры, что позволит значительно повысить эффективность работы и снизить нагрузку на преподавательский состав.
% 
%Данный проект представляет собой техническое задание на разработку системы автоматизированной проверки работ студентов, функциональность которой включает в себя развертывание приложений для управления проектами и задачами Redmine и серверов системы контроля версий Git, а также их администрирование.
%
%Система должна предоставлять следующие возможности: авторизация данных пользовательских аккаунтов; выдача заданий студентам по вариантам; проверка оформления отчетов и расчетно-пояснительных записок студентов на основе ГОСТ 7.32-2017 <<СИБИД. Отчет о научно-исследовательской работе. Структура и правила оформления>>; архивация работ и заданий. Техническое задание выполнено на основе ГОСТ 19.201–78 «ЕСПД. Техническое задание. Требования к содержанию и оформлению».

\subsection*{Краткое описание предметной области}

В условиях современного образовательного процесса кафедры <<Программное обеспечение ЭВМ и информационные технологии>> МГТУ им. Н.Э. Баумана преподаватели сталкиваются с необходимостью эффективного управления учебными заданиями, такими как лабораторные работы, курсовые проекты и рубежные контроли для студентов различных групп. 


На некоторых курсах кафедры для выдачи заданий используется приложение для управления проектами Redmine, а выполненные задания сохраняются на сервере системы контроля версий Git GitLab. 


Существующая система кафедры имеет ряд недостатков.

\begin{enumerate}
	\item Процесс развертывания системы осуществляется вручную, занимает от двух до трех часов.
	\item Экземпляры Redmine и Gitlab развернуты в единственном экземпляре для всех дисциплин кафедры. 
	\item Отсутствует автоматическая проверка отчетов и расчетно-пояснительных записок.
	\item При добавлении нового студента в группу все ранее выданные работы ему нужно выдавать вручную.
	\item Не поддерживается шифрование трафика, что повышает риски безопасности.
	\item По нормативным документам МГТУ им. Н.~Э. Баумана, работы студентов нужно хранить:
	\begin{itemize}
		\item лабораторные работы~---~2 года;
		\item курсовые работы~---~3 года;
		\item выпускные квалификационные работы~---~5 лет;
		\item работы не могут быть удалены, пока не будет отчислен последний студент группы;
	\end{itemize}
	текущая реализация системы не поддерживает автоматическую архивацию работ студентов.
\end{enumerate}

 Разрабатываемая \textbf{Платформа}  должна решить описанные выше недостатки.

%В настоящее время процесс развертывания и администрирования системы осуществляется вручную, что не только неэффективно, но и создает дополнительные сложности для преподавателей кафедры. Отсутствие автоматизированной системы управления заданиями и группами студентов приводит к снижению качества образовательного процесса и увеличению нагрузки на преподавателей. 

%Кроме того, в текущей реализации системы не поддерживается шифрование трафика, что повышает риски безопасности.

В связи с этим возникает необходимость разработки современной административной \textbf{Платформы}, которая позволит автоматизировать процессы развертывания приложений для управления проектами, серверов системы контроля версий Git, распределения заданий, управления группами и обеспечения конфиденциальности данных. Такая \textbf{Платформа} должна быть удобной в использовании, легко разворачиваемой и адаптируемой под текущие нужды кафедры, что позволит значительно повысить эффективность работы и снизить нагрузку на преподавательский состав.


%Предметная область проекта охватывает автоматизацию процессов управления учебными заданиями на кафедре <<Программное обеспечение ЭВМ и информационные технологии>>, включая лабораторные работы, курсовые проекты и рубежные контроли. Система предназначена для распределения заданий между студентами различных групп с возможностью разграничения прав доступа. Функциональность системы включает в себя создание, настройку и автоматическое назначение заданий учащимся, отслеживание жизненного цикла задания, а также хранение текстов заданий и репозиториев с работами студентов.
%
%Ключевым аспектом является минимизация трудозатрат преподавателей на развертывание приложений для управления проектами и задачами Redmine и серверов системы контроля версий Git. Это позволит сократить временные затраты и минимизировать ошибки, так как в настоящее время это занимает от двух до трех часов. Система должна обеспечивать управление группами студентов, поддержку прав доступа, а также возможность быстрого развертывания. 

\subsection*{Существующие аналоги}

В настоящее время у \textbf{Платформы} отсутствуют прямые аналоги, так как она разрабатывается с учетом уникальных требований и специфики образовательного процесса МГТУ им. Н. Э. Баумана. Существующие облачные платформы, такие как Timeweb Cloud, Yandex Cloud, хотя и предоставляют базовые возможности для развертывания приложений для управления проектами и задачами Redmine и серверов системы контроля версий Git, не способны обеспечить  интеграцию между друг другом. Поэтому требуется разработка специализированного решения, которое будет учитывать все особенности и потребности учебного процесса.

\subsection*{Описание системы}

Главное назначение \textbf{Платформы}~---~развертывание приложений для управления проектами Redmine и приложений системы контроля версий Git. Экземпляры приложений должны быть интегрированы друг с другом: при загрузке решений заданий состояния задач в Redmine должны изменяться автоматически. Кроме того, экземпляры приложений могут запускаться на разных серверах, \textbf{Платформа} должна управлять распределением ресурсов по серверам. 

На рисунке~\ref{img:topology.pdf} отображена топология \textbf{Платформы}. \textbf{Платформа} состоит из следующих частей:

\begin{enumerate}
	\item сервис заданий;
	\item сервис пользователей;
	\item сервис конфигураций развертывания;
	\item сервис архивации;
	\item сервис развертывания приложения контроля версий;
	\item сервис развертывания Redmine;
	\item сервис проверки работ;
	\item экземпляры приложений контроля версий;
	\item экземпляры Redmine.
\end{enumerate}

\newpage

\img{1\textwidth}{topology.pdf}{Топология \textbf{Платформы}}

\subsubsection*{Сервис заданий}

Часть \textbf{Платформы}, реализующая следующие функции:
\begin{itemize}
    \item хранение информации о заданиях;
    \item создание задач в экземплярах приложений Redmine;
    \item создание репозиториев в приложениях системы контроля версий Git;
    \item выдача заданий студентам в бесконечном конвейере. 
\end{itemize}

Используется СУБД PostgreSQL.

\subsubsection*{Сервис пользователей}

Часть \textbf{Платформы}, которая хранит информацию о пользователях и реализует следующие функции:
\begin{itemize}
	\item аутентификация (проверка сессии) пользователя;
	\item авторизация пользователя (вход, или «логин»);
	\item выход из сессии («логаут»).
\end{itemize}

Используется СУБД PostgreSQL.

\subsubsection*{Сервис конфигураций развертывания}

Часть \textbf{Платформы}, которая хранит информацию об уже развернутых экземлярах приложений и реализует следующие функции:
\begin{itemize}
	\item конфигурирование экземпляров приложений Redmine;
	\item конфигурирование экземпляров приложений контроля версий Git;
	\item формирование очереди задач на развертывание экземпляров приложений для сервисов  развертывания.
\end{itemize}


Используется СУБД PostgreSQL.

\subsubsection*{Сервис архивации}

Часть \textbf{Платформы}, реализующая следующие функции:
\begin{itemize}
	\item архивирование содержимого приложений системы контроля версий Git; 
	\item архивирование содержимого приложений управления задачами Redmine;
	\item восстановление состояния \textbf{Платформы} из архива.
\end{itemize}

Используется файловое хранилище.

\subsubsection*{Сервис развертывания приложения контроля версий}

Часть \textbf{Платформы}, реализующая следующие функции:
\begin{itemize}
	\item развертывание приложений контроля версий;
	\item актуализация состояний экземпляров приложений контроля версий в соответствии с данными сервиса конфигурации;
	\item отключение экземпляров приложений контроля версий.
\end{itemize}

\subsubsection*{Сервис развертывания Redmine}

Часть \textbf{Платформы}, реализующая следующие функции:
\begin{itemize} 
	\item развертывание приложений управления задачами Redmine;
	\item актуализация состояний экземпляров приложений управления задачами Redmine в соответствии с данными сервиса конфигурации;
	\item отключение экземпляров приложений управления задачами Redmine.
\end{itemize}

\subsubsection*{Сервис проверки работ}

Часть \textbf{Платформы}, реализующая следующие функции:
\begin{itemize} 
	\item проверка отчетов студентов на соответствие ГОСТ 7.32-2017 <<СИБИД. Отчет о научно-исследовательской работе. Структура и правила оформления>>;
	\item формирование отчетов с указанием ошибок по результатам проверок.
\end{itemize}

\subsubsection*{Экземпляры приложений контроля версий}

Готовые приложения контроля версий Git с открытым исходным кодом, которые были настроены и развернуты с помощью сервисов конфигурации развертывания и развертывания приложений контроля версий Git.

Используется СУБД PostgreSQL и файловое хранилище.

\subsubsection*{Экземпляры Redmine}

Готовые приложения управления задачами Redmine с открытым исходным кодом, которые были настроены и развернуты с помощью сервисов конфигурации развертывания и развертывания Redmine.

Используется СУБД PostgreSQL.

%Система должна предоставлять следующие возможности: развертывание приложений для управления проектами и задачами Redmine и серверов системы контроля версий Git, авторизация данных пользовательских аккаунтов; выдача заданий студентам по вариантам; проверка оформления отчетов и расчетно-пояснительных записок студентов на основе ГОСТ 7.32-2017 <<СИБИД. Отчет о научно-исследовательской работе. Структура и правила оформления>>; архивация работ и заданий. 

\subsection*{Основания для разработки}

Разработка ведётся в рамках выполнения лабораторных работ по курсу
«Методология программной инженерии» на кафедре «Программное обеспечение ЭВМ и информационные технологии» факультета «Информатика и
системы управления» МГТУ им. Н. Э. Баумана.

\subsection*{Назначение разработки}

Назначение разрабатываемой \textbf{Платформы}~---~обеспечить автоматизированное управление учебными заданиями (лабораторными работами, курсовыми проектами и рубежными контролями) для студентов. Система позволит упростить процессы распределения заданий, администрирование и обеспечить корректное хранение и обработку данных. 

Целевая аудитория~---~студенты и преподаватели
МГТУ им. Н. Э. Баумана.

\subsection*{Требования к системе}

\begin{enumerate}
	\item Разрабатываемое программное обеспечение должно обеспечивать функционирование системы в режиме 24/7/365 со среднегодовым временем доступности не менее 99.9\%. Допустимое время, в течение которого система не доступна, за год должна составлять $24\cdot365\cdot0.001=8.76$ часов.
	\item Время восстановления системы после сбоя не должно превышать 30 минут.
	\item \textbf{Платформа} должна поддерживать возможность «горячего» переконфигурирования системы. Необходимо поддержать возможность добавления нового узла во время работы системы без рестарта.
	\item  \textbf{Платформа} должна обеспечивать разделение пользователей на три роли:
	\begin{itemize} 
		\item \textbf{Администратор};
		\item \textbf{Преподаватель};
		\item \textbf{Студент}.	
	\end{itemize}
	\item \textbf{Платформа} должна обеспечивать аутентификацию пользователей с использованием учетной записи МГТУ им.~Н.~Э.~Баумана (CAS).
\end{enumerate}

\subsection*{Требования к функциональным характеристикам}

\begin{enumerate}
	\item \textbf{Платформа} должна обеспечивать возможность запуска в современных браузерах: не менее 85\% пользователей Интернета должны иметь возможность пользоваться порталом без какой-либо деградации функционала.
	\item Время развертывания нового экземпляра приложения для управления проектами и задачами Redmine или серверы системы контроля версий Git не более 10 минут.
	\item Интервал времени между выдачей задания преподавателем и назначением его студенту составляет не более 5 минут.
	\item \textbf{Платформа} должна обеспечивать время отклика на запрос не более 5 секунд.
\end{enumerate}

\subsection*{Функциональные требования к Платформе с точки зрения Администратора}
\begin{enumerate} 
	\item развертывание приложения для управления проектами и задачами Redmine без участия \textbf{Администратора} системы;
	\item развертывание сервера системы контроля версий Git без участия \textbf{Администратора} системы;
	\item установку статусов (<<выдано>>, <<проверка>>, <<доработка>>, <<выполнено>>) выполнения работы \textbf{Студентов} на основе данных в системе контроля версий;
	\item создание учебных дисциплин;
	\item назначение ролей пользователям;
	\item архивацию данных приложения для управления проектами и задачами Redmine с выгрузкой всех данных в ZIP-архив;
	\item архивацию данных сервера системы контроля версий Git с выгрузкой всех данных в ZIP-архив.
\end{enumerate}

\subsection*{Функциональные требования к Платформе с точки зрения Преподавателя}

\begin{enumerate}
			\item выдача заданий \textbf{Студентам} с автоматическим распределением вариантов;
	\item просмотр состояния выполнения работ \textbf{Студентами}.
\end{enumerate}

\subsection*{Функциональные требования к Платформе с точки зрения Студента}

	\begin{enumerate} 
		\item просмотр выданных заданий по своему варианту;
		\item загрузка выполненных заданий на сервер системы контроля версий Git;
		\item просмотр текущей успеваемости по курсу.
	\end{enumerate}

\subsection*{Входные параметры системы}

\begin{enumerate}
	\item Аккаунт пользователя
	\begin{itemize} 
		\item Имя и фамилия пользователя (не более 256 символов каждое поле).
		\item Учетная запись МГТУ им. Н.~Э. Баумана (логин и пароль);
		\item Электронная почта (в формате example@domain.com).
		\item Пароль (не менее 8 символов, как минимум одна заглавная и одна строчная буква, одна цифра, только латинские символы, без пробелов).
		\item Роль пользователя (администратор, преподаватель, студент).
	\end{itemize}
	\item Данные о заданиях
	\begin{itemize} 
		\item Название задания (не более 256 символов).
		\item Описание задания (не более 4096 символов).
		\item Варианты заданий (не более 256 символов каждый).
		\item Файл с заданием в формате PDF (размер файла не более 10 МБ).
		\item Критерии проверки (например, требования к оформлению отчетов по ГОСТ 7.32-2017).
		\item Сроки выполнения задания (дата и время в формате YYYY-MM-DD HH:MM).
	\end{itemize}
	\item Данные о студентах и дисциплинах
	\begin{itemize} 
		\item Список студентов (имя, фамилия, группа, курс).
		\item Список учебных дисциплин (название дисциплины, код дисциплины).
		\item Распределение студентов по группам и дисциплинам.
	\end{itemize}
	\item Данные о репозиториях
	\begin{itemize} 
		\item Ссылка на репозиторий Git.
		\item Коммиты и запросы на слияния студентов (автор, дата, описание изменений).
		\item Статус выполнения задания (<<выдано>>, <<проверка>>, <<доработка>>, <<выполнено>>).
	\end{itemize}
	\item Данные для проверки отчетов
	\begin{itemize} 
	\item Отчеты и расчетно-пояснительные записки в формате PDF (размер файла не более 10 МБ).
	\item Файлы с выполненными заданиями (размер каждого файла не более 10 МБ).
	\end{itemize}
	\item Конфигурационные данные
		\begin{itemize} 
		\item Настройки для развертывания Redmine (название экземпляра, описание, дата создания).
		\item Настройки для развертывания сервера системы контроля версий Git (название экземпляра, описание, дата создания).
		\item Параметры архивации данных (периодичность, пути сохранения).
	\end{itemize}
\end{enumerate}

\subsection*{Выходные параметры системы}

\begin{enumerate}
	\item Результаты развертывания
	\begin{itemize} 
		\item Статус развертывания приложения Redmine (успешно/неуспешно, время развертывания).
		\item Статус развертывания сервера Git (успешно/неуспешно, время развертывания, сообщения об ошибках).
		\item Ссылки на развернутые приложения (URL для доступа к Redmine и к серверу Git).
	\end{itemize}
	\item Результаты аутентификации
	\begin{itemize} 
		\item Успешная или неуспешная аутентификация пользователя.
		\item Токен доступа для авторизованных пользователей.
	\end{itemize}
	\item Список заданий, распределенных между студентами (название задания, вариант, срок выполнения).
	\item Статусы выполнения заданий (<<выдано>>, <<проверка>>, <<доработка>>, <<выполнено>>).
	\item Результаты проверки отчетов
	\begin{itemize} 
		\item Статус проверки отчета (<<соответствует ГОСТ 7.32-2017>>, <<не соответствует ГОСТ 7.32-2017>>).
		\item Список замечаний по оформлению отчета в формате JSON (См. приложение~\ref{appendix:a}).
	\end{itemize}
	\item Архивированные данные
	\begin{itemize} 
		\item ZIP-архивы с данными Redmine и Git (размер архива не более 10 ГБ).
		\item Логи архивации с указанием времени и статуса выполнения.
	\end{itemize}
	\item Отчеты и аналитика
	\begin{itemize} 
		\item Отчеты о текущей успеваемости студентов (в формате PDF или CSV).
		\item Статистика по выполнению заданий (количество выполненных и невыполненных работ). Подробнее в приложении~\ref{appendix:b}
	\end{itemize}
	\item Уведомления
	\begin{itemize} 
		\item Уведомления на почту для студентов о новых заданиях или изменениях в существующих.
		\item Уведомления на почту для преподавателей о загрузке новых работ или изменении статуса заданий.
	\end{itemize}
\end{enumerate}

\subsection*{Требования к составу и параметрам технических средств}

\begin{enumerate}
	\item Один экземпляр приложения для управления проектами и задачами Redmine должен потреблять не более 8~ГБ оперативной памяти, 40~ГБ дискового пространства и 4 единицы процессорного времени.
	\item Один экземпляр сервера системы контроля версий Git должен потреблять не более 8~ГБ оперативной памяти, 60~ГБ дискового пространства и 8 единиц процессорного времени.
	\item Все серверные приложения должны потреблять суммарно не более 4~ГБ оперативной памяти и 100~ГБ дискового пространства.
\end{enumerate}

\subsection*{Требования к надежности}

\textbf{Платформа} должна работать в соответствии с данным техническим заданием без перезапуска. Необходимо использовать «зеркалируемые серверы» для всех подсистем, которые будут держать нагрузку в случае сбоя до тех пор, пока основной сервер не восстановится.

\subsection*{Требования к документации}


Исполнитель должен подготовить и передать Заказчику следующие документы:

\begin{itemize} 
	\item руководство \textbf{Администратора} по использованию \textbf{Платформы};
	\item руководство \textbf{Преподавателя} по использованию \textbf{Платформы};
	\item руководство \textbf{Студента} по использованию \textbf{Платформы}.
\end{itemize}


\subsection*{Сценарии функционирования Платформы}

\subsubsection*{Авторизация пользователя в административной панели (для Администратора и Преподавателя)}
\begin{enumerate}
\item Пользователь переходит на страницу административной панели.
\item Вводит учётные данные, нажимает кнопку <<Войти>>.
\item Пользователь перенаправляется на страницу административной панели.
\end{enumerate}

\subsubsection*{Авторизация пользователя в административной панели (для Студента)}
\begin{enumerate}
	\item Пользователь переходит на страницу административной панели.
	\item Вводит учётные данные, нажимает кнопку <<Войти>>.
	\item Пользователю выводится сообщение об ошибке о недостаточности прав доступа.
\end{enumerate}

\subsubsection*{Создание нового экземпляра приложения системы контроля версий Git и приложения управления проектами Redmine (для Администратора)}
\begin{enumerate}
\item Администратор переходит на страницу административной панели с созданиям связки приложения системы контроля версий Git и приложения управления проектами Redmine.
\item Администратор вводит названия экземпляров, например \texttt{iu7-oii-2025}.
\item Администратор нажимает кнопку <<Создать>>.
\item После развертывания Администратор в административной панель URL экземпляра приложения системы контроля версий Git и приложения управления проектами Redmine.
\end{enumerate}

\subsubsection*{Выдача задания Студентам (для Преподавателя)}
\begin{enumerate}
\item Преподаватель переходит на страницу административной панели с созданиям заданий.
\item Преподаватель выбирает экземпляр приложения управления проектами Redmine, в котором будет выдано задание.
\item Преподаватель вводит название задания, текст каждого варианта задания, прикрепляет файлы.
\item Преподаватель выбирает группу, которой будет выдано задание.
\item Администратор нажимает кнопку <<Создать>>.
\end{enumerate}

\newpage
