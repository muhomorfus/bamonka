\chapter{Математическая постановка задачи}

Введем следующие обозначения для постановки задачи:

\begin{itemize}
\item $n$~---~количество рабочих узлов;
\item $m$~---~количество задач;
\item $p$~---~количество ресурсов, которые нужно учесть при распределении;
\item $R_{m \times p}$~---~матрица, в которой описаны запросы на ресурсы задачами, $R_{ij}$~---~величина ресурса $j$, запрашиваемого задачей $i$;
\item $A_{n \times p}$~---~матрица, в которой описаны доступные ресурсы узлов, $A_{kj}$~---~величина ресурса $j$, доступного на узле $k$;
\item $S_{m \times n}$~---~распределение задач по узлам (булева матрица), $A_{ik} = 1$~---~задаче $i$ распределена на узел $k$, иначе $0$.
\end{itemize}

Тогда условие неотрицательности запросов на ресурсы можно описать:

\begin{equation}
	R \ge 0.
\end{equation}

Условие неотрицательности доступных на узлах ресурсов можно описать:

\begin{equation}
	A \ge 0.
\end{equation}

На узле может быть запущено несколько задач. Однако одна задача может быть запущена только на одном узле:

\begin{equation}
	\sum_{k = 1}^{n} S_{ik} \le 1, i = \overline{1, m}.
\end{equation}

Условие того, что после распределения задач на узлах останется неотрицательное количество свободных ресурсов:

% Привести пример матрицы

\begin{equation}
	A - S^\mathsf{T} \times R \ge 0.
\end{equation}

Целевая функция:

\begin{equation}
	F(s) = \sum_{i = 1}^{m} \sum_{k = 1}^{n} S_{ik} \rightarrow max.
\end{equation}

Ограничения задачи:

\begin{equation}
	\begin{cases}
		R \ge 0,
		\\
		A \ge 0,
		\\
		\sum_{k = 1}^{n} S_{ik} \le 0, i = \overline{1, m},
		\\
		A - S^\mathsf{T} \times R \ge 0.
	\end{cases}
\end{equation}