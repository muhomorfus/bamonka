\chapter{Сравнение методов}

\section{Выбор критериев для сравнения}

Для сравнения вышеперечисленных методов использовались следующие критерии.

\begin{itemize}
	\item Учет запросов на ресурсы задачей~---~метод не может эффективно планировать задачи, если не учитывает потребности приложения в ресурсах.
	\item Учет доступных ресурсов рабочих узлов~---~метод не может эффективно планировать задачи, если не учитывает доступные на рабочих узлах ресурсы.
	\item Учет доминантного ресурса~---~на практике, методы, не учитывающие доминантный ресурс, могут неэффективно распределять приложения по рабочим узлам.
	\item Возможность перераспределения~---~вычислительные кластеры, в которых запускаются приложения, являются очень динамичной средой, поэтому распределение, оптимальное в какой-либо момент, может перестать таковым быть. Это может быть связано, например, с окончанием работы какого-либо приложения, значительно нагружающего систему. По этой причине, возможность перераспределения является важной составляющей распределения ресурсов.
\end{itemize}

\section{Сравнение методов}

В таблице~\ref{tab:compare} приведено сравнение методов распределения ресурсов в вычислительном кластере.

\begin{table}[H]
\caption{Сравнение методов распределения ресурсов в вычислительном кластере}
\label{tab:compare}
\begin{tabular}{|l|c|c|c|c|}
\hline
Метод                                                                       & \begin{tabular}[c]{@{}c@{}}Учет \\ запросов \\ на ресурсы\end{tabular} & \begin{tabular}[c]{@{}c@{}}Учет \\ доступных \\ ресурсов\end{tabular} & \begin{tabular}[c]{@{}c@{}}Учет \\ доминант-\\ного\\ ресурса\end{tabular} & \begin{tabular}[c]{@{}c@{}}Возможность\\ перераспре-\\деления\end{tabular} \\ \hline
\begin{tabular}[c]{@{}l@{}}Случайное \\ распределение\end{tabular}          & –                                                                      & –                                                                     & –                                                              & –                                                              \\ \hline
\begin{tabular}[c]{@{}l@{}}Наименее \\ запрашиваемый\end{tabular}           & –                                                                      & –                                                                     & –                                                              & –                                                              \\ \hline
\begin{tabular}[c]{@{}l@{}}Наиболее \\ запрашиваемый\end{tabular}           & –                                                                      & –                                                                     & –                                                              & –                                                              \\ \hline
Сбалансированное                                                            & +                                                                      & +                                                                     & –                                                              & –                                                              \\ \hline
\begin{tabular}[c]{@{}l@{}}Сбалансированное \\ с доминантным \\ресурсом\end{tabular} & +                                                                      & +                                                                     & +                                                              & –                                                              \\ \hline
\begin{tabular}[c]{@{}l@{}}Генетический \\ алгоритм\end{tabular}            & +                                                                      & +                                                                     & –                                                              & –                                                              \\ \hline
\begin{tabular}[c]{@{}l@{}}Нейронные \\ сети\end{tabular}                   & +                                                                      & +                                                                     & +                                                              & –                                                              \\ \hline
Выселение                                                                   & –                                                                      & –                                                                     & –                                                              & +                                                              \\ \hline
\end{tabular}
\end{table}

На основании приведенного сравнения было решено выбрать комбинацию методов сбалансированного распределения по доминантному ресурсу и выселения для дальнейшего изучения.