\chapter*{ВВЕДЕНИЕ}
\addcontentsline{toc}{chapter}{ВВЕДЕНИЕ}

На данный момент многие организации выбирают микросервисный подход к разработке информационных систем. Для запуска приложений используется технология контейнеризации~\cite{кочер2022микросервисы}. Системы оркестрации контейнеров, такие как Kubernetes, который фактически стал стандартом индустрии, упрощают развертывание контейнеризированных приложений. Ключевой задачей оркестратора является распределение физических ресурсов вычислительного кластера между контейнерами~\cite{carrion2022kubernetes}.

Целью данной работы является проведение сравнительного анализа оптимизационных методов для задачи распределения ресурсов в вычислительном кластере. Для выполнения этой цели необходимо решить следующие задачи.

\begin{enumerate}
	\item Рассмотреть существующие методы решения задачи распределения ресурсов в вычислительном кластере. 
	\item Выбрать критерии для сравнения и провести сравнительный анализ. 
	\item Выбрать критерии оптимизации работы метода распределения ресурсов в вычислительном кластере. 
	\item Привести формализацию поставленной задачи в виде математической модели.
\end{enumerate}