\begin{definitions}
\begin{itemize}
	\item Микросервис~---~независимо развертываемый компонент, предоставляющий сервис для реализации конкретной функциональной части приложения~\cite{gostms}.
	\item Контейнеризация~---~технология, позволяющая изолировать выполнение процессов между собой. При этом каждый контейнер использует ядро хостовой операционной системы~\cite{luksa2017kubernetes}.
	\item Виртуализация~---~технология, позволяющая изолировать выполнение операционных между собой. Каждая виртуальная машина имеет свой экземпляр ядра~\cite{luksa2017kubernetes}.
	\item Оркестратор~---~система, управляющая набором приложений~\cite{сизов2024организация}.
	\item Kubernetes~---~система для запуска контейнеризированных приложений с открытым исходным кодом~\cite{kubernetes}.
	\item Кластер~---~набор компьютеров, работающих как единое целое~\cite{luksa2017kubernetes}.
	\item Манифест~---~текстовое описание объекта в Kubernetes в формате YAML~\cite{kubernetesresources}.
\end{itemize}
\end{definitions}